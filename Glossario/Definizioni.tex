\usepackage{glossaries}
\usepackage[utf8x]{inputenc}
\usepackage[italian]{babel}

\makeglossaries

\newglossaryentry{implementazione}
{
	name={Implementazione},
	description={Realizzazione concreta di una procedura a partire dalla sua definizione logica},
	nonumberlist 
}
\newglossaryentry{interfaccia grafica}
{
name={Interfaccia Grafica},
description={Parte frontale del software posta tra l'utente e la parte logica per aiutare l' utilizzo del softaware stesso},
nonumberlist 
}
\newglossaryentry{browser}
{
name={Browser},
description={Applicazione per il recupero, la presentazione e la navigazione di risorse},
nonumberlist 
}
\newglossaryentry{file system}
{
name={File System},
description={Nell'interfaccia grafica di un computer, il drag-and-drop indica una successione di tre azioni, consistenti nel cliccare su un oggetto virtuale per trascinarlo in un'altra posizione, dove viene rilasciato.},
nonumberlist 
}
\newglossaryentry{drag and drop}
{
name={Drag and Drop},
description={Nell'interfaccia grafica di un computer, il drag-and-drop indica una successione di tre azioni, consistenti nel cliccare su un oggetto virtuale per trascinarlo in un'altra posizione, dove viene rilasciato.},
nonumberlist 
}
\newglossaryentry{collaudo}
{
	name={Collaudo},
	description={Controllo, verifica dei requisiti tecnologici ed economici in rapporto a una tabella di caratteristiche singolarmente o universalmente prestabilite},
	nonumberlist 
}
\newglossaryentry{proponente}
{
	name={Proponente},
	description={Chi propone e presenta qualcosa a qualcuno affinché venga accettato, approvato},
	nonumberlist 
}
\newglossaryentry{suite}
{
	name={Suite},
	description={Pacchetto di programmi complementari, in grado di interagire e di scambiarsi reciprocamente i dati},
	nonumberlist 
}
\newglossaryentry{debugging}
{
	name={Debugging},
	description={Attività che consiste nell'individuazione e correzione da parte del programmatore di uno o più errori (bug) rilevati nel software},
	nonumberlist 
}
\newglossaryentry{profiling}
{
	name={Profiling},
	description={Attività di raccolta di informazioni su qualcosa al fine di dare una descrizione di ciò},
	nonumberlist 
}
\newglossaryentry{code coverage}
{
	name={Code Coverage},
	description={Misura utilizzata per descrivere il grado di esecuzione del codice sorgente di un programma quando viene eseguita una particolare suite di test},
	nonumberlist 
}
\newglossaryentry{feedback}
{
	name={Feedback},
	description={Valutazione, giudizio sul lavoro o sul comportamento},
	nonumberlist 
}
\newglossaryentry{committente}
{
	name={Committente},
	description={Organizzazione o persona che commissiona un lavoro. Essa detta tempi e costi con i quali il team può portare avanti il lavoro commissionato},
	nonumberlist 
}
\newglossaryentry{LTS}
{
	name={LTS},
	description={Acronimo di Long Term Service è una versione di Ubuntu rilasciata ogni due anni con contenuti e aggiornamenti molto più testati e affidabili delle normali versioni di Ubuntu. Garantisce inoltre gli aggiornamenti per un periodo di almeno 2 anni prima di obbligare l'utente al cambio di versione del sistema operativo},
	nonumberlist 
}
\newglossaryentry{hosting}
{
	name={Hosting},
	description={Servizio di rete che consiste nell'allocare su un server web le pagine di un sito o di un'applicazione web consentendo agli utenti di accedere ad esse tramite la rete internet},
	nonumberlist 
}
\newglossaryentry{processo}
{
	name={Processo},
	description={Insieme di attività correlate che trasformano ingressi in uscite secondo regole fissate},
	nonumberlist 
}
\newglossaryentry{TexStudio}
{
	name={TexStudio},
	description={Editor per la stesura di documenti \LaTeX},
	nonumberlist 
}
\newglossaryentry{Lucidchart}
{
	name={Lucidchart},
	description={Piattaforma web per realizzare diagrammi illustrativi},
	nonumberlist 
}
\newglossaryentry{requisiti}
{
	name={Requisiti},
	description={Condizione richiesta per un determinato scopo},
	nonumberlist 
}
\newglossaryentry{requisito di vincolo}
{
	name={Requisito di Vincolo},
	description={Requisito da rispettare obbligatoriamente},
	nonumberlist 
}
\newglossaryentry{requisito funzionale}
{
	name={Requisito Funzionale},
	description={Requsito che aggiunge funzionalità},
	nonumberlist 
}
\newglossaryentry{requisito prestazionale}
{
	name={Requisito Prestazionale},
	description={Requsito che porta ad un miglioramento di prestazione del software},
	nonumberlist 
}
\newglossaryentry{requisito di qualita}
{
	name={Requisito di Qualita},
	description={Requisito che porta ad un miglioramento della qualità del software},
	nonumberlist 
}
\newglossaryentry{caso d'uso}
{
	name={Casi d'uso},
	description={Insieme di scenari possibili che si possono verificare nel	utilizzo del software},
	nonumberlist 
}
\newglossaryentry{responsabile di progetto}
{
	name={Responsabile di Progetto},
	description={Figura responsabile della gestione di un progetto, più precisamente dell'istanziazione di processi nel progetto, della stima dei costi e delle risorse necessarie, della pianificazione delle attività e della loro assegnazione, del controllo delle attività e verifica dei risultati},
	nonumberlist 
}
\newglossaryentry{analista}
{
	name={Analista},
	description={Figura che, a partire dal bisogno del cliente, individua il problema (di cui conosce il dominio) da fornire al progettista},
	nonumberlist 
}
\newglossaryentry{Dominio Applicativo}
{
	name={Dominio Applicativo},
	description={Contesto in cui una applicazione software opera, soprattutto con riferimento alla natura e al significato delle informazioni che devono essere manipolate},
	nonumberlist 
}
\newglossaryentry{amministratore}
{
	name={Amministratore},
	description={Figura che ha il compito di controllare che ad ogni istante della vita del progetto le risorse (umane, materiali, economiche e strutturali) siano presenti e operanti. inoltre, gestisce la documentazione e controlla il versionamento e la configurazione},
	nonumberlist 
}
\newglossaryentry{verificatore}
{
	name={Verificatore},
	description={Colui che verifica, addetto alla verifica dei prodotti(es: documenti, software)},
	nonumberlist 
}
\newglossaryentry{verifica}
{
	name={Verifica},
	description={Essa attiene alla coerenza, completezza e correttezza del prodotto. È un processo che si applica ad ogni "segmento" temporale di un progetto (ad ogni prodotto intermedio) per accertare che le attività svolte in tale segmento non abbiano introdotto errori nel prodotto
	},
	nonumberlist 
}
\newglossaryentry{validazione}
{
	name={Validazione},
	description={Essa accerta la conformità di un prodotto alle attese, fornisce una prova oggettiva di come le specifiche del prodotto siano conformi al suo scopo e alle esigenze degli utenti},
	nonumberlist 
}
\newglossaryentry{requisito}
{
	name={Requisito},
	description={Vincolo da rispettare o bisogno da soddisfare},
	nonumberlist 
}
\newglossaryentry{manuale utente}
{
	name={Manuale utente},
	description={Documento destinato all'utilizzatore finale del prodotto, che si presuppone privo di competenze tecniche specifiche. Contiene le informazioni utili al corretto utilizzo di un prodotto},
	nonumberlist 
}
\newglossaryentry{programmatore}
{
	name={Programmatore},
	description={Chi è incaricato della stesura dei programmi per calcolatori sulla base delle specifiche di programma assegnate dai progettisti},
	nonumberlist 
}
\newglossaryentry{piano di qualifica}
{
	name={Piano di Qualifica},
	description={Documento che dovrà illustrare la strategia complessiva di verifica e validazione proposta dal fornitore per pervenire al collaudo del sistema con la massima efficienza ed efficacia},
	nonumberlist 
}
\newglossaryentry{qualita}
{
	name={Qualita},
	description={Insieme di caratteristiche di un entità che ne comportano la capacità di soddisfare richieste esplicite o implicite di un committente},
	nonumberlist 
}
\newglossaryentry{Telegram}
{
	name={Telegram},
	description={Servizio di messaggistica istantanea basato su cloud},
	nonumberlist 
}
\newglossaryentry{dispositivo mobile}
{
	name={Dispositivo Mobile},
	description={Tutti quei dispositivi elettronici che sono pienamente utilizzabili seguendo la mobilità dell'utente},
	nonumberlist 
}
\newglossaryentry{Wrike}
{
	name={Wrike},
	description={Applicazione web per il project management},
	nonumberlist 
}
\newglossaryentry{task}
{
	name={Task},
	description={Attività da svolgere entro una scadenza prefissata. Essa può essere suddivisa in più compiti o subtask anche essi soggetti a scadenza il cui insieme porta al completamento del task},
	nonumberlist 
}
\newglossaryentry{GitHub}
{
	name={GitHub},
	description={Servizio di hosting per progetti software},
	nonumberlist 
}
\newglossaryentry{Git}
{
	name={Git},
	description={Software di controllo di versione distribuito},
	nonumberlist
}
\newglossaryentry{Slack}
{
	name={Slack},
	description={Strumento di collaborazione aziendale utilizzato per inviare messaggi in modo istantaneo ai membri del team},
	nonumberlist
}
\newglossaryentry{Gitkraken}
{
	name={Gitkraken},
	description={Interfaccia grafica per la piattaforma Git che agevola all'utente le operazioni da eseguire sulla repository},
	nonumberlist
}
\newglossaryentry{Repository}
{
	name={Repository},
	description={Archivi web nei quali vengono raggruppati file e documenti condivisibili con un gruppo di persone},
	nonumberlist
}
\newglossaryentry{casi d'uso}
{
	name={casi d'uso},
	description={Lista di azione o passi nei processi di ingegneria del software per descrivere, in maniera esaustiva e non ambigua, la raccolta dei requisiti al fine di produrre software di qualità},
	nonumberlist
}
\newglossaryentry{capitolato}
{
	name={Capitolato},
	description={Documento reddatto dal committente in cui vengono elencate le caratteristiche del prodotto richiesto},
	nonumberlist
}
\newglossaryentry{prodotto}
{
	name={Prodotto},
	description={Software richiesto dal committente},
	nonumberlist
}
\newglossaryentry{frontend grafico}
{
	name={Frontend Grafico},
	description={Interfaccia presentata all'utente che permette l'interazione con la logica del programma},
	nonumberlist
}
\newglossaryentry{plugin}
{
	name={Plugin},
	description={Componente software che aggiunge una specifica feature ad un programma preesistente},
	nonumberlist
}
\newglossaryentry{file .json}
{
	name={File .json},
	description={Formato di file leggero adatto allo scambio dati, autodescrittivo e indipendente dalla lingua usata},
	nonumberlist
}
\newglossaryentry{Speect}
{
	name={Speect},
	description={Sistema di traduzione da testo a voce multilingua},
	nonumberlist
}
\newglossaryentry{utterance}
{
	name={Utterance},
	description={Dato di passaggio tra gli stati di Speect},
	nonumberlist
}

\newglossaryentry{utterance type}
{
name={Utterance Type},
description={Definizione di una pipeline di utterance processor},
nonumberlist
}

\newglossaryentry{utterance processor}
{
	name={Utterance Processor},
	description={Unità di analisi del Utterance. Essa cambia a seconda del tipo di input, della lingua e della voce con la quale si vuole sintetizzare},
	nonumberlist
}

\newglossaryentry{Linux Ubuntu}
{
	name={Linux Ubuntu},
	description={Sistema operativo per computer},
	nonumberlist
}
\newglossaryentry{CMAKE}
{
	name={CMAKE},
	description={Software di gestione che facilita la compilazione e i test di un software},
	nonumberlist
}
\newglossaryentry{GCC}
{
	name={GCC},
	description={Compilatore prodotto da GNU Project supportato da vari linguaggi di programmazione},
	nonumberlist
}
\newglossaryentry{QT}
{
	name={QT},
	description={Strumento per la creazione di interfaccie grafiche multipiattaforma},
	nonumberlist
}
\newglossaryentry{attivita}
{
	name={attivita},
	description={Ciò che voglio fare per attuare un processo},
	nonumberlist
}
\newglossaryentry{consuntivo}
{
	name={consuntivo},
	description={Rendiconto finale di un periodo di attività},
	nonumberlist
}
\newglossaryentry{grafo HRG}
{
	name={grafo HRG},
	description={Heterogeneous Relation Graph: struttura dati per memorizzare le informazioni di un utterance in Speect},
	nonumberlist
}
\newglossaryentry{modello incrementale}
{
	name={modello incrementale},
	description={Modello di sviluppo di un progetto software basato sulla successione dei seguenti passi principali:
	\begin{itemize}
    	\item pianificazione;
    	\item analisi dei requisiti;
   		\item progetto;
   		\item implementazione;
   		\item prove;
   		\item valutazione.
	\end{itemize}
	Questo ciclo può essere ripetuto diverse volte, denominate "iterazioni", fino a che la valutazione del prodotto diviene soddisfacente rispetto ai requisiti richiesti.},
	nonumberlist
}
\newglossaryentry{file Voice}
{
	name={file Voice},
	description={File di inizializzazione di Speect},
	nonumberlist
}
\newglossaryentry{WAV}
{
	name={WAV},
	description={Tipologia di file per salvare un contenuto audio},
	nonumberlist
}
\newglossaryentry{Agile}
{
	name={Agile},
	description={Insieme di metodi di sviluppo del software emersi a partire dai
		primi anni 2000 e fondati su insieme di principi comuni, direttamente o
		indirettamente derivati dai principi del "Manifesto per lo sviluppo agile del
		software". Tale manifesto si può riassumere in quattro punti:
		\begin{enumerate}
			\item le persone e le interazioni sono più importanti dei processi e degli strumenti;
			\item è più importante avere software funzionante che documentazione;
			\item bisogna collaborare con i clienti oltre che rispettare il contratto;
			\item bisogna essere pronti a rispondere ai cambiamenti oltre che aderire alla pianificazione;
		\end{enumerate}
	},
	nonumberlist
}
\newglossaryentry{open source}
{
	name={open source},
	description={Software non protetto da copyright e liberamente modificabile dagli utenti.},
	nonumberlist
}
\newglossaryentry{Dominio Tecnologico}
{
	name={Dominio Tecnologico},
	description={Insieme di tecnologie da impiegare nello sviluppo del prodotto software},
	nonumberlist
}
\newglossaryentry{ISO/IEC 15504}
{
	name={ISO/IEC 15504},
	description={Insieme di documenti tecnici standard per il processo di sviluppo di software per computer e le relative funzioni di gestione aziendale},
	nonumberlist
}
\newglossaryentry{indice Gulpease}
{
	name={indice Gulpease},
	description={Indice di leggibilità di un testo tarato sulla lingua italiana},
	nonumberlist
}
\newglossaryentry{sviluppo}
{
	name={sviluppo},
	description={Attività o una serie di attività mirate a costruire un programma},
	nonumberlist
}
\newglossaryentry{stakeholders}
{
	name={stakeholders},
	description={Soggetto (o un gruppo di soggetti) influente nei confronti di un'iniziativa economica, che sia un'azienda o un progetto},
	nonumberlist
}
\newglossaryentry{Design Pattern}
{
	name={Design Pattern},
	description={Concetto che può essere definito "una soluzione progettuale generale ad un problema ricorrente". Si tratta di una descrizione o modello logico da applicare per la risoluzione di un problema che può presentarsi in diverse situazioni durante le fasi di progettazione e sviluppo del software, ancor prima della definizione dell'algoritmo risolutivo della parte computazionale},
	nonumberlist
}
\newglossaryentry{UML}
{
	name={UML},
	description={Linguaggio di modellizzazione e specifica basato sul paradigma orientato agli oggetti. E' usato per descrivere soluzioni analitiche e progettuali in modo sintetico e comprensibile a un vasto pubblico},
	nonumberlist
}
\newglossaryentry{deployment}
{
	name={deployment},
	description={Consegna o rilascio al cliente, con relativa installazione e messa in funzione o esercizio, di una applicazione o di un sistema software tipicamente all'interno di un sistema informatico aziendale},
	nonumberlist
}
\newglossaryentry{package}
{
	name={package},
	description={Collezione di classi e interfacce correlate},
	nonumberlist
}
\newglossaryentry{Google Drive}
{
	name={Google Drive},
	description={Servizio, in ambiente cloud computing, di memorizzazione e sincronizzazione online. Il servizio comprende il file hosting, il file sharing e la modifica collaborativa di documenti },
	nonumberlist
}
\newglossaryentry{issue}
{
	name={issue},
	description={Unità di lavoro per realizzare un miglioramento in un sistema di dati},
	nonumberlist
}
\newglossaryentry{commit}
{
	name={commit},
	description={Insieme di modifiche correlate in un repository},
	nonumberlist
}
\newglossaryentry{baseline}
{
	name={baseline},
	description={Versione formalmente approvata di un elemento di configurazione, indipendentemente dal supporto, formalmente designato e fissato in un momento specifico durante il ciclo di vita dell'elemento di configurazione},
	nonumberlist
}
\newglossaryentry{PDF}
{
	name={PDF},
	description={Formato di file basato su un linguaggio di descrizione di pagina sviluppato da Adobe Systems nel 1993 per rappresentare documenti in modo indipendente dall'hardware e dal software utilizzati per generarli o per visualizzarli},
	nonumberlist
}
\newglossaryentry{Valgrind}
{
	name={Valgrind},
	description={Strumento per il debug di problemi di memoria, la ricerca dei memory leak ed il profiling del software. È un software libero scritto in linguaggio C per i sistemi operativi GNU/Linux},
	nonumberlist
}
\newglossaryentry{SonarQube}
{
	name={SonarQube},
	description={Piattaforma open source per il controllo continuo della qualità del codice per eseguire revisioni automatiche con analisi statiche del codice per rilevare bug, odori di codice e vulnerabilità di sicurezza},
	nonumberlist
}
\newglossaryentry{Better Code Hub}
{
	name={Better Code Hub},
	description={Servizio di analisi del codice sorgente web-based che controlla il codice per la conformità},
	nonumberlist
}
\newglossaryentry{IDE}
{
	name={IDE},
	description={Ambiente di sviluppo integrato per la realizzazione di programmi per sistemi informatici},
	nonumberlist
}
\newglossaryentry{Qt}
{
	name={Qt},
	description={Libreria multipiattaforma per lo sviluppo di programmi con interfaccia grafica tramite l'uso di widget (congegni o elementi grafici)},
	nonumberlist
}
\newglossaryentry{Cmake}
{
	name={Cmake},
	description={Software libero multipiattaforma per l'automazione dello sviluppo il cui nome è un'abbreviazione di cross platform make},
	nonumberlist
}
\newglossaryentry{Qt Creator}
{
	name={Qt Creator},
	description={Ambiente di sviluppo integrato multipiattaforma C ++, JavaScript e QML che fa parte dell'SDK per il framework di sviluppo dell'applicazione Qt GUI. Include un debugger visivo e un layout GUI integrato e un designer di moduli.},
	nonumberlist
}
\newglossaryentry{Tender}
{
	name={Tender},
	description={Applicativo web per la gestione dei requisiti ed i casi d'uso },
	nonumberlist
}
\newglossaryentry{Astah}
{
	name={Astah},
	description={Strumento di modellazione UML},
	nonumberlist
}