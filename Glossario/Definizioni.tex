
\usepackage{glossaries}
\usepackage[utf8x]{inputenc}
\usepackage[italian]{babel}

\makeglossaries

\newglossaryentry{implementazione}
{
	name={Implementazione},
	description={la realizzazione concreta di una procedura a partire dalla sua definizione logica},
	nonumberlist 
}
\newglossaryentry{collaudo}
{
	name={Collaudo},
	description={controllo, verifica dei requisiti tecnologici ed economici in rapporto a una tabella di caratteristiche singolarmente o universalmente prestabilite},
	nonumberlist 
}
\newglossaryentry{proponente}
{
	name={Proponente},
	description={chi propone e presenta qualcosa a qualcuno affinché venga accettato, approvato},
	nonumberlist 
}
\newglossaryentry{suite}
{
	name={Suite},
	description={pacchetto di programmi complementari, in grado di interagire e di scambiarsi reciprocamente i dati},
	nonumberlist 
}
\newglossaryentry{debugging}
{
name={Debugging},
description={l'attività che consiste nell'individuazione e correzione da parte del programmatore di uno o più errori (bug) rilevati nel software},
nonumberlist 
}
\newglossaryentry{profiling}
{
name={Profiling},
description={l'attività di raccolta di informazioni su qualcosa al fine di dare una descrizione di ciò},
nonumberlist 
}
\newglossaryentry{code coverage}
{
name={Code Coverage},
description={una misura utilizzata per descrivere il grado di esecuzione del codice sorgente di un programma quando viene eseguita una particolare suite di test},
nonumberlist 
}
\newglossaryentry{feedback}
{
name={Feedback},
description={valutazione, giudizio sul lavoro o sul comportamento},
nonumberlist 
}
\newglossaryentry{committente}
{
name={Committente},
description={organizzazione o persona che commissiona un lavoro. Essa detta tempi e costi con i quali il team può portare avanti il lavoro commissionato},
nonumberlist 
}
\newglossaryentry{LTS}
{
name={LTS},
description={acronimo di Long Term Service è una versione di Ubuntu rilasciata ogni due anni con contenuti e aggiornamenti molto più testati e affidabili delle normali versioni di Ubuntu. Garantisce inoltre gli aggiornamenti per un periodo di almeno 2 anni prima di obbligare l'utente al cambio di versione del sistema operativo},
nonumberlist 
}
\newglossaryentry{hosting}
{
name={Hosting},
description={servizio di rete che consiste nell'allocare su un server web le pagine di un sito o di un'applicazione web consentendo agli utenti di accedere ad esse tramite la rete internet},
nonumberlist 
}
\newglossaryentry{processo}
{
name={Processo},
description={Insieme di attività correlate che trasformano ingressi in uscite secondo regole fissate},
nonumberlist 
}
\newglossaryentry{TexStudio}
{
name={TexStudio},
description={editor per la stesura di documenti \LaTeX},
nonumberlist 
}
\newglossaryentry{Lucidchart}
{
name={Lucidchart},
description={piattaforma web per realizzare diagrammi illustrativi},
nonumberlist 
}
\newglossaryentry{requisiti}
{
name={Requisiti},
description={condizione richiesta per un determinato scopo},
nonumberlist 
}
\newglossaryentry{requisito di vincolo}
{
name={Requisito di Vincolo},
description={è un requisito da rispettare obbligatoriamente},
nonumberlist 
}
\newglossaryentry{requisito funzionale}
{
name={Requisito Funzionale},
description={è un requsito che aggiunge funzionalità},
nonumberlist 
}
\newglossaryentry{requisito prestazionale}
{
name={Requisito Prestazionale},
description={è un requsito che porta ad un miglioramento di prestazione del software},
nonumberlist 
}
\newglossaryentry{requisito di qualita}
{
	name={Requisito di Qualità},
	description={è un requisito che porta ad un miglioramento della qualità del software},
	nonumberlist 
}
\newglossaryentry{caso d'uso}
{
name={Casi d'uso},
description={insieme di scenari possibili che si possono verificare nel	utilizzo del software},
nonumberlist 
}
\newglossaryentry{responsabile di progetto}
{
name={Responsabile di Progetto},
description={è la figura responsabile della gestione di un progetto, più precisamente dell'istanziazione di processi nel progetto, della stima dei costi e delle risorse necessarie, della pianificazione delle attività e della loro assegnazione, del controllo delle attività e verifica dei risultati},
nonumberlist 
}
\newglossaryentry{analista}
{
name={Analista},
description={è la figura che, a partire dal bisogno del cliente, individua il problema (di cui conosce il dominio) da fornire al progettista},
nonumberlist 
}
\newglossaryentry{dominio applicativo}
{
name={Dominio Applicativo},
description={il contesto in cui una applicazione software opera, soprattutto con riferimento alla natura e al significato delle informazioni che devono essere manipolate},
nonumberlist 
}
\newglossaryentry{amministratore}
{
name={Amministratore},
description={è la figura che ha il compito di controllare che ad ogni istante della vita del progetto le risorse (umane, materiali, economiche e strutturali) siano presenti e operanti. inoltre, gestisce la documentazione e controlla il versionamento e la configurazione},
nonumberlist 
}
\newglossaryentry{verificatore}
{
name={Verificatore},
description={colui che verifica, addetto alla verifica dei prodotti(es: documenti, software)},
nonumberlist 
}
\newglossaryentry{verifica}
{
name={Verifica},
description={essa attiene alla coerenza, completezza e correttezza del prodotto. È un processo che si applica ad ogni "segmento" temporale di un progetto (ad ogni prodotto intermedio) per accertare che le attività svolte in tale segmento non abbiano introdotto errori nel prodotto
},
nonumberlist 
}
\newglossaryentry{validazione}
{
name={Validazione},
description={essa accerta la conformità di un prodotto alle attese, fornisce una prova oggettiva di come le specifiche del prodotto siano conformi al suo scopo e alle esigenze degli utenti},
nonumberlist 
}
\newglossaryentry{requisito}
{
name={Requisito},
description={vincolo da rispettare o bisogno da soddisfare},
nonumberlist 
}
\newglossaryentry{manuale utente}
{
name={Manuale utente},
description={documento destinato all'utilizzatore finale del prodotto, che si presuppone privo di competenze tecniche specifiche. Contiene le informazioni utili al corretto utilizzo di un prodotto},
nonumberlist 
}
\newglossaryentry{programmatore}
{
name={Programmatore},
description={chi è incaricato della stesura dei programmi per calcolatori sulla base delle specifiche di programma assegnate dai progettisti},
nonumberlist 
}
\newglossaryentry{piano di qualifica}
{
name={Piano di Qualifica},
description={documento che dovrà illustrare la strategia complessiva di verifica e validazione proposta dal fornitore per pervenire al collaudo del sistema con la massima efficienza ed efficacia},
nonumberlist 
}
\newglossaryentry{qualita}
{
name={Qualità},
description={insieme di caratteristiche di un entità che ne comportano la capacità di soddisfare richieste esplicite o implicite di un committente},
nonumberlist 
}
\newglossaryentry{Telegram}
{
name={Telegram},
description={servizio di messaggistica istantanea basato su cloud},
nonumberlist 
}
\newglossaryentry{dispositivo mobile}
{
name={Dispositivo Mobile},
description={tutti quei dispositivi elettronici che sono pienamente utilizzabili seguendo la mobilità dell'utente},
nonumberlist 
}
\newglossaryentry{Wrike}
{
name={Wrike},
description={applicazione web per il project management},
nonumberlist 
}
\newglossaryentry{task}
{
name={Task},
description={attività da svolgere entro una scadenza prefissata. Essa può essere suddivisa in più compiti o subtask anche essi soggetti a scadenza il cui insieme porta al completamento del task},
nonumberlist 
}
\newglossaryentry{GitHub}
{
name={GitHub},
description={servizio di hosting per progetti software},
nonumberlist 
}
\newglossaryentry{Git}
{
name={Git},
description={software di controllo di versione distribuito},
nonumberlist
}
\newglossaryentry{Slack}
{
name={Slack},
description={strumento di collaborazione aziendale utilizzato per inviare messaggi in modo istantaneo ai membri del team},
nonumberlist
}
\newglossaryentry{Gitkraken}
{
name={Gitkraken},
description={interfaccia grafica per la piattaforma Git che agevola all'utente le operazioni da eseguire sulla repository},
nonumberlist
}
\newglossaryentry{Repository}
{
name={Repository},
description={archivi web nei quali vengono raggruppati file e documenti condivisibili con un gruppo di persone},
nonumberlist
}

\newglossaryentry{casi d'uso}
{
	name={casi d'uso},
	description={lista di azione o passi nei processi di ingegneria del software per descrivere, in maniera esaustiva e non ambigua, la raccolta dei requisiti al fine di produrre software di qualità},
	nonumberlist
}


\newglossaryentry{capitolato}
{
	name={Capitolato},
	description={documento reddatto dal committente in cui vengono elencate le caratteristiche del prodotto richiesto},
	nonumberlist
}

\newglossaryentry{prodotto}
{
	name={Prodotto},
	description={software richiesto dal committente},
	nonumberlist
}

\newglossaryentry{frontend grafico}
{
	name={Frontend Grafico},
	description={interfaccia presentata all'utente che permette l'interazione con la logica del programma},
	nonumberlist
}

\newglossaryentry{plugin}
{
	name={Plugin},
	description={componente software che aggiunge una specifica feature ad un programma preesistente},
	nonumberlist
}

\newglossaryentry{file .json}
{
	name={File .json},
	description={formato di file leggero adatto allo scambio dati, autodescrittivo e indipendente dalla lingua usata},
	nonumberlist
}

\newglossaryentry{Speect}
{
name={Speect},
description={sistema di traduzione da testo a voce multilingua},
nonumberlist
}

\newglossaryentry{utterance}
{
name={Utterance},
description={dato di passaggio tra gli stati di Speect},
nonumberlist
}



\newglossaryentry{Linux Ubuntu}
{
	name={Linux Ubuntu},
	description={Sistema operativo per computer},
	nonumberlist
}

\newglossaryentry{CMAKE}
{
	name={CMAKE},
	description={Software di gestione che facilita la compilazione e i test di un software},
	nonumberlist
}

\newglossaryentry{GCC}
{
	name={GCC},
	description={Compilatore prodotto da GNU Project supportato da vari linguaggi di programmazione},
	nonumberlist
}

\newglossaryentry{QT}
{
	name={QT},
	description={Strumento per la creazione di interfaccie grafiche multipiattaforma},
	nonumberlist
}

\newglossaryentry{attività}
{
name={attività},
description={ciò che voglio fare per attuare un processo},
nonumberlist
}
\newglossaryentry{consuntivo}
{
name={consuntivo},
description={rendiconto finale di un periodo di attività},
nonumberlist
}
\newglossaryentry{grafo HRG}
{
name={grafo HRG},
description={Heterogeneous Relation Graph: struutura dati per memorizzare le informazioni di un utterance in Speect},
nonumberlist
}
\newglossaryentry{modello incrementale}
{
name={modello incrementale},
description={modello di sviluppo di un progetto software basato sulla successione dei seguenti passi principali:
\begin{itemize}
    \item pianificazione;
    \item analisi dei requisiti;
    \item progetto;
    \item implementazione;
    \item prove;
    \item valutazione.
\end{itemize}
Questo ciclo può essere ripetuto diverse volte, denominate "iterazioni", fino a che la valutazione del prodotto diviene soddisfacente rispetto ai requisiti richiesti.},
nonumberlist
}
\newglossaryentry{prodotto}
{
name={prodotto},
description={risultato dell'attività di progetto, in questo caso software},
nonumberlist
}
