\documentclass[openany,12pt,a4paper]{report}
\usepackage{StileDoc}
\usepackage{hyperref}

\title{Studio di fattibilità}
\author{Matteo Rizzo}
\date{\today}

\begin{document}

\maketitle

% INDICE
 
\tableofcontents{}

% INTRODUZIONE

\chapter{Introduzione}

\section{Scopo del documento}

Lo studio di fattibilità ha l’obiettivo di descrivere le motivazioni che hanno portato
il gruppo alla scelta del  \textit{capitolato\ped{G}} C3 e alla relativa esclusione degli altri
capitolati proposti.

\section{Ambiguità}

Al fine di evitare ogni ambiguità relativa al linguaggio utilizzato nei documenti, viene
fornito il Glossario v1.0.0, contenente la definizione dei termini in corsivo marcati con
il pedice "G".

\section{Riferimenti normativi}

\begin{itemize}

    \item{Norme di progetto:} documento \textit{Norme di progetto v1.0.0}.

\end{itemize}

\section{Riferimenti informativi}

\begin{itemize}
    \item{Capitolato d'appalto C1:} AJarvis - Assistente Virtuale Cerimonie Agile \\ \url{http://www.math.unipd.it/~tullio/IS-1/2017/Progetto/C1.pdf}
    
    \item{Capitolato d'appalto C2:} BlockCV: blockchain per gestione di CV certificati \\ \url{http://www.math.unipd.it/~tullio/IS-1/2017/Progetto/C2.pdf}
    
    \item{Capitolato d'appalto C3:} DeSpeect: interfaccia grafica per Speect \\ \url{http://www.math.unipd.it/~tullio/IS-1/2017/Progetto/C3.pdf}
    
    \item{Capitolato d'appalto C4:} ECoRe: Enterprise Content Recommendation \\ \url{http://www.math.unipd.it/~tullio/IS-1/2017/Progetto/C4.pdf}
    
    \item{Capitolato d'appalto C5:} Ironworks \\ \url{http://www.math.unipd.it/~tullio/IS-1/2017/Progetto/C5.pdf}
    
    \item{Capitolato d'appalto C6:} Marvin: dimostratore di Uniweb su Ethereum \\ \url{http://www.math.unipd.it/~tullio/IS-1/2017/Progetto/C6.pdf}
    
    \item{Capitolato d'appalto C7:} OpenAPM: cruscotto di Application Performance Management \\ \url{http://www.math.unipd.it/~tullio/IS-1/2017/Progetto/C7.pdf}
    
    \item{Capitolato d'appalto C8:} TuTourSelf: piattaforma di prenotazioni per artisti in tournee \\ \url{http://www.math.unipd.it/~tullio/IS-1/2017/Progetto/C8.pdf}
\end{itemize}

% CAPITOLATO SCELTO

\chapter{Capitolato scelto}

\section{Capitolato C3}

\begin{itemize}
    \item{Nome:} DeSpeect: interfaccia grafica per Speect;
    \item{Proponente:} MIVOQ S.R.L.;
    \item{Commitente:} Prof.Tullio Vardanega e Prof. Riccardo Cardin.
\end{itemize}

\subsection{Descrizione}

\subsection{Dominio applicativo}

\subsection{Dominio tecnologico}

\begin{itemize}
    \item{};
    
    \item{}.
\end{itemize}

\subsection{Aspetti positivi}

\begin{itemize}
    \item ;
    
    \item  .
\end{itemize}
\subsection{Fattori di rischio}

\begin{itemize}
    \item ;
    
    \item .
\end{itemize}

\subsection{Valutazione finale}

\chapter{Analisi degli altri capitolati}

% CAPITOLATO 1

\section{Capitolato C1}

\begin{itemize}
    \item{Nome:} AJarvis - Assistente Virtuale Cerimonie Agile;
    \item{Proponente:} Zero12;
    \item{Commitente:} Prof.Tullio Vardanega e Prof. Riccardo Cardin.
\end{itemize}

\subsection{Descrizione}

Produzione di un’applicazione di Machine Learning in grado di ascoltare gli standup giornalieri sullo stato di avanzamento dei progetti, comprenderne i dialoghi, analizzarne il contenuto per fornire un’analisi dello standup estraendo dal contesto gli argomenti emersi.

\subsection{Dominio applicativo}

Il progetto si inserisce nell'ambito dell'automatizzazione della creazione di report sulle riunioni standup aziendali nel contesto della progettazione \textit{Agile\ped{G}}.

\subsection{Dominio tecnologico}

\begin{itemize}
    \item{Interfacce Web di registrazione e di reportistica delle analisi:} HTML5, CSS3, Javascript, AngularJS, PHP;
    
    \item{Servizi Cloud:} tecnologie Google Cloud Platform per l’analisi dei dati e traduzione di voce in testo.
\end{itemize}

\subsection{Aspetti positivi}

\begin{itemize}
    \item Proposta d'uso di tecnologie moderne e innovative;
    
    \item Possibilità di applicazione in un ambito aziendale concreto.
\end{itemize}
\subsection{Fattori di rischio}

\begin{itemize}
    \item Complessità elevata nel produrre un prodotto di qualità utilizzando le tecnologie richieste, in particolare nel riconoscimento di più voci sovrapposte;
    
    \item Il gruppo ritiene significativo che un prodotto così interessante sia condiviso sotto licenze \textit{open source\ped{G}}, ma così non è.
\end{itemize}

\subsection{Valutazione finale}

Nonostante l'interesse suscitato dalle tecnologie proposte, i loro limiti intrinseci, la gravosità dei fattori di rischio e l'alta competizione nell'appalto ci spingono alla scelta di un altro capitolato.

% CAPITOLATO 2

\section{Capitolato C2}

\begin{itemize}
    \item{Nome:} BlockCV: blockchain per gestione di CV certificati;
    \item{Proponente:} Ifin Sistemi Srl;
    \item{Commitente:} Prof.Tullio Vardanega e Prof. Riccardo Cardin.
\end{itemize}

\subsection{Descrizione}

Creazione di un sistema distribuito per la pubblicazione di CV certificati e per la ricerca di proposte di lavoro basato su una permissioned blockchain.

\subsection{Dominio applicativo}

Si vuole ottenere un modello applicabile nell'attuale sistema lavorativo, in grado di gestire le operazioni di base che accompagnano il lavoratore dalla fase iniziale di ricerca di una prima occupazione con conseguente creazione e pubblicazione di un CV, seguendone via via l’evoluzione della carriera attraverso i vari aggiornamenti del CV, piuttosto che l’iscrizione a proposte di lavoro e successivi cambi di occupazione.

\subsection{Dominio tecnologico}

\begin{itemize}
    \item{Piattaforma blockchain:} Hyperledger Fabric, MongoDB, Cassandra;
    
    \item{Codifica:} Java EE;
    
    \item{Interfaccia grafica:} Framework Play o suite di componenti Vaadin Elements.
\end{itemize}

\subsection{Aspetti positivi}

\begin{itemize}
    \item Utilizzo di una tecnologia innovativa e in forte sviluppo;
    
    \item Obiettivo del progetto interessante e con forti risvolti pratici.
\end{itemize}

\subsection{Fattori di rischio}

\begin{itemize}
    \item Tecnologia reputata piuttosto complessa da padroneggiare adeguatamente;
    
    \item Possibili difficoltà nell'interazione con gli enti che andrebbero a certificare le competenze curricolari.
\end{itemize}

\subsection{Valutazione finale}

Nonostante l'interesse suscitato dalle tecnologie proposte, l'elevata complessità del progetto e la dubbia efficacia del sistema di certificazione delle competenze curricolari da parte dell'ente erogatore ci spingono alla scelta di un altro capitolato.

% CAPITOLATO 4

\section{Capitolato C4}

\begin{itemize}
    \item{Nome:} ECoRe: Enterprise Content Recommendation;
    \item{Proponente:} SIAV SPA;
    \item{Commitente:} Prof.Tullio Vardanega e Prof. Riccardo Cardin.
\end{itemize}

\subsection{Descrizione}

Realizzazione di un servizio proattivo in grado di suggerire all’utente che accede a contenuti aziendali altri contenuti di interesse che potrebbero essere utili nello svolgimento del proprio lavoro. Tale utilità sarà stabilita sulla base del comportamento dell’utente stesso.

\subsection{Dominio applicativo}

Il progetto si inserisce nell'ambito dell'enterprise search, ovvero dell’applicazione delle conoscenze e dei metodi del reperimento dell’informazione nel contesto della ricerca di informazioni all’interno di una organizzazione aziendale, e dei sistemi di raccomandazione, strumenti software in grado di suggerire contenuti utili all’utente.

\subsection{Dominio tecnologico}

\begin{itemize}
    \item{Motore di ricerca:} Apache SolR o ElasticSearch;
    
    \item{Libreria di apprendimento automatico con sezione dedicata alla raccomandazione:} Apache Mahout;
    
    \item{Misure di sicurezza:} Identity and Access Management Keycloak;
    
    \item{Importazione di dati provenienti da fonti web:} Apache Nutch;
    
    \item{Documentazione:} Evernote.
\end{itemize}

\subsection{Aspetti positivi}

\begin{itemize}
    \item Progetto ambizioso che porterebbe alla realizzazione di un prodotto estremamente utile in ambito aziendale e non;
    
    \item Esplorazione di un gran numero di tecnologie diverse.
\end{itemize}
\subsection{Fattori di rischio}

\begin{itemize}
    \item Il numero di tecnologie richieste dalla realizzazione del progetto è molto elevato e la loro natura variegata;
    
    \item Lo use case del progetto è volutamente generico e non propriamente definito.
\end{itemize}

\subsection{Valutazione finale}

La notevole vastità di tecnologie richieste per la realizzazione del progetto e l'estrema generalità dello use case dello stesso ci spingono alla scelta di un altro capitolato.

% CAPITOLATO 5

\section{Capitolato C5}

\begin{itemize}
    \item{Nome:} Ironworks;
    \item{Proponente:} Zucchetti Software srl;
    \item{Commitente:} Prof.Tullio Vardanega e Prof. Riccardo Cardin.
\end{itemize}

\subsection{Descrizione}

Realizzazione di un software di costruzione di \textit{diagrammi di robustezza\ped{G}} con la relativa generazione di codice Java per le entità persistenti.

\subsection{Dominio applicativo}

Il progetto si inserisce nell'ambito dell'automatizzazione della codifica allo scopo di aumentare la qualità e la velocità di produzione del codice.

\subsection{Dominio tecnologico}

\begin{itemize}
    \item{Disegnatore diagrammi UML:} ArgoUML, StarUML, Software Idea Designer ecc...;
    
    \item{Codifica parte lato server:} Java con server Tomcat o Javascript con server Node.Js.;
    
    \item{Interfaccia grafica:} HTML5, CSS3, Javascript.
\end{itemize}

\subsection{Aspetti positivi}

\begin{itemize}
    \item Le tecnologie proposte per la realizzazione del progetto sono numerose ma abbordabili;
    
    \item Lo scopo del progetto è reputato molto utile e interessante.
\end{itemize}

\subsection{Fattori di rischio}

\begin{itemize}
    \item Dubbi sulla capacità di garantire codice autogenerato di effettiva qualità.
\end{itemize}

\subsection{Valutazione finale}

Il progetto proposto ha raccolto l'interesse della totalità dei membri del gruppo, tuttavia l'eccessiva competizione nell'appalto ci spinge alla scelta di un altro capitolato.

% CAPITOLATO 6

\section{Capitolato C6}

\begin{itemize}
    \item{Nome:} Marvin: dimostratore di Uniweb su Ethereum;
    \item{Proponente:} RedBabel;
    \item{Commitente:} Prof.Tullio Vardanega e Prof. Riccardo Cardin.
\end{itemize}

\subsection{Descrizione}

L'obiettivo del progetto è la realizzazione di un sottoinsieme delle funzionalità di Uniweb come una Applicazione Decentralizzata che funzioni su di una Ethereum Virtual Machine.  

\subsection{Dominio applicativo}

Il progetto si inserisce nell'ambito di una sperimentazione della tecnologia BlockChain attraverso Ethereum.

\subsection{Dominio tecnologico}

\begin{itemize}
    \item{Interfaccia grafica:} HTML5, CCS3, Javascript ecc..;
    
    \item{Interazione con Ethereum:} Truffle11 development framework;
    
    \item{Misure di sicurezza:} Metamask.
\end{itemize}

\subsection{Aspetti positivi}

\begin{itemize}
    \item Le tecnologie proposte per la realizzazione del progetto sono interessanti, innovative e in forte sviluppo;
\end{itemize}

\subsection{Fattori di rischio}

\begin{itemize}
    \item Il progetto stimola all'interno del gruppo un interesse relativamente basso a causa della mancata concretezza dello scopo del progetto, al di là dei fini dimostrativi in relazione alla tecnologia proposta;
    
    \item Le tecnologie necessarie alla realizzazione del progetto appaiono di un elevato grado di difficoltà di apprendimento.
\end{itemize}

\subsection{Valutazione finale}

Nonostante l'interesse suscitato dalle tecnologie proposte, l'elevata complessità delle stesse e la mancanza di uno scopo concreto del prodotto ci spingono alla scelta di un altro capitolato.

% CAPITOLATO 7

\section{Capitolato C7}

\begin{itemize}
    \item{Nome:} OpenAPM: cruscotto di Application Performance Management;
    \item{Proponente:} Iks - Kirey Group;
    \item{Commitente:} Prof.Tullio Vardanega e Prof. Riccardo Cardin.
\end{itemize}

\subsection{Descrizione}

Realizzare uno Strumento di APM basato su tecnologie Open Source.

\subsection{Dominio applicativo}

Il progetto si inserisce nell'ambito degli APM(Application Performance Managment), strumenti per il monitoraggio e la gestione di performance ed availability delle applicazioni con l’obiettivo di individuare e diagnosticare in modo semplice problematiche complesse che impattano sul servizio erogato.

\subsection{Dominio tecnologico}

\begin{itemize}
    \item{Agent:} Java, PHP, Node.js, ElasticSearch;
    
    \item{Server:} Kibana, D3.js.
\end{itemize}

\subsection{Aspetti positivi}

\begin{itemize}
    \item Il progetto propone l'utilizzo di alcune tecnologie interessanti, innovative e \textit{open source\ped{G}};
    
    \item Lo scopo del progetto ha notevole utilità pratica.
\end{itemize}

\subsection{Fattori di rischio}

\begin{itemize}
    \item La magnitudine delle sfaccettature del progetto potrebbe portare al conseguimento di una qualità complessiva scadente.
\end{itemize}

\subsection{Valutazione finale}

Nonostante l'interesse suscitato dalle tecnologie indicate e dallo scopo del progetto, la presenza di proposte ancor più interessanti ci spingono alla scelta di un altro capitolato.

% CAPITOLATO 8

\section{Capitolato C8}

\begin{itemize}
    \item{Nome:} TuTourSelf: piattaforma di prenotazioni per artisti in tournee;
    \item{Proponente:} TuTourSelf S.r.l.;
    \item{Commitente:} Prof.Tullio Vardanega e Prof. Riccardo Cardin.
\end{itemize}

\subsection{Descrizione}

Realizzare una piattaforma web (con focus particolare sull'esperienza utente) che permetta agli artisti indipendenti di tutto il mondo di organizzare in poco tempo il proprio tour comunicando direttamente con i locali disponibili.

\subsection{Dominio applicativo}

Il progetto si inserisce nell'ambito delle Single Page Applications (SPA) con particolare attenzione all'esperienza utente.

\subsection{Dominio tecnologico}

\begin{itemize}
    \item{Front-end:} HTML5, CSS, Javascript, REACT;
    
    \item{Back-end:}  PHP, Ruby, Python, Java, MySQL.
\end{itemize}

\subsection{Aspetti positivi}

\begin{itemize}
    \item Lo scopo del progetto è reputato molto interessante;
    
    \item La difficoltà complessiva della realizzazione del progetto sembra essere relativamente bassa, date anche le conoscenze pregresse nell'ambito possedute dai membri del gruppo.
\end{itemize}

\subsection{Fattori di rischio}

\begin{itemize}
    \item Il capitolato stimola all'interno del gruppo un interesse relativamente basso a causa delle sue dichiarate finalità di produrre una demo del prodotto vero e proprio, su cui raccogliere dati e fare analisi.
\end{itemize}

\subsection{Valutazione finale}

Nonostante l'interesse suscitato dallo scopo del progetto e la sua evidente fattibilità, il suo scopo e l'alta competizione nell'appalto ci spingono alla scelta di un altro capitolato.

\end{document}
