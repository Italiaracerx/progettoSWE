\documentclass[openany,12pt,a4paper]{report}
\usepackage{StileDoc}

\title{Glossario - Studio di fattibilità}
\author{Matteo Rizzo}
\date{\today}

\begin{document}

\chapter{Glossario (Studio di fattibilità)}

\begin{itemize}
    \item{Capitolato}: atto allegato a un contratto d'appalto che intercorre tra il cliente ed una ditta appaltatrice in cui vengono indicate modalità, costi e tempi di realizzazione dell'opera oggetto del contratto;
    
    \item{Agile}: Insieme di metodi di sviluppo del software emersi a partire dai
    primi anni 2000 e fondati su insieme di principi comuni, direttamente o
    indirettamente derivati dai principi del "Manifesto per lo sviluppo agile del
    software". Tale manifesto si può riassumere in quattro punti:
    \begin{enumerate}
        \item le persone e le interazioni sono più importanti dei processi e degli strumenti;
        \item è più importante avere software funzionante che documentazione;
        \item bisogna collaborare con i clienti oltre che rispettare il contratto;
        \item bisogna essere pronti a rispondere ai cambiamenti oltre che aderire alla pianificazione;
    \end{enumerate}
    
    \item{Open source}: software non protetto da copyright e liberamente modificabile dagli utenti.
\end{itemize}

\end{document}