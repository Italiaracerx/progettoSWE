
\documentclass[openany,12pt,a4paper]{report}

\usepackage{helvet} 
\usepackage[utf8x]{inputenc}
\usepackage[italian]{babel}
\usepackage[]{graphicx}
\usepackage{fancyhdr}
\usepackage{lipsum}
\usepackage{titlesec}
\usepackage{datetime}
\setlength{\headsep}{50pt}  
\fancypagestyle{plain}{
	\fancyhf{}
	\fancyhead[L]{\includegraphics[height=1.5cm]{Logos.png}}
	\fancyhead[R]{\leftmark}
	\fancyfoot[L]{\today \    \currenttime}
	\fancyfoot[R]{\thepage}
}
\titleformat{\chapter}{\normalfont\huge}{\thechapter.}{20pt}{\huge}

\setlength{\headheight}{48pt}
\pagestyle{plain}





\begin{document}
	
\chapter{Titolo Capitolo}

\section{Sezione}
\label{Sezione}
Sintesi circa del pdf di 7 pagine 
\subsection{Sotto Sezione}
\lipsum[20]


\section{Liste e Tabelle}
\lipsum[15]
%usare solo in casi eccezionali chiude la pagina e inizia in quella successiva normalmente latex si arrangia ma ogni tanto è esteticamente orrendo
\newpage

\subsection{Lista Enumerata}
\begin{enumerate}
	\item{Punto1:} Oggetto 1 
	\item{Punto2:} Oggetto 1 
	\item{Punto3:} Oggetto 1 
	\item{Punto4:} Oggetto 1 
	\item{Punto5:} Oggetto 1 
\end{enumerate}

\subsection{Lista Puntata}
\begin{itemize}
	\item{Punto1:} Oggetto 1 
	\item{Punto2:} Oggetto 1 
	\item{Punto3:} Oggetto 1 
	\item{Punto4:} Oggetto 1 
	\item{Punto5:} Oggetto 1 
\end{itemize}


\subsection{Tabella}
\begin{tabular}{ | l |  r | }
	\hline
	l per allineare a destra & r per allineare a sinistra \\
	\hline
	\& per finire il testo della cella  &\textbackslash\textbackslash per finire la riga \\
	\hline
	hline traccia una linea orizzontale \\
	\hline
	
\end{tabular}


\paragraph{}per evitare che il testo vada si affianchi alla tabella


\chapter{Titolo Capitolo2}


\section{Titolo Documento}
Il titolo del documento si cambia sopra dentro fancypagestyle
\section{Riferimenti}
Per fare riferimenti ad altre parti usa il comando ref{nomelabel} e nella parte riferita laber{nomelabel}
\ref{Sezione} 
\section{Immagini}
\begin{figure}[!ht]
	\centering
	\includegraphics[width=10cm,height=10cm]{Logos.png}
	\caption{testo Immagine}
\end{figure}

\paragraph{}
richiede graphicx che serve anche per lo header

\chapter{Fine}
\section{Piu informazioni}
manuali latex ArteLaTeX.pdf(180 pagine) e easy latex.pdf(37 pagine)


% Non mettere piu sezioni senza testo interno senno latex muore ,e non ha molto senso comunque
%	\section{Se Fai Questo Latex esplode}	
%	\section{Se Fai Questo Latex esplode}	
%	\section{Se Fai Questo Latex esplode}	
%	\section{Se Fai Questo Latex esplode}	
%	\section{Se Fai Questo Latex esplode}	
%	\section{Se Fai Questo Latex esplode}	
%	\section{Se Fai Questo Latex esplode}	
%	\section{Se Fai Questo Latex esplode}	
%	\section{Se Fai Questo Latex esplode}	
%	\section{Se Fai Questo Latex esplode}	
%	\section{Se Fai Questo Latex esplode}	
%	\section{Se Fai Questo Latex esplode}	
%	\section{Se Fai Questo Latex esplode}	
%	\section{Se Fai Questo Latex esplode}	
%	\section{Se Fai Questo Latex esplode}	
%	\section{Se Fai Questo Latex esplode}	
%	\section{Se Fai Questo Latex esplode}	
%	\section{Se Fai Questo Latex esplode}	
%	\section{Se Fai Questo Latex esplode}	
%	\section{Se Fai Questo Latex esplode}	
%	\section{Se Fai Questo Latex esplode}	
%	\section{Se Fai Questo Latex esplode}	
%	\section{Se Fai Questo Latex esplode}	
%	\section{Se Fai Questo Latex esplode}	
%	\section{Se Fai Questo Latex esplode}	
%	\section{Se Fai Questo Latex esplode}	
%	\section{Se Fai Questo Latex esplode}	
%	\section{Se Fai Questo Latex esplode}	
%	\section{Se Fai Questo Latex esplode}	
%	\section{Se Fai Questo Latex esplode}	
\end{document}                       