\documentclass[./PianodiProgetto.tex]{subfiles}

\begin{document}

\chapter{Consuntivo e Preventivo a finire}
In questa sezione verranno presentati i consuntivi dei vari periodi con una
breve valutazione degli stessi. Verrà inoltre presentato un preventivo a finire
che terrà conto dei soli periodi rendicontati. I valori presentati saranno:
\begin{itemize}
\item \textbf{Positivi:} se il preventivo è superiore ai valori del consuntivo;
\item \textbf{Negativi:} se il preventivo è inferiore ai valori del consuntivo.
\end{itemize}
\section{Periodo di Analisi}
Essendo il periodo di Analisi considerato periodo di investimento, il consuntivo viene presentato a scopo informativo ma non conteggiato nel preventivo a finire.
\subsection{Consuntivo}
Di seguito è presentata la tabella contenente i dati del consuntivo per il
periodo di Analisi.

\begin{table}[H]
	\centering
	\begin{tabular}{|c|c|c|c|c|}
		\hline
	 	 & \multicolumn{2}{c|}{Ore} & \multicolumn{2}{c|}{Costo in \euro{}}  \\ \hline
		Ruolo&Preventivo&Consuntivo&Preventivo&Consuntivo \\ \hline
		Responsabile&24&24&720,00&720,00  \\ \hline
		Amministratore&20&19(+1)&400,00&380,00(+20)  \\ \hline
		Analista&61&67(-6)&1525,00&1675(-150)  \\ \hline
		Progettista& & & &  \\ \hline
		Programmatore& & & &  \\ \hline
		Verificatore&46&46&690,00&690,00  \\ \hline
		Totale&151&156&3335,00&3465,00  \\ \hline
		Differenza& \multicolumn{2}{c|}{-5 Ore} & \multicolumn{2}{c|}{-130,00 \euro{}} \\ \hline
	\end{tabular}
	\caption{Prospetto orario ed economico a consuntivo del periodo di Analisi}
\end{table}

\subsection{Conclusione}
Nell'esecuzione del primo periodo di Analisi è stato necessario usare più
ore del previsto per il ruolo di Analista mentre si è riusciti a risparmiare nel ruolo di \textit{Amministratore}. Questo è dovuto
probabilmente ad una sottostima del carico di lavoro presentato dalla \textit{Analisi
dei Requisiti}. Le ore di verifica invece, si sono dimostrate sufficienti a svolgere
le attività previste. Il risultato del periodo è complessivamente di cinque ore
lavorative oltre il previsto, con una spesa aggiuntiva di 130,00 \euro{}.
\section{Preventivo a finire}
Viene qui presentata una tabella contenente l'attuale preventivo a finire.
Vengono inseriti i valori del periodo di Analisi e Consolidamento dei requisiti
a scopo riassuntivo, tuttavia essi non verranno conteggiati nel calcolo delle
ore rendicontate. Se il valore del consuntivo non fosse ancora presente, verrà
usato il valore del preventivo.

\begin{table}[H]
	\centering
	\begin{tabular}{|c|c|c|}
		\hline
		Periodo&Preventivo \euro{}&Consuntivo \euro{} \\ \hline
		Analisi&3335,00&3465,00  \\ \hline
		Consolidamento dei requisiti&1010,00&Non presente  \\ \hline
		\multicolumn{3}{|c|}{Rendicontato}  \\ \hline
		Progettazione architetturale&4320,00&Non presente  \\ \hline
		Progettazione di dettaglio e codifica&6593,00&Non presente  \\ \hline
		Validazione e collaudo&2504,00&Non presente  \\ \hline
		Totale&17762,00&17892,00 \\ \hline
		Rendicontato&13417,00&13417,00 \\ \hline
	\end{tabular}
	\caption{Preventivo a finire}
\end{table}

\appendix


\appendix

\chapter{Rilevazione dei rischi}

\section{Analisi}

\begin{itemize}
	\item \textbf{Codice:} RT0;
	\begin{itemize}
		\item \textbf{Descrizione:} alcuni membri del gruppo non erano pratici con Git e \LaTeX;
		\item \textbf{Contromisure:} è stato spiegato loro l'utilizzo di queste tecnologie.
	\end{itemize} 
	\item \textbf{Codice:} RG1;
	\begin{itemize}
		\item \textbf{Descrizione:} a causa delle festività invernali e/o problemi personali, alcuni membri del gruppo si sono assentati per alcuni giorni;
		\item \textbf{Contromisure:} l'eventuali assenze sono state comunicate con largo anticipo, ciò ha permesso di organizzare il lavoro in modo ottimale tenendo conto delle assenze.
	\end{itemize} 
\end{itemize}

\chapter{Organigramma}

\section{Redazione}
\begin{table}[H]
	\centering
	\begin{tabular}{|c|c|c|}
		\hline
		Nome&Data&Firma \\ \hline
		Matteo Rizzo& 12-12-2017 &\includegraphics[scale=0.5]{img/firme/RizzoMatteo} \\
		\hline
	\end{tabular}
	\caption{Redazione}
\end{table}

\section{Approvazione}
\begin{table}[H]
	\centering
	\begin{tabular}{|c|c|c|}
		\hline
		Nome&Data&Firma \\ \hline
		Matteo Rizzo& 12-12-2017 & \includegraphics[scale=0.5]{img/firme/RizzoMatteo} \\
		\hline
	\end{tabular}
	\caption{Approvazione}
\end{table}

\section{Accettazione dei componenti}
\begin{table}[H]
	\centering
	\begin{tabular}{|c|c|c|}
		\hline
		Nome&Data&Firma \\ \hline
		Marco Focchiatti& 12-12-2017 &\includegraphics[scale=0.5]{img/firme/FocchiattiMarco} \\ \hline
		Samuele Modena& 12-12-2017 &\includegraphics[scale=0.5]{img/firme/ModenaSamuele} \\ \hline
		Matteo Rizzo& 12-12-2017 &\includegraphics[scale=0.5]{img/firme/RizzoMatteo} \\ \hline
		Giulio Rossetti&  12-12-2017 &\includegraphics[scale=0.5]{img/firme/RossettiGiulio} \\ \hline
		Kevin Silvestri& 12-12-2017 &\includegraphics[scale=0.5]{img/firme/SilvestriKevin} \\ \hline
		Manfredi Smaniotto& 12-12-2017 &\includegraphics[scale=0.5]{img/firme/SmaniottoManfredi} \\ \hline
		Cristiano Tessarolo& 12-12-2017 &\includegraphics[scale=0.5]{img/firme/TessaroloCristiano} \\  
		\hline
	\end{tabular}
	\caption{Accettazione dei componenti}
\end{table}

\section{Componenti}
\begin{table}[H]
	\begin{tabular}{|c|c|c|}
	\hline
	Nome&Matricola&Indirizzo email \\ \hline
	Marco Focchiatti&1121294&marco.focchiatti@studenti.unipd.it  \\ \hline
	Samuele Modena&1099080&samuele.modena@studenti.unipd.it \\ \hline
	Matteo Rizzo&1123496&matteo.rizzo.4@studenti.unipd.it \\ \hline
	Giulio Rossetti&1122603&giulio.rossetti@studenti.unipd.it \\ \hline
	Kevin Silvestri&1094138&kevin.silvestri@studenti.unipd.it \\ \hline
	Manfredi Smaniotto&1123057&manfredi.smaniotto@studenti.unipd.it \\ \hline
	Cristiano Tessarolo&1119924&cristiano.tessarolo@studenti.unipd.it \\  
	\hline
	\end{tabular}
\caption{Elenco dei componenti}
\end{table}

\section{Definizione dei ruoli}
I membri del gruppo, durante lo svolgimento del progetto, andranno a ricoprire diversi ruoli. Questi ultimi rappresentano figure aziendali specializzate, alle quali corrisponde un costo orario espresso in euro. \\
Durante tutta la durata del progetto ogni componente del gruppo dovrà ricoprire almeno una volta ogni ruolo. Al fine di evitare il conflitto di interesse va certificato che non vi siano intervalli di tempo in cui una risorsa sia anche verificatrice di se stessa.

\end{document}