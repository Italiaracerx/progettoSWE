\documentclass[../PianodiProgetto.tex]{subfiles}

\begin{document}
	
	\chapter{Introduzione}
	\section{Scopo del documento}
	Questo documento ha lo scopo di fornire le specifiche di pianificazione che il gruppo Graphite seguirà durante lo svolgimento del progetto. Nello specifico si prefigge di:
	
	\begin{itemize}
		\item Descrivere nel dettaglio la pianificazione dei tempi e delle \glossario{attività}{attivita};
		\item Fornire un preventivo delle risorse da utilizzare;
		\item Fornire il \glossario{consuntivo}{consuntivo} delle risorse;
		\item Analizzare i possibili fattori di rischio che potrebbero minare il corretto svolgimento del progetto.
	\end{itemize}
	
	\section{Scopo del prodotto}
	
	L'obiettivo di questo progetto è realizzare un'interfaccia grafica per Speect, una libreria per la creazione di sistemi di sintesi vocale, che agevoli l'ispezione del suo stato interno durante il funzionamento e la scrittura di test per le sue funzionalità.
	Nello specifico l'interfaccia dovrà avere le seguenti funzioni:
	\begin{itemize}
		\item Visualizzazione del \glossario{grafo HRG}{grafo HRG}, ovvero il grafo dei risultati delle componenti di analisi linguistica;
		\item Caricamento/Modifica/Salvataggio del grafo;
		\item Manipolazione della struttura dati interna e la configurazione di Speect;
		\item Caricamento ed esecuzione manuale di singole componenti di analisi;
		\item Possibilità di visualizzare percorsi su un grafo;
		\item Possibilità di confrontare visivamente e/o automaticamente due stati della struttura interna di Speect;
		\item Permettere lo sviluppo di test in grado di caricare un particolare stato interno e verificare che gli effetti dell’esecuzione del plugin siano quelli attesi.
	\end{itemize}
	
	\noindent La progettazione e l'implementazione dovranno tener conto della manutenibilità e dell'estensibilità dell'applicazione, cercando di conciliare i vari casi d'uso (sviluppatore che vuole ispezionare lo stato interno della libreria, sviluppatore che vuole creare dei test, sviluppatore che vuole creare configurazioni particolari). 
	
	\section{Glossario}
	Al fine di evitare ogni ambiguità di linguaggio e massimizzare la comprensione dei documenti, i termini tecnici, di dominio, gli acronimi e le parole che necessitano di essere chiarite, sono riportate nel documento \textit{Glossario v1.0.0}. Ogni termine presente nel glossario è marcato da una "G" maiuscola in pedice.
	
	\section{Riferimenti}
	
	\subsection{Normativi}
	\begin{itemize}
		\item \textbf{Norme di Progetto:} \textit{Norme di Progetto v1.0.0};
		\item \textbf{\textit{Capitolato\ped{G}} d'appalto C3:}\\ \url{http://www.math.unipd.it/~tullio/IS-1/2017/Progetto/C3.pdf};
		\item \textbf{Regole del progetto:}\\ \url{http://www.math.unipd.it/~tullio/IS-1/2017/Dispense/P01.pdf};
		\item \textbf{Vincoli Organigramma e Offerta tecnico-economica:}\\ \url{http://www.math.unipd.it/~tullio/IS-1/2017/Progetto/RO.html}.
	\end{itemize}
	
	\subsection{Informativi}
	\begin{itemize}
		\item \textbf{Slide del corso:} \url{http://www.math.unipd.it/~tullio/IS-1/2017/};
		\item \textbf{Studio di Fattibilità:} \textit{Studio di Fattibilità v1.0.0};
		\item \textbf{Analisi dei Requisiti:} \textit{Analisi dei Requisiti v1.0.0}.
	\end{itemize}
	
	\section{Scadenze}
	Le scadenze che il gruppo Graphite ha deciso di rispettare sono le seguenti:
	\begin{itemize}
		\item \textbf{Revisione dei Requisiti(RR):} 26-01-2018;
		\item \textbf{Revisione di Progettazione(RP):} 19-03-2018;
		\item \textbf{Revisione di Qualifica(RQ):} 23-04-2018;
		\item \textbf{Revisione di Accettazione(RA):} 14-05-2018.
	\end{itemize}
	
	\chapter{Analisi dei rischi}
	
	\section{Visione generale dell'analisi dei rischi}
	
	Per una buona riuscita del \glossario{prodotto}{prodotto} è necessario effettuare un'approfondita analisi dei rischi che si possono incontrare durante lo svolgimento dell'intero progetto. I rischi vengono dunque brevemente descritti, analizzati per probabilità di frequenza e fattore di rischio e viene infine proposto un metodo di controllo e delle contromisure per ognuno di essi, i rischi effettivamente riscontrati vengono riportati in appendice §A "Rilevazione dei rischi" di questo documento.
	
	\section{Procedura di analisi dei rischi}
	
	La procedura utilizzata per l'analisi dei rischi è la seguente:
	
	\begin{enumerate}
		\item \textbf{Identificazione}: vengono individuati i potenziali rischi che si ritiene possano presentarsi durante
		l’avanzamento di un processo e se ne identifica la tipologia. I rischi possono essere inerenti a: 
		\begin{itemize}
			\item \textbf{Progetto}: relativi a pianificazione, strumenti, costi e risorse;
			\item \textbf{Prodotto}: relativi a conformità ai requisiti e alle aspettative del committente in termini qualitativi.
		\end{itemize}
		\item \textbf{Analisi}: valutazione della probabilità dell’occorrenza del rischio e delle possibili ricadute
		sul progetto di quest'ultima;
		\item \textbf{Pianificazione del controllo}: definizione di metodi di controllo dei rischi;
		\item \textbf{Mitigazione}: definizione di accorgimenti atti a mitigare gli effetti deleteri di un rischio nel caso in cui dovesse verificarsi. Ciò è richiesto solo per rischi
		difficilmente controllabili e gestibili.
	\end{enumerate}

	\section{Catalogazione dei rischi}
	
	Ogni rischio verrà catalogato secondo il seguente codice identificativo:
	
	\begin{center}
		R[Tipologia].[Identificativo]
	\end{center}

	Dove:
	\begin{itemize}
		\item \textbf{Tipologia}: indica il tipo di rischio in esame, e può assumere uno dei seguenti valori:
		\begin{itemize}
			\item \textbf{T}: rischi inerenti le tecnologie;
			\item \textbf{S}: rischi inerenti gli strumenti;
			\item \textbf{G}: rischi inerenti il gruppo di progetto;
			\item \textbf{R}: rischi inerenti i requisiti;
			\item \textbf{O}: rischi inerenti l'organizzazione.
		\end{itemize}
		\item \textbf{Identificativo}: indica un codice incrementale per il rischio.
	\end{itemize}

	Ogni rischio è inoltre corredato da:
	
	\begin{itemize}
		\item \textbf{Nome}: nome descrittivo del rischio;
		\item \textbf{Descrizione}: breve descrizione;
		\item \textbf{Probabilità di occorrenza}: indica la probabilità che il rischio si verifichi effettivamente;
		\item \textbf{Livello di gravità}: indica l'impatto che il verificarsi del rischi avrebbe sul progetto;
		\item \textbf{Strategia di rilevazione}: indica una possibile strategia con cui tracciare il rischio;
		\item \textbf{Contromisure}: indica eventuali contromisure per mitigare gli effetti deleteri del verificarsi del rischio.
	\end{itemize}
	
	\newpage
	\section{Descrizione dei rischi}

\setlength\LTleft{-35.5mm}
\raggedleft
	\begin{longtable}{|p{15mm}|p{23.5mm}|p{38mm}|p{22mm}|p{19mm}|p{30mm}|p{30mm}|}
		\hline \textbf{Codice} & \textbf{Nome} & \textbf{Descrizione} & \textbf{Probabilità di occorrenza} & \textbf{Livello di gravità} & \textbf{Strategie di rilevazione} & \textbf{Contromisure} \\
		
		\hline\multicolumn{7}{|c|}{Rischi legati alle tecnologie} \\
		
		\hline RT0 & Inesperienza delle tecnologie da utilizzare & Le tecnologie adottate per sviluppare il prodotto sono solamente in parte note ai componenti del gruppo, ciò non toglie che vi possano essere delle mancanze & Bassa & Medio-alto & Viene verificato il grado di conoscenza di ciascun componente controllando settimanalmente l'efficacia dei contenuti aggiunti da esso e verificando che essi soddisfino, nelle loro parti funzionanti o visibili, i criteri di qualità scelti & Ciascun componente è tenuto a documentarsi in maniera autonoma sulle tecnologie adottate e a richiedere informazioni ai colleghi nel caso siano sorti dei dubbi durante il proprio lavoro. Se necessario vengono sfruttati canali telematici con i detentori delle tecnologie (come ad esempio le e-mail) per ricevere spiegazioni riguardanti le tecnologie utilizzate \\
		\hline

		\newpage
	
		\hline \textbf{Codice} & \textbf{Nome} & \textbf{Descrizione} & \textbf{Probabilità di occorrenza} & \textbf{Livello di gravità} & \textbf{Strategie di rilevazione} & \textbf{Contromisure} \\
		
		\hline\multicolumn{7}{|c|}{Rischi legati agli strumenti} \\
		
		\hline RS0 & Guasti hardware e problematiche software & La strumentazione usata dal gruppo potrebbe avere dei malfunzionamenti o guastarsi durante lo sviluppo del progetto, rendendo difficoltoso l'avanzamento del progetto o fermarlo & Bassa & Basso & Ogni componente segnala al \textit{Responsabile di progetto} ogni eventuale malfunzionamento  alla propria strumentazione. In caso di mancata segnalazione viene presunto che ogni strumento utile al progetto sia perfettamente funzionante & Affinchè il lavoro non venga perduto ogni componente del gruppo deve salvare il proprio lavoro sul \glossario{repository}{Repository} \glossario{GitHub}{GitHub} dedicato al progetto, se tale operazione non è possibile allora deve salvarlo su una periferica di archiviazione di massa esterna. In caso di guasti alla propria strumentazione si provvede alla continuazione su altri strumenti in cui viene importata la repository precedentemente creata \\
		\hline
	
	\newpage
	
		\hline \textbf{Codice} & \textbf{Nome} & \textbf{Descrizione} & \textbf{Probabilità di occorrenza} & \textbf{Livello di gravità} & \textbf{Strategie di rilevazione} & \textbf{Contromisure} \\
		
		\hline RS1 & Inesperienza degli strumenti da utilizzare &  L'approccio al metodo di lavoro risulta nuovo. Sono richieste capacità di pianificazione e di analisi che il gruppo non possiede a causa dell'inesperienza. Alcune conoscenze richieste richiedono tempo per essere apprese & Alta & Alto & Quando un componente del gruppo stabilisce che è necessario utilizzare un nuovo strumento lavorativo deve segnalarlo al  \textit{Responsabile}. Una volta approvato l'utilizzo di tale strumento, il gruppo si documenta ed entro una settimana comunica al \textit{Responsabile di progetto} eventuali perplessità o difficoltà incontrate nelle ore di studio personale & Gli eventuali dubbi sollevati dai componenti del gruppo vengono colmati con la conoscenza accumulata dagli altri componenti del gruppo. Nel caso in cui non sia possibile risolvere tali mancanze il \textit{Responsabile di progetto} è incaricato di chiedere delucidazioni puntuali ai detentori degli strumenti utilizzati, in modo tale che tali lacune non vengano lasciate scoperte \\
		
		\hline

		\newpage
	
		\hline \textbf{Codice} & \textbf{Nome} & \textbf{Descrizione} & \textbf{Probabilità di occorrenza} & \textbf{Livello di gravità} & \textbf{Strategie di rilevazione} & \textbf{Contromisure} \\
		
		\hline\multicolumn{7}{|c|}{Rischi legati al gruppo di progetto} \\
		
		\hline RG0 & Problemi personali dei membri del gruppo & Ogni membro del gruppo ha degli impegni personali o potrebbe avere dei problemi di salute. Questo implica la possibilità che qualche componente non sia disponibile in certi momenti creando eventualmente dei rallentamenti al progetto & Media & Medio & Dopo aver ricevuto eventuali segnalazioni di impegni personali dei vari componenti del gruppo, il \textit{Responsabile di progetto} stila il calendario delle scandenze fino alla successiva revisione di progetto. Vengono previsti dei tempi di \glossario{slack}{Slack} affinchè anche le assenze per malattia non compromettano l'avanzamento dei lavori & Nel caso un componente del gruppo non sia disponibile per un breve periodo di tempo, egli dovrà comunicarlo tempestivamente al \textit{Responsabile}, il quale organizzerà e ripartirà il carico di lavoro ai rimanenti componenti del gruppo \\ \hline

	\newpage

		\hline \textbf{Codice} & \textbf{Nome} & \textbf{Descrizione} & \textbf{Probabilità di occorrenza} & \textbf{Livello di gravità} & \textbf{Strategie di rilevazione} & \textbf{Contromisure} \\

		\hline RG1 & Problemi tra membri del gruppo & La maggior parte dei membri del gruppo non si conosceva prima di questo progetto. Questo può portare a delle incomprensioni causate anche da diversi punti di vista & Bassa & Alto & Il \textit{Responsabile del progetto} monitora lo stato di collaborazione fra i vari componenti del gruppo durante le varie fasi controllando durante le proprie ore lavorative il livello di coesione dei singoli & Nel caso si verifichino forti dissidi, il \textit{Responsabile} dovrà prendere in mano la situazione e cercare di trovare una soluzione di comune accordo con i membri in contrasto \\ 

		\hline

		\newpage
	
		\hline \textbf{Codice} & \textbf{Nome} & \textbf{Descrizione} & \textbf{Probabilità di occorrenza} & \textbf{Livello di gravità} & \textbf{Strategie di rilevazione} & \textbf{Contromisure} \\

		\hline \multicolumn{7}{|c|}{Rischi legati ai requisiti} \\ 

		\hline  RR0 & Comprensione dei requisiti & Data l'inesperienza dei componenti del gruppo nell'analisi dei requisiti, è possibile un'errata comprensione dei requisiti comportando un'offerta non conforme alle richieste & Medio-bassa & Alto & Le persone aventi dei dubbi comunicano le proprie difficoltà a dei colleghi addetti all'analisi dei requisiti, i quali si occupano di risolvere le perplessità sollevate. I verificatori possono al contempo segnalare la scarsa chiarezza di una sezione del testo da essi letto affinchè vengano subito corretti gli errori commessi da un collega, consigliando ad esso una rivisitazione degli argomenti & Per ridurre al minimo gli errori dell'analisi, vengono organizzati incontri con il proponente al fine di sciogliere ogni dubbio e capire a fondo le sue esigenze. A seguito di tali incontri viene reddatto un verbale affinchè le informazioni vengano fissate e possano essere rilette rapidamente \\ \hline

		\newpage
	
		\hline \textbf{Codice} & \textbf{Nome} & \textbf{Descrizione} & \textbf{Probabilità di occorrenza} & \textbf{Livello di gravità} & \textbf{Strategie di rilevazione} & \textbf{Contromisure} \\

		\hline \multicolumn{7}{|c|}{Rischi legati all'organizzazione} \\
		
		\hline RO0 & Sottostima dei tempi necessari & Data l'inesperienza dei componenti del gruppo nella pianificazione di progetto e l'attuazione della stessa su una arco di tempo medio-lungo, può verificarsi la sottostima dei tempi necessari alla realizzazione del progetto & Media & Alto & Vengono tracciati i progressi all'interno di una attività tramite strumenti di project management atti all'organizzazione puntuale del lavoro tra una revisione di progetto ed un altra & Vengono inviate a cadenza giornaliera delle mail dal software di project management affinchè sia esplicito il tempo rimanente per completare la propria attività. Prevedendo in anticipo dei ritardi vengono attribuiti dei periodi di \glossario{slack}{Slack} affinchè il ritardo nel terminare una attività non interferisca con l'inizio di un'altra \\

		\hline

	\end{longtable}

\end{document}