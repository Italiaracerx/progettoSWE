\documentclass[../PianodiProgetto.tex]{subfiles}

\begin{document}
	
	\chapter{Ciclo di sviluppo}
	
	Il modello di ciclo di sviluppo adottato dal gruppo è il \glossario{modello incrementale}{modello incrementale}. Le motivazioni che hanno portato alla scelta di questo modello sono:
	\begin{itemize}
		\item Possibilità di suddividere il lavoro in più sottoattività sviluppate in modo parallelo. Questo permette maggior controllo sull’avanzamento del progetto;
		\item Lo sviluppo avviene per incrementi, dove ogni incremento rilascia parte delle funzionalità richieste;
		\item I requisiti vengono suddivisi in livelli di priorità. Quelli a priorità maggiore verranno soddisfatti con i primi incrementi, questo permette più attività di verifica e quindi maggiore stabilità ad ogni iterazione;
		\item I primi incrementi possono essere usati come prototipo per aiutare a definire i requisiti degli incrementi successivi;
		\item Minimizzare i rischi di ritardo rispetto ai tempi stabiliti in quanto i cicli hanno durata breve e sono precedentemente pianificati.
	\end{itemize}
	Alla fine della prima fase si avrà un prototipo funzionante con le implementazioni dei requisiti obbligatori. Tramite incrementi successivi verranno integrate le funzionalità opzionali.
	
	\chapter{Pianificazione}
	
	La pianificazione del lavoro stata costruita sulla base delle scadenze elencate alla sezione 1.5 del seguente documento. In base ad esse, si è suddiviso lo sviluppo nei periodi seguenti:
	\begin{itemize}
		\item Analisi;
		\item Consolidamento dei requisiti;
		\item Consolidamento delle tecnologie;
		\item Progettazione e Codifica;
		\item Validazione e collaudo.
	\end{itemize}
	\newpage
	\section{Analisi}
	Inizia con la formazione del gruppo e termina con la scadenza per la consegna della Revisione dei Requisiti.
	Durante questo periodo viene fatta un'analisi del capitolato scelto, in particolare vengono svolte le seguenti attività:
	\begin{itemize}
		\item \textbf{Individuazione delle norme:} vengono individuati gli strumenti e le norme relative ai vari processi per il corretto svolgimento del progetto;
		\item \textbf{Analisi dei capitolati:} in questa attività vengono studiati i vari capitolati, analizzando i vari pro e contro, scegliendo quale sviluppare;
		\item \textbf{Analisi dei Requisiti:} viene effettuata un'analisi approfondita dei requisiti del capitolato che il gruppo ha deciso di sviluppare;
		\item \textbf{Piano di Progetto:} viene scelto il modello di sviluppo e fatta la pianificazione per la realizzazione del progetto suddividendo le risorse disponibili, analizzando i vari rischi in cui si può incombere, realizzando infine il preventivo; 
		\item \textbf{Pianificazione della qualità:} attività in cui si individuano gli obiettivi di qualità che si vuole raggiungere.	
	\end{itemize}
	
	\subsection{Analisi - Diagramma di Gantt}
	\begin{figure}[H]
		\includegraphics[width=1\linewidth]{pianificazione/AnalisiGantt.jpg}	
		\caption{Diagramma di Gantt di Analisi}\label{fig:1}	
	\end{figure}
	\newpage
		\section{Consolidamento dei requisiti} Inizia dopo la consegna dei documenti per la Revisione dei Requisiti e termina con la presentazione della Revisione dei Requisiti. Questo periodo consiste nel migliorare, consolidare ed eventualmente ampliare quanto fatto nell'Analisi dei Requisiti, in particolare i requisiti individuati.
	\subsection{Consolidamento dei requisiti - Diagramma di Gantt}
	\begin{figure}[H]
		\includegraphics[width=\linewidth]{pianificazione/ConsolidamentoRequisitiGantt}	
		\caption{Diagramma di Gantt di Consolidamento dei requisiti}\label{fig:2}
	\end{figure}
	\newpage
	\section{Consolidamento delle tecnologie} Inizia dopo l'esito della Revisione dei Requisiti e termina con la consegna per la Revisione di Progettazione. In questo periodo viene svolto:

	\begin{itemize}	
		\item \textbf{Incremento e Verifica}: vengono incrementati e verificati, se necessario, i documenti già redatti, correggendo i difetti emersi nell'esito della Revisione dei Requisiti;
		\item \textbf{Consolidamento dei requisiti}: vengono ulteriormente affinati, ed eventualmente ampliati, i requisiti analizzati durante il periodo di Analisi secondo le indicazioni ricevute durante l'esito;
		\item \textbf{Technology Baseline:} vengono analizzate nel dettaglio le tecnologie scelte per il progetto individuando eventuali rischi delle stesse e mitigati mediante la realizzazione di un \glossario{Proof of Concept}{Proof of Concept} utile per il prodotto finale;
	\end{itemize}

	\subsection{Consolidamento delle tecnologie - Diagramma di Gantt}
	\begin{figure}[H]
		\includegraphics[width=\linewidth]{pianificazione/ConsolidamentoTecnologieGantt}	
		\caption{Diagramma di Gantt di Consolidamento delle tecnologie}\label{fig:3}
	\end{figure}
	\newpage
	\section{Progettazione e Codifica}
	Inizia dopo la fine del periodo di Consolidamento e termina con la consegna per la Revisione di Qualifica. Durante questo periodo vengono svolte le seguenti attività:
	\begin{itemize}
		\item \textbf{Incremento e Verifica}: vengono incrementati e verificati, se necessario, i documenti già redatti, correggendo i difetti emersi nell'esito della Revisione di Progettazione;	
		\item \textbf{Progettazione:} vengono individuati i design pattern e costruita l'architettura del prodotto realizzando diagrammi di classi e di sequenza, il lavoro svolto rappresenterà la \glossario{Product Baseline}{Product Baseline};
		\item \textbf{Codifica:} consiste nella stesura del primo ciclo di codice per la creazione del prodotto che soddisfi i requisiti obbligatori individuati, aggiunti uno ad uno in modo incrementale;
	\end{itemize}
	\subsection{Progettazione e Codifica - Diagramma di Gantt}
	\begin{figure}[H]
		\includegraphics[width=\linewidth]{pianificazione/ProgettazioneCodificaGantt.jpg}	
		\caption{Diagramma di Gantt di Progettazione e Codifica}\label{fig:4}	
	\end{figure}
	\newpage
	\section{Validazione e collaudo}
	Inizia dopo la Revisione di Qualifica e termina con la Revisione di Accettazione del 14-05-2018. Durante questo periodo viene svolto:
	\begin{itemize}
		\item \textbf{Incremento e Verifica}: vengono incrementati e verificati, se necessario, i documenti già redatti e l'architettura del prodotto, correggendo i difetti emersi nell'esito della Revisione di Qualifica;
		\item \textbf{Codifica:} viene affinato il codice prodotto durante il primo ciclo, verificato, ed eventualmente incrementato con l'aggiunta di alcuni requisiti opzionali e/o desiderabili;		
		\item \textbf{Validazione e Collaudo:} viene verificata la conformità del prodotto rispetto ai requisiti, testandolo per assicurarsi il corretto funzionamento ed il raggiungimento di determinati vincoli qualificativi.
		 
	\end{itemize}
	\subsection{Validazione e collaudo - Diagramma di Gantt}
	\begin{figure}[H]
		\includegraphics[width=\linewidth]{pianificazione/ValidazioneCollaudoGantt.jpg}	
		\caption{Diagramma di Gantt di Validazione e collaudo}\label{fig:5}	
	\end{figure}
\end{document}