\documentclass[../AnalisideiRequisiti.tex]{subfiles}

\begin{document}
	% Il comando UserCase accetta primo una label nel caso serva un link verso di lui \refer{label} poi 
	% attore primario
	% attore secondario
	% Descrizione
	% Precondi
	% Post
	% Scenari alternativi
	% Flusso Eventi 
	% e li scrive se un campo e vuoto verra ignorato e anche la sua identazione rimossa es: se non si mette attore secondario lasciando {} non verra scritta la riga attore secondario GL HF 
	\chapter{Casi d'uso}
	\section{UC1:File}
	\UserCase
	{UC1}
	{Utente}
	{}
	{L'Utente vuole visualizare il menu File}
	{Il programma è correttamente avviato}
	{Viene mostrato il menu a tendina}
	{}
	{}
	\section{UC2:Caricamento JSon}
	\UserCase
	{UC2}
	{Utente}
	{Speect}
	{L'Utente vuole caricare un file JSon}
	{Il programma è correttamente avviato}
	{Viene inizializzato Speect con il file JSon selezionato e aggiornata la GUI}
	{L'attore visualizza un messaggio di errore relativo al file \refer{UC2.1}}
	{}
	\section{UC2.1:Errore Caricamento JSon}
	\UserCase
	{UC2.1}
	{Utente}
	{Speect}
	{Durante l'inizializzazione Speect fallisce ritornando un errore }
	{L'Utente carica un file JSon non corretto}
	{L'errore è visualizzato a schermo e viene ripristinato lo stato precedente}
	{}
	{}
	\section{UC3:Esportazione stato del Grafo}
	\UserCase
	{UC3}
	{Utente}
	{Speect}
	{L'utente vuole esportare lo stato del grafo}
	{Esiste un grafo esportabile}
	{Il grafo viene esportato in file}
	{ L'utente visualizza un messaggio di errore \refer{UC3.1}}
	{}
\section{UC3.1:Errore Esportazione Grafo}
\UserCase
{UC3.1}
{Speect}
{Utente}
{Avviene un errore durante l'esportazione}
{L'Utente ha cercato di esportare un grafo}
{Viene visualizzato l'errore e nessuna operazione viene eseguita}
{}
{}
\section{UC4:Importa Grafo}
\UserCase
{UC4}
{Utente}
{Speect}
{L'utente vuole importare lo stato del grafo}
{Esiste un grafo esportabile}
{Il grafo viene importato da file}
{L'utente visualizza un messaggio di errore \refer{UC4.1}}
{}
\section{UC4.1:Errore Importazione Grafo}
\UserCase
{UC4.1}
{Speect}
{Utente}
{Avviene un errore durante l'importazione}
{L'Utente ha cercato di importare un file scorretto}
{Viene visualizzato l'errore e nessuna operazione viene eseguita}
{}
{}
\section{UC5:Salvataggio Audio Prodotto}
\UserCase
{UC5}
{Utente}
{Speect}
{L'utente vuole salvare l'audio}
{Speect ha processato il file .json senza dare errori}
{L'audio è salvato in un file}
{L'utente visualizza un messaggio di errore \refer{UC5.1}}
{}
\section{UC5.1:Errore Salvataggio Audio}
\UserCase
{UC5.1}
{Utente}
{Speect}
{Avviene un errore durante il salvataggio dell'audio}
{L'Utente ha cercato di salvare un file audio}
{Viene visualizzato l'errore e nessuna operazione viene eseguita}
{}
{}
\section{UC6:Exit}
\UserCase
{UC6}
{Utente}
{}
{L'Utente vuole chiudere l'applicazione }
{L'applicazione sta funzionando}
{L'applicazione viene terminata}
{}
{
}

\section{UC7:Selezione Utterance}
\UserCase
{UC7}
{Utente}
{}
{L'Utente seleziona la Utterance desiderata }
{Il file JSon è caricato correttamente}
{Vengono mostrati gli Utterance Processors utilizzati da Speect per tale Utterance}
{}
{}

\section{UC8:Submit di Testo}
\UserCase
{UC8}
{Utente}
{Speect}
{L'Utente preme il tasto di Esecuzione }
{Il file JSon è stato caricato e i campi dati sono compilati}
{Speect elabora il testo selezionato e viene visualizzato il grafo}
{L'utente visualizza un messaggio di errore \refer{UC8.1}}
{}
\section{UC8.1:Errore Esecuzione}
\UserCase
{UC8.1}
{Speect}
{Utente}
{L'Utente visualizza a schermo l errore di esecuzione di Speect }
{Speect ha fallito a eseguire la utterance e ha ritornato un errore}
{Viene visualizzato un messaggio di errore all' utente}
{}
{}
\section{UC9:Modifica Visualizzazione Layer}
\UserCase
{UC9}
{Utente}
{}
{L'Utente sceglie quali relazioni mostrare}
{Esiste un grafo visualizzato coerentemente alle opzioni selezionate}
{Vengono mostrati tutti i layer di relazione selezionati}
{}
{}

\section{UC10:Ricerca Path}
\UserCase
{UC10}
{Utente}
{Speect}
{L'Utente scrive un path utilizzabile da speect per visualizzare tale nodo}
{Esiste un grafo HRG(scusa Manfredi mi è venuto in mente adesso il nome e sono di fretta fixo dopo se non vuoi farlo tu)
	corretto}
{Se il path porta ad un nodo definito esso viene evidenziato}
{}
{}

\section{UC11:Selezione Nodo}
\UserCase
{UC11}
{Utente}
{}
{L'utente preme su un nodo per vedere le sue caratteristiche}
{Viene visualizzato a schermo un grafo corretto con almeno un nodo cliccabile}
{}
{}

\end{document}

\textbf{}