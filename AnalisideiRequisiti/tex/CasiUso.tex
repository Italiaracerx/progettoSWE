\documentclass[../AnalisideiRequisiti.tex]{subfiles}

\begin{document}
	% Il comando UserCase accetta primo una label nel caso serva un link verso di lui \refer{label} poi 
	% attore primario
	% attore secondario
	% Descrizione
	% Precondi
	% Post
	% Scenari alternativi
	% Flusso Eventi 

	\chapter{Casi d'uso}
	\section{UC0:Pagina Iniziale}
	\UserCase
	{UC0}
	{Utente}
	{}
	{L'attore visualizza la pagina iniziale di DeSpeect nella quale può accedere al menu File e caricare il file JSon}
	{Il programma è correttamente avviato e visualizza la pagina iniziale}
	{L'attore apre il menu file}
	{}
	{}
	
	\section{UC1:Menu File}
	\UserCase
	{UC1}
	{Utente}
	{}
	{L'Utente vuole visualizare il menu File}
	{Il programma mostra Pagina Iniziale \refer{UC0}}
	{Viene selezionata una voce del menu}
	{}
	{	\begin{itemize}
		\item{} L'attore può Caricare un file JSon \refer{UC2}
		\item{} L'attore può Salvare le modifiche al file JSon \refer{UC11}
		\item{} L'attore può Salvare l'audio prodotto da Speect \refer{UC4}
		\item{} L'attore può chiudere l'applicazione \refer{UC5}
		\item{} L'attore può Caricare un grafo \refer{UC8}
		\item{} L'attore può Salvare un grafo \refer{UC9}
		\item{} L'attore può Ricercare un Path nel grafo \refer{UC10}
		\end{itemize}
	}

	\section{UC2:Caricamento JSon}
	\UserCase
	{UC2}
	{Utente pre Inizializzazione,Utente}
	{Speect}
	{L'attore vuole caricare un file JSon}
	{L'attore ha selezionato la voce nel menu}
	{Viene inizializzato Speect con il file JSon selezionato e aggiornata la GUI}
	{Speect fallisce l'inizializzazione l'attore visualizza il messaggio del errore relativo al file \refer{UC2.1}}
	{
		\begin{itemize}
			\item{} Viene aperto il file browser per Caricare
			\item{} L'attore seleziona il file tramite il file browser \refer{UC3}
			\item{} L'attore preme Carica
			\item{} Il file viene dato a Speect che prova l'inizializzazione
			\item{} Viene visualizzato il percorso del file nella apposito spazio (link a figura)
		\end{itemize}
	}
	\section{UC2.1:Errore Caricamento JSon}
	\UserCase
	{UC2.1}
	{Utente}
	{Speect}
	{Durante l'inizializzazione Speect fallisce ritornando un errore }
	{L'attore carica un file JSon non corretto}
	{L'errore è visualizzato a schermo e viene ripristinato lo stato precedente e ridato controllo all'attore}
	{}
	{}

	
	\section{UC3:File Browser}
	\UserCase
	{UC3}
	{Utente}
	{}
	{L'attore deve navigare nel file system alla ricerca di un file o di una cartella}
	{L'attore deve selezionare un file o raggiungere una cartella}
	{L'attore seleziona il file da caricare o la cartella in cui salvare}
	{L'attore non ha i permessi per aprire il file viene visualizzato l'errore di navigazione \refer{UC3.3}}
	{
		\begin{itemize}
			\item{} Il file browser viene visualizzato dall'attore
			\item{} L'attore naviga nel file \glossario{file system}{file system} cercando il suo file \refer{UC3.1}
			\item{} L'attore seleziona un file \refer{UC3.2}
		\end{itemize}
	}
	\section{UC3.1:Navigazione}
	\UserCase
	{UC3.1}
	{Utente}
	{}
	{L'attore vuole navigare nel suo }
	{Il file \glossario{browser}{browser} è aperto}
	{L'attore naviga nel file system}
	{}
	{
		\begin{itemize}
			\item{} L'attore può aprire cartelle \refer{UC3.1.1}
			\item{} L'attore può creare nuove cartelle \refer{UC3.1.2}
			\item{} L'attore può tornare alla cartella padre \refer{UC3.1.3}
		\end{itemize}
	}	
	\section{UC3.1.1:Aprire cartelle}
	\UserCase
	{UC3.1.1}
	{Utente}
	{}
	{L'attore vuole aprire una cartella e ha i permessi per farlo}
	{Il file browser visualizza la cartella}
	{Viene aperta la cartella e visualizzato il suo contenuto}
	{L'attore non ha i permessi necessari, viene visualizzato l'errore di navigazione \refer{UC3.3}}
	{
		\begin{itemize}
			\item{} L'attore fa un doppio click sulla cartella
			\item{} La cartella viene aperta
			\item{} L'attore visualizza il contenuto della cartella
		\end{itemize}
	}
	\section{UC3.1.2:Creare cartelle}
	\UserCase
	{UC3.1.2}
	{Utente}
	{}
	{L'attore vuole creare una cartella}
	{L'attore ha i permessi per creare una cartella}
	{Viene creata una cartella}
	{L'attore non ha i permessi necessari, viene visualizzato l'errore di navigazione \refer{UC3.3}}
	{
		\begin{itemize}
		\item{} L'attore preme su il tasto per creare la cartella
		\item{} La cartella viene creata
		\item{} L'utente visualizza la nuova cartella
		\end{itemize}
	}

	\section{UC3.1.3:Ritorna al Padre}
\UserCase
{UC3.1.3}
{Utente}
{}
{L'attore preme sul tasto per tornare alla cartella padre}
{Esiste una cartella padre}
{Viene visualizzato il contenuto della cartella padre}
{L'attore non ha i permessi necessari, viene visualizzato l'errore di navigazione \refer{UC3.3}}
{}
	\section{UC3.2:Scegliere un file}
\UserCase
{UC3.2}
{Utente}
{}
{L'attore selezionare un file}
{Il file browser è aperto correttamente}
{Il file scelto viene evidenziato}
{}
{
	\begin{itemize}
		\item{} L'attore clicca sull' elemento
		\item{} L'elemento selezionato viene evidenziato
	\end{itemize}
}
\section{UC3.3:Errore di Navigazione}
\UserCase
{UC3.3}
{Utente}
{}
{L'attore cerca di fare un operazione senza i permessi necessari}
{L'attore ha cercato di effettuare un operazione senza i permessi necessari}
{Viene visualizzato l'errore e nessuna operazione viene svolta}
{}
{}

\section{UC4:Salvataggio Audio Prodotto}
\UserCase
{UC4}
{Utente}
{}
{L'attore vuole salvare l'audio}
{L'attore ha premuto su Salva Audio}
{L'audio è salvato in un file}
{L'utente visualizza un messaggio di errore \refer{UC4.2}}
{
		\begin{itemize}
		\item{} Viene aperto il file browser per il salvataggio
		\item{} L'attore seleziona la cartella di destinazione tramite il file browser \refer{UC3}
		\item{} L'attore scrive il nome del file nella barra di testo
		\item{} L'attore seleziona l'estensione attraverso menu a tendina\refer{UC4.1}
		\item{} L'attore preme su Salva 
		\item{} Il file viene salvato nella destinazione con estensione scelta altrimenti con estensione di default .WAV
\end{itemize}
}
\section{UC4:Menu Estensioni}
\UserCase
{UC4.1}
{Utente}
{}
{L'attore vuole selezionare l'estensione}
{L'attore ha premuto sul menu a tendina}
{L'audio viene selezionata l'estensione}
{}
{
\begin{itemize}
	\item{} L'attore clicca su una voce del menu
	\item{} La voce viene selezionata e rimpiazzata al posto della precedente
	\item{} Viene chiuso il menu a tendina

\end{itemize}
}		
\section{UC4.2:Errore Salvataggio Audio}
\UserCase
{UC4.2}
{Utente}
{Speect}
{Avviene un errore durante il salvataggio dell'audio}
{L'Utente ha cercato di salvare un file audio}
{Viene visualizzato l'errore e nessuna operazione viene eseguita}
{}
{}

\section{UC5:Exit}
\UserCase
{UC5}
{Utente}
{}
{L'Utente vuole chiudere l'applicazione }
{L'applicazione sta funzionando}
{L'applicazione viene terminata}
{}
{
}

\section{UC6:Selezione Utterance}
\UserCase
{UC6}
{Utente}
{}
{L'attore vuole selezionare la Utterance desiderata}
{Un file JSon è stato caricato \refer{UC2} correttamente }
{Vengono mostrati gli Utterance Processors utilizzati da Speect per tale Utterance type}
{}
{
	\begin{itemize}
		\item{} L'attore clicca sul menu a tendina
		\item{} Viene aperto il menu a tendina
		\item{} L'attore clicca sul Utterance type desiderata
		\item{} Vengono mostrati a schermo i nomi degli Utterance Processor utilizzati negli appositi spazi (link figura), colorati con lo stesso colore delle relazioni da loro prodotte,una select box adiacente selezionata e un pulsante
		
	\end{itemize}
}
\section{UC6.1:Modifica Utterance Processor}
\UserCase
{UC6.1}
{Utente}
{}
{L'attore vuole cambiare l'ordine degli Utterance Processor}
{Un file JSon è stato caricato \refer{UC2} correttamente }
{Il file JSon viene aggiornato}
{}
{
	\begin{itemize}
		\item{} L'attore clicca sul Utterance Processor \refer{UC6.1.1}
		\item{} L'attore modifica tramite i pulsanti forniti	
		\item{} Le operazioni vengono eseguite
		\item{} Se esisteva un grafo esso non viene modificato
		
	\end{itemize}
}
\section{UC6.1.1:Selezione Utterance Processor}
\UserCase
{UC6.1.1}
{Utente}
{}
{L'attore vuole selezionare un utterance processor per spostarlo}
{Un file JSon è stato caricato \refer{UC2} correttamente }
{Vengono visualizzati i bottoni per modificare tale Utterance Processor}
{}
{
	\begin{itemize}
		\item{} L'attore clicca sul nome dell'Utterance Processor
		\item{} Vengono visualizzati due bottoni che permettono lo spostamento grafico del Utterance Processor e un bottone che ne permette la rimozione 
		
		
	\end{itemize}
}
\section{UC7:Esecuzione}
\UserCase
{UC7}
{Utente}
{Speect}
{L'Utente vuole eseguire Speect}
{Il file JSon è stato caricato correttamente}
{Speect elabora il testo selezionato e viene visualizzato il grafo}
{L'utente visualizza un messaggio di errore \refer{UC7.1}}
{\begin{itemize}
		\item{} L'attore seleziona l'utterance type
		\item{} L'attore può compilare il campo di testo
		\item{} L'attore preme sul tasto di esecuzione
		\item{} Se il campo di testo è vuoto viene utilizzato il grafo come dato per l'utterance
		\item{} Vengono eseguiti gli utterance processor designati dall' utterance type
		\item{} Viene mostrato il grafo risultante dall'esecuzione\refer{UC7.2}
	\end{itemize}
}

\section{UC7.1:Errore Esecuzione}
\UserCase
{UC7.1}
{Speect}
{Utente}
{L'Utente visualizza a schermo l errore di esecuzione di Speect }
{Speect ha fallito a eseguire la utterance e ha ritornato un errore}
{Viene visualizzato un messaggio di errore all' utente}
{}
{}
\section{UC7.2:Visualizzazione del grafo}
\UserCase
{UC7.2}
{Utente}
{Speect}
{L'utente visualizza il grafo}
{Speect ha terminato l'esecuzione con successo}
{Viene visualizzato a schermo un grafo corretto con almeno un nodo cliccabile}
{}
{
	\begin{itemize}
		\item{} L'attore può selezionare un nodo \refer{UC7.2.1}
		\item{} L'attore può spostare un nodo \refer{UC7.2.2}
	\end{itemize}
}

\section{UC7.2.1:Selezione Nodo}
\UserCase
{UC7.2.1}
{Utente}
{}
{L'utente vuole selezionare un nodo per visualizzarne i dettagli}
{Un grafo con almeno un nodo è mostrato a schermo}
{Viene evidenziato il nodo del grafo e vengono mostrate le sue informazioni nella finestra apposita}
{}
{
	\begin{itemize}
		\item{} L'attore clicca una volta sul nodo
		\item{} Il nodo viene evidenziato con un contorno giallo
		\item{} Nel riquadro apposito(ref a figura) vengono visualizzati i dati del grafo:
		\begin{enumerate}
			\item{} Name
			\item{} Part of Speech
		\end{enumerate}
	\end{itemize}
}
\section{UC7.2.2:Spostare Nodo}
\UserCase
{UC7.2.2}
{Utente}
{}
{L'utente vuole spostare graficamente un nodo}
{L'utente ha cliccato su un nodo\refer{UC7.2.1}}
{Il nodo viene spostato}
{}
{\begin{itemize}
		\item{} L'utente trascina il nodo cliccando senza rilasciare
		\item{} Il nodo si sposta
		\item{} L'utente rilascia il click
		\item{} Il nodo rimane nella nuova posizione
\end{itemize}
}

\section{UC7.2.3:Modificare Name}
\UserCase
{UC7.2.3}
{Utente}
{}
{L'utente vuole modificare il nome del nodo selezionato}
{L'utente ha cliccato su un nodo\refer{UC7.2.1}}
{Il nodo cambia nome}
{}
{\begin{itemize}
		\item{} L'utente seleziona la casella di testo del nome
		\item{} L'utente cancella il nome precedente
		\item{} L'utente rimuove il focus dalla casella di testo
		\item{} Il nome viene aggiornato
		\item{} Il grafo viene aggiornato \refer{UC7}
		\item{} Il grafo viene ristampato a schermo \refer{UC7.2}
\end{itemize}}

\section{UC7.2.4:Modifica Visualizzazione Layer}
\UserCase
{UC7.2.4}
{Utente}
{}
{L'Utente vuole filtrare i layer del grafico}
{Un utterance type è stato scelto \refer{UC6}}
{Vengono mostrati tutti i layer di relazione selezionati}
{}
{\begin{itemize}
		\item{} L'utente deseleziona una select box adiacente ad un utterance processor
		\item{} Lo strato prodotto da tale utterance processor viene nascosto
\end{itemize}
}
\section{UC7.2.5:Esecuzione Singola Utterance}
\UserCase
{UC7.2.5}
{Utente}
{}
{L'Utente vuole eseguire una singola utterance}
{Un utterance type è stato scelto \refer{UC6} }
{Viene eseguita l'utterance partendo dal grafo già presente o dal campo di testo scritto}
{}
{\begin{itemize}
				\item{} L'attore seleziona l'utterance type
		\item{} L'attore può compilare il campo di testo
		\item{} L'attore preme sul tasto di esecuzione adiacente al utterance da eseguire \refer{UC6}
	\end{itemize}
}

\section{UC8:Esportazione stato del Grafo}
\UserCase
{UC8}
{Utente}
{Speect}
{L'utente vuole esportare lo stato del grafo}
{Esiste un grafo esportabile}
{Il grafo viene esportato in file}
{ L'utente visualizza un messaggio di errore \refer{UC8.1}}
{
	{\begin{itemize}
			\item{} Viene aperto il file browser
			\item{} L'utente seleziona il file da importare
			\item{} L'utente preme su Importa Grafo
\end{itemize}}}

\section{UC8.1:Errore Esportazione Grafo}
\UserCase
{UC8.1}
{Speect}
{Utente}
{Avviene un errore durante l'esportazione}
{L'Utente ha cercato di esportare un grafo}
{Viene visualizzato l'errore e nessuna operazione viene eseguita}
{}
{}
\section{UC9:Importa Grafo}
\UserCase
{UC9}
{Utente}
{Speect}
{L'utente vuole importare lo stato del grafo}
{Esiste un grafo e l'utente ha cliccato carica grafo}
{Il grafo viene importato da file}
{L'utente visualizza un messaggio di errore \refer{UC9.1}}
{
	{\begin{itemize}
			\item{} Viene aperto il file browser
			\item{} L'utente seleziona il file da importare
			\item{} L'utente preme su Importa Grafo
\end{itemize}}}
\section{UC9.1:Errore Importazione Grafo}
\UserCase
{UC9.1}
{Speect}
{Utente}
{Avviene un errore durante l'importazione}
{L'Utente ha cercato di importare un file scorretto}
{Viene visualizzato l'errore e nessuna operazione viene eseguita}
{}
{}


\section{UC10:Ricerca Path}
\UserCase
{UC10}
{Utente}
{Speect}
{L'utente vuole cercare un percorso nel grafo}
{Esiste un grafo corretto e l'utente ha premuto Ricerca Path nel menu}
{Se il path porta ad un nodo definito esso viene selezionato \refer{UC7.2.1}}
{Viene visualizzato un errore \refer{UC10.1}}
{
	\begin{itemize}
		\item{} Viene visualizzata una finestra con una casella di testo e un pulsante
		\item{} L'attore inserisce il percorso da cercare
		\item{} L'attore preme il pulsante di Ricerca
		\item{} Se il percorso finisce in un nodo esso viene selezionato \refer{UC7.2.1}
 	\end{itemize}
}

\section{UC10.1:Errore Ricerca Path}
\UserCase
{UC10.1}
{Utente}
{Speect}
{L'utente vuole cercare un percorso nel grafo}
{Il percorso inserito dall'utente è sintatticamente errato e l'utente ha il pulsante di Ricerca}
{Viene visualizzato un errore a schermo e si riapre la finestra di Ricerca \refer{UC10}}
{}
{}

\section{UC11:Salvataggio modifiche file JSon}
\UserCase
{UC12}
{Utente}
{}
{L'Utente ha modificato gli Utterance Processor e vuole salvare il nuovo file JSon}
{Esiste un file Json correttamente aperto \refer{UC2} e l'utente ha modificato gli utterance processor \refer{UC6.1}}
{Le modifiche vengono Salvate}
{Viene visualizzato un errore \refer{12.1}}
{\begin{itemize}
		\item{} L'attore apre il menu file \refer{UC1}
		\item{} L'attore preme su Salva File JSon
\end{itemize}
}
\section{UC11.1:Errore Salvataggio modifiche file JSon}
\UserCase
{UC12}
{Utente}
{}
{L'Utente ha provato a salvare il file JSon}
{L'operazione di salvataggio fallisce }
{Viene visualizzato la causa del fallimento e si ripristina lo stato precedente all'errore}
{}
{}

\end{document}
