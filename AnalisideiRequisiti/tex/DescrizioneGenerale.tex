\documentclass[../AnalisideiRequisiti.tex]{subfiles}

\begin{document}
	
\chapter{Descrizione generale}

\section{Obiettivo del prodotto}

Lo scopo del progetto consiste nel creare un applicativo software di supporto all’utilizzo di Speect(GLOSSARIO). L’applicazione che dobbiamo creare è una interfaccia grafica che aiuti i programmatori nello sviluppo dei plug-in per Speect. Nell’interfaccia utente si deve poter visualizzare e modificare i grafi delle \glossario{utterance}{utterance} di Speect. 


\section{Funzioni del prodotto}
L’interfaccia grafica permetterà di:
\begin{itemize}
	\item{} Caricare i file JSON(GLOSSARIO) utili all’inizializzazione di Speect;
	\item{} Mostrare i grafi delle varie utterance;
	\item{} Aggiunta, modifica e eliminazione degli archi dei nodi; 
	\item{} La modifica dei campi dei nodi;
	\item{} Disporre graficamente i nodi per permettere una lettura semplificata;
	\item{} Ritornare il file audio generato da Speect;
	\item{} Permettere una stampa grafica dei grafi;(???)
	\item{} Poter visualizzare passo passo i grafi delle varie utterance in modo sequenziale, cioè l’utente potrà decidere quando eseguire e visualizzare il grafo della successiva utterance.	
\end{itemize}


\section{Caratteristiche degli utenti}
Il software si rivolge prevalentemente a programmatori esperti che si occupano di sviluppare plug-in per Speect. L’utente deve avere aver installato nulla propria macchina Speect e deve conoscere la struttura di grafi ritornata dalle varie utterance.


\section{Piattaforma di esecuzione}
Sarà possibile eseguire il software su tutte le macchine desktop con sistema operativo Linux, dovranno essere presenti \glossario{CMAKE}{CMAKE}, \glossario{GCC}{GCC} e le librerie di \glossario{QT}{QT}. Comunque verranno utilizzate tecnologie presenti anche su sistemi Windows in questo modo sarà possibile la compilazione, però non verrà fornito un manuale di installazione per quest’ultima piattaforma.

\section{Vincoli generali}
Il software realizzato dovrà rispettare vari requisiti:
\begin{itemize}
	\item{} Requisiti obbligatori:
	\begin{itemize}
		\item{}Realizzazione di una interfaccia grafica per Speect in grado di visualizzare dei risultati delle componenti di analisi linguistica;
		\item{}	Documentazione tecnica del software;
		\item{}Salvare e caricare lo stato dell’analisi.
	\end{itemize}
	\item{} Requisiti desiderabili:
	\begin{itemize}
		\item{} Poter eseguire passo passo le varie utterance;
		\item{}	Modificare gli archi che collegano i vari nodi dei grafi delle utterance.
	\end{itemize}
	\item{} Requisiti opzionali:
	\begin{itemize}
		\item{} Possibilità di visualizzare percorsi su un grafo;(IN CHE SENSO? PAG 7 CAPITOLATO)
		\item{}	Possibilità di confrontare visivamente due stati della struttura interna di Speect;
		\item{} Possibilità di confrontare automaticamente due stati della struttura interna di Speect;
		\item{}	Possibilità di manipolare la configurazione di Speect:
		\begin{itemize}
			\item{} Caricamento;
			\item{}	Modifica;
			\item{} Salvataggio.
		\end{itemize}
	\end{itemize}
	
\end{itemize}

\end{document}