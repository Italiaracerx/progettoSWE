\documentclass[../AnalisideiRequisiti.tex]{subfiles}

\begin{document}
	
\chapter{Descrizione generale}

\section{Obiettivo del prodotto}

Lo scopo del progetto consiste nel creare un applicativo software di supporto allo sviluppo di \glossario{Speect}{Speect}. L’applicazione da creare è una interfaccia grafica che aiuti i programmatori nello sviluppo dei plug-in per Speect. Nell’interfaccia utente si deve poter visualizzare e modificare i grafi delle \glossario{utterance}{utterance} di Speect. 


\section{Funzioni del prodotto}
L’interfaccia grafica permetterà di:
\begin{itemize}
	\item{} Caricare i \glossario{file .json}{file .json} utili all’inizializzazione di Speect;
	\item{} Mostrare i grafi delle varie utterance;
	\item{} Aggiunta, modifica e eliminazione degli archi dei nodi;
	\item{} La modifica dei campi dei nodi;
	\item{} Disporre graficamente i nodi per permettere una lettura semplificata;
	\item{} Ritornare il file audio generato da Speect;
	\item{} Permettere una stampa grafica dei grafi;
	\item{} Poter visualizzare passo passo i grafi delle varie utterance in modo sequenziale, cioè l’utente potrà decidere quando eseguire e visualizzare il grafo della successiva utterance.	
\end{itemize}


\section{Caratteristiche degli utenti}
Il software si rivolge a programmatori esperti che si occupano di sviluppare plug-in per Speect. L'utente deve possedere una buona conoscenza di Speect e delle sue componenti.

\section{Piattaforma di esecuzione}
Sarà possibile eseguire il software su tutte le macchine desktop con sistema operativo Linux, dovranno essere presenti \glossario{CMAKE}{CMAKE}, \glossario{GCC}{GCC} e le librerie di \glossario{QT}{QT}. Verranno comunque utilizzate tecnologie presenti anche su sistemi Windows in questo modo sarà possibile la compilazione, però non verrà fornito un manuale di installazione per quest’ultima piattaforma.

\section{Vincoli generali}
Il software realizzato dovrà rispettare vari requisiti:
\begin{itemize}
	\item{} Requisiti obbligatori:
	\begin{itemize}
		\item{}Realizzazione di una interfaccia grafica per Speect in grado di:
		\begin{enumerate}
			\item{} Caricare un \glossario{file Voice}{file Voice} con estensione JSON;
			\item{} Inserire un input di testo, che verrà utilizzato in fase di compilazione;
			\item{} Selezionare il tipo di \glossario{utterance type}{utterance type} di compilazione;
			\item{} Compilazione mediante Speect dato input di testo e l'utterance type;
			\item{} Visualizzazione grafica del grafo prodotto dalla compilazione;
			\item{} Spostare un nodo graficamente;
			\item{} Selezionato un nodo dall'interfaccia grafica, visualizzare le informazioni del nodo;
			\item{} Possibilità di salvare un file audio con estensione \glossario{WAV}{WAV} generato a seguito di una compilazione di Speect.
		\end{enumerate}
		\item{}	Documentazione tecnica del software;
	\end{itemize}
	\item{} Requisiti desiderabili:
	\begin{itemize}
		\item{} Selezione file JSon tramite \glossario{Drag and Drop}{drag and drop}
		\item{} Permettere all'utente di selezionare le relazioni del grafo da visualizzare;
		\item{} Permettere la riproduzione del file audio prodotto;
		\item{} Permettere di nascondere i nodi senza dati;
		\item{} Cambiare il colore degli strati del grafo.
	\end{itemize}
	\item{} Requisiti facoltativo:
	\begin{itemize}
		\item{} Evidenziare un nodo dato un percorso riferito al grafo;
		\item{} Poter eseguire passo passo le varie utterance;
		\item{}	Modificare gli archi che collegano i vari nodi dei grafi delle utterance;
		\item{} Caricare e salvare lo stato di un grafo precedentemente realizzato;
		\item{} Poter compilare partendo da un grafo caricato;
		\item{}	Possibilità di confrontare visivamente due stati della struttura interna di Speect;
		\item{} Possibilità di confrontare automaticamente due stati della struttura interna di Speect;
		\item{} Modificare il file Voice di estensione JSON caricato nell'applicazione.
	\end{itemize}
	
\end{itemize}

\section{Interfaccia Grafica}
In questa parte verrà presentato, in linea generale, il funzionamento dell'interfaccia grafica. Le interfacce proposte nelle immagini che seguono non rappresentano le finestre che andremo a implementare, ma semplicemente vogliono essere una linea guida per comprendere al meglio le varie funzionalità dell'applicazione; quindi l'estetica di \textit{\glossario{DeSpeect}{DeSpeect}} potrebbe differire dalle immagini che seguono. Le istruzioni che seguono non intendono essere in alcun modo una guida all'utilizzo dell'applicazione.
	\subsection{Schermata principale}
		\begin{figure}
			\caption{Esempio pagina principale}
			\centering
			\includegraphics[width=\textwidth]{../img/paginainiziale.png}
			\label{fig:GUI}
		\end{figure}	
		\noindent Nell'interfaccia grafica saranno presenti due pulsanti per caricare il \glossario{file Voice JSon}{file voice .json}, uno in alto a sinistra di nome "Load Voice" (vedi \ref{fig:GUI}) e uno di nome "Load Voice JSon" all'interno della voce "File" nella barra del menu (vedi \ref{fig:menufile}).

		\begin{figure}
			\caption{Esempio voce File nella barra del menu}
			\centering
			\includegraphics[]{../img/menu-file.png}
			\label{fig:menufile}
		\end{figure}
		\noindent A seguito del caricamento del file Voice Json il menù a tendina "Utterance Type" (vedi \ref{fig:GUI} in alto a destra) verrà popolato con l'elenco delle varie \glossario{utterance type}{utterance type} contenute nel file appena caricato. 
		
		\noindent Una volta selezionata la \glossario{utterance type}{utterance type} desiderata, il programma popolerà l'elenco "Utterance Processor" (vedi \ref{fig:GUI} appena sotto a sinistra della linea orizzontale che separa la parte alta dell'applicazione dal resto) con una lista di \glossario{utterance processor}{utterance processor} contenuti nella utterance type selezionata. 
		
		\noindent In seguito, l'utente potrà compilare l'area di testo sottostante il pulsante "Load Voice" (vedi \ref{fig:GUI}) con il testo che desidera mandare in elaborazione a \glossario{Speect}{Speect}.
		
		\noindent Proseguendo verso destra, nella figura \ref{fig:GUI}, l'utente avrà la possibilità di eseguire tutti gli \glossario{utterance processor}{utterance processor} contenuti nella utterance type premendo il pulsante "Run all processor" (vedi \ref{fig:GUI}), in alternativa, potrà eseguirli sequenzialmente uno alla volta con la possibilità di tornare al passo precedente (vedi pulsanti contenuti nell'area nominata "Run single processor" \ref{fig:GUI}). 
		
		\noindent Man mano che gli utterance processor vengono eseguiti, nell'area centrale bianca (vedi \ref{fig:GUI}) verrà disegnato il \glossario{grafo HRG}{grafo HRG}.
		
		\noindent Nella sezione degli utterance processor (vedi \ref{fig:GUI}) l'utente avrà la possibilità di:
		
		\begin{itemize}
			\item{}decidere se visualizzare il grafo HRG di una determinata utterance mediante una spunta;
			\item{}modificare l'ordine di esecuzione delle utterance agendo sulle freccie a lato della singola utterance processor;
			\item{}rimuovere una determinata utterance processor dall'elenco e quindi dalla utterance type.
		\end{itemize}
	
		\noindent Cliccando un nodo del grafo HRG l'utente lo evidenzierà con un cerchio di colore giallo e potrà visualizzare, nella parte inferiore dell'interfaccia grafica, le sue proprietà, tra cui il percorso del nodo.
		
		\noindent Attraverso la voce "File" della barra del menu (vedi \ref{fig:menufile}) l'utente potrà:
		\begin{itemize}
			\item{Load Voice JSon:} caricare il file inizializzazione di Speect;
			\item{Save Voice JSon:} salvare il file inizializzazione di Speect;
			\item{Load HRG Graph:} caricare e visualizzare nell'apposita area un grafo HRG;
			\item{Save HRG Graph:} salvare lo stato di un grafo HRG;
			\item{Save Audio file:} salvare il file audio prodotto dall'esecuzione di Speect;
			\item{Search from path:} evidenziare il nodo nel grafo HRG e di conseguenza potrà vedere le sue proprietà;
			\item{Exit:} Uscire dall'applicazione.
		\end{itemize}
		\noindent Attraverso la voce "Help" della barra del menu (vedi \ref{fig:menuhelp}) l'utente potrà:
		\begin{itemize}
			\item{Manual:} visualizzare il manuale utente;
			\item{Licence:} visualizzare la licenza di DeSpeect;
		\end{itemize}
		
		\begin{figure}
			\caption{Esempio voce Help nella barra di menu}
			\centering
			\includegraphics[]{../img/menu-help.png}
			\label{fig:menuhelp}
		\end{figure}
	\subsection{Schermata caricamento file}
		Mediante questa schermata (\ref{fig:filebrowser-load}) l'utente potrà navigare attraverso il \glossario{file system}{file system} e selezionare il file che desidera aprire. 
		Nell'area centrale bianca vengono visualizzati i file e le cartelle del percorso specificato nella barra "Path". Facendo doppio clic su una cartella, contenuta nel blocco centrale, verrà cambiato il percorso del "Path" e quindi verrà visualizzato il suo contenuto.
		Il primo pulsante in alto a sinistra è utilizzato per raggiungere la directory padre e visualizzarne il contenuto nell'area dedicata.
		Il pulsante in alto a destra permette all'utente di creare una nuova cartella nel percorso indicato nella barra "Path". 
		Una volta individuato il file da aprire l'utente avrà tre modi per aprirlo:
		\begin{enumerate}
			\item{} Doppio clic sopra il file;
			\item{} Un clic sopra il file e premere il pulsante "Open";
			\item{} Scrivere il nome, compreso di estensione, del file e premere il pulsante "Open".
		\end{enumerate}
		\begin{figure}
			\caption{Esempio finestra caricamento file}
			\centering
			\includegraphics[width=\textwidth]{../img/filebrowser-load.png}
			\label{fig:filebrowser-load}
		\end{figure}

	\subsection{Schermata salvataggio file}
		Mediante questa schermata (\ref{fig:filebrowser-save}) l'utente potrà navigare attraverso il \glossario{file system}{file system} e posizionarsi all'interno della cartella nella quale vuole salvare il file.
		Nell'area centrale bianca vengono visualizzati i file e le cartelle del percorso specificato nella barra "Path". Facendo doppio clic su una cartella, contenuta nel blocco centrale, verrà cambiato il percorso del "Path" e quindi verrà visualizzato il suo contenuto.
		Il primo pulsante in alto a sinistra è utilizzato per raggiungere la directory padre e visualizzarne il contenuto nell'area dedicata.
		Il pulsante in alto a destra permette all'utente di creare una nuova cartella nel percorso indicato nella barra "Path". 
		Una volta individuato il punto in cui salvare il file bisogna:
		\begin{itemize}
			\item{} Scrivere il nome del file nell'apposito campo senza riportare l'estensione;
			\item{} Selezionare l'estensione del file;
			\item{} Premere il pulsante "Save".
		\end{itemize}
		\begin{figure}
			\caption{Esempio finestra salvataggio file}
			\centering
			\includegraphics[width=\textwidth]{../img/filebrowser-save.png}
			\label{fig:filebrowser-save}
		\end{figure}

\end{document}