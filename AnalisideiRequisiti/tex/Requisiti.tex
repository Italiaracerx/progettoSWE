\documentclass[../AnalisideiRequisiti.tex]{subfiles}

\begin{document}
	\chapter{Requisiti}
	\section{Descrizione} 
	I requisiti devono essere suddivisi per importanza e classificati come segue:
	
	\begin{center}
		R[Importanza][Tipologia][Codice]
	\end{center}
	
	\begin{itemize}
		\item \textbf{Importanza:} Ogni requisito può appartenere solo ad una delle classi di Importanza elencate di seguito:
		\begin{itemize}
			\item \textbf{O (Requisito Obbligatorio):} requisito fondamentale per la corretta realizzazione del progetto;
			\item \textbf{D (Requisito Desiderabile):} requisito non fondamentale al progetto ma il cui soddisfacimento comporterebbe una maggiore completezza del prodotto;
			\item \textbf{F (Requisito Facoltativo):} requisito non richiesto per il corretto funzionamento del prodotto ma che se incluso arricchirebbe il progetto. Prima di soddisfare il requisito è necessaria un’analisi di tempi e costi per evitare ritardi nella consegna e/o costi superiori a quelli preventivati.
		\end{itemize}
		\item \textbf{Tipologia:} Di seguito sono riportate le tipologie di requisito:
		\begin{itemize}
			\item \textbf{V:} Identifica un \glossario{requisito di vincolo}{requisito di vincolo};
			\item \textbf{F:} Identifica un \glossario{requisito funzionale}{requisito funzionale};
			\item \textbf{P:} Identifica un \glossario{requisito prestazionale}{requisito prestazionale};
			\item \textbf{Q:} Identifica un \glossario{requisito di qualità}{requisito di qualita}.
		\end{itemize}
		\item \textbf{Codice:} Ogni requisito è formato da un codice numerico che lo indentifica in modo univoco.
	\end{itemize}
	\newpage
	\section{Requisiti Funzionali}
	\begin{longtable}{| p{2cm} | p{2.5cm} |p{5cm} | p{2.5cm} |}
		\hline
		\textbf{Codice} & \textbf{Importanza} & \textbf{Descrizione} & \textbf{Fonti}\\
		\hline
		\endhead
		\newline ROF0&
		\newline Obbligatorio&
		\newline L'utente può avviare DeSpeect visualizzandone la pagina iniziale&
		\newline \refer{UC0} \newline \refer{UC0.1} \newline Capitolato
		\\[1em]
		\hline
		\newline ROF1&
		\newline Obbligatorio&
		\newline L'utente può accedere al menu file&
		\newline \refer{UC1} \newline Interno
		\\[1em]	
		
		\hline
			
		\newline ROF2&
		\newline Obbligatorio&
		\newline L'utente può caricare un file Json&
		\newline \refer{UC1} \newline \refer{UC2} \newline Capitolato
		\\[1em]	
			\hline	
			
		\newline ROF2.1&
		\newline Obbligatorio&
		\newline L'utente può visualizzare il percorso del file JSon caricato&
		\newline \refer{UC2} \newline  VE-15-12-2017
		\\[1em]	
		\hline	
		
		\newline RFF2.2&
		\newline Facoltativo&
		\newline L'utente può modificare il file Json cambiando l'ordine o rimuovendo gli Utterance Processor nell'Utterance Type&
	 	\newline \refer{UC6.2} \newline \refer{UC6.3} \newline Capitolato
		\\[1em]	
		\hline
				
		\newline RFF2.2.1&
		\newline Facoltativo&
		\newline L'utente può salvare nel file JSon le modifiche agli Utterance Processor&
			\newline \refer{UC1} \newline \refer{UC11} \newline  VE-15-12-2017
		\\[1em]	
		\hline
		
		\newline RFF2.2.1.1&
		\newline Facoltativo&
		\newline Il sistema deve visualizzare un errore nel caso il salvataggio fallisca e ripristinare uno stato funzionante&
		\newline \refer{UC11.1} \newline Interno
		\\[1em]	
		\hline

		\newline ROF3&		\newline Obbligatorio&
		\newline L'utente può inizializzare Speect con il file json&
		\newline \refer{UC2} \newline  VE-15-12-2017
		\\[1em]	
			\hline	
		
		\newline ROF3.1&\newline Obbligatorio&
		\newline Il sistema deve visualizzare un errore in caso Speect fallisca l'inizializzazione&
		\newline \refer{UC2.1} \newline Interno
		\\[1em]		
		\hline
		
		\newline ROF4&\newline Obbligatorio&
		\newline L'utente può salvare l'audio risultante con estensione WAV&
		\newline \refer{UC4} \newline Interno
		\\[1em]
			\hline
		
		\newline ROF4.1&\newline Obbligatorio&
		\newline L'utente può selezionare dove salvare il file&
		\newline \refer{UC4} \newline \refer{UC3.1} \newline \refer{UC3.1.1} \newline Interno
		\\[1em]
		
		\hline	
		\newline ROF4.1.1&\newline Obbligatorio&
		\newline L'utente può scrivere il nome del file da salvare&
		\newline \refer{UC4} \newline Interno
		\\[1em]
		
		\hline
		\newline ROF4.2&\newline Obbligatorio&
		\newline Il sistema deve visualizzare un errore in caso il salvataggio dell'audio fallisca&
		\newline \refer{UC4.1} \newline Interno
		\\[1em]
		\hline
		
		\newline RFF4.3&\newline Facoltativo&
		\newline L'utente può ascoltare l'audio prima di salvarlo&
		\newline Interno
		\\[1em]
		\hline
		
		\newline ROF5&\newline Obbligatorio&
		\newline L'utente può cercare il file tramite file browser&
		\newline \refer{UC3} \newline \refer{UC3.1} \newline  VE-15-12-2017
		\\[1em]
		\hline
		
		\newline ROF5.1&\newline Obbligatorio&
		\newline L'utente può selezionare un file tramite file browser&
		\newline \refer{UC3} \newline \refer{UC3.2} \newline Interno
		\\[1em]	
		\hline
		
		\newline ROF5.2&\newline Obbligatorio&
		\newline Il sistema visualizza un errore se si cerca di accedere ad una cartella senza i permessi necessari&
		\newline \refer{UC3.1.2} \newline  VE-15-12-2017
		\\[1em]	
		\hline
		
		\newline ROF5.3&\newline Obbligatorio&
		\newline Il sistema visualizza un errore se si cerca di accedere ad un file non supportato&
		\newline \refer{UC3.2.1} \newline  VE-15-12-2017
		\\[1em]	
		\hline
		
		\newline ROF5.4&\newline Obbligatorio&
		\newline Il sistema visualizza un errore se si cerca di accedere ad un file senza i permessi necessari&
		\newline \refer{UC3.2.2} \newline  VE-15-12-2017
		\\[1em]	
		\hline
		
		\newline RDF5.5&\newline Desiderabile&
		\newline Il file browser mostra solo file di estensione corretta&
		\newline Interno
		\\[1em]
		\hline
		\newline RDF5.6&\newline Desiderabile&
		\newline L'utente può modificare il nome di una cartella tramite file browser&
		\newline Interno
		\\[1em]
		\hline
		\newline RDF5.7&\newline Desiderabile&
		\newline L'utente può modificare il nome di un file tramite file browser&
		\newline Interno
		\\[1em]
		\hline
		\newline ROF6&\newline Obbligatorio&
		\newline L'utente può selezionare la Utterance Type&
		\newline \refer{UC12} \newline  VE-15-12-2017
		\\[1em]
		\hline
				
		\newline RDF6.1&\newline Desiderabile&
		\newline L'utente può modificare gli Utterance Processor di un Utterance Type&
		\newline \newline \refer{UC6}  \newline \refer{UC6.2} \newline \refer{UC6.3} \newline  VE-15-12-2017
		\\[1em]
		\hline	
				
		\newline RDF6.1.1&\newline Desiderabile&
		\newline L'utente può spostare gli Utterance Processor di un Utterance Type&
		\newline \refer{UC6.1} \newline \refer{UC6.2} \newline Interno
		\\[1em]
		\hline	
				
		\newline RDF6.1.2&\newline Desiderabile&
		\newline L'utente può rimuovere gli Utterance Processor di un Utterance Type&
		\newline \refer{UC6.1} \newline \refer{UC6.3} \newline Interno
		\\[1em]
		\hline	
		
		\newline ROF7&\newline Obbligatorio&
		\newline L'utente può inserire un testo da tradurre in voce&
		\newline \refer{UC7} \newline Capitolato
		\\[1em]
		
		\hline
		\newline ROF8&\newline Obbligatorio&
		\newline L'utente può eseguire il testo inserito&
		\newline \refer{UC7} \newline Capitolato
		\\[1em]
		\hline
		\newline ROF8.1&\newline Obbligatorio&
		\newline Il sistema visualizza l'errore di esecuzione se Speect fallisce l'esecuzione&
		\newline \refer{UC7.1} \newline Interno
		\\[1em]
		\hline
		
		\newline ROF9&\newline Obbligatorio&
		\newline L'utente può visualizzare il grafo ottenuto eseguendo Speect&
		\newline \refer{UC7.2} \newline Capitolato
		\\[1em]
		\hline
		
			
		\newline ROF9.1&\newline Obbligatorio&
		\newline L'utente può visualizzare l'informazione generale di ogni nodo sul grafo&
		\newline \refer{UC7.2} \newline Capitolato
		\\[1em]
		\hline
		
		\newline ROF9.2&\newline Obbligatorio&
		\newline L'utente vede ogni relazione del grafo di un colore diverso, relativo al colore in legenda&
		\newline  VE-03-01-2018  \newline Capitolato
		\\[1em]
		\hline
		
		\newline RDF9.2.1&\newline Desiderabile&
		\newline L'utente può cambiare il colore delle relazioni in legenda&
		\newline  VE-03-01-2018
		\\[1em]
		\hline
		
		\newline ROF9.3&\newline Obbligatorio&
		\newline L'utente può selezionare il nodo del grafo tramite click&
		\newline \refer{UC13.1} \newline Capitolato
		\\[1em]
		\hline
		
			\newline ROF9.3.1&\newline Obbligatorio&
		\newline L'utente può visualizzare tutte le informazioni del nodo selezionato&
		\newline \refer{UC13.1} \newline Capitolato
		\\[1em]
		\hline
			
		\newline RDF9.3.1.1&\newline Desiderabile&
		\newline L'utente può modificare il name del nodo selezionato&
		\newline \refer{UC13.3} \newline  VE-15-12-2017 \newline Capitolato
		\\[1em]
		\hline
		
			\newline RDF9.3.1.2&\newline Desiderabile&
		\newline L'utente può modificare il PoS del nodo selezionato&
		\newline \refer{UC13.4} \newline  VE-15-12-2017 \newline Capitolato
		\\[1em]
		\hline
		
		\newline RDF9.4&\newline Desiderabile&
		\newline L'utente può testare se un percorso porta ad un nodo esistente&
		\newline \refer{UC10} \newline  VE-15-12-2017 \newline Capitolato
		\\[1em]
		\hline
		
		\newline RDF9.4.1&\newline Desiderabile&
		\newline L'utente può evidenziare un nodo del grafo tramite percorso partendo da un nodo selezionato&
		\newline \refer{UC10} \newline \refer{UC13.1} \newline  VE-03-01-2018 \newline Capitolato
		\\[1em]
		\hline
		
		\newline RDF9.4.2&\newline Desiderabile&
		\newline Il sistema visualizza un errore se il path porta fuori dal grafo e riapre la ricerca&
		\newline \refer{UC10.1} \newline Interno
		\\[1em]
		\hline
		
		\newline ROF9.5&\newline Obbligatorio&
		\newline I nodi selezionati dall'utente vengono evidenziati&
		\newline \refer{UC13.1} \newline  VE-15-12-2017 \newline Capitolato
		\\[1em]
		\hline
		
		\newline RDF9.5.1&\newline Desiderabile&
		\newline L'utente può modificare il colore con il quale si evidenzia il focus&
		\newline Interno
		\\[1em]
		\hline
		
		\newline ROF9.6&\newline Obbligatorio&
		\newline L'utente può spostare i nodi del grafo graficamente&
		\newline \refer{UC13.2} \newline VE-03-01-2018
		\\[1em]
		\hline
		
		\newline ROF9.7&\newline Obbligatorio&
		\newline L'utente può visualizzare gli strati di relazione del grafo selezionati&
		\newline \refer{UC13.5} \newline Capitolato
		\\[1em]
		\hline
	
		\newline RFF9.8&\newline Facoltativo&
		\newline L'utente può modificare gli archi dei nodi del grafo&
		\newline  VE-15-12-2017 \newline Capitolato
		\\[1em]
		\hline
		
		\newline RFF9.8.1&\newline Facoltativo&
		\newline L'utente può cancellare gli archi dei nodi del grafo&
		\newline Interno \newline Capitolato
		\\[1em]
		\hline
		
		\newline RFF9.8.2&\newline Facoltativo&
		\newline L'utente può aggiungere archi a dei nodi del grafo&
		\newline Interno \newline Capitolato
		\\[1em]
		\hline
		
		\newline ROF9.9&\newline Obbligatorio&
		\newline L'utente può modificare il grafo ottenuto eseguendo Speect&
		\newline \refer{UC13} \newline Capitolato
		\\[1em]
		\hline	
		
		\newline RFF10&\newline Facoltativo&
		\newline L'utente può eseguire ogni Utterance Processor singolarmente&
		\newline \refer{UC7.3} \newline Capitolato
		\\[1em]
		\hline
	
		
		\newline RFF11&\newline Facoltativo&
		\newline L'utente può salvare il grafo&
		\newline \refer{UC8} \newline  VE-15-12-2017 
		\\[1em]
		\hline


		\newline RFF11.1&\newline Facoltativo&
		\newline Il sistema deve visualizzare un errore se non riesce a salvare il grafo&
		\newline \refer{UC8.1} \newline  VE-15-12-2017 
		\\[1em]
		\hline
		
		\newline RFF12&\newline Facoltativo&
		\newline L'utente può caricare un grafo&
		\newline \refer{UC9} \newline  VE-15-12-2017
		\\[1em]
		\hline
		
		\newline RFF12.1&\newline Facoltativo&
		\newline Il sistema deve visualizzare un errore se non riesce a caricare il grafo&
		\newline \refer{UC9.1} \newline  VE-15-12-2017 
		\\[1em]
		\hline
		
		\newline RFF12.2&\newline Facoltativo&
		\newline L'utente può confrontare due strati di relazione automaticamente&
		\newline Capitolato
		\\[1em]
		\hline
		
		
		\newline RFF13&\newline Facoltativo&
		\newline L'utente può eseguire Speect dato un grafo&
		\newline \refer{UC7} \newline  VE-15-12-2017
		\\[1em]
		\hline
		
	
		
		\newline ROF14&\newline Obbligatorio&
		\newline L'utente può chiudere l'applicazione&
		\newline \refer{UC5} \newline \refer{UC5.1} \newline Interno
		\\[1em]
		\hline
		
		
	\end{longtable}
	\section{Requisiti di Qualità}
			\begin{longtable}{| p{2cm} | p{2.5cm} |p{5cm} | p{2.5cm} |}
			\hline
			\textbf{Codice} & \textbf{Importanza} & \textbf{Descrizione} & \textbf{Fonti}\\
			\hline
			\endhead
				
			
			\newline ROQ0&\newline Obbligario&
			\newline Deve essere fornito un manuale utente&
			\newline Capitolato
			\\[1em]
			\hline
			\newline ROQ0.1&\newline Obbligario&
			\newline Il manuale deve essere in lingua italiana&
			\newline Interno
			\\[1em]
			\hline
			\newline
			RDQ1&\newline Desiderabile&
			\newline 
			L'applicazione deve essere rilasciata con licenze opensource&
			\newline Capitolato \newline Interno
			\\[1em]
			\hline	
			\newline
			RDQ1.1&\newline Desiderabile&
			\newline 
			L'applicazione deve essere rilasciata con licenze BSD/MIT &
			\newline Capitolato
			\\[1em]
			\hline
	\end{longtable}
\newpage
	\section{Requisiti di Vincolo}
			\begin{longtable}{| p{2cm} | p{2.5cm} |p{5cm} | p{2.5cm} |}
			\hline
			\textbf{Codice} & \textbf{Importanza} & \textbf{Descrizione} & \textbf{Fonti}\\
			\hline
			\endhead
				\newline ROV0&\newline Obbligario&
			\newline 
			L'applicativo deve usare Speect modificato da Mivoq &
			\newline Capitolato
			\\[1em]
			\hline	
			\newline 
			ROV1&\newline Obbligario&
			\newline 
			L'applicativo deve essere sviluppato con QT 5.9 LTS &
			\newline Capitolato
			\newline Interno
			\\[1em]
			\hline
			\newline 
			ROV2&\newline Obbligario&
			\newline 
			L'applicativo deve essere utilizzabile su sistema operativo Linux Ubuntu 16.04 LTS&
			\newline Capitolato
			\\[1em]
			\hline
			\newline
			RDV2.1&\newline Desiderabile&
			\newline 
			L'applicativo deve essere utilizzabile su sistema operativo Windows 7 e successivi&
			\newline Capitolato
			\\[1em]
			\hline
	\end{longtable}
\newpage
	\section{Tracciamento fonte-requisiti}
	\begin{longtable}{| p{4cm} | p{4cm} |}
		\hline
		\textbf{Fonte} & \textbf{Requisti} \\
			\hline
		\endhead
		\newline Capitolato & \newline ROF0 \newline ROF2 \newline RFF2.2 \newline ROF7 \newline ROF8 \newline ROF9 \newline ROF9.1 \newline ROF9.2 \newline ROF9.3 \newline ROF9.3 \newline ROF9.3.1 \newline ROF9.3.1.1 \newline ROF9.3.1.2 \newline RDF9.4 \newline RDF9.4.1 \newline RDF9.5 \newline ROF9.7 \newline RFF9.8 \newline RFF9.8.1 \newline RFF9.8.2 \newline ROF9.9 \newline RFF10 \newline RFF12.2 \newline ROQ0 \newline RDQ1 \newline RDQ1.1 \newline ROV0 \newline ROV1 \newline ROV2 \newline ROV2.1 \\[1em]
	\hline	
		\newline Interno &  \newline ROF1 \newline RFF2.2.1.1 \newline ROF3.1 \newline ROF4 \newline ROF4.1  \newline ROF4.1.1  \newline ROF4.2  \newline RFF4.3 \newline RDF5.5  \newline RDF5.6  \newline RDF5.7 \newline ROF6.1 \newline RDF6.1.1 \newline RDF6.1.2 \newline RDF9.4.2 \newline ROF 9.6  \newline ROF 8.1  \newline RDF9.5.1  \newline RFF9.8.1  \newline RFF9.8.2 \newline ROF14 \newline ROQ0.1  \newline RDQ1  \newline ROV1 \\[1em]
	\hline
		\newline  VE-15-12-2017 & \newline ROF2.1 \newline RFF2.2.1 \newline ROF5 \newline ROF5.2 \newline ROF5.3 \newline ROF5.4 \newline ROF6 \newline RDF6.1 \newline RDF9.3.1.1 \newline RDF9.3.1.2 \newline RDF9.4 \newline ROF9.5 \newline RFF9.8 \newline RFF11 \newline RFF11.1 \newline RFF12 \newline RFF12.1 \newline RFF13 \\[1em]
	\hline
		\newline VE-03-01-2018 & \newline ROF9.2 \newline RDF9.2.1 \newline RDF9.4.1 \newline ROF9.6 \\[1em]

	\hline
		\newline UC0 &  \newline ROF0 \\[1em]
	\hline
		\newline UC0.1 &  \newline ROF0 \\[1em]
	\hline
		\newline UC1 &  \newline ROF1 \newline ROF2 \newline RFF2.2.1   \\[1em]	
		\hline
		\newline UC2 &  \newline ROF2 \newline ROF2.1 \newline ROF3 \\[1em]	
		\hline
		\newline UC2.1 &  \newline ROF3.1 \\[1em]	
		\hline		
		\newline UC3 &  \newline ROF5 \newline ROF5.1 \\[1em]	
		\hline
		\newline UC3.1 &  \newline ROF4.1 \newline ROF5  \\[1em]
		\hline
		\newline UC3.1.1 &  \newline ROF4.1 \\[1em]
		\hline
		\newline UC3.1.2 &  \newline ROF5.2 \\[1em]
		\hline
		\newline UC3.2 &  \newline ROF5.1 \\[1em]
		\hline
		\newline UC3.2.1 &  \newline ROF5.3 \\[1em]
		\hline
		\newline UC3.2.2 &  \newline ROF5.4 \\[1em]
		\hline
		\newline UC4 &  \newline ROF4 \newline ROF4.1 \newline ROF4.1.1 \\[1em]
		\hline
		\newline UC4.1 &  \newline ROF4.1 \\[1em]
		\hline
		\newline UC5 &  \newline ROF14 \\[1em]
		\hline
		\newline UC5.1 &  \newline ROF14 \\[1em]
		\hline
		\newline UC6 &  \newline RDF6.1 \\[1em]
		\hline
		\newline UC6.1 &  \newline RDF6.1.1 \newline RDF6.1.2 \\[1em]
		\hline
		\newline UC6.2 &  \newline RFF2.2 \newline RDF6.1 \newline RDF6.1.1 \\[1em]
		\hline
		\newline UC6.3 &  \newline RFF2.2 \newline RDF6.1 \newline RDF6.1.2 \\[1em]
		\hline
		\newline UC7 &  \newline ROF7 \newline ROF8 \newline RFF13 \\[1em]
		\hline
		\newline UC7.1 &  \newline ROF8.1 \\[1em]
		\hline
		\newline UC7.2 &  \newline ROF9 \newline ROF9.1 \\[1em]
		\hline
		\newline UC7.3 &  \newline RFF10 \\[1em]
		\hline
		\newline UC8 &  \newline RFF11 \\[1em]
		\hline
		\newline UC8.1 &  \newline RFF11.1 \\[1em]
		\hline
		\newline UC9 &  \newline RFF12 \\[1em]
		\hline
		\newline UC9.1 &  \newline RFF12.1 \\[1em]
		\hline
		\newline UC10 &  \newline RDF9.4 \newline RDF9.4.1 \newline RDF9.4.2 \\[1em]
		\hline
		\newline UC10.1 &  \newline RDF9.4.2 \\[1em]
		\hline
		\newline UC11 &  \newline RFF2.2.1 \newline RFF2.2.1.1\\[1em]
		\hline
		\newline UC11.1 &  \newline RFF9.2.1.1 \\[1em]
		\hline
		\newline UC12 &  \newline ROF6 \\[1em]
		\hline			
		\newline UC13 &  \newline ROF9.9 \\[1em]
		\hline			
		\newline UC13.1 &  \newline ROF9.3 \newline ROF9.3.1 \newline RDF9.1 \newline RDF9.4.1 \newline ROF9.5 \\[1em]
		\hline
		\newline UC13.2 &  \newline ROF9.6 \\[1em]
		\hline
		\newline UC13.3 &   \newline RDF9.3.1.1 \\[1em]
		\hline
		\newline UC13.4 &  \newline RDF9.3.1.2 \\[1em]
		\hline
		\newline UC13.5 &  \newline ROF9.7 \\[1em]
		\hline
		
		\caption{Tracciamento Fonte Requisiti}
	\end{longtable}
\newpage
	\section{Tracciamento requisito-fonti}
	\begin{longtable}{| p{4cm} | p{4cm} |}
	
	\hline
\textbf{Requisiti} & \textbf{Fonte} \\
\hline
\endhead
	\newline ROF0&
	\newline \refer{UC0} \newline \refer{UC0.1} \newline Capitolato
	\\[1em]
	\hline
	\newline ROF1&
	\newline \refer{UC1} \newline Interno
	\\[1em]	
	
	\hline
	
	\newline ROF2&
	\newline \refer{UC1} \newline \refer{UC2} \newline Capitolato
	\\[1em]	
	\hline	
	
	\newline ROF2.1&
	\newline \refer{UC2} \newline  VE-15-12-2017
	\\[1em]	
	\hline	
	
	\newline RFF2.2&
	\newline \refer{UC6.2} \newline \refer{UC6.3} \newline Capitolato
	\\[1em]	
	\hline
	
	\newline RFF2.2.1&
	\newline \refer{UC1} \newline \refer{UC11} \newline  VE-15-12-2017
	\\[1em]	
	\hline
	
	\newline RFF2.2.1.1&
	\newline \refer{UC11.1} \newline Interno
	\\[1em]	
	\hline
	
	\newline ROF3&	
	\newline \refer{UC2} \newline  VE-15-12-2017
	\\[1em]	
	\hline	
	
	\newline ROF3.1&
	\newline \refer{UC2.1} \newline Interno
	\\[1em]		
	\hline
	
	\newline ROF4&
	\newline \refer{UC4} \newline Interno
	\\[1em]
	\hline
	
	\newline ROF4.1&
	\newline \refer{UC4} \newline \refer{UC3.1} \newline \refer{UC3.1.1} \newline Interno
	\\[1em]
	
	\hline	
	\newline ROF4.1.1&
	\newline \refer{UC4} \newline Interno
	\\[1em]
	
	\hline
	\newline ROF4.2&
	\newline \refer{UC4.1} \newline Interno
	\\[1em]
	\hline
	
	\newline RFF4.3&
	
	\newline Interno
	\\[1em]
	\hline
	
	\newline ROF5&
	
	\newline \refer{UC3} \newline \refer{UC3.1} \newline  VE-15-12-2017
	\\[1em]
	\hline
	
	\newline ROF5.1&
	
	\newline \refer{UC3} \newline \refer{UC3.2} \newline Interno
	\\[1em]	
	\hline
	
	
	\newline ROF5.2&
	
	\newline \refer{UC3.1.2} \newline  VE-15-12-2017
	\\[1em]	
	\hline
	
	\newline ROF5.3&
	
	\newline \refer{UC3.2.1} \newline  VE-15-12-2017
	\\[1em]	
	\hline
	
	\newline ROF5.4&
	
	\newline \refer{UC3.2.2} \newline  VE-15-12-2017
	\\[1em]	
	\hline
	
	\newline RDF5.5&
	
	\newline Interno
	\\[1em]
	\hline
	\newline RDF5.6&
	
	\newline Interno
	\\[1em]
	\hline
	\newline RDF5.7&
	
	\newline Interno
	\\[1em]
	\hline
	\newline ROF6&
	
	\newline \refer{UC12} \newline  VE-15-12-2017
	\\[1em]
	\hline
	
	\newline RDF6.1&
	
	\newline \refer{UC6} \newline \refer{UC6.2} \newline \refer{UC6.3} \newline  VE-15-12-2017
	\\[1em]
	\hline	
	
	\newline RDF6.1.1&
	
	\newline \refer{UC6.1} \newline \refer{UC6.2} \newline Interno
	\\[1em]
	\hline	
	
	\newline RDF6.1.2&
	
	\newline \refer{UC6.1} \newline \refer{UC6.3} \newline Interno
	\\[1em]
	\hline	
	
	\newline ROF7&
	
	\newline \refer{UC7} \newline Capitolato
	\\[1em]
	
	\hline
	\newline ROF8&
	
	\newline \refer{UC7} \newline Capitolato
	\\[1em]
	\hline
	\newline ROF8.1&
	
	\newline \refer{UC7.1} \newline Interno
	\\[1em]
	\hline
	
	\newline ROF9&
	
	\newline \refer{UC7.2} \newline Capitolato
	\\[1em]
	\hline
	
	
	
	\newline ROF9.1&
	
	\newline \refer{UC7.2} \newline Capitolato
	\\[1em]
	\hline
	
	\newline ROF9.2&
	
	\newline  VE-03-01-2018  \newline Capitolato
	\\[1em]
	\hline
	
	\newline RDF9.2.1&
	
	\newline  VE-03-01-2018
	\\[1em]
	\hline
	
	\newline ROF9.3&
	
	\newline \refer{UC13.1} \newline Capitolato
	\\[1em]
	\hline
	
	\newline ROF9.3.1&
	
	\newline \refer{UC13.1} \newline Capitolato
	\\[1em]
	\hline
	
	\newline RDF9.3.1.1&
	
	\newline \refer{UC13.3} \newline  VE-15-12-2017 \newline Capitolato
	\\[1em]
	\hline
	
	\newline RDF9.3.1.2&
	
	\newline \refer{UC13.4} \newline  VE-15-12-2017 \newline Capitolato
	\\[1em]
	\hline
	
	\newline RDF9.4&
	
	\newline \refer{UC10} \newline  VE-15-12-2017 \newline Capitolato
	\\[1em]
	\hline
	
	\newline RDF9.4.1&
	
	\newline \refer{UC10} \newline \refer{UC13.1} \newline  VE-03-01-2018 \newline Capitolato
	\\[1em]
	\hline
	
	\newline RDF9.4.2&
		\newline \refer{UC10.1} \newline Interno
	\\[1em]
	\hline
	
	\newline ROF9.5&
	\newline \refer{UC13.1} \newline  VE-15-12-2017 \newline Capitolato
	\\[1em]
	\hline
	
	\newline RDF9.5.1
	&\newline Interno
	\\[1em]
	\hline
	
	\newline ROF9.6&
	\newline \refer{UC13.2} \newline VE-03-01-2018
	\\[1em]
	\hline
	
	\newline ROF9.7&
	\newline \refer{UC13.5} \newline Capitolato
	\\[1em]
	\hline
	
	\newline RFF9.8&
	\newline VE-15-12-2017 \newline Capitolato
	\\[1em]
	\hline
	
	\newline RFF9.8.1&
	\newline Interno \newline Capitolato
	\\[1em]
	\hline
	
	\newline RFF9.8.2&
	\newline Interno \newline Capitolato
	\\[1em]
	\hline
	
	\newline ROF9.9&
	\newline \refer{UC13} \newline Capitolato
	\\[1em]
	\hline
	
	\newline RFF10&
	\newline \refer{UC7.3} \newline Capitolato
	\\[1em]
	\hline
	
	
	
	
	\newline RFF11&
	\newline \refer{UC8} \newline  VE-15-12-2017 
	\\[1em]
	\hline
	
	
	\newline RFF11.1&
	\newline \refer{UC8.1} \newline  VE-15-12-2017 
	\\[1em]
	\hline
	
	\newline RFF12&
	\newline \refer{UC9} \newline  VE-15-12-2017
	\\[1em]
	\hline
	\newline RFF12.1&
	\newline \refer{UC9.1} \newline  VE-15-12-2017 
	\\[1em]
	\hline
	
	\newline RFF12.2&
	\newline Capitolato
	\\[1em]
	\hline
	
	
	\newline RFF13&
	\newline \refer{UC7} \newline  VE-15-12-2017
	\\[1em]
	\hline
	
	
	
	\newline ROF14&	\newline \refer{UC5} \newline \refer{UC5.1} \newline Interno
	\\[1em]
	\hline
	
	

\newline 
ROQ0&\newline Capitolato
\\[1em]
\hline
\newline 
ROQ0.1&\newline Interno
\\[1em]
\hline
\newline
RDQ1&\newline Capitolato \newline Interno
\\[1em]
\hline	
\newline
RDQ1.1&\newline Capitolato
\\[1em]
\hline
\newline 
ROV0&\newline Capitolato
\\[1em]
\hline	
\newline 
ROV1&\newline Capitolato \newline Interno
\\[1em]
\hline
\newline 
ROV2&\newline Capitolato
\\[1em]
\hline
\newline
RDV2.1&\newline Capitolato
\\[1em]
\hline
\caption{Tracciamento Requisito Fonti}
\end{longtable}
\section{Riepilogo dei Requisiti}
	\begin{longtable}{| c | c | c | c | c |}
	\hline
	\textbf{Tipo} & \textbf{Obbligatorio}& \textbf{Facoltativo} &\textbf{Desiderabile} &\textbf{Totale} \\
	\hline
	\endhead
\textbf{Funzionale} &29& 15 & 13 & 57\\

\hline\textbf{Prestazionale} &0& 0& 0 &0  \\

\hline\textbf{Qualità} &2& 0&2&4 \\

\hline\textbf{Vincolo}&3&0& 1 & 4 \\

\hline\textbf{Totale}&34& 15& 16 & 65 \\

\hline
\caption{Riepilogo Requisiti}
\end{longtable}
\end{document}