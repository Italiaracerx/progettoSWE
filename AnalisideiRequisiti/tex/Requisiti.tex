\documentclass[../AnalisideiRequisiti.tex]{subfiles}

\begin{document}
	\chapter{Requisiti}
	\section{Requisiti Funzionali}
	\begin{longtable}{| p{3cm} | p{6cm} | p{3cm} |}
		\hline
		\textbf{Codice} & \textbf{Descrizione} & \textbf{Fonti}\\
		\hline
		\endhead
		\newline ROF0&
		\newline L'utente può visualizzare la pagina di DeSpeect&
		\newline UC0 \newline Capitolato
		\\[1em]
		\hline
		\newline ROF1&
		\newline L'utente può accedere al menu file&
		\newline UC1
		\\[1em]	
		
		\hline
			
		
		\newline ROF2&
		\newline L'utente può caricare un file Json&
		\newline UC2 \newline Capitolato
		\\[1em]	
			\hline	
			
		\newline ROF2.1&
		\newline L'utente può visualizzare il percorso del file JSon caricato&
		\newline Capitolato
		\\[1em]	
		\hline	
		
		\newline RFF2.2&
		\newline L'utente può modificare il file Json&
		\newline Capitolato
		\\[1em]	
		\hline
				
		\newline RFF2.2.1&
		\newline L'utente può spostare due utterance processor modificando il file Json&
		\newline VI-15-12-17
		\\[1em]	
		\hline
		
	
		\newline ROF3&
		\newline L'utente può inizializzare Speect con il file json&
		\newline UC2 \newline VI-15-12-17
		\\[1em]	
			\hline	
		
		\newline ROF3.1&
		\newline Il sistema deve visualizzare un errore in caso speect fallisca l'inizializzazione&
		\newline UC3.1
		\\[1em]		
		\hline
		
		\newline ROF4&
		\newline L'utente può salvare l'audio risultante con estensione WAV&
		\newline UC1 \newline VI-15-12-17 \newline Capitolato
		\\[1em]
			\hline
			
		\newline RFF4.1&
		\newline L'utente può salvare l'audio risultante con estensione diversa&
		\newline VI-15-12-17
		\\[1em]
		
		\hline
		\newline ROF4.2&
		\newline Il sistema deve visualizzare un errore in caso il salvataggio dell'audio fallisca&
		\newline UC4.1 \newline VI-15-12-17
		\\[1em]
		\hline
		
		\newline RFF4.3&
		\newline L'utente può ascoltare l'audio prima di salvarlo&
		\newline interno
		\\[1em]
		\hline
		
		\newline ROF5&
		\newline L'utente può cercare il file tramite file browser&
		\newline UC3 \newline UC3.1 \newline VI-15-12-17
		\\[1em]
		\hline
		
			\newline ROF5.1&
		\newline L'utente può aprire cartelle tramite file browser&
		\newline UC3.1 \newline UC3.1.1
		\\[1em]
		\hline
		
		\newline ROF5.2&
		\newline L'utente può tornare alla cartella padre&
		\newline UC3.1 \newline UC3.1.3
		\\[1em]
		\hline
		
		\newline RDF5.3&
		\newline L'utente può creare cartelle tramite file browser&
		\newline UC3 \newline UC3.2
		\\[1em]
		\hline
		
		\newline ROF5.4&
		\newline L'utente può selezionare un file tramite file browser&
		\newline UC3 \newline UC3.2
		\\[1em]	
		\hline
		
		\newline RDF5.5&
		\newline L'utente può selezionare un file tramite file browser&
		\newline VI-15-12-17
		\\[1em]	
		\hline
		
		\newline ROF5.6&
		\newline Il sistema visualizza un errore se si cerca di accedere ad una cartella senza i permessi necessari&
		\newline UC3.3 \newline VI-15-12-17
		\\[1em]	
		\hline
		\newline RDF5.7&
		\newline il file browser mostra solo file di estensione corretta&
		\newline interno
		\\[1em]
		\hline
		\newline ROF6&
		\newline L'utente può selezionare la utterance type&
		\newline UC6 
		\\[1em]
		\hline
				
		\newline ROF6.1&
		\newline L'utente può visualizzare gli utterance processor di un utterance type&
		\newline UC6 
		\\[1em]
		\hline
			
		\newline ROF7&
		\newline L'utente può inserire un testo da tradurre in voce&
		\newline UC7 \newline Capitolato
		\\[1em]
		
		\hline
		\newline ROF8&
		\newline L'utente può eseguire il testo inserito&
		\newline UC7 \newline Capitolato
		\\[1em]
		\hline
		\newline ROF8.1&
		\newline Il sistema visualizza l'errore di esecuzione se Speect fallisce l'esecuzione&
		\newline UC7.1 \newline Capitolato
		\\[1em]
		\hline
		
			\newline ROF9&
		\newline L'utente può visualizzare il grafo delle utterance&
		\newline UC7.2 \newline Capitolato
		\\[1em]
		\hline
		
			
		
			\newline RDF9.1&
		\newline L'utente può visualizzare l'informazione generale di ogni nodo&
		\newline UC7.2 \newline Capitolato
		\\[1em]
		\hline
		
		\newline ROF9.2&
		\newline L'utente può vedere ogni strato del grafo dello stesso colore del utterance processor che l'ha prodotto &
		\newline UC6 \newline Capitolato
		\\[1em]
		\hline
		
		\newline RDF9.2.1&
		\newline L'utente può cambiare il colore degli utterance processor&
		\newline UC7 \newline VI-15-12-17
		\\[1em]
		\hline
		
		\newline ROF9.3&
		\newline L'utente può selezionare il nodo del grafo tramite click&
		\newline UC7.2.1 \newline Capitolato
		\\[1em]
		\hline
		
			\newline ROF9.3.1&
		\newline L'utente può visualizzare tutte le informazioni del nodo selezionato&
		\newline UC7.2.1 \newline Capitolato
		\\[1em]
		\hline
			
		\newline RDF9.3.1.1&
		\newline L'utente può modificare le informazioni del nodo selezionato&
		\newline VI-15-12-17 \newline Capitolato
		\\[1em]
		\hline
		
		\newline RDF9.4&
		\newline L'utente può testare se un percorso porta ad un nodo esistente&
		 \newline VI-15-12-17 \newline Capitolato
		\\[1em]
		\hline
		
		\newline RDF9.4.1&
		\newline L'utente può selezionare il nodo del grafo tramite percorso nel grafo&
		\newline UC7.2.1 \newline VI-15-12-17 \newline Capitolato
		\\[1em]
		\hline
		
		\newline ROF9.5&
		\newline Il nodo selezionato dall'utente viene evidenziato con un contorno giallo&
		\newline UC7.2.1 \newline VI-15-12-17 \newline Capitolato
		\\[1em]
		\hline
		
		\newline RDF9.5.1&
		\newline L'utente può modificare il colore con il quale si evidenzia il focus&
		\newline interno
		\\[1em]
		\hline
		
		\newline ROF9.6&
		\newline L'utente può spostare i nodi del grafo graficamente&
		\newline UC7.2.2 \newline VI-15-12-17 \newline Capitolato
		\\[1em]
		\hline
		
		\newline RDF9.7&
		\newline L'utente può filtrare il grafo per strati di relazione&
		\newline Capitolato
		\\[1em]
		\hline
	
		\newline RFF9.8&
		\newline L'utente può modificare gli archi dei nodi del grafo&
		\newline VI-15-12-17 \newline Capitolato
		\\[1em]
		\hline
		
		\newline RFF9.8.1&
		\newline L'utente può cancellare gli archi dei nodi del grafo&
		\newline VI-15-12-17
		\\[1em]
		\hline
		
		\newline RFF9.8.2&
		\newline L'utente può aggiungere archi a dei nodi del grafo&
		\newline VI-15-12-17 
		\\[1em]
		\hline
	
		
		\newline RFF10&
		\newline L'utente eseguire ogni utterance processor autonomamente&
		\newline VI-15-12-17 \newline Capitolato
		\\[1em]
		\hline
	
		
	
		
		\newline RFF11&
		\newline L'utente può salvare lo stato del grafo&
		\newline VI-15-12-17 
		\\[1em]
		\hline
		
		
		\newline RFF12&
		\newline L'utente può caricare lo stato del grafo&
		\newline VI-15-12-17
		\\[1em]
		\hline
		
			\newline RFF12.1&
		\newline L'utente può caricare  graficamente due grafi diversi&
		\newline VI-15-12-17
		\\[1em]
		\hline
		
			\newline RFF12.1.1&
		\newline L'utente può confrontare due grafi automaticamente&
		\newline VI-15-12-17
		\\[1em]
		\hline
		
		
		\newline RFF13&
		\newline L'utente può eseguire Speect dato un grafo&
		\newline VI-15-12-17
		\\[1em]
		\hline
		
	
		
		\newline ROF14&
		\newline L'utente può chiudere l'applicazione&
		\newline UC5
		\\[1em]
		\hline
		
		
		
	\end{longtable}
\end{document}