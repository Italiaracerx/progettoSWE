\documentclass[../AnalisideiRequisiti.tex]{subfiles}

\begin{document}
	\chapter{Requisiti}
	\section{Requisiti Funzionali}
	\begin{longtable}{| p{2cm} | p{2.5cm} |p{5cm} | p{2.5cm} |}
		\hline
		\textbf{Codice} & \textbf{Importanza} & \textbf{Descrizione} & \textbf{Fonti}\\
		\hline
		\endhead
		\newline ROF0&
		\newline Obbligatorio&
		\newline L'utente può visualizzare la pagina di DeSpeect&
		\newline \refer{UC0} \newline Capitolato
		\\[1em]
		\hline
		\newline ROF1&
		\newline Obbligatorio&
		\newline L'utente può accedere al menu file&
		\newline \refer{UC1} \newline Interno
		\\[1em]	
		
		\hline
			
		\newline ROF2&
		\newline Obbligatorio&
		\newline L'utente può caricare un file Json&
		\newline \refer{UC1} \newline \refer{UC2} \newline Capitolato
		\\[1em]	
			\hline	
			
		\newline ROF2.1&
		\newline Obbligatorio&
		\newline L'utente può visualizzare il percorso del file JSon caricato&
		\newline \refer{UC2} \newline Capitolato 
		\\[1em]	
		\hline	
		
		\newline RFF2.2&
		\newline Facoltativo&
		\newline L'utente può modificare il file Json cambiando l'ordine degli utterance processor nel utterance type&
		\newline \refer{UC1} \newline \refer{UC6.1.1} \newline Capitolato
		\\[1em]	
		\hline
				
		\newline RFF2.2.1&
		\newline Facoltativo&
		\newline L'utente può salvane nel file JSon le modifiche agli utterance processor&
		\newline \refer{UC11} \newline VI-15-12-17
		\\[1em]	
		\hline
		
		\newline RFF2.2.1.1&
		\newline Facoltativo&
		\newline Il sistema deve visualizzare un errore nel caso il salvataggio fallisca e ripristinare uno stato funzionante&
		\newline \refer{UC11.1} 
		\\[1em]	
		\hline

		\newline ROF3&		\newline Obbligatorio&
		\newline L'utente può inizializzare Speect con il file json&
		\newline \refer{UC2} \newline VI-15-12-17
		\\[1em]	
			\hline	
		
		\newline ROF3.1&\newline Obbligatorio&
		\newline Il sistema deve visualizzare un errore in caso speect fallisca l'inizializzazione&
		\newline \refer{UC3.1}
		\\[1em]		
		\hline
		
		\newline ROF4&\newline Obbligatorio&
		\newline L'utente può salvare l'audio risultante con estensione WAV&
		\newline \refer{UC4} \newline Capitolato
		\\[1em]
			\hline
			
		\newline RFF4.1&\newline Facoltativo&
		\newline L'utente può salvare l'audio risultante con estensione diversa&
		\newline \refer{UC4} \newline \refer{UC4.1} \newline VI-15-12-17 
		\\[1em]
		\hline
		
		\newline ROF4.2&\newline Obbligatorio&
		\newline L'utente può selezionare dove salvare il file&
		\newline \refer{UC4} \newline \refer{UC3.1} \newline Interno
		\\[1em]
		
		\hline	
		\newline ROF4.2.1&\newline Obbligatorio&
		\newline L'utente può selezionare il nome del file da salvare&
		\newline \refer{UC4} \newline Interno
		\\[1em]
		
		\hline
		\newline ROF4.3&\newline Obbligatorio&
		\newline Il sistema deve visualizzare un errore in caso il salvataggio dell'audio fallisca&
		\newline \refer{UC4.1} \newline Interno
		\\[1em]
		\hline
		
		\newline RFF4.4&\newline Facoltativo&
		\newline L'utente può ascoltare l'audio prima di salvarlo&
		\newline Interno
		\\[1em]
		\hline
		
		\newline ROF5&\newline Obbligatorio&
		\newline L'utente può cercare il file tramite file browser&
		\newline \refer{UC3} \newline \refer{UC3.1} \newline VI-15-12-17
		\\[1em]
		\hline
		
			\newline ROF5.1&\newline Obbligatorio&
		\newline L'utente può aprire cartelle tramite file browser&
		\newline \refer{UC3.1} \newline \refer{UC3.1.1}
		\\[1em]
		\hline
		
		\newline ROF5.2&\newline Obbligatorio&
		\newline L'utente può tornare alla cartella padre&
		\newline \refer{UC3.1} \newline \refer{UC3.1.3}
		\\[1em]
		\hline
		
		\newline RDF5.3&\newline Desiderabile&
		\newline L'utente può creare cartelle tramite file browser&
		\newline \refer{UC3} \newline \refer{UC3.2}
		\\[1em]
		\hline
		
		\newline ROF5.4&\newline Obbligatorio&
		\newline L'utente può selezionare un file tramite file browser&
		\newline \refer{UC3} \newline \refer{UC3.2}
		\\[1em]	
		\hline
		
		
		\newline ROF5.5&\newline Obbligatorio&
		\newline Il sistema visualizza un errore se si cerca di accedere ad una cartella senza i permessi necessari&
		\newline \refer{UC3.3} \newline VI-15-12-17
		\\[1em]	
		\hline
		\newline RDF5.6&\newline Desiderabile&
		\newline il file browser mostra solo file di estensione corretta&
		\newline Interno
		\\[1em]
		\hline
		\newline RDF5.7&\newline Desiderabile&
		\newline L'utente può modificare il nome di una cartella tramite file browser&
		\newline Interno
		\\[1em]
		\hline
		\newline RDF5.8&\newline Desiderabile&
		\newline L'utente può modificare il nome di un file tramite file browser&
		\newline Interno
		\\[1em]
		\hline
		\newline ROF6&\newline Obbligatorio&
		\newline L'utente può selezionare la utterance type&
		\newline \refer{UC6} 
		\\[1em]
		\hline
				
		\newline RDF6.1&\newline Desiderabile&
		\newline L'utente può modificare gli utterance processor di un utterance type&
		\newline \refer{UC6.1}
		\\[1em]
		\hline	
				
		\newline RDF6.1.1&\newline Desiderabile&
		\newline L'utente può spostare gli utterance processor di un utterance type&
		\newline \refer{UC6.1} \newline \refer{UC6.1.1}
		\\[1em]
		\hline	
				
		\newline RDF6.1.2&\newline Desiderabile&
		\newline L'utente può cancellare gli utterance processor di un utterance type&
		\newline \refer{UC6.1} \newline \refer{UC6.1.1}
		\\[1em]
		\hline	
		
		\newline ROF7&\newline Obbligatorio&
		\newline L'utente può inserire un testo da tradurre in voce&
		\newline \refer{UC7} \newline Capitolato
		\\[1em]
		
		\hline
		\newline ROF8&\newline Obbligatorio&
		\newline L'utente può eseguire il testo inserito&
		\newline \refer{UC7} \newline Capitolato
		\\[1em]
		\hline
		\newline ROF8.1&\newline Obbligatorio&
		\newline Il sistema visualizza l'errore di esecuzione se Speect fallisce l'esecuzione&
		\newline \refer{UC7.1} \newline Capitolato
		\\[1em]
		\hline
		
			\newline ROF9&\newline Obbligatorio&
		\newline L'utente può visualizzare il grafo delle utterance&
		\newline \refer{UC7.2} \newline Capitolato
		\\[1em]
		\hline
		
			
		
			\newline ROF9.1&\newline Obbligatorio&
		\newline L'utente può visualizzare l'informazione generale di ogni nodo sul grafo&
		\newline \refer{UC7.2} \newline Capitolato
		\\[1em]
		\hline
		
		\newline ROF9.2&\newline Obbligatorio&
		\newline L'utente vede ogni relazione del grafo di un colore diverso, relativo alla legenda&
		\newline \refer{UC6} \newline Capitolato
		\\[1em]
		\hline
		
		\newline RDF9.2.1&\newline Desiderabile&
		\newline L'utente può cambiare il colore degli utterance processor&
		\newline \refer{UC7} \newline VI-15-12-17
		\\[1em]
		\hline
		
		\newline ROF9.3&\newline Obbligatorio&
		\newline L'utente può selezionare il nodo del grafo tramite click&
		\newline \refer{UC7.2.1} \newline Capitolato
		\\[1em]
		\hline
		
			\newline ROF9.3.1&\newline Obbligatorio&
		\newline L'utente può visualizzare tutte le informazioni del nodo selezionato&
		\newline \refer{UC7.2.1} \newline Capitolato
		\\[1em]
		\hline
			
		\newline RDF9.3.1.1&\newline Desiderabile&
		\newline L'utente può modificare il name del nodo selezionato&
		\newline \refer{UC7.2.3} \newline VI-15-12-17 \newline Capitolato
		\\[1em]
		\hline
		
			\newline RDF9.3.1.2&\newline Desiderabile&
		\newline L'utente può modificare il PoS del nodo selezionato&
		\newline \refer{UC7.2.6} \newline VI-15-12-17 \newline Capitolato
		\\[1em]
		\hline
		
		\newline RDF9.4&\newline Desiderabile&
		\newline L'utente può testare se un percorso porta ad un nodo esistente&
		\newline \refer{UC10} \newline VI-15-12-17 \newline Capitolato
		\\[1em]
		\hline
		
		\newline RDF9.4.1&\newline Desiderabile&
		\newline L'utente può selezionare il nodo del grafo tramite percorso nel grafo&
		\newline \refer{UC10} \newline \refer{UC7.2.1} \newline VI-15-12-17 \newline Capitolato
		\\[1em]
		\hline
		
		\newline ROF9.5&\newline Obbligatorio&
		\newline Il nodo selezionato dall'utente viene evidenziato con un contorno giallo&
		\newline \refer{UC7.2.1} \newline VI-15-12-17 \newline Capitolato
		\\[1em]
		\hline
		
		\newline RDF9.5.1&\newline Desiderabile&
		\newline L'utente può modificare il colore con il quale si evidenzia il focus&
		\newline Interno
		\\[1em]
		\hline
		
		\newline ROF9.6&\newline Obbligatorio&
		\newline L'utente può spostare i nodi del grafo graficamente&
		\newline \refer{UC7.2.2} \newline VI-15-12-17 \newline Capitolato
		\\[1em]
		\hline
		
		\newline RDF9.7&\newline Desiderabile&
		\newline L'utente può filtrare il grafo per strati di relazione&
		\newline \refer{UC7.2.4} \newline \refer{UC6} \newline Capitolato
		\\[1em]
		\hline
	
		\newline RFF9.8&\newline Facoltativo&
		\newline L'utente può modificare gli archi dei nodi del grafo&
		\newline VI-15-12-17 \newline Capitolato
		\\[1em]
		\hline
		
		\newline RFF9.8.1&\newline Facoltativo&
		\newline L'utente può cancellare gli archi dei nodi del grafo&
		\newline Interno
		\\[1em]
		\hline
		
		\newline RFF9.8.2&\newline Facoltativo&
		\newline L'utente può aggiungere archi a dei nodi del grafo&
		\newline Interno
		\\[1em]
		\hline
	
		
		\newline RFF10&\newline Facoltativo&
		\newline L'utente può eseguire ogni utterance processor autonomamente&
		\newline \refer{UC7.2.5} \newline Capitolato
		\\[1em]
		\hline
	
		
	
		
		\newline RFF11&\newline Facoltativo&
		\newline L'utente può salvare lo stato del grafo&
		\newline \refer{UC8} \newline VI-15-12-17 
		\\[1em]
		\hline


		\newline RFF11.1&\newline Facoltativo&
		\newline Il sistema deve visualizzare un errore se non riesce a salvare il grafo&
		\newline \refer{UC8} \newline \refer{UC8.1} \newline VI-15-12-17 
		\\[1em]
		\hline
		
		\newline RFF12&\newline Facoltativo&
		\newline L'utente può caricare lo stato di un grafo&
		\newline \refer{UC9} \newline VI-15-12-17
		\\[1em]
		\hline
				\newline RFF12.1&\newline Facoltativo&
		\newline Il sistema deve visualizzare un errore se non riesce a caricare il grafo&
		\newline \refer{UC9} \newline \refer{UC9.1} \newline VI-15-12-17 
		\\[1em]
		\hline
			\newline RFF12.2&\newline Facoltativo&
		\newline L'utente può caricare  graficamente due grafi diversi&
		\newline Capitolato
		\\[1em]
		\hline
		
			\newline RFF12.2.1&\newline Facoltativo&
		\newline L'utente può confrontare due grafi automaticamente&
		\newline Capitolato
		\\[1em]
		\hline
		
		
		\newline RFF13&\newline Facoltativo&
		\newline L'utente può eseguire Speect dato un grafo&
		\newline \refer{UC7} \newline \refer{UC7.2.5} \newline VI-15-12-17
		\\[1em]
		\hline
		
	
		
		\newline ROF14&\newline Obbligatorio&
		\newline L'utente può chiudere l'applicazione&
		\newline \refer{UC5}
		\\[1em]
		\hline
		
		
	\end{longtable}
	\section{Requisiti di Qualità}
			\begin{longtable}{| p{2cm} | p{2.5cm} |p{5cm} | p{2.5cm} |}
			\hline
			\textbf{Codice} & \textbf{Importanza} & \textbf{Descrizione} & \textbf{Fonti}\\
			\hline
			\endhead
				
			
			\newline ROQ0&\newline Obbligario&
		\newline Deve essere fornito un manuale utente&
		\newline Capitolato
		\\[1em]
		\hline
			\newline ROQ0.1&\newline Obbligario&
			\newline Il manuale deve essere in lingua italiana&
			\newline interno
			\\[1em]
			\hline
	\end{longtable}

	\section{Requisiti di Vincolo}
			\begin{longtable}{| p{2cm} | p{2.5cm} |p{5cm} | p{2.5cm} |}
			\hline
			\textbf{Codice} & \textbf{Importanza} & \textbf{Descrizione} & \textbf{Fonti}\\
			\hline
			\endhead
				\newline ROV0&\newline Obbligario&
			\newline 
			L'applicativo deve usare Speect modificato da Mivoq &
			\newline Capitolato
			\\[1em]
			\hline	
			\newline 
			ROV1&\newline Obbligario&
			\newline 
			L'applicativo deve essere sviluppato con QT &
			\newline Capitolato
			\newline Interno
			\\[1em]
			\hline
			\newline 
			ROV2&\newline Obbligario&
			\newline 
			L'applicativo deve essere utilizzabile su sistema operativo Linux Ubuntu 16.04 LTS&
			\newline Capitolato
			\\[1em]
			\hline
			\newline
			RDV2.1&\newline Desiderabile&
			\newline 
			L'applicativo deve essere utilizzabile su sistema operativo Windows 7 e successivi&
			\newline Capitolato
			\\[1em]
			\hline
			\newline
			RDV3&\newline Desiderabile&
			\newline 
			L'applicazione deve essere rilasciata con licenze opensource&
			\newline Capitolato
			\\[1em]
			\hline	
			\newline
			RDV3.1&\newline Desiderabile&
			\newline 
			L'applicazione deve essere rilasciata con licenze BSD/MIT &
			\newline Capitolato
			\\[1em]
			\hline
	\end{longtable}
\end{document}