
\documentclass[openany,12pt,a4paper]{report}
\usepackage{StileDoc}

\title{Strumenti - Norme di progetto - Verifica}
\author{Silvestri Kevin}
\date{\today}

\begin{document}

\chapter{Strumenti}

\section{Strumenti}
\subsection{Verifica ortografica}
Viene utilizzata la verifica dell’ortografia in tempo reale, strumento integrato in TexStudio
che sottolinea in rosso le parole errate secondo la lingua italiana.
\subsection{Analisi statica}
Per l’analisi statica del codice vengono utilizzati i seguenti strumenti:
\begin{itemize}
\item \textbf{Valgrind:} La \glossario{suite} di strumenti Valgrind fornisce numerosi strumenti di \glossario{debugging} e di \glossario{profiling} che aiutano a rendere i programmi più veloci e più corretti. Il più popolare di questi strumenti è chiamato Memcheck. È in grado di rilevare molti errori relativi alla memoria comuni nei programmi C e C ++ e che possono causare arresti anomali e comportamenti imprevedibili. Valgrind è accessibile al seguente link: \\ \centerline{\url{http://valgrind.org/}}
\end{itemize}
\subsection{Analisi dinamica}
Per l’esecuzione dei test di analisi dinamica vengono utilizzati i seguenti strumenti:
\begin{itemize}
\item \textbf{SonarQube:} è una piattaforma open source per la gestione della qualità del codice. SonarQube è un’applicazione web che produce reports sul codice duplicato, sugli standards di programmazione, i tests di unità, il \glossario{code coverage}, la complessità, i bugs potenziali, i commenti, la progettazione e l’architettura. SonarQube è accessibile al seguente link: \\ \centerline{\url{https://www.sonarqube.org/}}
\end{itemize}
\subsection{Metriche}
Per il controllo delle varie metriche vengono utilizzati i seguenti strumenti:
\begin{itemize}
\item \textbf{Better Code Hub:} Better Code Hub è un servizio di analisi del codice sorgente web-based che controlla il codice per la conformità rispetto a 10 linee guida per l'ingegneria del software e fornisce un \glossario{feedback} immediato per capire dove concentrarsi per miglioramenti di qualità. Better Code Hub è accessibile al seguente link:\\ \centerline{\url{https://bettercodehub.com/}}
\end{itemize}

\end{document}