\begin{document}

\section{Strumenti}

\subsection{IDE}

Il gruppo di progetto lavora sul seguente IDE:

\begin{itemize}
	\item \textbf{Qt}: per la codifica relativa al software richiesto dal progetto il gruppo usa l'\glossario{IDE}{IDE} \glossario{Qt}{Qt}, in virtù della sua diffusione, potenza e semplicità d'uso. Qt è reperibile al seguente link: \\ \centerline{\url{https://www.qt.io/}}.
\end{itemize}

\subsection{Compilazione}

Per la compilazione vengono utilizzati i seguenti strumenti:

\begin{itemize}
	\item \textbf{GCC}: il compilatore che verrà usato per la compilazione del software è il \glossario{GCC}{GCC} (GNU Compiler Collection). Il compilatore è reperibile al seguente link: \\ \centerline{\url{https://gcc.gnu.org/}};
	
	\item \textbf{Cmake}: per semplificare la compilazione verrà usato \glossario{Cmake}{Cmake}. Lo strumento è reperibile al seguente link: \\ \centerline{\url{https://cmake.org/}}.
\end{itemize}

\subsection{GUI}

Per progettare l'interfaccia grafica vengono utilizzati i seguenti strumenti:

\begin{itemize}
	\item \textbf{Qt Creator}: per progettare l'interfaccia grafica viene utilizzato \glossario{Qt Creator}{Qt Creator}. Questo strumento permette di realizzare interfacce grafiche mediante le librerie grafiche \glossario{Qt}{Qt}, diventate in questo ambito quasi uno standard per piattaforme Linux Based. Qt Creator è disponibile al seguente link: \\  \centerline{\url{https://www.qt.io/qt-features-libraries-apis-tools-and-ide/}}. 
\end{itemize}

\subsection{Tracciamento requisiti e casi d'uso}

Per catalogare i requisiti e i casi d'uso il gruppo utilizzerà l'applicativo web \glossario{Trender}{Trender}, software in grado di velocizzare e automatizzare la gestione dei requisiti e dei casi d'uso. Lo strumento è accessibile al seguenti link: \\ \centerline{\url{https://github.com/campagna91/Trender}}

\subsection{Diagrammi UML}

Per la produzione di diagrammi UML utilizzeremo \glossario{Astah}{Astah}, software in grado di velocizzare la produzione dei diagrammi. Lo strumento è accessibile al seguenti link: \\ \centerline{\url{http://astah.net/}}
	
\end{document}