
\documentclass[openany,12pt,a4paper]{report}
\usepackage{StileDoc}

\title{Strumenti - Norme di progetto - Processi primari - Sviluppo}
\author{Cristiano Tessarolo}
\date{\today}

\begin{document}

\chapter{Strumenti}

\section{Strumenti}
\subsection{Compilazione}
	Per la compilazione vengono usati i seguenti strumenti:
	\begin{itemize}
		\item Il compilatore che verrà usato per la compilazione del software è il \glossaryentry{GCC}{GCC} (GNU Compiler Collection). Il compilatore è accessibile al seguente link: \\ \centerline{\url{https://gcc.gnu.org/}}
		\item Per semplificare la compilazione verrà usato \glossaryentry{Cmake}{Cmake}. Lo strumento è accessibile al seguenti link: \\ \centerline{\url{https://cmake.org/}}
	\end{itemize}
\subsection{GUI}
	Per progettare l'interfaccia grafica utilizzemo \glossaryentry{QT Creator}{QT Creator} disponibile al seguente link: \\  \centerline{\url{https://www.qt.io/qt-features-libraries-apis-tools-and-ide/#ide}}. Questo strumento ci permettere di realizzare un'intefaccia grafica mediante le librerie grafiche \glossaryentry{QT}{QT}, diventate quasi uno standard nelle piattaforme Linux Based.
	
\subsection{Tracciento requisiti e casi d'uso}
	Per catalogare i requisiti e i casi d'uso il gruppo utilizzerà l'applicativo web \glossaryentry{Tender}{Tender}, software in grado di velocizzare e automatizzare la gestione dei requisiti e dei casi d'uso.
	
	\subsection{Diagrammi UML}
	Per la produzione di diagrammi UML utilizzeremo \glossaryentry{Astah}{Astah}, software in grado di velocizzare la produzione dei diagrammi. Lo strumento è accessibile al seguenti link: \\ \centerline{\url{http://astah.net/}}	
\end{document}