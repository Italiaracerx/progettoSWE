\documentclass[openany,12pt,a4paper]{report}

\title{Strumentazione}
\author{Matteo Rizzo}
\date{\today}

\begin{document}

\maketitle

% INDICE
 
\tableofcontents{}

% INTRODUZIONE

\chapter{Introduzione}

\section{Scopo}

Lo scopo del documento è quello di esporre in maniera dettagliata l'insieme degli strumenti che il gruppo Graphite andrà ad utilizzare per documentare e codificare il prodotto designato dal capitolato scelto.

% DOCUMENTAZIONE

\chapter{Documentazione}

\section{Scopo}

Il capitolo vuole esporre gli strumenti utilizzati per redarre la documentazione relativa al progetto. 

\section{Descrizione}

Per redarre la documentazione, il gruppo utilizza i seguenti strumenti:

\begin{itemize}
    \item \textbf{LaTex}: per la stesura della documentazione viene utilizzato il linguaggio \LaTeX{}, data la sua flessibilità, potenza e facilità d'uso;
    
    \item \textbf{TexStudio}: per la stesura del codice \LaTeX{} viene utilizzato l’editor \glossario{TexStudio}{TexStudio}, in virtù delle innumerevoli feature che offre gratuitamente e del fatto che si tratta di un'applicazione cross-platform supportata dai maggiori sistemi operativi;
    
    \item \textbf{LucidChart}: per la realizzazione di diagrammi illustrativi per i documenti viene utilizzata la piattaforma web collaborativa \glossario{Lucidchart}{Lucidchart};
    	
    \item \textbf{Tracciamento requisiti e casi d'uso}: per catalogare i requisiti e i casi d'uso in modo veloce e automatizzato il gruppo utilizza l'applicativo web \glossario{Tender}{Tender};
	
	\item \textbf{Diagrammi UML}: per la produzione di diagrammi UML viene utilizzato \glossario{Astah}{Astah}. Lo strumento è accessibile al seguenti link: \\ \centerline{\url{http://astah.net/}}	
\end{itemize}

% ORGANIZZAZIONE INTERNA

\chapter{Organizzazione interna}

\section{Scopo}

\section{Descrizione}
	
	\begin{itemize}
	    \item \textbf{Telegram}: \glossario{Telegram}{Telegram} è una applicazione di messaggistica nata come applicazione mobile e successivamente portata anche su Windows, Mac e varie distribuzioni Linux. Rispetto agli altri sistemi di messaggistica, Telegram consente un facile passaggio di immagini e documenti in più formati, mantenendo inoltre nel proprio cloud storage tali file per un agevole recupero su qualsiasi dispositivo. È possibile creare gruppi di utenti la cui chat ha il valore aggiunto di poter contenere sistemi automatici per l'organizzazione di sondaggi e la comunicazione di messaggi importanti da tenere in sovraimpressione;
	    
	    \item \textbf{GitHub}: \glossario{Github}{Github} è un servizio di \glossario{hosting}{hosting} per progetti software. Il sito è principalmente utilizzato dagli sviluppatori, che caricano il codice sorgente dei loro programmi e lo rendono scaricabile dagli utenti, ma può essere utilizzato anche per la condivisione e la modifica di file di testo e documenti revisionabili. Un utente può interagire con lo sviluppatore tramite un sistema di issue tracking, pull request e commenti che permette di migliorare il codice della repository, risolvendo bug o aggiungendo funzionalità.
	\end{itemize}

% SVILUPPO

\chapter{Sviluppo}

\section{Scopo}
Il capitolo vuole esporre gli strumenti utilizzati per sviluppare il software richiesto dal progetto dal progetto. 
\section{Descrizione}
Per lo sviluppo software relativo al progetto, il gruppo utilizza i seguenti strumenti:

\subsection{Sistema operativo}
Il gruppo di progetto lavora sui seguenti sistemi operativi:
\begin{itemize}
	\item Ubuntu 17.10 x64;
	\item Ubuntu 16.04 \glossario{LTS}{LTS} x64;
	\item Windows 10 Home x64;
	\item Windows 10 Pro x64;
	\item Windows 7 Home Premium.
\end{itemize}

\subsection{IDE}
Il gruppo di progetto lavora sul seguente IDE:
\begin{itemize}
    \item \textbf{Qt}: per la codifica relativa al software richiesto dal progetto il gruppo usa l'\glossario{IDE}{IDE} \glossario{Qt}{Qt}, in virtù della sua diffusione, potenza e semplicità d'uso. Qt è reperibile al seguente link: \\ \centerline{https://www.qt.io/}.
\end{itemize}

\subsection{Compilazione}
Per la compilazione vengono utilizzati i seguenti strumenti:

\begin{itemize}
	\item \textbf{GCC}: il compilatore che verrà usato per la compilazione del software è il \glossario{GCC}{GCC} (GNU Compiler Collection). Il compilatore è reperibile al seguente link: \\ \centerline{\url{https://gcc.gnu.org/}};
	
	\item \textbf{Cmake}: per semplificare la compilazione verrà usato \glossario{Cmake}{Cmake}. Lo strumento è reperibile al seguente link: \\ \centerline{\url{https://cmake.org/}}.
\end{itemize}

\subsection{GUI}
Per progettare l'interfaccia grafica vengono utilizzati i seguenti strumenti:

\begin{item}
    \item \textbf{Qt Creator}: per progettare l'interfaccia grafica viene utilizzato \glossario{Qt Creator}{Qt Creator}. Questo strumento permette di realizzare interfacce grafiche mediante le librerie grafiche Qt, diventate in questo ambito quasi uno standard per piattaforme Linux Based. Qt Creator è disponibile al seguente link: \\  \centerline{\url{https://www.qt.io/qt-features-libraries-apis-tools-and-ide/#ide}}.
\end{item}

% VERIFICA

\chapter{Verifica}

\section{Scopo}

Il capitolo vuole esporre gli strumenti utilizzati al fine di eseguire controlli e verifiche sul codice e la documentazione relativi al progetto. 

\section{Descrizione}

Per eseguire controlli e verifiche, il gruppo utilizza i seguenti strumenti:

\subsection{Verifica ortografica}
Per verificare la correttezza dell'ortografia nell'ambito della documentazione vengono utilizzati i seguenti strumenti:

\begin{itemize}
    \item \textbf{Verifica dell'ortografia}: per eseguire controlli ortografici sulla documentazione viene utilizzata la verifica dell’ortografia in tempo reale, strumento integrato in TexStudio che sottolinea in rosso le parole errate secondo la lingua italiana.
\end{itemize}

\subsection{Analisi statica}
Per l’analisi statica del codice vengono utilizzati i seguenti strumenti:
    \begin{itemize}
        \item \textbf{Valgrind:} La \glossario{suite} di strumenti Valgrind fornisce numerosi strumenti di \glossario{debugging} e di \glossario{profiling} che aiutano a rendere i programmi più performanti e più corretti. Il più popolare di questi strumenti è chiamato Memcheck, ed è in grado di rilevare molti errori relativi alla memoria comuni nei programmi C e C ++ e che possono causare arresti anomali e comportamenti imprevedibili. Valgrind è accessibile al seguente link: \\ \centerline{\url{http://valgrind.org/}}
    \end{itemize}
    
\subsection{Analisi dinamica}
Per l’esecuzione dei test di analisi dinamica vengono utilizzati i seguenti strumenti:
    \begin{itemize}
        \item \textbf{SonarQube:} è una piattaforma open source per la gestione della qualità del codice. SonarQube è un’applicazione web che produce report sul codice duplicato, sugli standard di programmazione, i test di unità, il \glossario{code coverage}, la complessità, i bug potenziali, i commenti, la progettazione e l’architettura. SonarQube è accessibile al seguente link: \\ \centerline{\url{https://www.sonarqube.org/}}
    \end{itemize}
    
\subsection{Metriche}
Per il controllo delle varie metriche vengono utilizzati i seguenti strumenti:
    \begin{itemize}
        \item \textbf{Better Code Hub:} Better Code Hub è un servizio di analisi del codice sorgente web-based che controlla il codice per la conformità rispetto a 10 linee guida per l'ingegneria del software e fornisce un \glossario{feedback} immediato per capire dove concentrarsi per miglioramenti di qualità. Better Code Hub è accessibile al seguente link:\\ \centerline{\url{https://bettercodehub.com/}}
    \end{itemize}
    
\end{document}
