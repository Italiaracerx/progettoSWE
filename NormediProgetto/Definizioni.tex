
\usepackage{glossaries}
\usepackage[utf8x]{inputenc}
\usepackage[italian]{babel}

\makeglossaries
\newglossaryentry{Piano di Qualifica}
{
	name={piano di qualifica},
	description={documento che dovrà illustrare la strategia complessiva di verifica e validazione proposta dal fornitore per pervenire al collaudo del sistema con la massima efficienza ed efficacia.},
	nonumberlist 
}
\newglossaryentry{implementazione}
{
	name={implementazione},
	description={la realizzazione concreta di una procedura a partire dalla sua definizione logica.},
	nonumberlist 
}
\newglossaryentry{collaudo}
{
	name={collaudo},
	description={controllo, verifica dei requisiti tecnologici ed economici in rapporto a una tabella di caratteristiche singolarmente o universalmente prestabilite.},
	nonumberlist 
}
\newglossaryentry{proponente}
{
	name={proponente},
	description={chi propone e presenta qualcosa a qualcuno affinché venga accettato, approvato.},
	nonumberlist 
}
\newglossaryentry{suite}
{
	name={suite},
	description={pacchetto di programmi complementari, in grado di interagire e di scambiarsi reciprocamente i dati.},
	nonumberlist 
}
\newglossaryentry{debugging}
{
name={debugging},
description={l'attività che consiste nell'individuazione e correzione da parte del programmatore di uno o più errori (bug) rilevati nel software.},
nonumberlist 
}
\newglossaryentry{profiling}
{
name={profiling},
description={l'attività di raccolta di informazioni su qualcosa al fine di dare una descrizione di ciò.},
nonumberlist 
}
\newglossaryentry{code coverage}
{
name={code coverage},
description={una misura utilizzata per descrivere il grado di esecuzione del codice sorgente di un programma quando viene eseguita una particolare suite di test.},
nonumberlist 
}
\newglossaryentry{feedback}
{
name={feedback},
description={valutazione, giudizio sul lavoro o sul comportamento.},
nonumberlist 
}
\newglossaryentry{committente}
{
name={committente},
description={organizzazione o persona che commissiona un lavoro. Essa detta tempi e costi con i quali il team può portare avanti il lavoro commissionato;},
nonumberlist 
}
\newglossaryentry{Task}
{
name={task},
description={attività da svolgere entro una scadenza prefissata. Essa può essere suddivisa in più compiti o subtask anche essi soggetti a scadenza il cui insieme porta al completamento del task.},
nonumberlist 
}
\newglossaryentry{LTS}
{
name={LTS},
description={acronimo di Long Term Service è una versione di Ubuntu rilasciata ogni due anni con contenuti e aggiornamenti molto più testati e affidabili delle normali versioni di Ubuntu. Garantisce inoltre gli aggiornamenti per un periodo di almeno 2 anni prima di obbligare l'utente al cambio di versione del sistema operativo.},
nonumberlist 
}
\newglossaryentry{hosting}
{
name={hosting},
description={servizio di rete che consiste nell'allocare su un server web le pagine di un sito o di un'applicazione web consentendo agli utenti di accedere ad esse tramite la rete internet.},
nonumberlist 
}
\newglossaryentry{processo}
{
name={processo},
description={Insieme di attività correlate che trasformano ingressi in uscite secondo regole fissate.},
nonumberlist 
}
\newglossaryentry{Responsabile di Progetto}
{
name={responsabile di progetto},
description={è la figura responsabile della gestione di un progetto, più precisamente dell'istanziazione di processi nel progetto, della stima dei costi e delle risorse necessarie, della pianificazione delle attività e della loro assegnazione, del controllo delle attività e verifica dei risultati.},
nonumberlist 
}
\newglossaryentry{Amministratori}
{
name={amministratore},
description={è la figura che ha il compito di controllare che ad ogni istante della vita del progetto le risorse (umane, materiali, economiche e strutturali) siano presenti e operanti; inoltre, gestisce la documentazione e controlla il versionamento e la configurazione.},
nonumberlist 
}
\newglossaryentry{Analisti}
{
name={analista},
description={è la figura che, a partire dal bisogno del cliente, individua il problema (di cui conosce il dominio) da fornire al progettista.},
nonumberlist 
}
\newglossaryentry{analisti}
{
	name={analista},
	description={è la figura che, a partire dal bisogno del cliente, individua il problema (di cui conosce il dominio) da fornire al progettista.},
	nonumberlist 
}
\newglossaryentry{Verificatori}
{
name={verificatore},
description={figura designata alla verifica del lavoro prodotto dai programmatori.},
nonumberlist 
}
\newglossaryentry{Verifica}
{
name={verifica},
description={essa attiene alla coerenza, completezza e correttezza del prodotto. È un processo che si applica ad ogni "segmento" temporale di un progetto (ad ogni prodotto intermedio) per accertare che le attività svolte in tale segmento non abbiano introdotto errori nel prodotto.},
nonumberlist 
}
\newglossaryentry{Validazione}
{
name={validazione},
description={essa accerta la conformità di un prodotto alle attese, fornisce una prova oggettiva di come le specifiche del prodotto siano conformi al suo scopo e alle esigenze degli utenti.},
nonumberlist 
}
\newglossaryentry{Manuale utente}
{
name={Manuale utente},
description={documento destinato all'utilizzatore finale del prodotto, che si presuppone privo di competenze tecniche specifiche. Contiene le informazioni utili al corretto utilizzo di un prodotto;},
nonumberlist 
}
\newglossaryentry{Programmatore}
{
name={programmatore},
description={chi è incaricato della stesura dei programmi per calcolatori sulla base delle specifiche di programma assegnate dai progettisti;},
nonumberlist 
}
\newglossaryentry{TexStudio}
{
name={TexStudio},
description={editor per la stesura di documenti Latex;},
nonumberlist 
}
\newglossaryentry{Lucidchart}
{
name={Lucidchart},
description={piattaforma web per realizzare diagrammi illustrativi.},
nonumberlist 
}
\newglossaryentry{Dominio Applicativo}
{
	name={dominio applicativo},
	description={},
	nonumberlist 
}
\newglossaryentry{requisiti}
{
name={requisiti},
description={},
nonumberlist 
}
\newglossaryentry{requisito di vincolo}
{
name={requisito di vincolo},
description={},
nonumberlist 
}
\newglossaryentry{requisito funzionale}
{
name={requisiti di vincolo},
description={},
nonumberlist 
}
\newglossaryentry{requisito prestazionale}
{
name={requisito prestazionale},
description={},
nonumberlist 
}
\newglossaryentry{requisito di qualita}
{
	name={requisito di qualità},
	description={},
	nonumberlist 
}
\newglossaryentry{caso d'uso}
{
name={casi d'uso},
description={},
nonumberlist 
}
\newglossaryentry{responsabile di progetto}
{
name={casi d'uso},
description={},
nonumberlist 
}
\newglossaryentry{analista}
{
name={casi d'uso},
description={},
nonumberlist 
}
\newglossaryentry{dominio applicativo}
{
name={casi d'uso},
description={},
nonumberlist 
}
\newglossaryentry{amministratore}
{
name={casi d'uso},
description={},
nonumberlist 
}
\newglossaryentry{verificatore}
{
name={casi d'uso},
description={},
nonumberlist 
}
\newglossaryentry{verifica}
{
name={casi d'uso},
description={},
nonumberlist 
}
\newglossaryentry{validazione}
{
name={casi d'uso},
description={},
nonumberlist 
}
\newglossaryentry{requisito}
{
name={casi d'uso},
description={},
nonumberlist 
}
\newglossaryentry{manuale utente}
{
name={casi d'uso},
description={},
nonumberlist 
}
\newglossaryentry{programmatore}
{
name={casi d'uso},
description={},
nonumberlist 
}
\newglossaryentry{piano di qualifica}
{
name={casi d'uso},
description={},
nonumberlist 
}
\newglossaryentry{qualita}
{
name={casi d'uso},
description={},
nonumberlist 
}
\newglossaryentry{Telegram}
{
name={casi d'uso},
description={},
nonumberlist 
}
\newglossaryentry{dispositivo mobile}
{
name={casi d'uso},
description={},
nonumberlist 
}
\newglossaryentry{Asana}
{
name={casi d'uso},
description={},
nonumberlist 
}
\newglossaryentry{task}
{
name={casi d'uso},
description={},
nonumberlist 
}
\newglossaryentry{GitHub}
{
name={casi d'uso},
description={},
nonumberlist 
}