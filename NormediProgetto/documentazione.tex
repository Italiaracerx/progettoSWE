\documentclass[NormediProgetto.tex]{subfiles}


\begin{document}

\chapter{Processi di supporto}

\section{Documentazione}

Questo processo descrive dettagliatamente tutte le norme e le convenzioni adottate per la redazione della documentazione riguardante il progetto.

\subsection{Template}

È stato creato un template \LaTeX{} per uniformare la grafica dei documenti in modo da velocizzare il processo di documentazione.

\subsection{Struttura dei documenti}

\subsubsection{Prima pagina}

La prima pagina di ogni documento è strutturata nel seguente modo:

\begin{itemize}
    \item{\textbf{Logo del gruppo:}} primo elemento centrato in alto; 
    \item{\textbf{Titolo:}} nome del documento, centrato e posizionato sotto il logo; 
    \item{\textbf{Gruppo e progetto:}} nome del gruppo e del progetto, centrato e subito sotto al titolo del documento; 
    \item{\textbf{Indirizzo e-mail del gruppo:}} centrato e sotto il nome del gruppo e del progetto; 
    \item{\textbf{Tabella informativa:}}contiene le seguenti informazioni: 
    \begin{itemize} 
        \item versione del documento;
        \item nome e cognome di chi ha svolto la redazione del documento; 
        \item nome e cognome degli incaricati alla verifica del documento; 
        \item nome e cognome dell'incaricato all’approvazione del documento;
        \item tipo di uso; 
        \item  destinatari del documento.
        \end{itemize}
    \item {\textbf{Descrizione:}} sintetica del contenuto del documento.
\end{itemize}

\subsubsection{Registro delle modifiche}

Posizionato dopo la prima pagina, il registro delle modifiche deve contenere tutte le modifiche apportate al documento stesso, indicando per ognuna:

\begin{itemize}
\item versione del documento dopo la modifica;
\item data della modifica;
\item nome e cognome dell'autore della modifica;
\item ruolo dell'autore della modifica;
\item breve descrizione della modifica.
\end{itemize}

\subsubsection{Indice}

Ogni documento deve avere un indice che ne agevoli la consultazione e permetta una visione generale degli argomenti trattati nel documento. L'indice è strutturato in gerarchie ed è posizionato dopo il registro delle modifiche.

\subsubsection{Contenuto principale}

 Ad eccezione della prima, tutte le pagine devono contenere un’intestazione ed un piè di pagina. L’intestazione è strutturata nel seguente modo: 

\begin{itemize}
\item Logo del gruppo posto a sinistra;
\item Il titolo del capitolo posto a destra;
\end{itemize}

Il piè di pagina è così strutturato:

\begin{itemize}
\item Data e ora dell'ultima modifica del documento, posti a sinistra;
\item Numerazione progressiva della pagina posta a destra.
\end{itemize}

\subsubsection{Note a piè di pagina}

In caso di presenza in una pagina interna di note da esplicare, esse vanno indicate nella pagina corrente, in basso a sinistra. Ogni nota deve riportare un numero e una descrizione.

\subsection{Versionamento}

Ogni documento è accompagnato da un numero di versionamento, dove ogni versione corrisponde ad una riga nel registro delle modifiche, ed è espresso nel modo seguente:

\[\textbf{v $\biggl\{$A$\biggr\}$.$\biggl\{$B$\biggr\}$.$\biggl\{$C$\biggr\}$}\]

dove:

\begin{itemize}

\item{\textbf{A:}} è l'indice principale. Viene incrementato dal \textit{Responsabile di Progetto} all’approvazione del documento e 
corrisponde al numero di revisione.

\item{\textbf{B:}} è l'indice di verifica. Viene incrementato dal \textit{Verificatore\ped{G}} ad ogni verifica. Quando viene incrementato A, riparte da 0.

\item{\textbf{C:}} è l'indice di modifica. Viene incrementato dal redattore del documento ad ogni modifica. Quando viene incrementato B, riparte da 0.

\end{itemize}

\subsection{Norme tipografiche}

\subsubsection{Stile del testo}

\begin{itemize}

\item{\textbf{Glossario:}} ogni parola contenuta nel glossario deve essere marcata, alla sua prima occorrenza in ogni documento, in corsivo e con una G maiuscola a pedice (es: \textit{glossario\ped{G}});  

\item{\textbf{Grassetto:}} viene applicato ai titoli e agli elementi di un elenco puntato seguiti da una descrizione, può essere usato anche per mettere in risalto parole significative; 

\item{\textbf{Corsivo:}} Il corsivo dev’essere utilizzato per:
\begin{itemize}
\item citazioni;
\item parole inserite nel glossario;
\item attività del progetto;
\item ruoli del progetto;
\item riferimenti ad altri documenti;
\item parole particolari solitamente poco usate o conosciute.
\end{itemize}

\item{\textbf{Maiuscolo:}} deve essere usato solo per gli acronimi.

\end{itemize}

\subsubsection{Elenchi puntati}

 Gli elenchi puntati servono ad esprimere in modo sintetico un concetto, evitando frasi lunghe e discorsive. Ogni voce di un elenco puntato deve terminare con un punto e virgola, ad eccezione dell’ultima, che va terminata con un punto.

\subsubsection{Formati}

\begin{itemize}

\item{\textbf{Date:}}  \[\textbf{GG-MM-AAAA}\]
\begin{itemize}
\item{\textbf{GG:}} rappresenta il giorno del mese in cifre;
\item{\textbf{MM:}} rappresenta il mese in cifre;
\item{\textbf{AAAA:}} rappresenta l'anno in cifre per intero.

\end{itemize}

\item{\textbf{Orari:}} \[\textbf{HH:MM}\]
\begin{itemize}
\item{\textbf{HH:}} rappresenta l'ora;
\item{\textbf{MM:}} rappresenta i minuti.
\end{itemize}

\item{\textbf{Nomi ricorrenti:}}
\begin{itemize}
\item{\textbf{Ruoli di progetto:}} ogni nome di ruolo di progetto viene scritto in corsivo e con l’iniziale maiuscola;
\item{\textbf{Nomi dei documenti:}}  ogni nome di documento viene scritto in corsivo e con l’iniziale di ogni parola che non sia un articolo maiuscola;
\item{\textbf{Nomi propri:}} ogni nome proprio di persona deve essere scritto nella forma \textit{Nome Cognome}.
\end{itemize}

\item{\textbf{Link:}} i link dovranno essere scritti attraverso il comando \LaTeX{} \textit{href}.
\end{itemize}

\subsubsection{Sigle}

È previsto l’utilizzo delle seguenti sigle: 

\begin{itemize}
\item{\textbf{AR:}} \textit{Analisi dei Requisiti};
\item{\textbf{PP:}} \textit{Piano di Progetto};
\item{\textbf{NP:}} \textit{Norme di Progetto};
\item{\textbf{SF:}} \textit{Studio di Fattibilità};
\item{\textbf{PQ:}} \textit{Piano di Qualifica};
\item{\textbf{ST:}} \textit{Specifica Tecnica};
\item{\textbf{MU:}} \textit{Manuale utente\ped{G}};
\item{\textbf{DP:}} \textit{Definizione di Prodotto};
\item{\textbf{RR:}} Revisione dei requisiti;
\item{\textbf{RP:}} Revisione di progettazione;
\item{\textbf{RQ:}} Revisione di qualifica;
\item{\textbf{RA:}} Revisione di accettazione;
\item{\textbf{Re:}} \textit{Responsabile di Progetto};
\item{\textbf{Am:}} \textit{Amministratore di Progetto};
\item{\textbf{An:}} \textit{Analista};
\item{\textbf{Pt:}} \textit{Progettista};
\item{\textbf{Pr:}} \textit{Programmatore\ped{G}};
\item{\textbf{Ve:}} \textit{Verificatore}.
\end{itemize}

\subsection{Elementi grafici}

\subsubsection{Tabelle}

Ogni tabella deve possedere una didascalia in cui deve comparire il numero identificativo, per agevolarne il tracciamento, ed una breve descrizione del suo contenuto.

\subsubsection{Immagini}

Ogni immagine deve essere centrata e separata dai paragrafi prima e dopo di essa. Le immagini devono avere una didascalia analoga a quella delle tabelle. Tutti i diagrammi UML vengono inseriti come immagini.

\subsection{Classificazione dei documenti}

\subsubsection{Documenti informali}

Tutte le versioni dei documenti che non siano state approvate dal \textit{Responsabile di Progetto} sono ritenute informali e, in quanto tali, sono considerate esclusivamente ad uso interno.

\subsubsection{Documenti formali}

Una versione di un documento viene considerata formale quando è stata approvata dal \textit{Responsabile di Progetto}. Solo i documenti formali possono essere distribuiti all’esterno del gruppo. 

\subsection{Procedura di approvazione}

Ogni documento non formale completato dovrà essere sottoposto al \textit{Responsabile di Progetto}, che a sua volta si occuperà di incaricare i \textit{Verificatori} di controllarne la correttezza del contenuto e della forma. Se vengono individuati degli errori, i \textit{Verificatori} li riporteranno al \textit{Responsabile di Progetto}, che a sua volta incaricherà il redattore del documento di correggerli. Questo ciclo va ripetuto fino a che il documento non è considerato corretto. Successivamente sarà sottoposto al Responsabile di Progetto, che potrà approvarlo o meno. Quando approvato, il documento verrà considerato un documento formale. In caso contrario il \textit{Responsabile di Progetto} dovrà comunicare le motivazioni per cui il documento non è stato approvato, specificando le modifiche da apportare.

\subsection{Strumenti}

\subsubsection{\LaTeX{}}

Per la stesura della documentazione è stato utilizzato il linguaggio \LaTeX{} per la sua flessibilità e facilità d'uso.
Per la stesura del codice è stato utilizzato l’editor \textit{TexStudio\ped{G}}.

\subsubsection{Lucidchart}

Per la realizzazione di diagrammi illustrativi per i documenti viene utilizzata la piattaforma web \textit{Lucidchart\ped{G}}.

\end{document}