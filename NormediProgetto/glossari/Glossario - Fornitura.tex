
\documentclass[openany,12pt,a4paper]{report}
\usepackage{StileDoc}

\title{Glossario - Norme di progetto / Fornitura}
\author{Matteo Rizzo}
\date{\today}

\begin{document}

\chapter{Glossario (norme di progetto / fornitura)}

\begin{itemize}
    \item{Processo}: Insieme di attività correlate che trasformano ingressi in uscite secondo regole fissate.
    \item{Responsabile di Progetto}: è la figura responsabile della gestione di un progetto, più precisamente dell'istanziazione di processi nel progetto, della stima dei costi e delle risorse necessarie, della pianificazione delle attività e della loro assegnazione, del controllo delle attività e verifica dei risultati.
    
    \item{Amministratore}: è la figura che ha il compito di controllare che ad ogni istante della vita del progetto le risorse (umane, materiali, economiche e strutturali) siano presenti e operanti; inoltre, gestisce la documentazione e controlla il versionamento e la configurazione.
    
    \item{Analista}: è la figura che, a partire dal bisogno del cliente, individua il problema (di cui conosce il dominio) da fornire al progettista.
    
    \item{Verificatore}:
    \figura designata alla verifica del lavoro prodotto dai programmatori.
     è la item{Verifica}: essa attiene alla coerenza, completezza e correttezza del prodotto. È un processo che si applica ad ogni "segmento" temporale di un progetto (ad ogni prodotto intermedio) per accertare che le attività svolte in tale segmento non abbiano introdotto errori nel prodotto.

    \item{Validazione}: essa accerta la conformità di un prodotto alle attese, fornisce una prova oggettiva di come le specifiche del prodotto siano conformi al suo scopo e alle esigenze degli utenti.
\end{itemize}

\end{document}