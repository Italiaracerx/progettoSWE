\documentclass[openany,12pt,a4paper]{report}
\usepackage{StileDoc}

\title{Glossario - Processi Organizzativi}
\author{Manfredi Smaniotto}

\begin{document}
	\chapter{Glossario - Processi Organizzativi}
	
	\begin{itemize}
		\item{\textbf{Qualità:}} insieme di caratteristiche di un entità che ne comportano la capacità di soddisfare richieste esplicite o implicite di un committente;
		\item{\textbf{Committente:}} organizzazione o persona che commissiona un lavoro. Essa detta tempi e costi con i quali il team può portare avanti il lavoro commissionato;
		\item{\textbf{Task:}} attività da svolgere entro una scadenza prefissata. Essa può essere suddivisa in più compiti o subtask anche essi soggetti a scadenza il cui insieme porta al completamento del task.
		\item{\textbf{LTS:}} acronimo di Long Term Service è una versione di Ubuntu rilasciata ogni due anni con contenuti e aggiornamenti molto più testati e affidabili delle normali versioni di Ubuntu. Garantisce inoltre gli aggiornamenti per un periodo di almeno 2 anni prima di obbligare l'utente al cambio di versione del sistema operativo.
		\item {\textbf{Hosting:}} servizio di rete che consiste nell'allocare su un server web le pagine di un sito o di un'applicazione web consentendo agli utenti di accedere ad esse tramite la rete internet.
	\end{itemize}
	contenuto...
\end{document}