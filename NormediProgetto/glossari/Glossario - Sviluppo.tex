\documentclass[openany,12pt,a4paper]{report}
\usepackage{StileDoc}

\title{Glossario - Norme di progetto / Sviluppo}
\author{Cristiano Tessarolo}
\date{\today}

\begin{document}

\chapter{Glossario (norme di progetto / Sviluppo)}

\begin{itemize}
	
	\item{Analista}: è la figura che, a partire dal bisogno del cliente, individua il problema (di cui conosce il dominio) da fornire al progettista.
	
	\item{Verificatore}: è la figura che si occupa a verificare il lavoro svolto dai programmatori.
	
	\item{Requisito}: vincolo da rispettare o bisogno da soddisfare.
	
	\item{Requisito di Vincolo}: è un requisito da rispettare.
	
	\item{Requisito Funzionale}: è un requsito che aggiunge funzionalità.
	
	\item{Requisito prestazionale}: è un requsito che porta ad un miglioramento di prestazione del software.
	
	\item{Requisito di qualità}: è un requisito che porta ad un miglioramento della qualità del software.
	
	\item{Caso d'uso}: insieme di scenari possibili che si possono verificare nel	utilizzo del software.
	    
 
\end{itemize}

\end{document}