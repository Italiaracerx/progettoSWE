\documentclass[../NormediProgetto.tex]{subfiles}


\begin{document}

% NORME DI PROGETTO - PROCESSI DI SUPPORTO

\chapter{Processi di supporto}

% NORME DI PROGETTO - PROCESSI DI SUPPORTO - DOCUMENTAZIONE

\section{Documentazione}

\subsection{Scopo} 

Lo scopo del processo di documentazione è descrivere l'insieme di regole adottate dal gruppo per la redazione della documentazione di progetto. Il gruppo si prefigge di definire in questo capitolo norme atte alla stesura di una documentazione il più possibile formale, corretta, coerente e coesa.

% STRUTTURA DEI DOCUMENTI

\subsection{Struttura dei documenti}

\subsubsection{Template}

Il gruppo utilizza un template \LaTeX{} appositamente codificato per uniformare l'aspetto dei documenti e velocizzare il processo di documentazione. Il template include frontespizio, registro delle modifiche, indice, piè di pagina e intestazione, elementi del documento approfonditi di seguito. 

\subsubsection{Frontespizio}

Il frontespizio di ogni documento è strutturato nel seguente modo:

\begin{itemize}
    \item \textbf{Nome del documento:} indica il nome del documento (ex: Norme di progetto);
    \item \textbf{Logo del gruppo:} il logo identificativo del gruppo Graphite;
    \item \textbf{Informazioni sul documento:} tabella riassuntiva indicante le seguenti informazioni di spicco sul documento:
        \begin{itemize}
            \item \textbf{Versione:} indica un numero che identifica la versione corrente del documento. La struttura del numero di versionamento è approfondita di seguito in questa sezione;
            \item \textbf{Data di redazione:} indica l'ultima data in cui il documento è stato modificato;
            \item \textbf{Redattori:} indica il sottoinsieme dei membri del gruppo che ha partecipato alla redazione del documento;
            \item \textbf{Verificatori:} indica il sottoinsieme dei membri del gruppo che ha partecipato alla verifica del documento;
            \item \textbf{Distribuzione:} indica a chi è destinato il documento;
            \item \textbf{Uso:} indica se l'uso del documento è interno o esterno al gruppo.
        \end{itemize}
        
\end{itemize}

Ogni documento è accompagnato da un numero di versionamento, indicato nella tabella \textit{Informazioni sul documento}, dove ogni versione corrisponde ad una riga nel registro delle modifiche, che è espresso nel modo seguente:

\[\textbf{v $\biggl\{$A$\biggr\}$.$\biggl\{$B$\biggr\}$.$\biggl\{$C$\biggr\}$}\]

Dove:

\begin{itemize}
    \item{\textbf{A:}} è l'indice principale. Viene incrementato dal \textit{Responsabile di Progetto} all’approvazione del documento e 
    corrisponde al numero di revisione.
    \item{\textbf{B:}} è l'indice di verifica. Viene incrementato dal \glossario{\textit{Verificatore}}{verificatore} ad ogni verifica. Quando viene incrementato A, riparte da 0.
    \item{\textbf{C:}} è l'indice di modifica. Viene incrementato dal redattore del documento ad ogni modifica. Quando viene incrementato B, riparte da 0.
\end{itemize}

\subsubsection{Registro delle modifiche}

Segue il frontespizio il registro delle modifiche, che cataloga la totalità delle modifiche apportate al documento durante il suo sviluppo indicando per ognuna:

\begin{itemize}
    \item Versione del documento dopo la modifica;
    \item Data della modifica;
    \item Nome e cognome dell'autore della modifica;
    \item Ruolo dell'autore della modifica;
    \item Breve descrizione della modifica.
\end{itemize}

\subsubsection{Indice}

Ogni documento possiede un indice che ne agevola la consultazione e permette una visione generale degli argomenti trattati nello stesso. L'indice è strutturato gerarchicamente ed è collocato dopo il registro delle modifiche.

\subsubsection{Intestazione}

Fatta eccezione per il frontespizio, tutte le pagine del documento contengono un’intestazione. Essa contiene: 

\begin{itemize}
    \item \textbf{Logo del gruppo:} il logo identificativo del gruppo Graphite, collocato a sinistra;
    \item \textbf{Titolo del capitolo:} indicazione del titolo del capitolo corrente, collocata a destra;
\end{itemize}

\subsubsection{Piè di pagina}

Fatta eccezione per il frontespizio, tutte le pagine del documento contengono un piè di pagina. Esso contiene: 

\begin{itemize}
    \item \textbf{Data e ora:} indica data e ora dell'ultima modifica del documento, collocate a sinistra;
    \item \textbf{Numero di pagina:} numerazione progressiva delle pagine, collocato a destra.
\end{itemize}

\subsubsection{Note a piè di pagina}

In caso di presenza in una pagina interna di note da esplicare, esse vanno indicate nella pagina corrente, in basso a sinistra. Ogni nota deve riportare un numero, corrispondente alla relativa parola o frase che la riferisce, e una descrizione.

\subsubsection{Contenuto principale}

Il contenuto principale del documento è organizzato secondo la seguente struttura gerarchica:

\begin{enumerate}
    \item Capitolo
    \item Sezione
    \item Sottosezione
    \item Sottosottosezione
\end{enumerate}

\subsubsection{Verbali delle riunioni}

I verbali delle riunioni sono redatti in base ad un template \LaTeX{} ad hoc che include sempre:

\begin{itemize}
    \item Data della riunione
    
    \item Membri presenti
    
    \item Ordine del giorno
    
    \item Esito sintetico della discussione
\end{itemize}

% NORME TIPOGRAFICHE

\subsection{Norme tipografiche}

\subsubsection{Stile del testo}

\begin{itemize}
    
    \item \textbf{Glossario:} ogni parola contenuta nel glossario deve essere marcata, solo alla sua prima occorrenza in ogni documento, in corsivo e con una G maiuscola a pedice. Tale formattazione viene assicurata dal comando \LaTeX{} \{ \}glossario{termine}{riferimento al glossario} che stampa il termine con la giusta formattazione (\textit{termine}\ped{G}) e controlla che il suo riferimento sia presente nel glossario, restituendo un errore di compilazione in caso contrario;  
    
    \item \textbf{Grassetto:} il grassetto viene applicato ai titoli e agli elementi di un elenco puntato seguiti da una descrizione. Esso può inoltre essere usato per mettere in particolare risalto termini significativi; 
    
    \item \textbf{Corsivo:} il corsivo viene utilizzato per:
    \begin{itemize}
        \item citazioni;
        \item termini di glossario;
        \item attività del progetto;
        \item ruoli del progetto;
        \item riferimenti ad altri documenti.
    \end{itemize}
    
     Esso può inoltre essere usato per mettere in particolare risalto termini significativi;
    
    \item{\textbf{Maiuscolo:}} il maiuscolo viene usato solo per gli acronimi.

\end{itemize}

\subsubsection{Elenchi puntati}

Gli elenchi puntati servono ad esprimere in modo sintetico un concetto, evitando frasi lunghe e dispersive. Ogni voce di un elenco puntato termina con un punto e virgola, ad eccezione dell’ultima, che termina invece con un punto.

\subsubsection{Formati}

\begin{itemize}

\item{\textbf{Date:}}  \[\textbf{GG-MM-AAAA}\]
\begin{itemize}
\item{\textbf{GG:}} rappresenta il giorno del mese in cifre;
\item{\textbf{MM:}} rappresenta il mese in cifre;
\item{\textbf{AAAA:}} rappresenta l'anno in cifre per intero.

\end{itemize}

\item{\textbf{Orari:}} \[\textbf{HH:MM}\]
\begin{itemize}
\item{\textbf{HH:}} rappresenta l'ora;
\item{\textbf{MM:}} rappresenta i minuti.
\end{itemize}

\item{\textbf{Nomi ricorrenti:}}
\begin{itemize}
\item{\textbf{Ruoli di progetto:}} ogni nome di ruolo di progetto viene scritto in corsivo e con l’iniziale maiuscola;
\item{\textbf{Nomi dei documenti:}} ogni nome di documento viene scritto in corsivo e con l’iniziale di ogni parola che non sia un articolo maiuscola;
\item{\textbf{Nomi propri:}} ogni nome proprio di persona deve essere scritto nella forma \textit{Nome Cognome}.
\end{itemize}

\item{\textbf{Link:}} i link dovranno essere scritti attraverso il comando \LaTeX{} \textit{href} e sono distinti dal colore blu.
\end{itemize}

\subsubsection{Sigle}

È previsto l’utilizzo delle seguenti sigle: 

\begin{itemize}
    \item{\textbf{AR:}} \textit{Analisi dei Requisiti};
    \item{\textbf{PP:}} \textit{Piano di Progetto};
    \item{\textbf{NP:}} \textit{Norme di Progetto};
    \item{\textbf{SF:}} \textit{Studio di Fattibilità};
    \item{\textbf{PQ:}} \textit{Piano di Qualifica};
    \item{\textbf{ST:}} \textit{Specifica Tecnica};
    \item{\textbf{MU:}} \glossario{Manuale utente}{manuale utente};
    \item{\textbf{DP:}} \textit{Definizione di Prodotto};
    \item{\textbf{RR:}} Revisione dei requisiti;
    \item{\textbf{RP:}} Revisione di progettazione;
    \item{\textbf{RQ:}} Revisione di qualifica;
    \item{\textbf{RA:}} Revisione di accettazione;
    \item{\textbf{Re:}} \textit{Responsabile di Progetto};
    \item{\textbf{Am:}} \textit{Amministratore di Progetto};
    \item{\textbf{An:}} \textit{Analista};
    \item{\textbf{Pt:}} \textit{Progettista};
    \item{\textbf{Pr:}} \glossario{Programmatore}{programmatore};
    \item{\textbf{Ve:}} \textit{Verificatore}.
\end{itemize}

\subsection{Elementi grafici}

\subsubsection{Tabelle}

Ogni tabella è corredata da una didascalia in cui compare il suo numero identificativo (per agevolarne il tracciamento) ed una breve descrizione del suo contenuto.

\subsubsection{Immagini}

Ogni immagine deve essere centrata e separata dai paragrafi a lei precedenti e successivi. Le immagini sono corredate da una didascalia analoga a quella delle tabelle. Tutti i diagrammi UML vengono inseriti come immagini.

\subsection{Classificazione dei documenti}

Ogni documento formale è classificato come documento Interno od Esterno, con le seguenti differenze:

\begin{itemize}
    \item \textbf{Interno:} il documento ha utilizzo interno al gruppo Graphite;
    \item \textbf{Esterno:} il documento viene condiviso con i Committenti e con la Proponente.
\end{itemize}

\subsection{Nomenclatura dei documenti}

I documenti formali, fatta eccezione per i Verbali di riunione, adottano il seguente sistema di nomenclatura:

\centerline{NomeDelDocumento vX.Y.Z}

Dove:

\begin{itemize}
    \item \textbf{NomedelDocumento:} indica il nome del documento, senza spazi con lettera maiuscola per ogni parola che non sia un articolo o una preposizione;
    
    \item \textbf{vX.Y.Z:}  indica l'ultima versione del documento approvata dal \textit{Responsabile di Progetto}.
\end{itemize}

\subsection{Formato dei file}
I file legati alla documentazione vengono salvati in formato .tex durante il loro ciclo di vita. Quando un documento raggiunge lo stato di "Approvato" viene creato un file in formato PDF contenente la versione del documento approvata dal \textit{Responsabile}.

\end{document}
