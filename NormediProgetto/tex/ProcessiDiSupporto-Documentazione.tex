\documentclass[../NormediProgetto.tex]{subfiles}


\begin{document}

% NORME DI PROGETTO - PROCESSI DI SUPPORTO

\chapter{Processi di supporto}

% NORME DI PROGETTO - PROCESSI DI SUPPORTO - GESTIONE DELLA CONFIGURAZIONE

\section{Gestione della configurazione}

\subsection{Versionamento} 
Ogni componente versionabile del progetto, nonché la documentazione formale, è versionata mediante la tecnologia \glossario{Git}{Git}, nello specifico utilizzando il servizio gratuito GitHub.
Per la condivisione di materiale informale e/o non versionabile si predilige invece una cartella condivisa sullo spazio cloud offerto da \glossario{Google Drive}{Google Drive}.

\subsection{Controllo della configurazione}

Per identificare e tracciare le richieste di cambiamenti, analizzare e valutare le modifiche effettuate e approvare o meno quelle proposte, viene utilizzato il sistema di \glossario{issue}{issue} di Github. Le issue riportano nella loro descrizione la ragione della modifica effettuata, nonché la o le \glossario{commit}{commit} ad esse associate ed eventuali ulteriori issue correlate. 

\subsection{Stato della configurazione}

Successivamente ad ogni revisione, l'ultima commit effettuata viene marcata con un'etichetta di versione (essa costituirà la nuova \glossario{baseline}{baseline} di riferimento). L'esito delle revisioni vengono riportate in appendice al PQ con le relative correzioni da apportare ai prodotti oggetto della revisione.

\subsection{Rilasci e consegne}

I rilasci e le consegne dei prodotti software e della documentazione correlata sono sottoposti a verifica e validazione. Prima di ogni revisione viene consegnato al \textit{Committente}, secondo tempistiche stabilite dal \textit{Piano di Progetto v 1.0.0} e mezzi concordati, un archivio contenente tutta la documentazione richiesta in formato \glossario{PDF}{PDF}. La consegna è accompagnata da una Lettera di
Presentazione, anch'essa inclusa nell'archivio.

% NORME DI PROGETTO - PROCESSI DI SUPPORTO - GESTIONE DELLA QUALITà

\section{Gestione della qualità}

\subsection{Classificazione degli obiettivi}

Gli obiettivi di qualità stabiliti nel PQ rispettano la seguente notazione:

    \begin{center}
        OQ[Tipo][Oggetto]*[ID]: [Nome]
    \end{center}
    
Dove:

\begin{itemize}
    \item \textbf{Tipo:} indica se l'obiettivo si riferisce a prodotti o a processi. Può assumere i valori:
    
    \begin{itemize}
        \item \textbf{P:} per indicare i processi;
        \item \textbf{PP:} per indicare i prodotti;
    \end{itemize}
    
    \item \textbf{Oggetto:} indica, per gli obiettivi di prodotto, se si riferisce a documentazione o a software. Può assumere i valori:
    
    \begin{itemize}
        \item \textbf{D:} per indicare i documenti;
        \item \textbf{S:} per indicare il software;
    \end{itemize}
    
    \item \textbf{ID:} identifica univocamente l'obiettivo tramite un codice numerico incrementale;
    
    \item \textbf{Nome:} titolo dell'obiettivo;
    
    \item \textbf{Descrizione:} breve descrizione dell'obiettivo.
\end{itemize}

\subsection{Classificazione delle metriche}

Le metriche stabilite nel PQ rispettano la seguente notazione:

    \begin{center}
        M[Tipo][Oggetto]*[ID]: [Titolo]
    \end{center}
    
Dove:

\begin{itemize}
    \item \textbf{Tipo:} indica se la metrica si riferisce a prodotti o a processi. Può assumere i valori:
    
    \begin{itemize}
        \item \textbf{P:} per indicare i processi;
        \item \textbf{PP:} per indicare i prodotti;
    \end{itemize}
    
    \item \textbf{Oggetto:} indica, per le metriche di prodotto, se si riferisce a documentazione o a software. Può assumere i valori:
    
    \begin{itemize}
        \item \textbf{D:} per indicare i documenti;
        \item \textbf{S:} per indicare il software;
    \end{itemize}
    
    \item \textbf{ID:} identifica univocamente la metrica tramite un codice numerico incrementale;
    
    \item \textbf{Nome:} titolo della metrica.
\end{itemize}

% NORME DI PROGETTO - PROCESSI DI SUPPORTO - DOCUMENTAZIONE

\section{Documentazione}

\subsection{Scopo} 

Lo scopo del processo di documentazione è descrivere l'insieme di regole adottate dal gruppo per la redazione della documentazione di progetto. Il gruppo si prefigge di definire in questo capitolo norme atte alla stesura di una documentazione il più possibile formale, corretta, coerente e coesa.

% STRUTTURA DEI DOCUMENTI

\subsection{Struttura dei documenti}

\subsubsection{Template}

Il gruppo utilizza un template \LaTeX{} appositamente codificato per uniformare l'aspetto dei documenti e velocizzare il processo di documentazione. Il template include frontespizio, registro delle modifiche, indice, piè di pagina e intestazione, elementi del documento approfonditi di seguito. 

\subsubsection{Frontespizio}

Il frontespizio di ogni documento è strutturato nel seguente modo:

\begin{itemize}
    \item \textbf{Logo del gruppo:} il logo identificativo del gruppo;
    \item \textbf{Nome del documento:} indica il nome del documento (ex: Norme di progetto);
    \item \textbf{Informazioni sul documento:} tabella riassuntiva indicante le seguenti informazioni di spicco sul documento:
        \begin{itemize}
            \item \textbf{Versione:} indica un numero che identifica la versione corrente del documento. La struttura del numero di versionamento è approfondita di seguito in questa sezione;
            \item \textbf{Data di redazione:} indica l'ultima data in cui il documento è stato modificato;
            \item \textbf{Redattori:} indica il sottoinsieme dei membri del gruppo che ha partecipato alla redazione del documento;
            \item \textbf{Verificatori:} indica il sottoinsieme dei membri del gruppo che ha partecipato alla verifica del documento;
            \item \textbf{Distribuzione:} indica a chi è destinato il documento;
            \item \textbf{Uso:} indica se l'uso del documento è interno o esterno al gruppo.
        \end{itemize}
        
\end{itemize}

\subsection{Nomenclatura di versionamento dei documenti}
Ogni documento è accompagnato da un numero di versionamento, indicato nella tabella \textit{Informazioni sul documento}, dove ogni versione corrisponde ad una riga nel registro delle modifiche, che è espresso nel modo seguente:

\[\textbf{v $\biggl\{$A$\biggr\}$.$\biggl\{$B$\biggr\}$.$\biggl\{$C$\biggr\}$}\]

Dove:

\begin{itemize}
    \item{\textbf{A:}} è l'indice principale. Viene incrementato dal \textit{Responsabile di Progetto} all’approvazione del documento e 
    corrisponde al numero di revisione.
    \item{\textbf{B:}} è l'indice di verifica. Viene incrementato dal \glossario{\textit{Verificatore}}{verificatore} ad ogni verifica. Quando viene incrementato A, riparte da 0.
    \item{\textbf{C:}} è l'indice di modifica. Viene incrementato dal redattore del documento ad ogni modifica. Quando viene incrementato B, riparte da 0.
\end{itemize}

\subsubsection{Registro delle modifiche}

Dopo il frontespizio segue il registro delle modifiche, che cataloga le modifiche apportate al documento durante il suo sviluppo indicando per ognuna:

\begin{itemize}
    \item Versione del documento dopo la modifica;
    \item Data della modifica;
    \item Nome e cognome dell'autore della modifica;
    %\item Ruolo dell'autore della modifica;
    \item Breve descrizione della modifica.
\end{itemize}

\subsubsection{Indice}

Ogni documento possiede un indice che ne agevola la consultazione e permette una visione generale degli argomenti trattati nello stesso. L'indice è strutturato gerarchicamente ed è collocato dopo il registro delle modifiche.

\subsubsection{Intestazione}

Fatta eccezione per il frontespizio, tutte le pagine del documento contengono un’intestazione. Questa è costituita da: 

\begin{itemize}
    \item \textbf{Logo del gruppo:} il logo identificativo del gruppo Graphite, collocato a sinistra;
    \item \textbf{Titolo del capitolo:} indicazione del titolo del capitolo corrente, collocata a destra.
\end{itemize}

\subsubsection{Piè di pagina}

Fatta eccezione per il frontespizio, tutte le pagine del documento contengono un piè di pagina. Questo è costituito da: 

\begin{itemize}
    \item \textbf{Data e ora:} indica data e ora dell'ultima modifica del documento, collocate a sinistra;
    \item \textbf{Numero di pagina:} numerazione progressiva delle pagine, collocato a destra.
\end{itemize}

\subsubsection{Note a piè di pagina}

In caso di presenza in una pagina interna di note da esplicare, esse vanno indicate nella pagina corrente, in basso a sinistra. Ogni nota deve riportare un numero, corrispondente alla relativa parola o frase che la riferisce, e una descrizione.

\subsubsection{Contenuto principale}

Il contenuto principale del documento è organizzato secondo la seguente struttura gerarchica:

\begin{enumerate}
    \item Capitolo;
    \item Sezione;
    \item Sottosezione;
    \item Sottosottosezione.
\end{enumerate}

% NORME TIPOGRAFICHE

\subsection{Norme tipografiche}

\subsubsection{Stile del testo}

\begin{itemize}
    
    \item \textbf{Glossario:} ogni parola contenuta nel glossario deve essere marcata, solo alla sua prima occorrenza in ogni documento, in corsivo e con una G maiuscola a pedice. Tale formattazione viene assicurata dalla macro \textit{\textbackslash glossario\{termine\}\{riferimento al glossario\}} che stampa il termine con la giusta formattazione (\textit{termine}\ped{G}) e controlla che il suo riferimento sia presente nel glossario, restituendo un errore di compilazione in caso contrario;  
    
    \item \textbf{Grassetto:} il grassetto viene applicato ai titoli e agli elementi di un elenco puntato seguiti da una descrizione. Esso può inoltre essere usato per mettere in particolare risalto termini significativi; 
    
    \item \textbf{Corsivo:} il corsivo viene utilizzato per:
    \begin{itemize}
        \item citazioni;
        \item termini di glossario;
        \item attività del progetto;
        \item ruoli del progetto;
        \item riferimenti ad altri documenti.
    \end{itemize}
    
     Esso può inoltre essere usato per mettere in particolare risalto termini particolari solitamente poco usati o conosciuti;
    
    \item{\textbf{Maiuscolo:}} il maiuscolo viene usato solo per gli acronimi.

\end{itemize}

\subsubsection{Elenchi puntati}

Gli elenchi puntati servono ad esprimere in modo sintetico un concetto, evitando frasi lunghe e dispersive. Ogni voce di un elenco puntato termina con un punto e virgola, ad eccezione dell'ultima, che termina invece con un punto.

\subsubsection{Formati}

\begin{itemize}

\item{\textbf{Date:}}  \[\textbf{GG-MM-AAAA}\]
\begin{itemize}
\item{\textbf{GG:}} rappresenta il giorno del mese in cifre;
\item{\textbf{MM:}} rappresenta il mese in cifre;
\item{\textbf{AAAA:}} rappresenta l'anno in cifre per intero.

\end{itemize}

\item{\textbf{Orari:}} \[\textbf{HH:MM}\]
\begin{itemize}
\item{\textbf{HH:}} rappresenta l'ora;
\item{\textbf{MM:}} rappresenta i minuti.
\end{itemize}

\item{\textbf{Nomi ricorrenti:}}
\begin{itemize}
\item{\textbf{Ruoli di progetto:}} ogni nome di ruolo di progetto viene scritto in corsivo e con l’iniziale maiuscola;
\item{\textbf{Nomi dei documenti:}} ogni nome di documento viene scritto in corsivo e con l’iniziale di ogni parola che non sia un articolo maiuscola;
\item{\textbf{Nomi propri:}} ogni nome proprio di persona deve essere scritto nella forma \textit{Nome Cognome}.
\end{itemize}

\item{\textbf{Link:}} i link esterni devono essere scritti attraverso il comando \LaTeX{} \textit{url} e sono distinti dal colore blu. I link interni, per esempio quelli dell'indice, non vengono evidenziati.
\end{itemize}

\subsubsection{Sigle}

È previsto l’utilizzo delle seguenti sigle: 

\begin{itemize}
    \item{\textbf{AR:}} \textit{Analisi dei Requisiti};
    \item{\textbf{PP:}} \textit{Piano di Progetto};
    \item{\textbf{NP:}} \textit{Norme di Progetto};
    \item{\textbf{SF:}} \textit{Studio di Fattibilità};
    \item{\textbf{PQ:}} \textit{Piano di Qualifica};
    \item{\textbf{ST:}} \textit{Specifica Tecnica};
    \item{\textbf{MU:}} \glossario{Manuale utente}{manuale utente};
    \item{\textbf{DP:}} \textit{Definizione di Prodotto};
    \item{\textbf{RR:}} Revisione dei requisiti;
    \item{\textbf{RP:}} Revisione di progettazione;
    \item{\textbf{RQ:}} Revisione di qualifica;
    \item{\textbf{RA:}} Revisione di accettazione;
    \item{\textbf{Re:}} \textit{Responsabile di Progetto};
    \item{\textbf{Am:}} \textit{Amministratore di Progetto};
    \item{\textbf{An:}} \textit{Analista};
    \item{\textbf{Pt:}} \textit{Progettista};
    \item{\textbf{Pr:}} \glossario{Programmatore}{programmatore};
    \item{\textbf{Ve:}} \textit{Verificatore}.
\end{itemize}

\subsection{Elementi grafici}

\subsubsection{Tabelle}

Ogni tabella è corredata da una didascalia in cui compare il suo numero identificativo (per agevolarne il tracciamento) ed una breve descrizione del suo contenuto.

\subsubsection{Immagini}

Ogni immagine deve essere centrata e separata dai paragrafi a lei precedenti e successivi. Le immagini sono corredate da una didascalia analoga a quella delle tabelle. Tutti i diagrammi vengono inseriti come immagini.

\subsection{Classificazione dei documenti}

\subsubsection{Documenti informali}

Tutte le versioni dei documenti che non siano state approvate dal \textit{Responsabile di Progetto} sono ritenute informali e, in quanto tali, sono considerate esclusivamente ad uso interno.

\subsubsection{Documenti formali}

Una versione di un documento viene considerata formale quando è stata approvata dal \textit{Responsabile di Progetto}. Solo i documenti formali possono essere distribuiti all’esterno del gruppo.

\subsubsection{Documenti interni}

Un documento viene considerato interno quando il suo utilizzo è destinato al solo gruppo Graphite.

\subsubsection{Documenti esterni}

Un documento viene considerato esterno quando il documento viene condiviso con i Committenti e con la Proponente.

\subsubsection{Verbali}

Per ogni incontro interno o esterno è prevista la redazione di un verbale, contenente le seguenti informazioni:
\begin{itemize}
    \item \textbf{Informazioni sull'incontro:}
    \begin{itemize}
        \item \textbf{Luogo}: indica il luogo in cui si svolge la riunione. In caso di videochiamate, questo campo può essere riempito con la voce \textit{"Videochiamata"} e/o con il punto di ritrovo in cui si è svolta (ex: Aula AC150, Torre Archimede (Videochiamata));
        \item \textbf{Data:} indica la data in cui si è svolto l'incontro;
        \item \textbf{Orari:} indica orari di inizio e di fine dell'incontro;
        \item \textbf{Assenti:} indica, se dovessero essercene, i membri del gruppo assenti all'incontro;
        \item \textbf{Partecipanti esterni:} indica, se dovessero essercene, eventuali partecipanti esterni presenzianti all'incontro.
    \end{itemize}
    
    \item \textbf{Ragioni dell'incontro:} indica le ragioni che hanno spinto ad indire l'incontro in esame. In questa sezione viene inserito l'eventuale ordine del giorno;
    
    \item \textbf{Resoconto:} contiene annotazioni riguardo gli argomenti discussi e capisaldi dell'incontro;
    
    \item \textbf{Tracciamento delle decisioni:} indica eventuali decisioni prese durante l'incontro, tracciate mediante il seguente codice:
    
        \[V[Tipologia]-[Data incontro].[ID]\]
    
    Dove:
    
    \begin{itemize}
        \item \textbf{Tipologia:} indica se il verbale è interno oppure esterno;
        \item \textbf{Data incontro:} indica la data in cui si è svolto l'incontro in formato \textit{GGMMAAAA};
        \item \textbf{ID:} è un codice identificativo numerico incrementale.
    \end{itemize}
    
\end{itemize}

I verbali esterni dovranno avere la seguente nomenclatura:

\[VE-DATA\]

Mentre quelli interni la seguente:

\[VI-DATA\]

I verbali dovranno essere approvati dal \textit{Responsabile di Progetto} al termine di ogni incontro.

\subsection{Ciclo di vita del documento}

Ogni documento non formale completato deve essere sottoposto al \textit{Responsabile di Progetto}, che si occupa di incaricare i \textit{Verificatori} di controllarne la correttezza del contenuto e della forma. Se vengono individuati degli errori, i \textit{Verificatori} li riportano al \textit{Responsabile di Progetto}, che a sua volta incarica il redattore del documento di correggerli. Questo ciclo va ripetuto fino a che il documento non è considerato corretto. Successivamente esso viene sottoposto al Responsabile di Progetto, che può o meno approvarlo. Quando approvato, il documento è da considerarsi formale. In caso contrario il \textit{Responsabile di Progetto} deve comunicare le motivazioni per cui il documento non è stato approvato, specificando le modifiche da apportare.

\subsection{Nomenclatura dei documenti}

I documenti formali, fatta eccezione per i Verbali di riunione, adottano il seguente sistema di nomenclatura:

\[NomeDelDocumento vX.Y.Z\]

Dove:

\begin{itemize}
    \item \textbf{NomedelDocumento:} indica il nome del documento, senza spazi con lettera maiuscola per ogni parola che non sia un articolo o una preposizione;
    
    \item \textbf{vX.Y.Z:}  indica l'ultima versione del documento approvata dal \textit{Responsabile di Progetto}.
\end{itemize}

\subsection{Formato dei file}
I file legati alla documentazione vengono salvati in formato .tex durante il loro ciclo di vita. Quando un documento raggiunge lo stato di "Approvato" viene creato un file in formato PDF contenente la versione del documento approvata dal \textit{Responsabile}.

\end{document}
