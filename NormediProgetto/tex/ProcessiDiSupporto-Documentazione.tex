\documentclass[../NormediProgetto.tex]{subfiles}


\begin{document}

\chapter{Processi di supporto}

\section{Documentazione}

\subsection{Scopo} 

Lo scopo del processo di documentazione è descrivere l'insieme di regole adottate dal gruppo per la redazione della documentazione di progetto. 
	
\subsection{Descrizione}

Di seguito vengono descritte le norme adottate per la stesura di tutti i documenti riguardanti il progetto.

\subsection{Template}

Il gruppo utilizza un template \LaTeX{} appositamente codificato per uniformare l'aspetto dei documenti e velocizzare il processo di documentazione. Il template include frontespizio, registro delle mofiche, indice, piè di pagina e intestazione, elementi del documento approfonditi di seguito. 

\subsection{Struttura dei documenti}

\subsubsection{Frontespizio}

La prima pagina di ogni documento è strutturata nel seguente modo:

\begin{itemize}
    \item \textbf{Nome del documento:} indica il nome del documento (ex: Norme di progetto);
    \item \textbf{Logo del gruppo:} il logo identificativo del gruppo Graphite;
    \item \textbf{Informazioni sul documento:} tabella riassuntiva indicante le seguenti informazioni di spicco sul documento:
        \begin{itemize}
            \item \textbf{Versione:} indica un numero che identifica la versione corrente del documento. La struttura del numero di versionamento è approfondita di seguito in questa sezione;
            \item \textbf{Data di redazione:} indica l'ultima data in cui il documento è stato modificato;
            \item \textbf{Redattori:} indica il sottoinsieme dei membri del gruppo che ha partecipato alla redazione del documento;
            \item \textbf{Verificatori:} indica il sottoinsieme dei membri del gruppo che ha partecipato alla verifica del documento;
            \item \textbf{Distribuzione:} indica a chi è destinato il documento;
            \item \textbf{Uso:} indica se l'uso del documento è interno o esterno al gruppo.
        \end{itemize}
        
\end{itemize}

Ogni documento è accompagnato da un numero di versionamento, indicato nella tabella \textit{Informazioni sul documento}, dove ogni versione corrisponde ad una riga nel registro delle modifiche, che è espresso nel modo seguente:

\[\textbf{v $\biggl\{$A$\biggr\}$.$\biggl\{$B$\biggr\}$.$\biggl\{$C$\biggr\}$}\]

Dove:

\begin{itemize}
    \item{\textbf{A:}} è l'indice principale. Viene incrementato dal \textit{Responsabile di Progetto} all’approvazione del documento e 
    corrisponde al numero di revisione.
    \item{\textbf{B:}} è l'indice di verifica. Viene incrementato dal \glossario{\textit{Verificatore}}{verificatore} ad ogni verifica. Quando viene incrementato A, riparte da 0.
    \item{\textbf{C:}} è l'indice di modifica. Viene incrementato dal redattore del documento ad ogni modifica. Quando viene incrementato B, riparte da 0.
\end{itemize}

\subsubsection{Registro delle modifiche}

Segue il frontespizio il registro delle modifiche, che cataloga la totalità delle modifiche apportate al documento durante il suo sviluppo indicando per ognuna:

\begin{itemize}
    \item Versione del documento dopo la modifica;
    \item Data della modifica;
    \item Nome e cognome dell'autore della modifica;
    \item Ruolo dell'autore della modifica;
    \item Breve descrizione della modifica.
\end{itemize}

\subsubsection{Indice}

Ogni documento possiede un indice che ne agevola la consultazione e permette una visione generale degli argomenti trattati nello stesso. L'indice è strutturato gerarchicamente ed è collocato dopo il registro delle modifiche.

\subsubsection{Intestazione}

Fatta eccezione per il frontespizio, tutte le pagine del documento contengono un’intestazione. Essa contiene: 

\begin{itemize}
    \item \textbf{Logo del gruppo:} il logo identificativo del gruppo Graphite, collocato a sinistra;
    \item \textbf{Titolo del capitolo:} indicazione del titolo del capitolo corrente, collocata a destra;
\end{itemize}

\subsubsection{Piè di pagina}

Fatta eccezione per il frontespizio, tutte le pagine del documento contengono un piè di pagina. Esso contiene: 

\begin{itemize}
    \item \textbf{Data e ora:} indica data e ora dell'ultima modifica del documento, collocate a sinistra;
    \item \textbf{Numero di pagina:} numerazione progressiva delle pagine, collocato a destra.
\end{itemize}

\subsubsection{Note a piè di pagina}

In caso di presenza in una pagina interna di note da esplicare, esse vanno indicate nella pagina corrente, in basso a sinistra. Ogni nota deve riportare un numero, corrispondente alla relativa parola o frase che la riferisce, e una descrizione.

\subsubsection{Contenuto principale}

Il contenuto principale del documento è organizzato segue la seguente struttura gerarchica:

\begin{enumerate}
    \item Capitolo
    \item Sezione
    \item Sottosezione
    \item Sottosottosezione
\end{enumerate}

\subsection{Norme tipografiche}

\subsubsection{Stile del testo}

\begin{itemize}

\item{\textbf{Glossario:}} ogni parola contenuta nel glossario deve essere marcata, alla sua prima occorrenza in ogni documento, in corsivo e con una G maiuscola a pedice, questo verrà automaticamente fatto tramite la macro \{ \}glossario{termine}{riferimento al glossario} che stamperà il termine con la giusta formattazione (\textit{termine}\ped{G}) e controllerà che il suo riferimento sia presente nel glossario, dando un errore in compilazione se è assente;  

\item{\textbf{Grassetto:}} viene applicato ai titoli e agli elementi di un elenco puntato seguiti da una descrizione, può essere usato anche per mettere in risalto parole significative; 

\item{\textbf{Corsivo:}} Il corsivo dev’essere utilizzato per:
\begin{itemize}
\item citazioni;
\item parole inserite nel glossario;
\item attività del progetto;
\item ruoli del progetto;
\item riferimenti ad altri documenti;
\item parole particolari solitamente poco usate o conosciute.
\end{itemize}

\item{\textbf{Maiuscolo:}} deve essere usato solo per gli acronimi.

\end{itemize}

\subsubsection{Elenchi puntati}

 Gli elenchi puntati servono ad esprimere in modo sintetico un concetto, evitando frasi lunghe e discorsive. Ogni voce di un elenco puntato deve terminare con un punto e virgola, ad eccezione dell’ultima, che va terminata con un punto.

\subsubsection{Formati}

\begin{itemize}

\item{\textbf{Date:}}  \[\textbf{GG-MM-AAAA}\]
\begin{itemize}
\item{\textbf{GG:}} rappresenta il giorno del mese in cifre;
\item{\textbf{MM:}} rappresenta il mese in cifre;
\item{\textbf{AAAA:}} rappresenta l'anno in cifre per intero.

\end{itemize}

\item{\textbf{Orari:}} \[\textbf{HH:MM}\]
\begin{itemize}
\item{\textbf{HH:}} rappresenta l'ora;
\item{\textbf{MM:}} rappresenta i minuti.
\end{itemize}

\item{\textbf{Nomi ricorrenti:}}
\begin{itemize}
\item{\textbf{Ruoli di progetto:}} ogni nome di ruolo di progetto viene scritto in corsivo e con l’iniziale maiuscola;
\item{\textbf{Nomi dei documenti:}}  ogni nome di documento viene scritto in corsivo e con l’iniziale di ogni parola che non sia un articolo maiuscola;
\item{\textbf{Nomi propri:}} ogni nome proprio di persona deve essere scritto nella forma \textit{Nome Cognome}.
\end{itemize}

\item{\textbf{Link:}} i link dovranno essere scritti attraverso il comando \LaTeX{} \textit{href}.
\end{itemize}

\subsubsection{Sigle}

È previsto l’utilizzo delle seguenti sigle: 

\begin{itemize}
\item{\textbf{AR:}} \textit{Analisi dei Requisiti};
\item{\textbf{PP:}} \textit{Piano di Progetto};
\item{\textbf{NP:}} \textit{Norme di Progetto};
\item{\textbf{SF:}} \textit{Studio di Fattibilità};
\item{\textbf{PQ:}} \textit{Piano di Qualifica};
\item{\textbf{ST:}} \textit{Specifica Tecnica};
\item{\textbf{MU:}} \glossario{Manuale utente}{manuale utente};
\item{\textbf{DP:}} \textit{Definizione di Prodotto};
\item{\textbf{RR:}} Revisione dei requisiti;
\item{\textbf{RP:}} Revisione di progettazione;
\item{\textbf{RQ:}} Revisione di qualifica;
\item{\textbf{RA:}} Revisione di accettazione;
\item{\textbf{Re:}} \textit{Responsabile di Progetto};
\item{\textbf{Am:}} \textit{Amministratore di Progetto};
\item{\textbf{An:}} \textit{Analista};
\item{\textbf{Pt:}} \textit{Progettista};
\item{\textbf{Pr:}} \glossario{Programmatore}{programmatore};
\item{\textbf{Ve:}} \textit{Verificatore}.
\end{itemize}

\subsection{Elementi grafici}

\subsubsection{Tabelle}

Ogni tabella deve possedere una didascalia in cui deve comparire il numero identificativo, per agevolarne il tracciamento, ed una breve descrizione del suo contenuto.

\subsubsection{Immagini}

Ogni immagine deve essere centrata e separata dai paragrafi prima e dopo di essa. Le immagini devono avere una didascalia analoga a quella delle tabelle. Tutti i diagrammi UML vengono inseriti come immagini.

\subsection{Classificazione dei documenti}

\subsubsection{Documenti informali}

Tutte le versioni dei documenti che non siano state approvate dal \textit{Responsabile di Progetto} sono ritenute informali e, in quanto tali, sono considerate esclusivamente ad uso interno.

\subsubsection{Documenti formali}

Una versione di un documento viene considerata formale quando è stata approvata dal \textit{Responsabile di Progetto}. Solo i documenti formali possono essere distribuiti all’esterno del gruppo. 

\subsection{Procedura di approvazione}

Ogni documento non formale completato dovrà essere sottoposto al \textit{Responsabile di Progetto}, che a sua volta si occuperà di incaricare i \textit{Verificatori} di controllarne la correttezza del contenuto e della forma. Se vengono individuati degli errori, i \textit{Verificatori} li riporteranno al \textit{Responsabile di Progetto}, che a sua volta incaricherà il redattore del documento di correggerli. Questo ciclo va ripetuto fino a che il documento non è considerato corretto. Successivamente sarà sottoposto al Responsabile di Progetto, che potrà approvarlo o meno. Quando approvato, il documento verrà considerato un documento formale. In caso contrario il \textit{Responsabile di Progetto} dovrà comunicare le motivazioni per cui il documento non è stato approvato, specificando le modifiche da apportare.

\end{document}
