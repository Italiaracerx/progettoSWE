\documentclass[../NormediProgetto.tex]{subfiles}

\begin{document}
	
% NORME DI PROGETTO -> PROCESSI PRIMARI
	
\chapter{Processi primari}

% NORME DI PROGETTO -> PROCESSI PRIMARI -> FORNITURA

\section{Fornitura}

% FORNITURA - introduzione

\subsection{Scopo}

Il \glossario{processo}{processo} di fornitura ha lo scopo di trattare le norme e i termini che i membri del gruppo Graphite sono tenuti a rispettare per diventare fornitori della \glossario{proponente}{proponente} MIVOQ S.R.L. e dei \glossario{committenti}{committente} Prof. Tullio Vardanega e Prof. Riccardo Cardin per quanto concerne il prodotto "DeSpeect: interfaccia grafica per Speect".

\subsection{Rapporti di fornitura}

Durante l'intero progetto si intende instaurare con la proponente MIVOQ S.R.L., nella persona del referente Giulio Paci, un profondo e quanto più possibile costante rapporto di collaborazione orientato a:

\begin{itemize}
	\item Determinare aspetti chiave per soddisfare i bisogni del proponente;
	
	\item Stabilire scelte volte alla definizione e realizzazione del prodotto (vincoli sui requisiti);
	
	\item Stabilire scelte volte alla definizione ed esecuzione di processi (vincoli di progetto);
	
	\item Stimare i costi;
	
	\item Concordare la qualifica del prodotto.
\end{itemize}

% FORNITURA - Attività della fornitura

\subsection{Attività della fornitura}

Il processo di fornitura consiste delle seguenti attività:

\begin{itemize}
	\item Analisi dei capitolati;
	
	\item Definizione modello di sviluppo; 
	
	\item Pianificazione della Qualità.
\end{itemize}

\noindent Le suddette attività vengono di seguito esposte secondo scopo e descrizione.

% FORNITURA - Analisi dei Capitolati

\subsection{Analisi dei Capitolati}
\subsubsection{Scopo}

L'attività consiste di un'analisi dettagliata di ogni capitolato proposto, con il fine di evidenziare le ragioni che hanno portato il gruppo Graphite a optare per quello scelto.

\subsubsection{Descrizione} 
In seguito alla presentazione ufficiale dei Capitolati d'appalto, è compito del \glossario{Responsabile di Progetto}{responsabile di progetto} convocare una riunione interna al gruppo per valutare le proposte di progetto pervenute. Gli \glossario{Analisti}{analista} conducono quindi un'approfondita attività di analisi dei rischi e delle opportunità su ciascun capitolato.
Lo studio effettuato deve evidenziare le motivazioni che hanno portato il gruppo Graphite a proporsi come fornitore per il prodotto indicato, nonché analizzare ogni capitolato proposto valutandone:
 
\begin{itemize}
	\item \textbf{Dominio applicativo:} analisi del \glossario{Dominio Applicativo}{Dominio Applicativo}, cioè l'ambito di utilizzo del prodotto da sviluppare;
	
	\item \textbf{Dominio tecnologico:} analisi del \glossario{Dominio Tecnologico}{Dominio Tecnologico} richiesto dal capitolato e raggruppamento delle tecnologie da impiegare nello sviluppo del progetto. Tale analisi include valutazioni sulle conoscenze attuali e sulle possibilità di apprendimento in relazione alle tecnologie richieste per realizzare il prodotto proposto nel capitolato;
	
	\item \textbf{Aspetti positivi:} analisi sul costo in rapporto ai risultati previsti e all’interesse del gruppo rispetto alle tematiche del capitolato;
	
	\item \textbf{Fattori di rischio:} analisi delle criticità di realizzazione, quali ad esempio mancanza di conoscenze adeguate o difficoltà nell’individuazione di requisiti dettagliati.

\end{itemize}

% FORNITURA - Definizione del modello di sviluppo

\subsection{Definizione del modello di sviluppo}

\subsubsection{Scopo}

Lo scopo della attività è la pianificazione del progetto e del ciclo di vita del software a cui attenersi nel corso della realizzazione dello stesso effettuata dal \textit{Responsabile di Progetto} coadiuvato dagli \glossario{Amministratori}{amministratore}. 

\subsubsection{Descrizione}

Lo studio deve analizzare i seguenti punti:

\begin{itemize}
	    
	\item \textbf{Pianificazione:} pianificazione delle attività da svolgere nel corso del progetto che include delle scadenze temporali precise sulle stesse;
	
    \item \textbf{Rischi:} analisi dettagliata dei rischi che potrebbero insorgere nel corso del progetto e proposta di metodi per affrontarli. Tale analisi include la comprensione della probabilità che i rischi evidenziati si concretizzino e del livello di gravità ad essi associato;

    \item \textbf{Preventivo e Consuntivo:} stima della quantità di lavoro necessaria per ogni fase, sulla base della pianificazione effettuata. A tale stima consegue un preventivo per il costo totale del progetto da aggiornarsi ad ogni fase.
    
\end{itemize}

% FORNITURA - Pianificazione della Qualità

\subsection{Pianificazione della Qualità}

\subsubsection{Scopo}

Lo scopo della attività è lo studio di una strategia atta a garantire la qualità del materiale prodotto dal gruppo. 

\subsubsection{Descrizione}

Per assicurare il conseguimento di una qualità coerente con gli obiettivi fissati dal gruppo, l'attività dovrà definire:

\begin{itemize}
	    
	\item \textbf{Obiettivi di Qualità:} obiettivi di processo e di prodotto a cui attenersi;
	
    \item \textbf{Politica di qualità:} la strategia generale atta al conseguimento dei succitati obiettivi;

    \item \textbf{Gestione della revisione:} definizione delle modalità di comunicazione delle anomalie e delle procedure di controllo per la qualità di processo.
\end{itemize}
\end{document}
