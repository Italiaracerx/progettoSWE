\documentclass[../NormediProgetto.tex]{subfiles}

\begin{document}
	
% NORME DI PROGETTO -> PROCESSI PRIMARI
	
\chapter{Processi primari}

% NORME DI PROGETTO -> PROCESSI PRIMARI -> FORNITURA

\section{Fornitura}

% FORNITURA - introduzione

\subsection{Scopo}

Il \glossario{processo}{processo} di fornitura ha lo scopo di trattare le norme e i termini che i membri del gruppo Graphite sono tenuti a rispettare per diventare fornitori della proponente MIVOQ S.R.L. e dei committenti Prof. Tullio Vardanega e Prof. Riccardo Cardin per quanto concerne il prodotto "DeSpeect: interfaccia grafica per Speect".


% FORNITURA - rapporti di fornitura

\subsection{Rapporti di fornitura}

Durante l'intero progetto si intende instaurare con la Proponente MIVOQ S.R.L., nella persona del referente Giulio Paci, un profondo e quanto più possibile costante rapporto di collaborazione orientato a:

\begin{itemize}

    \item Determinare aspetti chiave per soddisfare i bisogni del proponente;
    
    \item Stabilire scelte volte alla definizione e realizzazione del prodotto (vincoli sui requisiti);
    
    \item Stabilire scelte volte alla definizione ed esecuzione di processi (vincoli di progetto);
    
    \item Stimare i costi;
    
    \item Concordare la qualifica del prodotto.

\end{itemize}

A seguito della consegna del prodotto, il gruppo Graphite non seguirà l'attività di manutenzione dello stesso.


% FORNITURA - Studio di fattibilità

\subsection{Studio di fattibilità}

\subsubsection{Scopo}

Il succitato documento consiste di un'analisi dettagliata di ogni capitolato proposto, con il fine di evidenziare le ragioni che hanno portato il gruppo Graphite a optare per quello scelto. 

\subsubsection{Descrizione}

In seguito alla presentazione ufficiale dei Capitolati d'appalto, è compito del \glossario{Responsabile di Progetto}{responsabile di progetto} convocare una riunione interna al gruppo per valutare le proposte di progetto pervenute. Gli \glossario{Analisti}{analista} conducono quindi un'approfondita attività di analisi dei rischi e delle opportunità su ciascun capitolato, che culmina nella stesura del documento \textit{Studio di Fattibilità v1.0.0}.
Tale documento include le motivazioni che hanno portato il gruppo Graphite a proporsi come fornitore per il prodotto indicato, nonché l'analisi di ogni capitolato proposto articolata in:

\begin{itemize}
    \item \textbf{Descrizione generale:} sintesi del prodotto da sviluppare secondo quanto stabilito dal capitolato d'appalto;
    
    \item \textbf{Dominio applicativo:} analisi del \glossario{Dominio Applicativo}{Dominio Applicativo}, cioè l'ambito di utilizzo del prodotto da sviluppare;
    
    \item \textbf{Dominio tecnologico:} analisi del \glossario{Dominio Tecnologico}{Dominio Tecnologico} richiesto dal capitolato e raggruppamento delle tecnologie da impiegare nello sviluppo del progetto. Tale analisi include valutazioni sulle conoscenze attuali e sulle possibilità di apprendimento in relazione alle tecnologie richieste per realizzare il prodotto proposto nel capitolato;
    
    \item \textbf{Aspetti positivi:} analisi sul costo in rapporto ai risultati previsti e all’interesse del gruppo rispetto alle tematiche del capitolato;
    
    \item \textbf{Fattori di rischio:} analisi delle criticità di realizzazione, quali ad esempio mancanza di conoscenze adeguate o difficoltà nell’individuazione di requisiti dettagliati;
    
    \item \textbf{Valutazione Finale:} sintesi delle motivazioni, rischi e criticità evidenziate per cui il capitolato in questione è stato respinto o accettato.
\end{itemize}

% FORNITURA - Piano di progetto

\subsection{Piano di progetto}

\subsubsection{Scopo}

Lo scopo del documento è l'esposizione dettagliata della pianificazione cui attenersi nel corso della realizzazione del progetto, redatta dal Responsabile di Progetto coadiuvato dagli \glossario{Amministratori}{amministratore}. 

\subsubsection{Descrizione}

Il documento contiene:

\begin{itemize}
    \item \textbf{Analisi dei rischi:} analisi dettagliata dei rischi che potrebbero insorgere nel corso del progetto e proposta di metodi per affrontarli. Tale analisi include la comprensione della probabilità che i rischi evidenziati si concretizzino e del livello di gravità ad essi associato;
    
    \item \textbf{Pianificazione:} relazione sulla pianificazione delle attività da svolgere nel corso del progetto che include delle scadenze temporali precise sulle stesse;
    
    \item \textbf{Preventivo e Consuntivo:} stima della quantità di lavoro necessaria per ogni fase, sulla base della pianificazione effettuata. A tale stima consegue un preventivo per il costo totale del progetto. Alla fine di ogni attività si redige inoltre un consuntivo di periodo per tracciare l’andamento rispetto a quanto preventivato.
\end{itemize}

% FORNITURA - Piano di qualifica

\subsection{Piano di qualifica}

\subsubsection{Scopo}

Lo scopo del documento è l'esposizione della strategia individuata dai \glossario{Verificatori}{verificatore} per la \glossario{Verifica}{verifica} e la \glossario{Validazione}{validazione} del materiale prodotto dal gruppo. 

\subsubsection{Descrizione}

Il documento contiene:

\begin{itemize}
    \item \textbf{Quadro generale della strategia di verifica:} definizione delle procedure di controllo sulla qualità di processo e di prodotto stabilite tenendo in considerazione le risorse a disposizione;
    
    \item \textbf \textbf{Misure e metriche:} definizione delle metriche stabilite per documenti, processi e software prodotto;
    
    \item \textbf{Gestione della revisione:} definizione delle modalità di comunicazione delle anomalie e delle procedure di controllo per la qualità di processo;
    
    \item \textbf{Resoconto delle attività di verifica:} sintesi del tracciamento delle attività di verifica e rapporto sulle metriche calcolate durante le stesse.
\end{itemize}

\end{document}
