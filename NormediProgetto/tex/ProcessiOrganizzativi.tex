\documentclass[./../NomeDocumento.tex]{subfiles}

\begin{document}
	
	\chapter{Processi Organizzativi}
	
	\section{Scopo}
	Lo scopo di questo processo è la creazione del documento \textit{Piano di Progetto}, utile ai membri del gruppo per organizzare e gestire i ruoli di ogni componente del progetto.
	
	\section {Descrizione}
	% inserire orari di lavoro
	Durante questo processo sono trattati:
	\begin{itemize}
		\item Ruoli di progetto;
		\item Comunicazioni;
		\item Incontri;
		\item Strumenti di coordinamento;
		\item Strumenti di versionamento;
		\item Rischi.
		
	\end{itemize}
	
	\section {Ruoli di progetto}
	
	Ogni ruolo viene ricoperto da ciascun componente del gruppo a turno, dando la possibilità ad ogni membro di fare esperienza in ognuno di essi. L'organizzazione e la pianificazione delle attività da svolgere in ogni ruolo è regolata dal documento \textit{Piano di Progetto v1.0.0}.
	\\ \noindent I ruoli si suddividono in:
	
	\subsection {Amministratore di Progetto}
	
	L’\textit{Amministratore di Progetto} deve controllare e amministrare tutto l’ambiente di lavoro con piena responsabilità sulla capacità operativa e sull’efficienza. 
	\\ \noindent Le sue mansioni sono: 
	
	\begin{itemize}
		\item ricerca di strumenti che migliorino l’ambiente di lavoro e che lo automatizzino ove possibile;
		\item gestione del versionamento;
		\item controllo di versioni e configurazioni del prodotto software; 
		\item risoluzione dei problemi di gestione dei processi; 
		\item controllo della \glossario{qualità}{qualita} sul prodotto.
	\end{itemize}
	
	\subsection {Responsabile di Progetto}
	
	Il \textit{Responsabile di Progetto} è il punto di riferimento sia per il \glossario{\textit{committente}}{committente} che per il \textit{fornitore}. Esso deve anche approvare le scelte prese dal gruppo e se ne assume la responsabilità. 
	\\ \noindent Le sue mansioni sono:
	
	\begin{itemize}
		\item coordinare e pianificare le attività di progetto;
		\item approvare la documentazione;
		\item effettuare uno studio e gestire in modo corretto i rischi;
		\item approvare l'offerta economica;
		\item gestire le risorse umane distribuendo in modo corretto i carichi di lavoro.
	\end{itemize}
	
	\subsection {Analista}
	
	L'\textit{Analista} si occupa dell'analisi dei problemi e del dominio applicativo. Normalmente questo ruolo non rimane attivo per tutta la durata del progetto, bensì concentra tutta la propria attività nelle fasi iniziali.
	\\ \noindent Le sue principali mansioni sono:
	
	\begin{itemize}
		\item comprensione del problema e della sua complessità;
		\item produzione dello \textit{Studio di Fattibilità} e dell'\textit{Analisi dei Requisiti}.
	\end{itemize}
	
	\subsection {Progettista}
	
	Il \textit{Progettista} gestisce gli aspetti tecnologici e tecnici del progetto.
	\\ \noindent Le sue mansioni sono:
	
	\begin{itemize}
		\item rendere facilmente mantenibile il progetto;
		\item effettuare scelte efficienti ed ottimizzate su aspetti tecnici del progetto.
	\end{itemize}
	
	\subsection {Verificatore}
	
	Il \textit{Verificatore} deve garantire una verifica completa ed esaustiva del progetto basandosi sulle sue solide conoscenze delle sue normative.
	\\ \noindent Le sue mansioni sono:
	\begin{itemize}
		\item controllare le attività del progetto secondo le normative prestabilite.
	\end{itemize}
	
	\subsection {Programmatore}
	
	Il \textit{Programmatore} è il responsabile della codifica del progetto e delle componenti di supporto, che serviranno per effettuare le prove di verifica e validazione sul prodotto.
	\\ \noindent Le sue mansioni sono:
	
	\begin{itemize}
		\item versionamento del codice prodotto;
		\item implementare le decisioni del \textit{Progettista};
		\item realizzazione degli strumenti per la verifica e la validazione del software;
		\item scrittura di un codice pulito e facile da mantenere, che rispetti le \textit{Norme di Progetto}.
	\end{itemize}
	
	\section {Procedure}
	
	\subsection{Gestione delle comunicazioni}
	
	\subsubsection{Comunicazioni interne}
	
	Le comunicazioni interne avvengono tramite un tool di messaggistica di nome \glossario{\textit{Telegram}}{Telegram}. Tale strumento offre la possibilità di comunicare con il resto del team in modo formale e informale mettendo in risalto la comunicazione più importante valida in quel momento e lasciando la possibilità di inviare file di testo, immagini o file audio utili alla riuscita del progetto.
	Non meno importante caratteristica è la portabilità di tale strumento su \glossario{dispositivi mobile}{dispositivo mobile} e sistemi operativi differenti senza dover usare tool esterni per il suo funzionamento.
	\\ \noindent per comunicazioni formali e per ricevere notifiche riguardanti le modifiche della \glossario{\textit{repository}}{Repository} verrà usato un ulteriore tool di messaggistica di nome \glossario{\textit{Slack}}{Slack}

%	\\ \noindent Per comunicazioni di tipo formale da mantenere più a lungo nel tempo in vista viene usato uno strumento di coordinamento di nome \glossario{\textit{Asana}}{Asana} che consente di aprire una o più conversazioni archiviabili una volta divenute poco attuali.
	
	\subsubsection{Comunicazioni esterne}
	
	Il \textit{Responsabile del Progetto} è tenuto a mantenere le comunicazioni esterne utilizzando una cartella di posta elettronica appositamente creata:
	\centerline{graphite.swe@gmail.com.}
	Il \textit{Responsabile del Progetto} deve mantenere informati i restanti componenti del gruppo riguardo alle discussioni con terzi utilizzando i canali di comunicazione interna. 
	\\ \noindent In ogni caso per evitare disguidi interni è previsto che alla ricezione di una mail sull'indirizzo di posta elettronica il messaggio venga inoltrato a tutti i componenti del gruppo.
	
	\subsection{Gestione degli incontri}
	
	\subsubsection{Incontri interni}
	
	Il \textit{Responsabile di progetto} è incaricato di organizzare gli incontri interni dando comunicazione di data e ora tramite i canali di comunicazione. Inoltre deve essere comunicato prima di ogni riunione l'ordine del giorno sempre dal \textit{Responsabile di progetto}.
	\\ \noindent Per mantenere sotto continuo controllo l'avanzamento dei lavori è previsto almeno un incontro settimanale con l'intero gruppo.
	\\ \noindent Ogni componente del gruppo ha diritto di presentare una richiesta di organizzazione di un incontro al \textit{Responsabile di progetto} il quale può accogliere o meno la proposta.
	
	\subsubsection{Incontri Esterni}
	
	Il \textit{Responsabile di Progetto} deve organizzare gli incontri esterni con il committente e comunicare al proprio gruppo data e ora in cui essi avvengono. Come per gli incontri interni, ogni componente del gruppo ha diritto ad una richiesta di organizzazione di un incontro esterno.
	\\ \noindent Per ogni incontro esterno deve essere preventivamente stilata una serie di domande tale da giustificare la richiesta di un appuntamento con il committente.
	
	\subsection{Gestione degli strumenti di coordinamento}
	
	\subsubsection{Ticketing}
	
	Per la suddivisione del carico di lavoro in \glossario{\textit{Task}}{task} equamente distribuiti tra tutti i componenti del gruppo viene utilizzata la piattaforma \glossario{\textit{Wrike}}{Wrike}. Tale compito è dato al \textit{Responsabile di Progetto}. \textit{Wrike} mostra in modo efficace nella propria interfaccia il quadro completo di tutti i Task inseriti con relativo status (completato, in corso o libero), persona assegnata e scadenza. Quando viene inserito, viene assegnato o cambia stato un Task viene inviata una mail ad ogni componente del gruppo e, per tutti i componenti che fanno uso dell'applicazione mobile, viene inviata una notifica sugli smartphone collegati a textit{Wrike}.
	\\ \noindent L'assegnazione dei Task avviene secondo il seguente schema:
	
	\begin{itemize}
		\item inserire un titolo al Task;
		\item dividere in più Subtask il Task e titolarli;
		\item indicare la persona a cui è stato assegnato ogni Subtask;
		\item inserire la data entro cui consegnare i documenti/file nella repository prefissata;
		\item inserire una descrizione che contiene un breve riassunto del compito assegnato e il ruolo assunto in quella fase del progetto.
	\end{itemize}
	
	\subsection{Gestione degli strumenti di versionamento}
	
	\subsubsection{Repository}
	
	Per il versionamento e il salvataggio dei file è previsto l'utilizzo di repository su \glossario{\textit{GitHub}}{GitHub}. L'\textit{Amministratore di Progetto} si deve occupare della creazione dei repository. In seguito l'\textit{Amministratore} inserirà tutti i componenti del gruppo, i quali dovranno essere in possesso di un account personale, come collaboratori.
	%da completare in seguito
	\\ \noindent È previsto l'utilizzo di [.. numero..] repository:
	
	\begin{itemize}	
		\item Documents: contiene tutta la documentazione dell'attività di progetto.
	\end{itemize}
	
	\subsubsection{Tipi di file e .gitignore}
	
	Nelle cartelle contenenti tutti i documenti saranno presenti solamente i file .tex, .pdf, .jpg, .png. Le estensioni dei file generati automaticamente dalla compilazione sosno stati aggiungi a .gitignore, e quindi vengono ignorati e resi invisibili a \glossario{\textit{Git}}{Git}.
	
	\subsubsection{Norme sui commit}
	
	Ogni volta che vengono effettuate delle modifiche ai file del repository, le quali poi vengono caricate su di esso, bisogna specificarne le motivazioni. Questo avviene utilizzando il comando \textit{commit} accompagnato da un messaggio riassuntivo e una descrizione in cui va specificato: 
	\begin{itemize}
		\item la lista dei file coinvolti;
		\item la lista delle modifiche effettuate, ordinate per ogni singolo file.
	\end{itemize}
	\subsection{Gestione dei rischi}
	
	Il \textit{Responsabile di Progetto} ha il compito di rilevare i rischi indicati nel \textit{Piano di Progetto v1.0.0.} Nel caso ne vengano individuati di nuovi dovrà aggiungerli nell'analisi dei rischi. 
	\\ \noindent La procedura da seguire per la gestione dei rischi è la seguente:
	\begin{itemize}
		\item registrare ogni riscontro dei rischi nel \textit{Piano di Progetto v1.0.0};
		\item aggiungere i nuovi rischi individuati nel \textit{Piano di Progetto v1.0.0};
		\item individuare problemi non calcolati e monitorare i rischi già previsti;
		\item ridefinire, se necessariom le strategie di progetto.
	\end{itemize}
	
	\section{Strumenti}
	
	\subsection{Sistema operativo}
	
	Il gruppo di progetto lavora sui seguenti sistemi operativi:
	\begin{itemize}
		\item Ubuntu 17.10 x64;
		\item Ubuntu 16.04 \glossario{\textit{LTS}}{LTS} x64;
		\item Windows 10 Home x64;
		\item Windows 10 Pro x64;
		\item Windows 7 Home Premium.
	\end{itemize}
	
	\subsection{Telegram}
	
	\textit{Telegram} è una applicazione di messaggistica nata come applicazione mobile e successivamente portata anche su Windows, Mac e varie distribuzioni Linux. Rispetto agli altri sistemi di messaggistica \textit{Telegram} consente un facile passaggio di immagini e documenti in più formati mantenendo inoltre nel proprio cloud storage tali file per un agevole recupero su qualsiasi dispositivo. È possibile creare gruppi di utenti la cui chat ha soprattutto il valore aggiunto di poter contenere sistemi automatici per l'organizzazione di sondaggi e la comunicazione di messaggi importanti da tenere in sovraimpressione.
	
	\subsection{GitHub}
	
	GitHub è un servizio di \glossario{\textit{hosting}}{hosting} per progetti software. 
	Il sito è principalmente utilizzato dagli sviluppatori, che caricano il codice sorgente dei loro programmi e lo rendono scaricabile dagli utenti. Può essere utilizzato anche per la condivisione e la modifica di file di testo e documenti revisionabili.
	\\ \noindent Un utente può interagire con lo sviluppatore tramite un sistema di issue tracking, pull request e commenti che permette di migliorare il codice della repository, risolvendo bug o aggiungendo funzionalità.
	
	\subsection{Asana}
	
	\textit{Asana} è un applicazione web disponibile anche per dispositivi mobile creata per aiutare i team a tracciare il loro lavoro. Si concentra nel permettere agli utenti di gestire progetti e Task online senza l'utilizzo di email.
	\\ \noindent Ogni team può creare un proprio spazio di lavoro contenente progetti suddivisi in Task. In ogni Task gli utenti possono aggiungere note, commenti, allegati e tags. Gli utenti possono seguire i progetti e i task e, nel caso in cui uno di essi cambi di stato, ricevere aggiornamenti riguardo ai cambiamenti nelle rispettive caselle email e ricevere notifiche al riguardo sui telefoni.

	\subsection{Wrike}

	\textit{Wrike} è un' applicazione web disponibile anche per dispositivi mobile creata per aiutare i team a tracciare il proprio lavoro. Le sue principali funzionalità si dividono in:
	\begin{itemize}
		\item possibilità di creare dei task, assegnarvi delle persone, darne una priorità e controllarne lo stato di completamento;
		\item capacità di creare in modo automatico diagrammi di Gantt basati sui task inseriti dall'utente;
		\item possibilità di condivisione file con la possibilità di farci delle modifiche online;
		\item possibilità di creazione di resoconti e discussioni riguardo a specifici argomenti;
		\item servizio di notifica e modifica rapida tramite l'applicazione per dispositivi mobile.
	\end{itemize}
	Questi servizi sono normalmente disponibili soltanto per la versione a pagamento del software, dando però la possibilità agli studenti di usufruirne con una limitazione di 15 persone per gruppo di lavoro.

	\subsection{Git}

	\textit{Git} è un software di controllo versione distribuito utilizzabile da interfaccia a riga di comando. Pensato per mantenere un grande progetto di sviluppo distribuito, Git supporta fortemente lo sviluppo non lineare del software.
	\\ \noindent Tramite strumenti appositi è possibile creare più diramazioni di sviluppo del software con la garanzia di poter mantenere in locale la cronologia di sviluppo completa.
	\\ \noindent In sostituzione ai comandi testuali molti membri del gruppo utilizzeranno \glossario{\textit{Gitkraken}}{Gitkraken} come interfaccia grafica per la gestione della repository.

	\subsection{Slack}

	\textit{Slack} è uno strumento di collaborazione aziendale utilizzato particolarmente per lo scambio di messaggi tra componenti di un team. L'idea di partenza di questo software è di essere un totale rimpiazzo degli scambi di E-mail e sms ma a questo si aggiunge la possibilità di aggiungere plugin per la notifica di attività annesse al proprio ambiente di lavoro.
	\\ \noindent Verranno in particolare sfruttati i plugin per la notifica di attività sulla repository o per i promemoria riguardo ai task presenti sulla piattaforma Wrike.
	

\end{document}