\documentclass[../NormeDiProgetto.tex]{subfiles}

\begin{document}
	
\section{Verifica}
\subsection{Scopo}
Si occupa di accertare che non vengano introdotti errori nel prodotto a seguito dell’esecuzione delle attività dei processi svolti nella fase in esame.
\subsection{Descrizione}
Il processo è suddiviso in due attività:
\begin{itemize}
\item \textbf{Analisi:} consiste nell’analisi del codice sorgente e la sua successiva esecuzione.
Viene effettuata tramite due tecniche, l’analisi statica e l’analisi dinamica;
\item \textbf{Test:} definisce tutti i test che vengono eseguiti sul prodotto software.
\end{itemize}
\subsection{Analisi}
\subsubsection{Analisi statica}
L’analisi statica è una tecnica che permette di individuare anomalie all’interno di documenti
e codice sorgente durante tutto il loro ciclo di vita. Si può realizzare tramite due
tecniche diverse:
\begin{itemize}
\item \textbf{Walkthrough:} viene svolta effettuando una lettura a largo spettro. Si tratta
di un’attività onerosa e collaborativa che richiede la cooperazione di più persone,
essendo una tecnica non efficiente. Verrà utilizzata principalmente durante la prima
parte del progetto, quando non tutti i membri del gruppo hanno piena padronanza
e conoscenza delle Norme di Progetto e del \glossario{Piano di Qualifica}{piano di qualifica}. Utilizzando questa
tecnica è possibile stilare una lista di controllo contenente gli errori più comuni.
\item \textbf{Inspection:} viene svolta una lettura mirata e strutturata, volta a localizzare gli
errori segnalati nella lista di controllo, con il minor costo possibile. Tramite l’acquisizione
di esperienza la lista di controllo viene progressivamente estesa, rendendo
l’inspection via via più efficacie. Normalmente è effettuata da una persona sola.
\end{itemize}
\subsubsection{Analisi dinamica}
L’analisi dinamica è una tecnica di analisi del prodotto software che richiede la sua esecuzione. Viene effettuata mediante dei test volti a verificare il funzionamento del prodotto e nel
caso in cui vengano riscontrate anomalie ne permette l’identificazione.
I test devono essere ripetibili, cioè deve essere possibile, dato lo stesso input e nello
stesso ambiente, risalire allo stesso output. Per ogni test devono dunque essere definiti
i seguenti parametri:
\begin{itemize}
\item \textbf{Ambiente:} il sistema hardware e software sul quale verrà eseguito il test del
prodotto;
\item \textbf{Stato iniziale:} lo stato iniziale dal quale il test viene eseguito;
\item \textbf{Input:} l’input inserito;
\item \textbf{Output:} l’output atteso;
\item \textbf{Istruzioni aggiuntive:} ulteriori istruzioni su come va eseguito il test e su come
vanno interpretati i risultati ottenuti.
\end{itemize}
\subsection{Test}
\subsubsection{Test di unità}
Il test di unità si pone come obiettivo primario l’isolare dal resto del codice la parte
più piccola di software testabile nell’applicazione, chiamata unità, per stabilire se essa
funziona esattamente come previsto.
\subsubsection{Test di sistema}
Il test di sistema determina la validazione del prodotto software finale e verifica dunque che esso soddisfi in modo completo i requisiti.
\subsubsection{Test di regressione}
Il test di regressione deve essere eseguito ad ogni modifica di un’\glossario{implementazione}{implementazione} del
sistema. A tal fine è necessario eseguire sul codice modificato i test esistenti, in modo da
stabilire se le modifiche apportate hanno alterato elementi precedentemente funzionanti.
\subsubsection{Test di accettazione}
Il test di accettazione prevede il \glossario{collaudo}{collaudo} del prodotto in presenza del \glossario{proponente}{proponente} e,
in caso del superamento di tale collaudo, ne consegue il rilascio ufficiale del prodotto
sviluppato.

\end{document}