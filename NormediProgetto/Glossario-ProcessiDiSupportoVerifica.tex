
\documentclass[openany,12pt,a4paper]{report}
\usepackage{StileDoc}

\title{Glossario - Norme di progetto - Verifica}
\author{Silvestri Kevin}
\date{\today}

\begin{document}

\chapter{Glossario}

\begin{itemize}
    \item \textbf{Piano di Qualifica}: documento che dovrà illustrare la strategia complessiva di verifica e validazione proposta dal fornitore per pervenire al collaudo del sistema con la massima efficienza ed efficacia.
    \item \textbf{implementazione}: la realizzazione concreta di una procedura a partire dalla sua definizione logica.
    \item \textbf{collaudo}: controllo, verifica dei requisiti tecnologici ed economici in rapporto a una tabella di caratteristiche singolarmente o universalmente prestabilite.
    \item \textbf{proponente}: chi propone e presenta qualcosa a qualcuno affinché venga accettato, approvato.
    \item \textbf{suite}: pacchetto di programmi complementari, in grado di interagire e di scambiarsi reciprocamente i dati.
    \item \textbf{debugging}: l'attività che consiste nell'individuazione e correzione da parte del programmatore di uno o più errori (bug) rilevati nel software.
    \item \textbf{profiling}: l'attività di raccolta di informazioni su qualcosa al fine di dare una descrizione di ciò.
    \item \textbf{code coverage}: una misura utilizzata per descrivere il grado di esecuzione del codice sorgente di un programma quando viene eseguita una particolare suite di test.
    \item \textbf{feedback}: valutazione, giudizio sul lavoro o sul comportamento.
\end{itemize}

\end{document}