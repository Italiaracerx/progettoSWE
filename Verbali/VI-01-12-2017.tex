\documentclass[openany,12pt,a4paper]{article}
\usepackage{subfiles}
\usepackage[]{graphicx}
\usepackage{float}
\graphicspath{{./img/}{./../img/}}
\usepackage{../StileDoc}
\title{Verbale Interno}
\author{}
\date{1 dicembre 2017}

\loadglsentries{../Glossario/Definizioni}

\begin{document}
	\makeatletter
	\begin{titlepage}
		\setlength{\headsep}{0pt}  
		\begin{center}
			\includegraphics[width=0.5\linewidth]{Logo.png}\\[1em]
			{\huge \bfseries  \@title }\\[10ex]
			\textbf{\Large Informazioni Documento} \\[2em]
			\bgroup
			\def\arraystretch{1.5}
			\begin{tabular}{l|l}
				\textbf{Data redazione} & \large \@date \\
				\textbf{Responsabile} &  \\
				\textbf{Redattori} &  \\
				\textbf{Verificatori} &  \\
				\textbf{Distribuzione} & Prof. Tullio Vardanega \\
				 & Prof. Riccardo Cardin \\
				 & Gruppo Graphite \\
				\textbf{Uso} & Interno \\
				\textbf{Recapito} & graphite.swe@gmail.com \\
			\end{tabular}
		\egroup
		\end{center}
	\end{titlepage}
	\makeatother

	\thispagestyle{empty}
	\newpage
	
	\section{Informazioni generali}
	
	\subsection{Informazioni sull'incontro}
	
	\begin{itemize} 
	    \item \textbf{Luogo:} Aula 1C150, Torre Archimede;
	    \item \textbf{Data:} 01-12-2017;
	    \item \textbf{Orari} 13:00 - 14:30;
	    \item \textbf{Assenti:} nessuno;
	    \item \textbf{Partecipanti esterni:} nessuno.
	\end{itemize}
	
	\subsection{Ragioni dell'incontro}
	
	Decisione ruoli, suddivisione lavoro e analisi dettagliata del capitolato scelto con formulazione delle domande da porre alla proponente.

	\section{Resoconto}
	
	Durante la riunione innanzitutto sono stati decisi i ruoli di ogni membro del gruppo e, di conseguenza, i compiti che ognuno di essi deve svolgere. Dopodichè la riunione si è concentrata sull'analisi e discussione del capitolato, a cui ogni membro aveva studiato personalmente nel periodo precedente alla riunione. Da ciò sono emersi dubbi, riguardanti alcuni punti non propriamente chiari al gruppo, che hanno portato alla decisione di fissare un incontro con la proponente del capitolato e la stesura  delle relative domande da porre.  
	
	
	
	\end{document}    