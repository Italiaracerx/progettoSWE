\documentclass[openany,12pt,a4paper]{article} 
\usepackage{subfiles} 
\usepackage[]{graphicx} 
\usepackage{float} 
\graphicspath{{./img/}{./../img/}} 
\usepackage{../StileDoc} 
\title{Verbale Esterno} 
\author{} 
\newcommand{\versione}{} 
\loadglsentries{../Glossario/Definizioni} 
 
\begin{document} 
  \makeatletter 
  \begin{titlepage} 
    \setlength{\headsep}{0pt}   
    \begin{center} 
      \includegraphics[width=0.5\linewidth]{Logo.png}\\[1em] 
      {\huge \bfseries  \@title }\\[10ex] 
      \textbf{\Large Informazioni Documento} \\[2em] 
      \bgroup 
      \def\arraystretch{1.5} 
      \begin{tabular}{l|l} 
        \textbf{Data approvazione} & 11 gennaio 2018 \\ 
        \textbf{Responsabile} & Samuele Modena \\ 
        \textbf{Redattore} & Manfredi Smaniotto \\ 
        \textbf{Verificatore} & Matteo Rizzo \\ 
        \textbf{Distribuzione} & Prof. Tullio Vardanega \\ 
         & Prof. Riccardo Cardin \\ 
         & Gruppo Graphite \\ 
        \textbf{Uso} & Esterno \\ 
        \textbf{Recapito} & graphite.swe@gmail.com \\ 
      \end{tabular} 
    \egroup 
    \end{center} 
  \end{titlepage} 
  \makeatother 
 
  \thispagestyle{empty} 
  \newpage 
   
  \tableofcontents 
  \newpage 
   
  \section{Informazioni generali} 
   
  \subsection{Informazioni sull'incontro} 
   
  \begin{itemize}  
      \item \textbf{Luogo:} E-mail (posta elettronica);
      \item \textbf{Data:} 03-01-2018; 
      \item \textbf{Orari} 21.00;
      \item \textbf{Partecipanti del gruppo:} Marco Focchiatti, Samuele Modena, Matteo Rizzo, Giulio Rossetti, Kevin Silvestri, Manfredi Smaniotto, Cristiano Tessarolo; 
      \item \textbf{Partecipanti esterni:} Dr. Giulio Paci. 
  \end{itemize} 
 
  \subsection{Ragioni dell'incontro} 
  Chiarimenti riguardanti la possibilità di modificare i nodi all'interno del grafico, quali differenze intercorrono tra le relazioni e gli utterance processors e cosa deve essere evidenziato sul grafo che viene costruito. 
 
  \section{Resoconto} 
 
  Via mail è stato richiesto di chiarire qualche dubbio riguardante l'interfaccia da produrre in seguito ad un iniziale stesura dell'analisi dei requisiti. 
  In particolare sono stati precisati i seguenti punti: 

   \begin{itemize} 
    \item deve essere possibile spostare graficamente i nodi presenti sul grafico, requisito che quindi deve essere obbligatorio e non semplicemente opzionale; 
    \item non è presente una relazione biunivoca tra le relazioni e gli utterance processor, quindi deve essere corretto un requisito (in particolare ROF9.2) in modo che non venga posto alcun legame tra questi 2 elementi; 
    \item non è necessario evidenziare tutti i nodi di un dato percorso selezionato, bensì è sufficiente che siamo messi in risalto il nodo di partenza e quello di arrivo. 
  \end{itemize} 
 
  \section{Tracciamento delle decisioni} 
   
  \begin{itemize} 
      \item \textbf{VE-03-01-2018.1:}  
      correzione del documento \textit{Analisi dei Requisiti} con le indicazioni ricevute. 
  \end{itemize} 
   
  \end{document}
 \ No newline at end of file 
