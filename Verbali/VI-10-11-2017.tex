\documentclass[openany,12pt,a4paper]{article}
\usepackage{subfiles}
\usepackage[]{graphicx}
\usepackage{float}
\graphicspath{{./img/}{./../img/}}
\usepackage{../StileDoc}
\title{Verbale Interno}
\author{}
\date{10 novembre 2017}

\loadglsentries{../Glossario/Definizioni}

\begin{document}
	\makeatletter
	\begin{titlepage}
		\setlength{\headsep}{0pt}  
		\begin{center}
			\includegraphics[width=0.5\linewidth]{Logo.png}\\[1em]
			{\huge \bfseries  \@title }\\[10ex]
			\textbf{\Large Informazioni Documento} \\[2em]
			\bgroup
			\def\arraystretch{1.5}
			\begin{tabular}{l|l}
				\textbf{Data redazione} & \large \@date \\
				\textbf{Responsabile} &  \\
				\textbf{Redattori} &  \\
				\textbf{Verificatori} &  \\
				\textbf{Distribuzione} & Prof. Tullio Vardanega \\
				 & Prof. Riccardo Cardin \\
				 & Gruppo Graphite \\
				\textbf{Uso} & Interno \\
				\textbf{Recapito} & graphite.swe@gmail.com \\
			\end{tabular}
		\egroup
		\end{center}
	\end{titlepage}
	\makeatother

	\thispagestyle{empty}
	\newpage
	
	\section{Informazioni generali}
	
	\subsection{Informazioni sull'incontro}
	
	\begin{itemize} 
	    \item \textbf{Luogo:} Aula 1C150, Torre Archimede;
	    \item \textbf{Data:} 10-11-2017;
	    \item \textbf{Orari} 12:30 - 14:00;
	    \item \textbf{Assenti:} nessuno;
	    \item \textbf{Partecipanti esterni:} nessuno.
	\end{itemize}
	
	\subsection{Ragioni dell'incontro}
	
	Presentazione tra i membri del gruppo e discusione dei capitolati appena presentati, con conseguente scelta del capitolato da svolgere.
	
	\section{Resoconto}
	
	Dopo aver effettuato le reciproche presentazioni tra i membri del gruppo, si è analizzato e discusso i vari pro e contro di ogni capitolato, a cui si rimanda al documento \textit{Studio di Fattibilità} per una visione dettagliata, e, tenendo conto di questi e delle preferenze espresse dai membri del gruppo, è stato scelto di svolgere il capitolato C3.
	
	
	
	\end{document}    