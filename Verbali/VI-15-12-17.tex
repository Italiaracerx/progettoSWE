\documentclass[openany,12pt,a4paper]{article}
\usepackage{subfiles}
\usepackage[]{graphicx}
\usepackage{float}
\graphicspath{{./img/}{./../img/}}
\usepackage{StileDoc}
\title{Analisi Dei Requisiti}
\author{}
\date{\today}

%\loadglsentries{../Glossario/Definizioni}

\begin{document}
	\makeatletter
	\begin{titlepage}
		\setlength{\headsep}{0pt}  
		\begin{center}
			{\huge \bfseries  \@title }\\[10ex]
			\includegraphics[width=0.5\linewidth]{Logo.png}\\[1em]
			\textbf{\Large Informazioni Documento} \\[2em]
			\bgroup
			\def\arraystretch{1.5}
			\begin{tabular}{l|l}
				\textbf{Versione} & 1.0.0 \\
				\textbf{Data redazione} & \large \@date \\
				\textbf{Redattori} &  \\
				\textbf{Verificatori} &  \\
				\textbf{Distribuzione} & Prof. Tullio Vardanega \\
				 & Prof. Riccardo Cardin \\
				 & Gruppo Graphite \\
				\textbf{Uso} & Interno \\
			\end{tabular}
		\egroup
		\end{center}
	\end{titlepage}
	\makeatother

	\thispagestyle{empty}
	\newpage
	
	\section{Informazioni generali}
	
	\subsection{Informazioni sull'incontro}
	
	\begin{itemize} 
	    \item \textbf{Luogo:} Aula P300, Complesso Paolotti (Videochiamata);
	    \item \textbf{Data:} 15-12-2017;
	    \item \textbf{Orari} 14:00 - 15:00;
	    \item \textbf{Assenti:} nessuno;
	    \item \textbf{Partecipanti esterni:} Giulio Paci.
	\end{itemize}
	
	\subsection{Ragioni dell'incontro}
	
	Chiarimenti su dubbi emersi riguardo i requisiti del prodotto richiesto nel capitolato d'appalto proposto da MIVOQ SRL ([C3]). Tale capitolato è reperibile al seguente link:
	\\ 
	\\ \url{http://www.math.unipd.it/~tullio/IS-1/2017/Progetto/C3.pdf}
	
	\section{Resoconto}
	
	Nel corso dell'incontro sono stati chiariti i dubbi sollevati dal gruppo sul prodotto da realizzare per il capitolato proposto da MIVOQ SRL, nello specifico sono stati delucidati i seguenti punti:
	
	\begin{itemize}
	    \item Illustrata interfaccia grafica dimostrativa inclusa nel capitolato d'appalto, con spiegazione del significato e delle funzionalità richieste dagli elementi che la compongono.
	    \item Chiarimento sul funzionamento generale della tecnologia \glossario{Speect}{Speect} di cui farà largo uso il progetto.
	    \item Chiarimento sul livello di manipolazione richiesto dei nodi del \glossario{grafo}{grafo} offerto in output all'utente.
	    \item Illustrazione della struttura dei file \glossario{jason}{jason} di cui fa uso Speect.
	\end{itemize}
	
	
	
	\end{document}    










                   