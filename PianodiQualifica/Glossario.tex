\documentclass[openany,12pt,a4paper]{report}
\usepackage{StileDoc}

\title{Glossario - Piano di qualifica}
\author{Matteo Rizzo}
\date{\today}

\begin{document}

\chapter{Glossario (piano di qualifica)}

\newglossaryentry{qualità}
{
	name={Qualità},
	description={insieme delle caratteristiche di un'entità che ne determinano la capacità di soddisfare esigenze espresse e implicite},
	nonumberlist 
}

\newglossaryentry{sistema qualità}
{
	name={Sistema Qualità},
	description={struttura organizzativa, responsabilità, procedure, e risorse messe in atto per il perseguimento della qualità},
	nonumberlist 
}

\newglossaryentry{pianificazione della qualità}
{
	name={Pianificazione della Qualità},
	description={le attività del sistema qualità mirate a fissare gli obiettivi di qualità, e i processi e le risorse necessarie per conseguirli},
	nonumberlist 
}

\newglossaryentry{grado di accoppiamento}
{
	name={Grado di accoppiamento},
	description={Per grado di accoppiamento si intende il grado con cui ciascuna componente di un programma fa affidamento su ciascuna delle altre componenti, dipendendo dunque da essa.},
	nonumberlist 
}

\newglossaryentry{ISO/IEC 15504}
{
	name={ISO/IEC 15504},
	description={ISO/IEC 15504 consiste di un insieme di norme tecniche per il processo di sviluppo del software e le relative funzioni di gestione aziendale. È uno degli standard internazionali dell'Organizzazione internazionale per la standardizzazione (ISO) e della Commissione elettrotecnica internazionale (IEC).},
	nonumberlist 
}

\newglossaryentry{SPICE}
{
	name={SPICE},
	description={SPICE (acronimo di Software Process Improvement and Capability Determination, dove per capability si intende la capacità intesa come abilità di un processo nel raggiungere un obiettivo) è un nome alternativo per indicare lo standard di qualità ISO/IEC 15504.} 
	nonumberlist 
}

\newglossaryentry{Ciclo di Deming}
{
	name={Ciclo di Deming},
	description={Il ciclo di Deming è un metodo di gestione iterativo in quattro fasi utilizzato per il controllo e il miglioramento continuo dei processi e dei prodotti.} 
	nonumberlist 
}

\newglossaryentry{Ciclo di PDCA}
{
	name={Ciclo di PDCA},
	description={Ciclo di PDCA, acronimo dall'inglese Plan–Do–Check–Act, in italiano "Pianificare - Fare - Verificare - Agire", è un nome alternativo per indicare il ciclo di Deming} 
	nonumberlist 
}

\newglossaryentry{ISO/IEC 9126}
{
	name={ISO/IEC 9126},
	description={Con la sigla ISO/IEC 9126 si individua una serie di normative e linee guida, sviluppate dall'ISO (Organizzazione internazionale per la normazione) in collaborazione con l'IEC (Commissione Elettrotecnica Internazionale), preposte a descrivere un modello di qualità del software.} 
	nonumberlist 
}

\newglossaryentry{Indice Gulpease}
{
	name={Indice Gulpease},
	description={L'Indice Gulpease è un indice di leggibilità di un testo tarato sulla lingua italiana.} 
	nonumberlist 
}

\newglossaryentry{Schedule Variance}
{
	name={Schedule Variance},
	description={La Schedule Variance è una metrica che indica se si è in linea, in anticipo o in ritardo rispetto alla schedulazione delle attività di progetto pianificate nella baseline.} 
	nonumberlist 
}

\newglossaryentry{Metrica di progetto}
{
	name={Metrica di progetto},
	description={Le metriche di progetto rappresentano un insieme di indicatori rivolto a tenere sotto controllo e prevedere l'andamento delle principali variabili critiche del progetto (i costi, i tempi, la qualità, le risorse, le variazioni di scopo, ecc.). Le metriche includono solitamente una serie di indicatori standard, ma possono essere estese con altri indicatori definiti ad hoc in base alla specifica natura del progetto.} 
	nonumberlist 
}

\newglossaryentry{Baseline}
{
	name={Baseline},
	description={La baseline di progetto costituisce il punto di riferimento rispetto al quale calcolare gli scostamenti delle principali variabili implicate nella gestione di un progetto.} 
	nonumberlist 
}

\newglossaryentry{Budget Variance}
{
	name={Budget Variance},
	description={La Budget Variance è una metrica di progetto standard che indica se si sta sforando, risparmiando o se si è in linea rispetto alla schedulazione delle attività di progetto pianificate nella baseline e al budget preventivato.} 
	nonumberlist 
}

\newglossaryentry{Complessità ciclomatica}
{
	name={Complessità ciclomatica},
	description={La Complessità Ciclomatica (o complessità condizionale) è una metrica software. Sviluppata da Thomas J. McCabe nel 1976, è utilizzata per misurare la complessità di un programma.} 
	nonumberlist 
}

\newglossaryentry{Grafo}
{
	name={Grafo},
	description={In matematica, la configurazione formata da un insieme di punti (vertici o nodi) e un insieme di linee (archi) che uniscono coppie di nodi.} 
	nonumberlist 
}

\newglossaryentry{Copertura del codice}
{
	name={Grafo},
	description={In informatica, la copertura del codice è una misura utilizzata per descrivere il grado in cui viene eseguito il codice sorgente di un programma quando viene eseguita una particolare suite di test.} 
	nonumberlist 
}

\newglossaryentry{Bug}
{
	name={Bug},
	description={Il termine inglese bug (in italiano baco), identifica in informatica un errore nella scrittura del codice sorgente di un programma software.} 
	nonumberlist 
}

\newglossaryentry{Package}
{
	name={Package},
	description={Con il termine package si indica un meccanismo per organizzare classi in gruppi logici, principalmente (ma non solo) in modo da definire namespace distinti per diversi contesti. Il package ha lo scopo di riunire classi (o entità analoghe, quali interfacce ed enumerazioni) logicamente correlate.} 
	nonumberlist 
}

\newglossaryentry{Accoppiamento}
{
	name={Accoppiamento},
	description={Per accoppiamento o dipendenza, in informatica e ingegneria del software, si intende il grado con cui ciascuna componente di un software dipende dalle altre componenti.} 
	nonumberlist 
}

\newglossaryentry{Issues}
{
	name={Issue},
	description={Per Issues si intende il sistema di tracciamento e gestione dei bug offerto dal servizio di hosting per progetti software Github. Un issue consiste del rilevamento, della discussione e dell'assegnazione della risoluzione a un membro del gruppo di un'anomalia software.} 
	nonumberlist 
}

\newglossaryentry{Repository}
{
	name={Repository},
	description={Letteralmente deposito, è un ambiente di un sistema informativo in cui vengono gestiti i metadati, attraverso tabelle relazionali. Si tratta di qualcosa di più sofisticato del classico dizionario dati, ed è un ambiente che può essere implementato attraverso numerose piattaforme hardware e sistemi di gestione dei database.} 
	nonumberlist 
}

\newglossaryentry{Best practice}
{
	name={Best practice},
	description={Esperienza, o linea guida ricavabile da un insieme di esperienze, che ha permesso di ottenere risultati eccellenti in un determinato ambito e che costituisce quindi un esempio da seguire, o quantomeno imitare.} 
	nonumberlist 
}

\newglossaryentry{Target}
{
	name={Target},
	description={Nella pratica pubblicitaria il segmento di pubblico a cui è diretta una determinata comunicazione commerciale.} 
	nonumberlist 
}
\end{document}