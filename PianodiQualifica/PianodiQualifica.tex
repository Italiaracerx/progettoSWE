
\documentclass[openany,12pt,a4paper]{report}
\usepackage{subfiles}
\usepackage[]{graphicx}
\usepackage{multirow}
\graphicspath{{./img/}{./../img/}}
\usepackage{../StileDoc}
\title{Piano di qualifica}
\author{}

\loadglsentries{../Glossario/Definizioni}
%Ultima versione documento

\newcommand{\versione}{1.0.0}


\begin{document}
	\makeatletter
	\begin{titlepage}
		\setlength{\headsep}{0pt}  
		\begin{center}
			\includegraphics[width=0.5\linewidth]{img/logo.png}\\[1em]
			{\huge \bfseries  \@title }\\[10ex]
			\textbf{\Large Informazioni Documento} \\[2em]
			\bgroup
			\def\arraystretch{1.5}
			\begin{tabular}{l|l}
				\textbf{Versione} & \versione{} \\
				\textbf{Data approvazione} & 12 Gennaio 2018 \\
				\textbf{Responsabile} & Samuele Modena\\
				\textbf{Redattori} &  Kevin Silvestri, Samuele Modena,\\
				& Matteo Rizzo \\
				\textbf{Verificatori} & Manfredi Smaniotto, Giulio Rossetti \\
				\textbf{Distribuzione} & Prof. Tullio Vardanega \\
				 & Prof. Riccardo Cardin \\
				 & Gruppo Graphite \\
				\textbf{Uso} & Esterno \\
				\textbf{Recapito} & graphite.swe@gmail.com \\
			\end{tabular}
		\egroup
		\end{center}
	\end{titlepage}
	\makeatother
 
\thispagestyle{empty}
\newpage

%REGISTRO DELLE MODIFICHE

\chapter*{Registro delle modifiche}
\setlength\LTleft{-22mm}
\begin{longtable}{|p{20mm}|p{20mm}|p{40mm}|p{30mm}|p{50mm}|}
	\hline
	\textbf{Versione} & \textbf{Data} & \textbf{Autore} & \textbf{Ruolo} & \textbf{Descrizione} \\

		\hline 1.0.0 & 12-01-2018 & Samuele Modena & Responsabile & Approvazione \\
		\hline 0.2.0 & 11-01-2018 & Giulio Rossetti & Verificatore & Verifica \\
		\hline 0.1.2 & 10-01-2018 & Samuele Modena & Verificatore & Stesura resoconto delle attività di verifica \\
		\hline 0.1.1 & 20-12-2017 & Matteo Rizzo & Verificatore & Aggiunta misure e metriche \\
		\hline 0.1.0 & 19-12-2017 & Manfredi Smaniotto & Verificatore & Verifica \\		
		\hline 0.0.6 & 18-12-2017 & Kevin Silvestri & Verificatore & Stesura standard di qualità \\
		\hline 0.0.5 & 17-12-2017 & Kevin Silvestri & Verificatore & Stesura gestione amministrativa \\	
		\hline 0.0.3 & 15-12-2017 & Matteo Rizzo & Verificatore & Stesura misure e metriche \\
		\hline 0.0.2 & 14-12-2017 & Samuele Modena & Verificatore & Stesura visione generale \\
		\hline 0.0.1 & 13-12-2017 & Matteo Rizzo & Verificatore & Stesura Introduzione \\
		\hline
		
	\end{longtable}


% INDICE
\tableofcontents

% INTRODUZIONE

\chapter{Introduzione}

    \section{Scopo del documento}
    
    Lo scopo del documento è quello di esporre le strategie, le tecnologie e le metriche che il gruppo Graphite adotta al fine di garantire le qualità di prodotto e di processo. Il documento ha dunque l'intento di chiarificare il \glossario{Sistema Qualità}{Sistema Qualita} instaurato e accettato dal gruppo in relazione al progetto corrente. Con l'obiettivo di rivelare e correggere in maniera efficace ed economica ogni errore, viene costantemente applicato un sistema di \glossario{verifica}{verifica} e \glossario{validazione}{validazione} ai processi e alle attività svolte. Si vuole inoltre sottolineare la natura incrementale del documento, che intende essere ampliato e migliorato in itinere.
    
    \section{Scopo del prodotto}
    
    Lo scopo del progetto è la realizzazione di un’interfaccia grafica per \glossario{Speect}{Speect} [Meraka Institute(2008-2013)], una libreria per la creazione di sistemi di sintesi vocale, che agevoli l’ispezione del suo stato interno durante il funzionamento e la scrittura di test per le sue funzionalità.
    
    \section{Glossario}
    
    Al fine di evitare ogni ambiguità relativa al linguaggio utilizzato nei documenti, viene fornito il \textit{Glossario v1.0.0}, contenente la definizione dei termini in corsivo marcati con il pedice "G".
    
    \section{Riferimenti normativi}
    
    \begin{itemize}
    
        \item \textbf{Norme di progetto:} documento \textit{Norme di progetto v1.0.0}.
        
        \item\textbf{Capitolato d'appalto C3:} DeSpeect: interfaccia grafica per Speect \\ \url{http://www.math.unipd.it/~tullio/IS-1/2017/Progetto/C3.pdf}
    
    \end{itemize}
    
    \section{Riferimenti informativi}
    
    \begin{itemize}
        \item \textbf{Piano di Progetto:} documento \textit{Piano di Progetto v 1.0.0};
        
        \item \textbf{Qualità di prodotto - Slide del corso:} 
        \\ \url{http://www.math.unipd.it/~tullio/IS-1/2017/Dispense/L13.pdf};
        
        \item \textbf{Qualità di processo - Slide del corso:} \\ \url{http://www.math.unipd.it/~tullio/IS-1/2017/Dispense/L15.pdf};
        
        \item \textbf{Libro del corso:} Software Engineering - Ian Sommerville - 9 th Edition (2010);
        
        \item \textbf{Standard ISO/IEC 15504:} 
        \\ \url{https://en.wikipedia.org/wiki/ISO/IEC_15504};
        
        \item \textbf{HM\&S - SPICE Process Assessment Model:} 
        \\ \url{http://www.spice121.com/cms/en/about-spice-1-2-1.html};
        
        \item \textbf{Standard ISO/IEC 9126:}
        \\ \url{https://en.wikipedia.org/wiki/ISO/IEC_9126};
        
        \item \textbf{Ciclo di Deming (PDCA):} 
        \\ \url{https://en.wikipedia.org/wiki/PDCA};
        
        \item \textbf{Complessità ciclomatica:} 
        \\ \url{https://it.wikipedia.org/wiki/Complessit\%e0 \textunderscore ciclomatica}.
    \end{itemize}

% PIANIFICAZIONE DELLA QUALITA'

\chapter{Visione generale delle strategie \\ di gestione della qualità}
    
    % STANDARD QUALITATIVI DI RIFERIMENTO
    
    \section{Definizione degli standard \\ qualitativi di riferimento}
    
    Vengono di seguito sinteticamente esposti gli standard di qualità a cui il gruppo intende aderire e le motivazioni di tale scelta.   
    
    \subsection{Qualità di processo - ISO/IEC 15504}
    
    \glossario{ISO/IEC 15504}{ISO/IEC 15504}, anche noto come \glossario{SPICE}{SPICE}, è lo standard scelto per la definizione degli obiettivi di processo. Si rimanda all'appendice A.1 per un'approfondita descrizione di tale standard.
    La scelta di SPICE è motivata dal fatto che esso fornisce gli strumenti utili a valutare la qualità di processo, parametro da tenere in grande considerazione per il conseguimento di un prodotto qualitativamente valido entro tempi prestabiliti. 
    Per poter applicare correttamente SPICE, viene utilizzato il \textit{ciclo di Deming} o \glossario{ciclo di PDCA}{Ciclo di PDCA}. Si rimanda all'appendice A.3 per un'approfondita descrizione del ciclo di Deming. Tale ciclo definisce un metodo di controllo orientato al miglioramento continuo del livello qualitativo dei processi, evitando nel contempo regressioni. Il ciclo di Deming si applica solo conoscendo lo stato di maturità attuale dei processi di interesse, definendo specifici obiettivi di miglioramento, e studiando i risultati delle azioni migliorative sperimentate. Esistono dunque  stringenti precondizioni alla sua applicabilità, ovvero l'attuazione di processi ripetibili e misurabili, qualità di processo che il gruppo ha intenzione di perseguire. L'affiancamento dello standard ISO al ciclo PDCA permette di:
    
    \begin{itemize}
    	\item Misurare costantemente le performance di processo;
    	\item Perseguire un miglioramento continuo di tali performance;
    	\item Rispettare tempi e costi indicati nel PP.
    \end{itemize} 
    
    \subsection{Qualità di prodotto - ISO/IEC 9126}
    
    \glossario{ISO/IEC 9126}{ISO/IEC 9126} è lo standard scelto per la definizione degli obiettivi di prodotto. Si rimanda all'appendice A.2 per un'approfondita descrizione di tale standard.
    La scelta di ISO/IEC 9126 è motivata dal fatto che esso definisce criteri di applicazioni nell'ambito di metriche per la qualità interna esterna e in uso del software, qui approfondite nella sezione \textit{Misure e metriche}, utili a valutare il grado di raggiungimento degli obiettivi prefissati.
    I prodotti realizzati durante lo svolgimento del progetto sono di due tipologie:
    
    \begin{itemize}
    	\item \textbf{Documentazione:} deve essere leggibile, comprensibile e corretta dal punto di vista ortografico, sintattico e semantico.
    	
    	\item \textbf{Software:} deve avere le seguenti caratteristiche:
    	\begin{itemize}
    		\item Possedere funzionalità che soddisfino i requisiti fissati;
    		\item \glossario{Manutenibilità}{manutenibilita};
    		\item Essere ampiamente testato;
    		\item \glossario{Robustezza}{robustezza};
    	\end{itemize}
    \end{itemize}
    
    % DEFINIZIONE DEGLI OBIETTIVI DI QUALITA'
    
    \section{Definizione degli obiettivi di qualità}    
    
    Vengono di seguito illustrati gli obiettivi fissati dal gruppo al fine di garantire la qualità di processo e di prodotto. Gli obiettivi di qualità sono univocamente identificati da un codice che ne agevola il tracciamento. La classificazione degli obiettivi è descritta in dettaglio nelle NP.
    
    \subsection{Obiettivi di qualità di processo}
    
    	\begin{longtable}{| p{2cm} | p{3.5cm} |p{5.5cm} | p{5.5cm} |}
    		\caption {Tabella obiettivi di qualità di processo} \label{tab:title} \\
    		\hline
    		\textbf{Codice} & \textbf{Nome} & \textbf{Descrizione} & \textbf{Metriche associate}\\
    		\hline
    		\endhead
    		
    		\newline OQP001&
    		\newline Miglioramento continuo&
    		\newline Capacità del processo di misurare e migliorare le proprie performance \newline &
    		\newline MP001: SPICE
    		\\[1em]
    		
    		\hline
    	\end{longtable}
    
    \subsection{Obiettivi di qualità di prodotto}
        
        \begin{longtable}{| p{2cm} | p{3.5cm} |p{5.5cm} | p{5.5cm} |}
        	\caption {Tabella obiettivi di qualità di prodotto} \label{tab:title} \\
        	\hline
        	\textbf{Codice} & \textbf{Nome} & \textbf{Descrizione} & \textbf{Metriche associate}\\
        	\hline
        	\endhead
        	
        	\newline OQPPD001 &
        	\newline Correttezza ortografica dei documenti &
        	\newline I documenti non devono riportare errori ortografici o grammaticali \newline &
        	\newline MPPD001: Errori ortografici corretti
        	\\[1em]
        	
        	\hline
        	
        	\newline OQPPD002 &
        	\newline Leggibilità dei documenti &
        	\newline I documenti devono essere leggibili e comprensibili da persone con licenza media o superiore \newline &
        	\newline MPPD002: Gulpease
        	\\[1em]
        	
        	\hline
        	
        	\newline OQPPS001 &
        	\newline Implementazione requisiti obbligatori &
        	\newline Il prodotto richiesto dalla Proponente deve implementare tutti i requisiti obbligatori descritti nella AR \newline &
        	\newline MPPS001: Copertura Requisiti Obbligatori
        	\\[1em]
        	
        	\hline
        	\newline OQPPS002 &
        	\newline Copertura del codice &
        	\newline Il prodotto richiesto dalla Proponente deve essere testato in ogni sua parte per garantire le funzionalità relative ai requisiti \newline &
        	\newline MPPS002: Linee di codice coperte dai test
        	\\[1em]
        	
        	\hline
        	\newline OQPPS003 &
        	\newline Superamento Test &
        	\newline La percentuale di superamento dei test del prodotto software dovrà essere almeno l'80\% del totale \newline &
        	\newline MPPS003: Percentuale superamento test
        	\\[1em]
        	
        	\hline
        	\newline OQPPS004 &
        	\newline Robustezza &
        	\newline Il prodotto richiesto dalla proponente deve affrontare situazioni anomale senza arrestare la sua esecuzione \newline &
        	\newline MPPS004: Failure Avoidance
        	\\[1em]
        	
        	\hline
        	\newline OQPPS005 &
        	\newline Manutenzione e comprensione del codice &
        	\newline Il codice del prodotto richiesto dalla proponente deve essere quanto più possibile comprensibile e manutenibile \newline &
        	\newline MPPS005: Numero di parametri per metodo;
        	\newline MPPS006: Numero di attributi per classe;
        	\newline MPPS007: Numero di metodi per classe;
        	\newline MPPS008: Complessità ciclomatica;
        	\newline MPPS009: Grado di instabilità;
        	\newline MPPS010: Altezza albero della gerarchia;
        	\newline MPPS011: Rapporto linee di commento / linee di codice.
        	\\[1em]
        	
        	\hline
        \end{longtable}
    
	% MISURE E METRICHE
	
	\section{Misure e metriche}
	
	Allo scopo di conseguire e monitorare gli obiettivi di qualità definiti, è necessario che il processo di verifica produca risultati quantificabili che sia possibile confrontare con delle costanti di riferimento. Vengono di seguito stabiliti i valori di riferimento per le metriche descritte in dettaglio nelle NP, indicanti se i livelli qualitativi di processo e di prodotto sono in linea con gli obiettivi prefissati o meno. 	
	
	% METRICHE PER I PROCESSI
	
	\subsection{Metriche per i processi}
	
	\begin{longtable}{| p{2cm} | p{3.5cm} |p{5.5cm} | p{5.5cm} |}
		\caption {Tabella metriche per i processi} \label{tab:title} \\
		\hline
		\textbf{Codice} & \textbf{Nome} & \textbf{Range di accettabilità} & \textbf{Obiettivi associati}\\
		\hline
		\endhead
		
		\newline MP001&
		\newline ISO/IEC 15504 (SPICE)&
		\newline \textbf{Accettazione:} [P] per ogni processo 
		\newline \textbf{Ottimale:} [L - F] per ogni processo&
		\newline OQP001: Miglioramento continuo
		\\[1em]
		
		\hline
	\end{longtable}
	
	% METRICHE PER DOCUMENTI
	
	\subsection{Metriche per i documenti}
	
	\begin{longtable}{| p{2cm} | p{3.5cm} |p{5.5cm} | p{5.5cm} |}
		\caption {Tabella metriche per i processi} \label{tab:title} \\
		\hline
		\textbf{Codice} & \textbf{Nome} & \textbf{Range di accettabilità} & \textbf{Obiettivi associati}\\
		\hline
		\endhead
		
		\newline MPPD001 &
		\newline Errori ortografici corretti &
		\newline \textbf{Accettazione:} 100\% degli errori corretti
		\newline \textbf{Ottimale:} 100\% degli errori corretti&
		\newline OQPPD001: Correttezza ortografica dei documenti
		\\[1em]
		
		\hline
		
		\newline MPPD002 &
		\newline Indice Gulpease &
		\newline \textbf{Accettazione:} [50, 100] 
		\newline \textbf{Ottimale:} [65, 100]&
		\newline OQPPD002: Leggibilità dei documenti  
		\\[1em]
		
		\hline
	\end{longtable}

	% METRICHE PER IL SOFTWARE
	
	\subsection{Metriche per il software}
	
	\begin{longtable}{| p{2cm} | p{3.5cm} |p{5.5cm} | p{5.5cm} |}
		\caption {Tabella metriche per i processi} \label{tab:title} \\
		\hline
		\textbf{Codice} & \textbf{Nome} & \textbf{Range di accettabilità} & \textbf{Obiettivi associati}\\
		\hline
		\endhead
		
		\newline MPPS001 &
		\newline Copertura requisiti obbligatori &
		\newline \textbf{Accettazione:} [65\% - 100\%]
		\newline \textbf{Ottimale:} [80\% - 100\%] &
		\newline OQPPS001: Implementazione requisiti obbligatori
		\\[1em]
		
		\hline
		
		\newline MPPS002 &
		\newline Copertura del codice &
		\newline \textbf{Accettazione:} [65\% - 100\%]
		\newline \textbf{Ottimale:} [85\% - 100\%] &
		\newline OQPPS002: Copertura del codice
		\\[1em]
		
		\hline
		
		\newline MPPS003 &
		\newline Percentuale superamento test &
		\newline \textbf{Accettazione:} [85\% - 100\%] 
		\newline \textbf{Ottimale:} [100\% - 100\%] &
		\newline OQPPS003: Superamento test
		\\[1em]
		
		\hline
		
		\newline MPPS004 &
		\newline Failure avoidance &
		\newline \textbf{Accettazione:} [80\% - 100\%] 
		\newline \textbf{Ottimale:} [90\% - 100\%] &
		\newline OQPPS004: Robustezza
		\\[1em]
		
		\hline
		
		\newline MPPS005 &
		\newline Numero di parametri per metodo &
		\newline \textbf{Accettazione:} [0, 5]
		\newline \textbf{Ottimale:} [0, 3] &
		\newline OQPPS005: Manutenzione e comprensione del codice
		\\[1em]
		
		\hline
		
		\newline MPPS006 &
		\newline Numero di attributi per classe &
		\newline \textbf{Accettazione:} [0, 15] 
		\newline \textbf{Ottimale:} [0, 8] &
		\newline OQPPS005: Manutenzione e comprensione del codice
		\\[1em]
		
		\hline
		
		\newline MPPS007 &
		\newline Numero di metodi per classe &
		\newline \textbf{Accettazione:} [0, 15] 
		\newline \textbf{Ottimale:} [0, 5] &
		\newline OQPPS005: Manutenzione e comprensione del codice 
		\\[1em]
		
		\hline
		
		\newline MPPS008 &
		\newline Complessità ciclomatica &
		\newline \textbf{Accettazione:} [0, 15] 
		\newline \textbf{Ottimale:} [0, 10] &
		\newline OQPPS005: Manutenzione e comprensione del codice 
		\\[1em]
		
		\hline
		
		\newline MPPS009 &
		\newline Grado di instabilità &
		\newline \textbf{Accettazione:} [0.0 - 0.8] 
		\newline \textbf{Ottimale:} [0.0 - 0.4]&
		\newline OQPPS005: Manutenzione e comprensione del codice 
		\\[1em]
		
		\hline
		
		\newline MPPS010 &
		\newline Altezza albero della gerarchia &
		\newline \textbf{Accettazione:} \newline $ \textrm{\textit{rapporto}} \geq 0.3\% $
		\newline \textbf{Ottimale:} $ \textrm{\textit{rapporto}} \geq 0.5\% $&
		\newline OQPPS005: Manutenzione e comprensione del codice 
		\\[1em]
		
		\hline
		
		\newline MPPS011 &
		\newline Rapporto tra linee di codice e linee di commento &
		\newline \textbf{Accettazione:} \newline $ \textrm{\textit{rapporto}} \geq 0.3\% $
		\newline \textbf{Ottimale:} $ \textrm{\textit{rapporto}} \geq 0.5\% $ &
		\newline OQPPS005: Manutenzione e comprensione del codice 
		\\[1em]
		
		\hline
	\end{longtable}
	
	\section{Definizione delle anomalie}
	
	L’identificazione delle anomalie ha come scopo la loro risoluzione e rappresenta un dato rilevante per il monitoraggio dello stato del prodotto. Distinguere e catalogare le anomalie permette di organizzare (in particolar modo di priorizzare) e affinare le correzioni da attuare per eliminarle. Di seguito
	vengono quindi elencate le definizioni di anomalie (secondo glossario IEEE 610.12-90) adottate dal gruppo:
	
	\begin{itemize}
		\item \textbf{Error:} differenza riscontrata tra risultato di una computazione e valore teorico atteso;
		\item \textbf{Fault:} un passo, un processo o un dato definito in modo erroneo che corrisponde a quanto viene definito come bug;
		\item \textbf{Failure:} il risultato di un fault;
		\item \textbf{Mistake:} azione umana che produce un risultato errato.
	\end{itemize}

	Nello specifico, rappresentano un'anomalia:
	
	\begin{itemize}
		\item La violazione delle norme tipografiche definite nelle NP;
		\item La presenza di contenuti non pertinenti l'argomento trattato o il documento in cui risiedono;
		\item Errori di codifica;
		\item Il mancato rispetto dei valori di accettazione fissati in questo documento;
		\item Incongruenze tra il prodotto e le sue funzionalità determinate nell'\textit{Analisi dei Requisiti}.
	\end{itemize}
    
	\section{Scadenze temporali}
	
	Vista la presenza delle scadenze temporali definite nel PP, si necessita di un sistema di controllo efficiente dei tempi. Le procedure di controllo che verranno attuate per individuare e correggere eventuali errori sono descritte
	nelle NP. Nel tentativo di prevenire l'insorgenza di errori stessi, ogni attività svolta detiene un periodo iniziale di
	studio sull'argomento, che riduce la quantità di interventi correttivi a posteriori.
	
	\section{Risorse}
	
	Il controllo eseguito per garantire il livello qualitativo di processi e prodotti necessita di risorse umane e tecnologiche. In relazione a questa attività, i ruoli di maggior rilievo sono il \textit{Responsabile} e il \textit{Verificatore}, che rispettivamente si occupano del controllo di qualità del processo e del prodotto risultante. Una descrizione dettagliata di tali ruoli è reperibile nel documento NP. 
	Le risorse tecnologiche comprendono tutti gli strumenti software e hardware che vengono utilizzati per attuare le procedure di verifica, automatizzate e non. Una descrizione dettagliata di tali risorse è reperibile nel documento NP. 
	
\chapter{Gestione amministrativa \\ della revisione}
	
	\section{Gestione dei processi di \\ verifica e validazione}
	
	Il processo di verifica viene istanziato per ogni processo in esecuzione quando questo raggiunge un livello di maturità significativo, e/o in seguito a modiche notevoli del suo stato. Per ogni processo viene verificata la qualità dello stesso e del suo esito. 
	Ognuno dei periodi descritti nel PP produce degli esiti diversi, pertanto le procedure di verifica saranno specializzate e i loro risultati riportati in un'apposita appendice al termine di questo documento. 
	Al processo di verifica segue quello di approvazione, nel quale il \textit{Responsabile} si accerta che i risultati prodotti siano conformi con quanto atteso e accettabili dal punto di vista qualitativo.
	
	\section{Comunicazione e risoluzione \\ delle anomalie}
	Tale attività ha lo scopo di individuare e risolvere tempestivamente le anomalie riscontrabili nel corso del progetto. Qualora venisse rilevata un'anomalia durante l'attività di verifica, questa dovrà essere tempestivamente segnalata tramite il sistema di ticketing come descritto nelle NP. Ciò permette una pronta segnalazione dell'anomalia, informando il Responsabile della stessa cosicché possa prendere i necessari provvedimenti.
\appendix

% STANDARD DI QUALITà

\chapter{Standard di qualità}

% SPICE

\section{Qualità di processo - ISO/IEC 15504}

\subsection{Introduzione allo standard}

Il modello ISO/IEC 15504, anche noto come SPICE (acronimo di Software Process Improvement and Capability Determination, dove per \textit{capability} si intende la capacità intesa come abilità di un processo nel raggiungere un obiettivo) è lo standard di riferimento per la valutazione oggettiva della qualità dei processi software e permette la misurazione indipendente della capacità di ogni processo tramite la classificazione di alcuni attributi, eseguita previo studio del range di risultati che la sua esecuzione restituisce. Perché possano contribuire al miglioramento dei processi, le singole valutazioni devono essere ripetibili, oggettive e fornire esiti comparabili. Gli attributi associati alle capacità di ogni processo sono:

\begin{itemize}
    \item \textbf{Process performance:(PP)} indica in quale misura sono raggiunti gli obiettivi fissati;
    \item \textbf{Performance management:(PM)} indica il grado di organizzazione con cui sono raggiunti gli obiettivi fissati;
    \item \textbf{Work product management:(WMP)} indica in quale misura i prodotti sono gestiti correttamente per quanto riguarda documentazione, controllo e verifica;
    \item \textbf{Process definition:(PDEF)} indica in quale misura il processo si appoggia agli standard; 
    \item \textbf{Process distribution:(PDIS)} indica in quale misura il processo standard viene effettivamente rilasciato e distribuito come un processo definito in grado di raggiungere sempre gli stessi risultati;
    \item \textbf{Process measurement:(PMS)} indica il grado in cui i risultati delle misure sono utilizzati per garantire che il processo raggiunga i suoi obiettivi;
    \item \textbf{Process control:(PC)} indica in quale misura il processo risulta stabile, capace e predicibile (entro certo limiti);
    \item \textbf{Process change:(PCH)} indica in quale misura le modifiche da apportare al processo sono identificate grazie ad una fase di analisi delle performance e allo studio di approcci innovativi;
    \item \textbf{Process improvement:(PI)} indica in quale misura i cambiamenti all'organizzazione, alle performance e alla definizione del processo hanno un impatto effettivo che porta a raggiungere importanti obiettivi di miglioramento al processo.
\end{itemize}

\subsection{Classificazione dei processi}

Gli attributi vengono misurati e classificati secondo uno dei seguenti livelli:

\begin{itemize}
    \item \textbf{N - not implemented:} il processo non possiede l'attributo o dimostra gravi carenze in merito;
    \item \textbf{P - partially implemented:} esiste un approccio sistematico volto al possesso di un attributo già parzialmente ottenuto, ma alcuni aspetti non sono ancora prevedibili;
    \item \textbf{L - largely implemented:} esiste un approccio sistematico volto al possesso di un attributo già significativamente ottenuto, ma l'attuazione varia nelle diverse unità;
    \item \textbf{F - fully implemented:} l'attributo è stato completamente conseguito grazie ad un approccio sistematico e l'attuazione è uguale in tutte le unità.
\end{itemize}

\noindent Secondo la classificazione degli attributi, ad un processo viene assegnato uno dei seguenti livelli di capacità:

\begin{itemize}
    \item \textbf{Incomplete:} il processo è incompleto in quanto non è stato implementato, o fallisce nel raggiungere il proprio obiettivo. Questo livello non ha alcun attributo associato;
    \item \textbf{Performed:} il processo è stato implementato e ha successo nel raggiungere il proprio obiettivo. L'attributo associato a questo livello è \textit{process performance};
    \item \textbf{Managed:} il processo, che già apparteneva al livello \textit{performed}, è implementato in maniera organizzata tramite pianificazione, controllo e correzione; i suoi prodotti sono sicuri. Gli attributi associati a questo livello sono \textit{performance management} e \textit{work product management};
    \item \textbf{Established:} il processo, che già apparteneva al livello \textit{managed}, è stato implementato come processo definito in grado di raggiungere sempre gli stessi risultati. Gli attributi associati a questo livello sono \textit{process definition} e \textit{process distribution};
    \item \textbf{Predictable:} il processo, che già apparteneva al livello \textit{established}, opera entro limiti definiti per raggiungere i propri risultati. Gli attributi associati a questo livello sono \textit{process control} e \textit{process measurement};
    \item \textbf{Optimizing:} il processo, che già apparteneva al livello \textit{predictable}, è oggetto di miglioramento continuo per raggiungere gli obiettivi di progetto. Gli attributi associati a questo livello sono \textit{process change} e \textit{process improvement}.
\end{itemize}

% ISO/IEC 9126
    
\section{Qualità di prodotto - ISO/IEC 9126}    

\subsection{Introduzione allo standard}

La sigla ISO/IEC 9126 individua una serie di normative e linee guida preposte a descrivere un modello di qualità del software. Nello specifico, esso definisce un modello (costituito da metriche qualitative che possono essere misurate in termini quantitativi) per:

\begin{itemize}
    \item \textbf{Qualità interna:} la qualità interna definisce metriche applicabili al codice sorgente utili a rilevarvi problemi che ne possano inficiare la qualità prima che il software venga eseguito. Essa viene rilevata tramite analisi statica e, idealmente, determina la qualità esterna;
    
    \item \textbf{Qualità esterna:} la qualità esterna definisce metriche applicabili al software in esecuzione utili a valutarne i comportamenti tramite test, rispetto agli obiettivi stabiliti. Essa viene rilevata tramite analisi dinamica e, idealmente, determina la qualità in uso;
    
    \item \textbf{Qualità in uso} la qualità in uso definisce metriche applicabili al solo prodotto finito e calato in reali condizioni di utilizzo.
\end{itemize}

\subsection{Modello della qualità interna e esterna del software}

\begin{itemize}
    \item \textbf{Funzionalità:} il software è tenuto a fornire funzionalità atte a soddisfare i bisogni evidenziati nell'\textit{Analisi dei Requisiti}, e che permettano di operare nel \glossario{dominio applicativo}{dominio applicativo} desiderato. Nello specifico, esso deve avere le seguenti caratteristiche:

    \begin{itemize}
        \item \textbf{Appropriatezza:} ovvero la capacità di fornire funzionalità appropriate in relazione ad attività specifiche, e che permettano di raggiungere gli obiettivi fissati;
        
        \item \textbf{Accuratezza:} ovvero la capacità di fornire risultati corretti con la precisione richiesta;
        
        \item \textbf{Interoperabilità:} ovvero la capacità di interagire con dati sistemi;
        
        \item \textbf{Sicurezza:} ovvero la capacità di proteggere informazioni e dati.
    \end{itemize}
    
    \item \textbf{Affidabilità:} il software è tenuto a mantenere un livello di prestazioni quando utilizzato in condizioni date situazioni critiche. Nello specifico, esso deve avere le seguenti caratteristiche:

    \begin{itemize}
        \item \textbf{Maturità:} ovvero la capacità di evitare errori durante l'esecuzione;
        
        \item \textbf{Robustezza:} ovvero la capacità di mantenere uno stato funzionante anche in caso di errori;
        
        \item \textbf{Recuperabilità:} ovvero la capacità di ripristinare prestazioni e dati in caso di errori o malfunzionamenti.
    \end{itemize}

    \item \textbf{Efficienza:} il software è tenuto a eseguire le proprie funzionalità minimizzando tempo, spazio e tutte le altre risorse di cui necessita per il suo corretto funzionamento;
    
    \item \textbf{Usabilità:}  il software è tenuto ad essere comprensibile, studiabile e pienamente utilizzabile dal suo \glossario{target}{Target} di utenza. Nello specifico, esso deve avere le seguenti caratteristiche:
    
    \begin{itemize}
        \item \textbf{Comprensibilità:} ovvero la capacità di essere inequivocabilmente chiaro rispetto alle proprie funzionalità e modalità di utilizzo;
        \item \textbf{Apprendibilità:} ovvero la capacità di rendere palesi, studiabili e dunque apprendili le proprie applicazioni;
        \item \textbf{Operabilità:} ovvero la capacità di essere pienamente utilizzabile e sotto il controllo dell'utente;
        \item \textbf{Attrattiva:} ovvero la capacità di risultare interessante, utile e attraente nei confronti dell'utente.
    \end{itemize}
    
    \item \textbf{Manutenibilità:} il software deve essere in grado di evolvere sulla base di a modifiche, correzioni e adattamenti. Nello specifico, esso deve avere le seguenti caratteristiche:
    
    \begin{itemize}
        \item \textbf{Analizzabilità:} ovvero la capacità di essere analizzato agevolmente al fine di individuarne errori;
        \item \textbf{Modificabilità:} ovvero la capacità di essere modificato agevolmente a livello di codice, progettazione o documentazione;
        \item \textbf{Stabilità:} ovvero la capacità di evitare effetti indesiderati in seguito ad un modifica;
        \item \textbf{Testabilità:} ovvero la capacità di poter essere agevolmente verificato e validato.
    \end{itemize}
    
    \item \textbf{Portabilità:} il software deve poter essere trasportato da un ambiente hardware o software ad un altro, seguendo le evoluzioni tecnologiche. Nello specifico, esso deve avere le seguenti caratteristiche:
    
    \begin{itemize}
        \item \textbf{Adattabilità:} ovvero la capacità di adattarsi a differenti ambienti senza la necessità di azioni specifiche;
        \item \textbf{Installabilità:} ovvero la capacità di essere installato in un dato ambiente;
        \item \textbf{Conformità:} ovvero la capacità di coesistere con altre applicazioni e di condividere efficientemente le risorse;
        \item \textbf{Sostituibilità:} ovvero la capacità di sostituire un altro software, che abbia lo stesso scopo, nello stesso ambiente.
    \end{itemize}

\end{itemize}

\subsection{Modello della qualità in uso del software}

Il software è tenuto a permettere agli utenti di conseguire obiettivi specifici con:

\begin{itemize}
    \item \textbf{Efficacia:} il software deve effettivamente permettere agli utenti di raggiungere l'obiettivo fissato;
    \item \textbf{Produttività:} il software deve utilizzare in maniera efficiente le risorse a lui necessarie;
    \item \textbf{Soddisfazione:} il software deve soddisfare i bisogni degli utenti;
    \item \textbf{Sicurezza:} il software deve detenere livelli di rischio accettabili rispetto a danni nei confronti di persone, apparecchiature e ambiente operativo.
\end{itemize}

% CICLO DI DEMING

\section{Ciclo di Deming}

Il ciclo di Deming (anche conosciuto come ciclo PDCA, l'acronimo di Plan-Do-Check-Act) è un metodo iterativo utilizzato per il controllo dei processi finalizzato al miglioramento continuo della loro qualità e, conseguentemente, della qualità dei prodotti. Ogni iterazione del ciclo consiste di quattro fasi:

\begin{enumerate}
    \item \textbf{Plan:} la fase di pianificazione degli obiettivi di miglioramento. Qui vengono definite le attività da svolgere, le risorse da assegnarvi e le scadenze utili allo scopo di raggiungere tali obiettivi;
    \item \textbf{Do:} la fase in cui ciò che è stato precedentemente pianificato viene messo in atto;
    \item \textbf{Check:} la fase di verifica in cui si accerta che la fase \textit{Do} sia stata eseguita rispettando la fase \textit{Plan} e che abbia ottenuto esiti positivi secondo date metriche;
    \item \textbf{Act:} la fase di attuazione, in cui i processi che hanno beneficiato delle correzioni e delle modifiche eseguite vengono resi standard.
\end{enumerate}

% SPECIFICA DEI TEST

\chapter{Specifica dei test}

% RESOCONTO DELLE ATTIVITà DI VERIFICA

\chapter{Resoconto delle attività \\ di verifica di periodo}

\section{Introduzione}

Nel periodo precedente alla consegna per una revisione vengono verificati i documenti redatti ed i processi eseguiti. I documenti sono verificati dai \textit{Verificatori} secondo i criteri per l'analisi statica definiti nel documento \textit{Norme di Progetto v1.0.0}, applicando il sistema \glossario{Walkthrough}{Walkthrough} ed \glossario{Inspection}{Inspection}. In primo luogo, viene verificato il documento nella sua interezza, cercando eventuali errori presenti e trattandoli nel modo seguente:

\begin{enumerate}
	\item Correzione di errori grammaticali o di eventuali violazioni delle norme tipografiche definite nelle \textit{Norme di Progetto v 1.0.0};
	\item  Segnalazione ed aggiunta alla lista di controllo degli errori più frequenti;
	\item Applicazione del ciclo PDCA allo scopo di migliorare e velocizzare le future verifiche.
\end{enumerate}

\noindent In secondo luogo, viene applicato il metodo Inspection mediante l'uso della lista di controllo stilata sulla base dei documenti già sottoposti a verifica, con particolare enfasi sugli errori più comuni.

\noindent Il tracciamento dei requisiti viene effettuato tramite il software \glossario{SWEgo}{SWEgo} e successivamente controllato manualmente per assicurarne la correttezza.
Vengono infine controllati prodotti software e documentali e relativi processi ponendo attenzione sul rispetto delle metriche proposte in questo documento.

\section{Periodo di analisi}

\subsection{Processi}

Essendo l'Analisi il primo periodo di progetto, prima di essa non esistevano processi all'interno del gruppo e dunque essi si collocavano ad un livello iniziale 0 secondo lo standard SPICE. Attuando tuttavia una valutazione retrospettiva, si nota come l'introduzione delle \textit{Norme di Progetto v1.0.0} abbiano portato al miglioramento di seguito illustrato:

\begin{table}[h]
	\begin{center}
		\setlength\LTleft{-22mm}
		\begin{longtable}{|p{35mm}|p{20mm}|p{20mm}|p{20mm}|p{20mm}|p{20mm}|p{20mm}|}
			\hline
			\textbf{Nome Processo} & \textbf{Attr. L1} & \textbf{Attr. L2} & \textbf{Attr. L3} & \textbf{Attr. L4} & \textbf{Attr. L5} & \textbf{SPICE}\\
			\hline
			\multirow{2}{*}{\textit{Fornitura}} & \multirow{2}{*}{PP: F} & PM: F & PDEF: F & PMS: P & PCH: N & Inizio: 0\\  
			\cline{3-7}
			&          & WPM: F & PDIS: F & PC: N & PI: N & Fine: 3 \\ 
			\hline
			\multirow{2}{*}{\textit{Sviluppo}} & \multirow{2}{*}{PP: F} & PM: F & PDEF: F & PMS: N & PCH: N & Inizio: 0\\  \cline{3-7}
			&          & WPM: F & PDIS: N & PC: N & PI: N & Fine: 2\\
			\hline\multirow{2}{*}{\textit{Documentazione}} & \multirow{2}{*}{PP: F} & PM: F & PDEF: F & PMS: F & PCH: P & Inizio: 0\\  \cline{3-7}
			&          & WPM: F & PDIS: F & PC: L & PI: N & Fine: 4\\ 
			\hline\multirow{2}{*}{\textit{Versionamento}} & \multirow{2}{*}{PP: F} & PM: F & PDEF: F & PMS: P & PCH: N & Inizio: 0\\  \cline{3-7}
			&          & WPM: F & PDIS: F & PC: P & PI: N & Fine: 3\\ 
			\hline\multirow{2}{*}{\textit{Verifica}} & \multirow{2}{*}{PP: F} & PM: F & PDEF: L & PMS: P & PCH: N & Inizio: 0\\  \cline{3-7}
			&          & WPM: F & PDIS: L & PC: N & PI: N & Fine: 3\\ 
			\hline       
		\end{longtable}
	\end{center}
	\caption{Valori SPICE, periodo di Analisi} 
\end{table}

\subsection{Prodotti}

\subsubsection{Documenti}

Segue riassunto del calcolo dell'Indice Gulpease [MPPD002] (al netto di tabelle e frontespizio) e di quello del numero di Errori ortografici corretti [MPPD001].

\begin{itemize}
	\item \textbf{Errori ortografici corretti [MPPD001]}: tramite le funzionalità di rilevazione d'errori di TexStudio sono stati rilevati e corretti complessivamente 14 errori all'interno dei documenti;
	
	\item \textbf{Indice Gulpease [MPPD002]}: Viene qui riportata una tabella contenente il valore Gulpease calcolato per ciascun documento.
	Per il calcolo di tale indice sono state escluse eventuali tabelle presenti nei documenti, le pagine di frontespizio e i diari delle modifiche, in quanto una loro inclusione avrebbe portato a valori errati. L'esito della misurazione è Positivo, se l'indice è maggiore o uguale a 50, o Negativo nel caso fosse inferiore a tale valore.
	
	\begin{table}[h]
		\begin{center}
			\setlength\LTleft{6mm}
			\begin{longtable}{|p{60mm}|p{30mm}|p{25mm}|}
				\hline  
				\textbf{Nome Documento} & \textbf{Valore Indice} & \textbf{Esito} \\ \hline    
				\textit{Glossario v 1.0.0} & 55 & Positivo\\ \hline    
				\textit{Norme di Progetto v 1.0.0} & 56 & Positivo\\ \hline    
				\textit{Studio di Fattibilità v 1.0.0} & 58 & Positivo\\ \hline    
				\textit{Piano di Progetto v 1.0.0} & 53 & Positivo\\ \hline    
				\textit{Analisi dei Requisiti v 1.0.0} & 55 & Positivo\\ \hline    
				\textit{Piano di Qualifica v 1.0.0} & 55 & Positivo\\ \hline    
				\textit{Verbale Interno 10-11-2017} & 55 & Positivo\\ \hline    
				\textit{Verbale Interno 1-12-2017} & 53 & Positivo\\ \hline    
				\textit{Verbale Esterno 15-12-2017} & 57 & Positivo\\ \hline    
				\textit{Verbale Esterno 3-01-2018} & 51 & Positivo\\ \hline   
				\textit{Lettera di Presentazione} & 80 & Positivo\\ \hline
			\end{longtable}
		\end{center}
		\caption{Valori Indice Gulpease, periodo di Analisi} 
	\end{table} 

Dalla tabella si evince che tutti i documenti presentano un indice nei limiti preferibili.

\end{itemize}

\chapter{Valutazioni per il miglioramento}

\section{Introduzione}

Questa appendice si propone di riepilogare le valutazioni orientate al miglioramento dell'intero processo produttivo in relazione al progetto corrente. Verranno dunque tracciati problemi riguardanti i seguenti ambiti:

\begin{itemize}
	\item \textbf{Organizzazione:} ovvero quei problemi inerenti l'organizzazione e la comunicazione all'interno del gruppo;
	\item \textbf{Ruoli:} ovvero quei problemi riguardanti il corretto svolgimento di un ruolo di progetto;
	\item \textbf{Strumenti:} ovvero quei problemi riguardanti il corretto utilizzo di strumentazione specifica.
\end{itemize}

\noindent Ogni problema viene sollevato sulla base dell'autovalutazione dei membri del gruppo e dall'esito di revisioni e confronti con Committente e Proponente, e ad esso viene associata una possibile soluzione.

\section{Valutazioni sull'organizzazione}

\section{Valutazioni sui ruoli}

\section{Valutazioni sugli strumenti}

\end{document}
