\documentclass[openany,12pt,a4paper]{report}
\usepackage{subfiles}
\usepackage[]{graphicx}
\usepackage{float}
\usepackage{multirow}
\graphicspath{{./img/}{./../img/}}
\usepackage{../StileDoc}
\title{Manuale Utente}
\author{}

\loadglsentries{../Glossario/Definizioni}
%Ultima versione documento

\newcommand{\versione}{0.0.0}


\begin{document}
	\makeatletter
	\begin{titlepage}
		\setlength{\headsep}{0pt}  
		\begin{center}
			\includegraphics[width=0.5\linewidth]{img/logo.png}\\[1em]
			{\huge \bfseries  \@title }\\[10ex]
			\textbf{\Large Informazioni Documento} \\[2em]
			\bgroup
			\def\arraystretch{1.5}
			\begin{tabular}{l|l}
				\textbf{Versione} & \versione{} \\
				\textbf{Data approvazione} & 10 Marzo 2018 \\
				\textbf{Responsabile} & Marco Focchiatti\\
				\textbf{Redattori} &  Manfredi Smaniotto, Marco Focchiatti,\\
				& Cristiano Tessarolo, Giulio Rossetti, \\
				& Kevin Silvestri \\
				\textbf{Verificatori} & Manfredi Smaniotto, Marco Focchiatti \\
				\textbf{Distribuzione} & Prof. Tullio Vardanega \\
				& Prof. Riccardo Cardin \\
				& Gruppo Graphite \\
				\textbf{Uso} & Esterno \\
				\textbf{Recapito} & graphite.swe@gmail.com \\
			\end{tabular}
			\egroup
		\end{center}
	\end{titlepage}
	\makeatother
	
	\thispagestyle{empty}
	\newpage
	
	%REGISTRO DELLE MODIFICHE
	
	\chapter*{Registro delle modifiche}
	\setlength\LTleft{-22mm}
	\begin{longtable}{|p{20mm}|p{20mm}|p{40mm}|p{30mm}|p{50mm}|}
		\hline
		\textbf{Versione} & \textbf{Data} & \textbf{Autore} & \textbf{Ruolo} & \textbf{Descrizione} \\
		
		\hline 2.0.0 & 10-03-2018 & Marco Focchiatti & Responsabile & Approvazione \\
		\hline 1.2.0 & 09-03-2018 & Giulio Rossetti & Verificatore & Verifica \\
		\hline 1.1.1 & 08-03-2018 & Kevin Silvestri & Verificatore & Stesura appendice §C.3 e §D  \\
		\hline 1.1.0 & 25-02-2018 & Marco Focchiatti & Verificatore & Verifica \\
		\hline 1.0.6 & 23-02-2018 & Manfredi Smaniotto & Verificatore & Stesura appendice §B \\
		\hline 1.0.5 & 22-02-2018 & Manfredi Smaniotto & Verificatore & Spostamento appendice §B in appendice §C \\
		\hline 1.0.4 & 16-02-2018 & Giulio Rossetti & Verificatore & Rivisto e modificato §3 \\
		\hline 1.0.3 & 13-02-2018 & Cristiano Tessarolo & Verificatore & Rivisti obiettivi di qualità (§2.2) e aggiunta politica della qualità (§2.4) \\
		\hline 1.0.2 & 11-02-2018 & Kevin Silvestri & Verificatore & Spostate definizioni metriche (§3) in NP\\
		\hline 1.0.1 & 09-02-2018 & Marco Focchiatti & Verificatore & Rivista struttura generale e ampliata §2 \\
		\hline 1.0.0 & 12-01-2018 & Samuele Modena & Responsabile & Approvazione \\
		\hline 0.2.0 & 11-01-2018 & Giulio Rossetti & Verificatore & Verifica \\
		\hline 0.1.2 & 10-01-2018 & Samuele Modena & Verificatore & Stesura appendice §B \\
		\hline 0.1.1 & 20-12-2017 & Matteo Rizzo & Verificatore & Aggiornata §3 \\
		\hline 0.1.0 & 19-12-2017 & Manfredi Smaniotto & Verificatore & Verifica \\		
		\hline 0.0.6 & 18-12-2017 & Kevin Silvestri & Verificatore & Stesura appendice §A \\
		\hline 0.0.5 & 17-12-2017 & Kevin Silvestri & Verificatore & Stesura §4 \\	
		\hline 0.0.4 & 15-12-2017 & Matteo Rizzo & Verificatore & Stesura §3 \\
		\hline 0.0.3 & 14-12-2017 & Samuele Modena & Verificatore & Stesura §2 \\
		\hline 0.0.2 & 13-12-2017 & Matteo Rizzo & Verificatore & Stesura §1 \\
		\hline 0.0.1 & 13-12-2017 & Matteo Rizzo & Verificatore & Creazione del template \\
		\hline
		
	\end{longtable}
	
	
	% INDICE
	\tableofcontents
	
	% INTRODUZIONE
	
	\chapter{Introduzione}
	
	\section{Scopo del documento}
	
	Il documento ha la finalità di spiegare le funzionalità e le modalità di utilizzo dell’applicazione
	\textit{"DeSpeect: un'interfaccia grafica per Speect"}. Nonostante la versione attuale rappresenti una prima bozza del documento, esso rappresenterà sia una guida che un riferimento completo per l’utilizzo del prodotto.
	
	\section{Scopo del prodotto}
	
	Lo scopo del progetto è la realizzazione di un’interfaccia grafica per \glossario{Speect}{Speect} [Meraka Institute(2008-2013)], una libreria per la creazione di sistemi di sintesi vocale, che agevoli l’ispezione del suo stato interno durante il funzionamento e la scrittura di test per le sue funzionalità.
	
	\section{Informazioni utili}
	
	La stesura di questo documento assume come utente target del prodotto un programmatore esperto nell'utilizzo di \textit{Speect} e dei linguaggi di programmazione C e C++.
	Per completezza, viene riportato in appendice §A un glossario che comprensivo di termini tecnici o riguardanti particolari funzionalità di \textit{DeSpeect}. Per identificare i termini
	presenti nel glossario, la loro prima occorrenza all’interno del documento è riportata in corsivo e
	marcata con una G al pedice.
	
	\section{Riferimenti}
	
	\subsection*{Riferimenti normativi}
	
	\begin{itemize}
		
		\item \textbf{Norme di progetto:} documento \textit{Norme di progetto v3.0.0}.
		\begin{itemize}
			\item §2.2 "Sviluppo";
			\item §3.1 "Documentazione";
			\item §4.7.3 "Strumenti di sviluppo".
		\end{itemize}
		
		\item\textbf{Capitolato d'appalto C3:} "DeSpeect: un'interfaccia grafica per Speect" \\ \url{http://www.math.unipd.it/~tullio/IS-1/2017/Progetto/C3.pdf}
		\subitem Capitolato d'appalto per il progetto \textit{DeSpeect: un'interfaccia grafica per Speect}.
		
	\end{itemize}
	
	\subsection*{Riferimenti informativi}
	
	\begin{itemize}
		\item \textbf{Documentazione Speect:} \\
		\url{http://speect.sourceforge.net/contents.html};
		\subitem Documentazione ufficiale della libreria di \textit{Text-To-Speech} di riferimento per il progetto.
		
		\item \textbf{Documentazione Qt:} \\
		\url{http://doc.qt.io/};
		\subitem Documentazione ufficiale del framework utilizzato per lo sviluppo dell'interfaccia grafica.
		
		\item \textit{Documentazione CMAKE:} \\
		\url{https://cmake.org/documentation/}.
		\subitem Documentazione ufficiale del framework utilizzato per la build del prodotto. 
	\end{itemize}

	\chapter{Requisiti di sistema}
	
	\chapter{Manuale d'uso}
	
	\section{Accesso all'applicazione}
	
	\section{L'interfaccia grafica}
	
	\subsection{Struttura dell'interfaccia grafica}
	
	\subsection{Visualizzare il manuale utente}
	
	\subsection{Uscire dall'applicazione}
	
	\section{Interagire con la voice}
	
	\subsection{Caricare la voice}
	
	\subsection{Generare l'audio relativo alla voice}
	
	\subsection{Salvare l'audio relativo alla voice}
	
	\section{Stampare il grafo}
	
	\subsection{Importare il grafo}
	
	\subsection{Selezionare gli utterance processors}
	
	\subsection{Visualizzare il grafo}
	
	\subsubsection{Visualizzare il grafo step-by-step}
	
	\subsubsection{Visualizzare l'intero grafo}
	
	\section{Interagire con il grafo}
	
	\subsection{Esportare il grafo generato}
	
	\subsection{Traslare elementi grafici}
	
	\subsubsection{Traslare nodi}
	
	\subsubsection{Traslare archi}
	
	\subsection{Interagire con le relation}
	
	\chapter{Risoluzione dei problemi}
	
	\section{Errori in DeSpeect}
	
	\subsection{Struttura dei codici di errore}
	
	\subsection{Log degli errori}
	
	\section{Problemi con il reperimento di Speect}
	
	\section{Segnalazione di bug}
	
	\textit{DeSpeect} potrebbe contenere bug o potrebbe essere desiderabile apportare modifiche e ampliamenti alle sue funzionalità. \\ È possibile segnalare malfunzionamenti o richieste di nuove funzionalità sotto forma di GitHub issue all’indirizzo:
	\begin{center}
		\url{https://github.com/graphiteSWE/DeSpeect}
	\end{center}
  oppure scrivendo direttamente all'indirizzo e-mail:
  \begin{center}
  	\url{graphite.swe@gmail.com}
  \end{center}
	
	\appendix
	
	\chapter{Glossario}
	
\end{document}
