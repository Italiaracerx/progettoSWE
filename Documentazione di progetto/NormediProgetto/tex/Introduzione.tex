\documentclass[./NormediProgetto.tex]{subfiles}

\begin{document}

% NORME DI PROGETTO --> INTRODUZIONE
	
\chapter{Introduzione}

\section{Scopo del Documento}

Questo documento definisce le norme che i membri del gruppo Graphite seguiranno durante lo svolgimento del progetto per assicurare un modo di lavorare comune che garantisca una collaborazione efficiente tra tutti i diversi
membri. Nello specifico, esso definisce:

\begin{itemize}
	\item Modalità di lavoro durante le fasi di progetto;
	\item Convenzioni per la stesura di documenti;
	\item Strumenti utilizzati dai membri del gruppo.
\end{itemize}
Si fa notare la natura incrementale del documento e come quindi esso non definisca una versione finale.
\section{Scopo del Prodotto}

Lo scopo del \glossario{\textit{prodotto}}{prodotto} è quello di fornire un \glossario{\textit{frontend grafico}}{frontend grafico} utilizzabile come strumento di supporto allo sviluppo di \glossario{\textit{plugin}}{plugin} sulla piattaforma \glossario{Speect}{Speect}. 
\\ \noindent Lo strumento darà la possibilità all'utente di salvare i grafi generati a schermo dall'applicazione.
\\ \noindent Il funzionamento dell'applicazione sarà garantito su un terminale \glossario{\textit{Linux Ubuntu}}{Linux Ubuntu} versione 16.04.

\section{Glossario}

Al fine di evitare ogni ambiguità linguistica e di massimizzare la comprensibilità dei documenti, i termini tecnici e di dominio, gli acronimi e le parole che necessitano di essere chiarite, sono riportati nel documento \textit{Glossario v3.0.0}.
Ogni termine presente nel glossario è marcato da una "G" maiuscola in pedice.

\section{Riferimenti}

\subsection*{Normativi}

\begin{itemize}
	\item \textit{Standard ISO/IEC 12207:1995:}\\
	 \url{http://www.math.unipd.it/~tullio/IS-1/2009/Approfondimenti/ISO_12207-1995.pdf}.
	 \subitem Definisce la struttura del presente documento.
\end{itemize}

\subsection*{Informativi}

\begin{itemize}
	\item \textbf{Piano di Progetto v3.0.0}: documento \textit{Piano di Progetto v3.0.0};
		\subitem Definisce nel dettaglio ruoli e periodi.
	\item \textbf{Piano di Qualifica v3.0.0}: documento \textit{Piano di Qualifica v3.0.0};
		\subitem Definisce nel dettaglio obiettivi e strategie di qualità.
	\item \textbf{Amministrazione di progetto - Slide del corso "Ingegneria del Software":}\\	\url{http://www.math.unipd.it/~tullio/IS-1/2017/Dispense/L07.pdf};
		\subitem Introduzione all'amministrazione di progetto.
	\item \textbf{Guide to the Software Engineering Body of Knowledge (SWEBOK), 2004:}\\
	\url{http://www.math.unipd.it/~tullio/IS-1/2007/Approfondimenti/SWEBOK.pdf}; 
		\begin{itemize}
			\item §7 "Software configuration management";
			\item §8 "Software Engineering management";
			\item §9 "Software Engineering process";
			\item §10 "Software Engineering Tools and methods".
		\end{itemize}
	\item \textbf{Software Engineering - Ian Sommerville - 9 th Edition (2010):}
		\subitem \textit{(Disponibile nel solo formato cartaceo)};
		\subitem §25 "Configuration management".
	\item \textbf{GNU GCC Coding Conventions:}\\ \url{https://gcc.gnu.org/codingconventions.html}.
		\subitem Definizione delle norme di codifica.
\end{itemize}

\end{document}
