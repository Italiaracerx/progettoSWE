\documentclass[./PianodiProgetto.tex]{subfiles}

\begin{document}

\chapter{Consuntivo di periodo e \\ preventivo a finire}
In questa sezione verranno presentati i consuntivi dei vari periodi con una
breve valutazione degli stessi. Verrà inoltre presentato un preventivo a finire
che terrà conto dei soli periodi rendicontati. I valori presentati saranno:
\begin{itemize}
\item \textbf{Positivi:} se il preventivo è superiore ai valori del consuntivo;
\item \textbf{Negativi:} se il preventivo è inferiore ai valori del consuntivo.
\end{itemize}

\section{Periodo di analisi}
Essendo il periodo di Analisi considerato periodo di investimento, il consuntivo viene presentato a scopo informativo ma non conteggiato nel preventivo a finire.

\subsection{Consuntivo di periodo}
Di seguito è presentata la tabella contenente i dati del consuntivo per il
periodo di Analisi.

\begin{table}[H]
	\centering
	\begin{tabular}{|c|c|c|c|c|}
		\hline
	 	 & \multicolumn{2}{c|}{Ore} & \multicolumn{2}{c|}{Costo in \euro{}}  \\ \hline
		Ruolo&Preventivo&Consuntivo&Preventivo&Consuntivo \\ \hline
		Responsabile&24&24&720,00&720,00  \\ \hline
		Amministratore&20&19 (+1)&400,00&380,00 (+20,00)  \\ \hline
		Analista&61&67 (-6)&1525,00&1675,00 (-150,00)  \\ \hline
		Progettista& & & &  \\ \hline
		Programmatore& & & &  \\ \hline
		Verificatore&46&46&690,00&690,00  \\ \hline
		Totale&151&156&3335,00&3465,00  \\ \hline
		Differenza& \multicolumn{2}{c|}{-5 Ore} & \multicolumn{2}{c|}{-130,00 \euro{}} \\ \hline
	\end{tabular}
	\caption{Prospetto orario ed economico a consuntivo del periodo di Analisi}
\end{table}

\begin{figure}[H]
	\centering
	\includegraphics[width=1\linewidth]{img/grafici/OreConsuntivo/consuntivo-ore-analisi}
	\caption{Confronto grafico delle ore nei consuntivi a seguito del periodo di Analisi}
	\label{fig:consuntivo-ore-analisi}
\end{figure}

\begin{figure}[H]
	\centering
	\includegraphics[width=1\linewidth]{img/grafici/CostiConsuntivi/consuntivo-costo-analisi}
	\caption{Confronto grafico dei costi nei consuntivi a seguito del periodo di Analisi}
	\label{fig:consuntivo-costi-analisi}
\end{figure}

\subsection{Variazioni della pianificazione}
Nell'esecuzione del primo periodo di Analisi è stato necessario usare più
ore del previsto per il ruolo di \textit{Analista} mentre si è riusciti a risparmiare nel ruolo di \textit{Amministratore}. Questo è dovuto
probabilmente ad una sottostima del carico di lavoro presentato dalla \textit{Analisi
dei Requisiti}. Le ore di verifica invece, si sono dimostrate sufficienti a svolgere
le attività previste. Il risultato del periodo è complessivamente di cinque ore
lavorative oltre il previsto, con una spesa aggiuntiva di 130,00 \euro{}.

\subsection{Preventivo a finire}
Viene qui presentata una tabella contenente l'attuale preventivo a finire.
Vengono inseriti i valori del periodo di Analisi e Consolidamento dei requisiti a scopo riassuntivo, tuttavia essi non verranno conteggiati nel calcolo delle ore rendicontate. Se il valore del consuntivo non fosse ancora presente, verrà usato il valore del preventivo.

\begin{table}[H]
	\centering
	\begin{tabular}{|c|c|c|}
		\hline
		Periodo&Preventivo \euro{}&Consuntivo \euro{} \\ \hline
		Analisi&3335,00&3465,00  \\ \hline
		Consolidamento dei requisiti&1010,00&Non presente  \\ \hline
		\multicolumn{3}{|c|}{Rendicontato}  \\ \hline
		Consolidamento delle tecnologie&4229,00&Non presente  \\ \hline
		Progettazione e codifica&6684,00&Non presente  \\ \hline
		Validazione e collaudo&2504,00&Non presente  \\ \hline
		&Preventivo \euro{}&Preventivo a finire \euro{}  \\ \hline
		Totale&17762,00&17892,00 \\ \hline
		Rendicontato&13417,00&13417,00 \\ \hline
	\end{tabular}
	\caption{Preventivo a finire - periodo di Analisi}
\end{table}

\section{Periodo di consolidamento dei requisiti}
Essendo il periodo di Consolidamento dei requisiti considerato periodo di investimento, il consuntivo viene presentato a scopo informativo ma non conteggiato nel preventivo a finire.

\subsection{Consuntivo di periodo}
Di seguito è presentata la tabella contenente i dati del consuntivo per il
periodo di Consolidamento dei requisiti.

\begin{table}[H]
	\centering
	\begin{tabular}{|c|c|c|c|c|}
		\hline
		& \multicolumn{2}{c|}{Ore} & \multicolumn{2}{c|}{Costo in \euro{}}  \\ \hline
		Ruolo&Preventivo&Consuntivo&Preventivo&Consuntivo \\ \hline
		Responsabile&5&5&150,00&150,00  \\ \hline
		Amministratore&8&8&160,00&160,00  \\ \hline
		Analista&19&19&475,00&475,00  \\ \hline
		Progettista& & & &  \\ \hline
		Programmatore& & & &  \\ \hline
		Verificatore&15&15&225,00&225,00  \\ \hline
		Totale&47&47&1010,00&1010,00  \\ \hline
		Differenza& \multicolumn{2}{c|}{0 Ore} & \multicolumn{2}{c|}{0,00 \euro{}} \\ \hline
	\end{tabular}
	\caption{Prospetto orario ed economico a consuntivo del periodo di Consolidamento dei requisiti}
\end{table}
\begin{figure}[H]
	\centering
	\includegraphics[width=1\linewidth]{img/grafici/OreConsuntivo/consuntivo-ore-consolidamento_requisiti}
	\caption{Confronto grafico delle ore nei consuntivi a seguito del periodo di Consolidamento dei requisiti}
	\label{fig:consuntivo-ore-consolidamento_requisiti}
\end{figure}

\begin{figure}[H]
	\centering
	\includegraphics[width=1\linewidth]{img/grafici/CostiConsuntivi/consuntivo-costo-consolidamento_requisiti}
	\caption{Confronto grafico dei costi nei consuntivi a seguito del periodo di Consolidamento dei requisiti}
	\label{fig:consuntivo-costi-consolidamento_requisiti}
\end{figure}
\subsection{Variazioni della pianificazione}
Nel periodo di Consolidamento dei requisiti il gruppo ha concluso le operazioni rispettando il prospetto orario. Le ore preventivate sono state rispettate per tutti i ruoli, questo ha portato ad un pareggio tra consuntivo e preventivo, come mostrato in tabella.

\subsection{Preventivo a finire}
Viene qui presentata una tabella contenente l'attuale preventivo a finire.
Vengono inseriti i valori del periodo di Analisi e Consolidamento dei requisiti a scopo riassuntivo, tuttavia essi non verranno conteggiati nel calcolo delle ore rendicontate. Se il valore del consuntivo non fosse ancora presente, verrà usato il valore del preventivo.

\begin{table}[H]
	\centering
	\begin{tabular}{|c|c|c|}
		\hline
		Periodo&Preventivo \euro{}&Consuntivo \euro{} \\ \hline
		Analisi&3335,00&3465,00  \\ \hline
		Consolidamento dei requisiti&1010,00&1010,00  \\ \hline
		\multicolumn{3}{|c|}{Rendicontato}  \\ \hline
		Consolidamento delle tecnologie&4229,00&Non presente  \\ \hline
		Progettazione e codifica&6684,00&Non presente  \\ \hline
		Validazione e collaudo&2504,00&Non presente  \\ \hline
		&Preventivo \euro{}&Preventivo a finire \euro{}  \\ \hline
		Totale&17762,00&17892,00 \\ \hline
		Rendicontato&13417,00&13417,00 \\ \hline
	\end{tabular}
	\caption{Preventivo a finire - periodo di Consolidamento dei requisiti}
\end{table}

\section{Periodo di consolidamento delle tecnologie}
\subsection{Consuntivo di periodo}
Di seguito è presentata la tabella contenente i dati del consuntivo per il
periodo di Consolidamento delle tecnologie.

\begin{table}[H]
	\centering                   
       \begin{tabular}{|c|c|c|c|c|}
		\hline
		& \multicolumn{2}{c|}{Ore} & \multicolumn{2}{c|}{Costo in \euro{}}  \\ \hline
		Ruolo&Preventivo&Consuntivo&Preventivo&Consuntivo \\ \hline
		Responsabile&10&10&300,00&300,00  \\ \hline
		Amministratore&10&10&200,00&200,00 \\ \hline
		Analista&20&19 (+1)&500,00&475,00 (+25,00)  \\ \hline
		Progettista&97&96 (+1)&2134,00&2112,00 (+22,00) \\ \hline
		Programmatore&13&15 (-2)&195,00&225,00 (-30,00)  \\ \hline
		Verificatore&60&60&900,00&900,00  \\ \hline
		Totale&210&210&4229,00&4212,00  \\ \hline
		Differenza& \multicolumn{2}{c|}{0 Ore} & \multicolumn{2}{c|}{+17,00 \euro{}} \\ \hline
	\end{tabular}
	\caption{Prospetto orario ed economico a consuntivo del periodo di Consolidamento delle tecnologie}
\end{table}

\begin{figure}[H]
	\centering
	\includegraphics[width=1\linewidth]{img/grafici/OreConsuntivo/consuntivo-ore-consolidamento_tecnologie}
	\caption{Confronto grafico delle ore nei consuntivi a seguito del periodo di Consolidamento dei tecnologie}
	\label{fig:consuntivo-ore-consolidamento_tecnologie}
\end{figure}

\begin{figure}[H]
	\centering
	\includegraphics[width=1\linewidth]{img/grafici/CostiConsuntivi/consuntivo-costo-consolidamento_tecnologie}
	\caption{Confronto grafico dei costi nei consuntivi a seguito del periodo di Consolidamento dei tecnologie}
	\label{fig:consuntivo-costi-consolidamento_tecnologie}
\end{figure}

\subsection{Variazioni della pianificazione}
Nel periodo di Consolidamento delle tecnologie si è presentata la necessità di ridistribuire alcune ore necessarie per la creazione del \textit{Proof of Concept}. E' stata però possibile, tramite la comunicazione con Mivoq S.R.L. una miglior definizione dei requisiti. Questo ha lasciato la possibilità di ridurre le ore destinate all'analisi e alla progettazione piuttosto che ridurre le ore destinate alla verifica. Si è così riusciti a concludere con un consuntivo di periodo in positivo con un risparmio di 17,00 \euro{}.

\subsection{Preventivo a finire}
Viene qui presentata una tabella contenente l'attuale preventivo a finire.
Vengono inseriti i valori del periodo di Analisi e Consolidamento dei requisiti a scopo riassuntivo, tuttavia essi non verranno conteggiati nel calcolo delle ore rendicontate. Se il valore del consuntivo non fosse ancora presente, verrà usato il valore del preventivo.

\begin{table}[H]
	\centering
	\begin{tabular}{|c|c|c|}
		\hline
		Periodo&Preventivo \euro{}&Consuntivo \euro{} \\ \hline
		Analisi&3335,00&3465,00  \\ \hline
		Consolidamento dei requisiti&1010,00&1010,00  \\ \hline
		\multicolumn{3}{|c|}{Rendicontato}  \\ \hline
		Consolidamento delle tecnologie&4229,00&4212,00  \\ \hline
		Progettazione e codifica&6684,00&Non presente  \\ \hline
		Validazione e collaudo&2504,00&Non presente  \\ \hline
		 &Preventivo \euro{}&Preventivo a finire \euro{}  \\ \hline
		Totale&17762,00&17875,00 \\ \hline
		Rendicontato&13417,00&13400,00 \\ \hline
	\end{tabular}
	\caption{Preventivo a finire - periodo di Consolidamento delle tecnologie}
\end{table}

\noindent A seguito dell'ultimo periodo terminato si è verificata una diminuzione del preventivo a finire, che permette di avere del lieve margine utilizzabile nel prossimo periodo nella codifica, tutto ciò rimanendo entro i costi inizialmente preventivati.

\section{Periodo di progettazione e codifica}
\subsection{Consuntivo di periodo}
Di seguito è presentata la tabella contenente i dati del consuntivo per il periodo di Progettazione e codifica.

\begin{table}[H]
	\centering
	\begin{tabular}{|c|c|c|c|c|}
		\hline
		& \multicolumn{2}{c|}{Ore} & \multicolumn{2}{c|}{Costo in \euro{}}  \\ \hline
		Ruolo&Preventivo&Consuntivo&Preventivo&Consuntivo \\ \hline
		Responsabile&14&14&420,00&420,00 \\ \hline
		Amministratore&8&8&160,00&160,00 \\ \hline
		Analista&5&6 (-1)&125,00&150,00 (-25,00)\\ \hline
		Progettista&132&137 (-5)&2904,00&3014,00 (-110,00) \\ \hline
		Programmatore&118&110 (+8)&1770,00&1650,00 (+120,00) \\ \hline
		Verificatore&87&87&1305,00&1305,00 \\ \hline
		Totale&364&362&6684,00&6699,00 \\ \hline
		Differenza& \multicolumn{2}{c|}{+2 Ore} & \multicolumn{2}{c|}{-15,00 \euro{}} \\ \hline
	\end{tabular}
	\caption{Prospetto orario ed economico a consuntivo del periodo di Progettazione e codifica}
\end{table}
\begin{figure}[H]
	\centering
	\includegraphics[width=1\linewidth]{img/grafici/OreConsuntivo/consuntivo-ore-progettazione}
	\caption{Confronto grafico delle ore nei consuntivi a seguito del periodo di Progettazione e codifica}
	\label{fig:consuntivo-ore-progettazione}
\end{figure}

\begin{figure}[H]
	\centering
	\includegraphics[width=1\linewidth]{img/grafici/CostiConsuntivi/consuntivo-costo-progettazione}
	\caption{Confronto grafico dei costi nei consuntivi a seguito del periodo di Progettazione e codifica}
	\label{fig:consuntivo-costi-progettazione}
\end{figure}
\subsection{Variazioni della pianificazione}
Nel periodo di Progettazione e codifica si è presentata la necessità di ridistribuire alcune ore necessarie per la creazione dell'allegato tecnico relativo alla \textit{Product Baseline}, infatti sono servite più ore nel ruolo di \textit{Progettista}. Questo però ha fatto risparmiare ore al \textit{Programmatore} che trovando un'architettura ben fatta ha saputo eseguire il suo lavoro in meno ore rispetto a quelle preventivate. \\
Tramite la comunicazione con Mivoq S.R.L. è stata possibile un'ulteriore definizione dei requisiti che ha comportato però un impiego maggiore dell'\textit{Analista} rispetto a quanto preventivato. \\
Si è così concluso con un consuntivo di periodo in negativo con un dispendio di 15,00 \euro{}. Viste le differenze esigue tra preventivo e consuntivo, queste non hanno impattato in maniera rilevante sulle attività pianificate e non avranno quindi conseguenze sulla pianificazione futura.

\subsection{Preventivo a finire}
Viene qui presentata una tabella contenente l'attuale preventivo a finire.
Vengono inseriti i valori del periodo di Analisi e Consolidamento dei requisiti a scopo riassuntivo, tuttavia essi non verranno conteggiati nel calcolo delle ore rendicontate. Se il valore del consuntivo non fosse ancora presente, verrà usato il valore del preventivo.

\begin{table}[H]
	\centering
	\begin{tabular}{|c|c|c|}
		\hline
		Periodo&Preventivo \euro{}&Consuntivo \euro{} \\ \hline
		Analisi&3335,00&3465,00  \\ \hline
		Consolidamento dei requisiti&1010,00&1010,00  \\ \hline
		\multicolumn{3}{|c|}{Rendicontato}  \\ \hline
		Consolidamento delle tecnologie&4229,00&4212,00  \\ \hline
		Progettazione e codifica&6684,00&6699,00  \\ \hline
		Validazione e collaudo&2504,00&Non presente  \\ \hline
		&Preventivo \euro{}&Preventivo a finire \euro{}  \\ \hline
		Totale&17762,00&17890,00 \\ \hline
		Rendicontato&13417,00&13415,00 \\ \hline
	\end{tabular}
	\caption{Preventivo a finire - periodo di Progettazione e codifica}
\end{table}

\section{Periodo di validazione e collaudo}
\subsection{Consuntivo di periodo}
Di seguito è presentata la tabella contenente i dati del consuntivo per il periodo di Validazione e collaudo.

\begin{table}[H]
	\centering
	\begin{tabular}{|c|c|c|c|c|}
		\hline
		& \multicolumn{2}{c|}{Ore} & \multicolumn{2}{c|}{Costo in \euro{}}  \\ \hline
		Ruolo&Preventivo&Consuntivo&Preventivo&Consuntivo \\ \hline
		Responsabile&12&10 (+2)&360,00&300,00  (+60,00)  \\ \hline
		Amministratore&14&11 (+3)&280,00&220,00 (+60,00)  \\ \hline
		Analista& & 1 (-1) & & 25,00 (-25,00) \\ \hline
		Progettista&22&18 (+4)&484,00&396,00 (+88,00)  \\ \hline
		Programmatore&23&29 (-6)&345,00&435,00 (-90,00)  \\ \hline
		Verificatore&69&75 (-6)&1035,00&1125,00 (-90,00)  \\ \hline
		Totale&140&144&2504,00&2501,00 \\ \hline
		Differenza& \multicolumn{2}{c|}{-4 Ore} & \multicolumn{2}{c|}{+3,00 \euro{}} \\ \hline
	\end{tabular}
	\caption{Prospetto orario ed economico a consuntivo del periodo di Validazione e collaudo}
\end{table}

\subsection{Variazioni della pianificazione}
Nel periodo di Validazione e collaudo si è presentata la necessità di ridistribuire alcune ore verso il \textit{Programmatore} e \textit{Verificatore}, necessarie per la codifica ed implementazione dei test e per la verifica dei documenti. In questo modo è stato possibile svolgere ulteriori controlli per assicurarsi in maniera più esaustiva che il prodotto rispettasse tutte le aspettative, in termini
di requisiti e di qualità, richieste dal proponente.\\
Si è rivelato necessario dedicare un'ora, non prevista, per la correzione degli ultimi errori all'interno della \textit{Analisi dei Requisiti}.\\
Si è invece potuto risparmiare sulle ore del \textit{Responsabile} e dell'\textit{Amministratore} grazie all'esperienza maturata dal gruppo durante il progetto, grazie al quale si è riusciti ad organizzarsi meglio.\\
Si è così concluso con un consuntivo di periodo in positivo con un risparmio di 3,00 \euro{}. In conclusione l'ultimo periodo si è svolto positivamente permettendo di concludere il
lavoro entro i tempi e i costi preventivati inizialmente.


\section{Consuntivo finale}

\subsection{Consuntivo delle ore rendicontate}
Viene qui riportata una tabella contenente le ore a consuntivo contenente le sole ore svolte
nei periodi rendicontati. Viene mostrato il numero di ore effettuate a consuntivo con un
segno + se sono state fatte in eccesso rispetto al preventivo e un segno - se in quantità
inferiore.

\setlength\LTleft{-22mm}
\begin{longtable}{|p{38mm}|p{15mm}|p{15mm}|p{15mm}|p{15mm}|p{15mm}|p{17mm}|p{18mm}|}
		\hline
		Nominativo&Re&Am&An&Pt&Pr&Ve&Ore totali\\ \hline
		Marco Focchiatti&2&5& &33 (+1)&22&41&103 (+1) \\ \hline
		Samuele Modena& &6&13 (+1)&34&22 (-2)&28 (+2)&103 (+1) \\ \hline
		Matteo Rizzo& &7 (-3)&11 (+3)&40 (-2)&22&23 (+3)&103 (+1) \\ \hline
		Giulio Rossetti&8 (-2)& &8 (+3)&38 (-2)&24 (+2)&25&103 (+1) \\ \hline
		Kevin Silvestri&8&2& &36 (+1)&20 (-2)&37 (+2)&103 (+1) \\ \hline
		Manfredi Smaniotto&8&8 (-1)& &35 (+1)&22 (+1)&30&103 (+1) \\ \hline
		Cristiano Tessarolo&8& & &35 (+1)&22 (+1)&38 (-1)&103 (+1) \\  \hline
		Ore totali ruolo&34 (-2)&28 (-4)&32 (+7)&251&154&222 (+6)&721 (+7) \\ \hline
	\caption{Distribuzione ore rendicontate}
\end{longtable}

\begin{figure}[H]
	\centering
	\includegraphics[width=1\linewidth]{img/grafici/OreConsuntivo/consuntivo-ore-verifica}
	\caption{Confronto grafico delle ore nei consuntivi a seguito del periodo di Verifica e collaudo}
	\label{fig:consuntivo-ore-verifica}
\end{figure}


\subsection{Consuntivo economico}
Viene qui presentata una tabella contenente il consuntivo conclusivo del progetto. Vengono inseriti i valori del periodo di Analisi e Consolidamento dei requisiti a scopo riassuntivo,
tuttavia essi non verranno conteggiati nel calcolo delle ore rendicontate.

\begin{table}[H]
	\centering
	\begin{tabular}{|c|c|c|}
		\hline
		Periodo&Preventivo \euro{}&Consuntivo \euro{} \\ \hline
		Analisi&3335,00&3465,00  \\ \hline
		Consolidamento dei requisiti&1010,00&1010,00  \\ \hline
		\multicolumn{3}{|c|}{Rendicontato}  \\ \hline
		Consolidamento delle tecnologie&4229,00&4212,00  \\ \hline
		Progettazione e codifica&6684,00&6699,00  \\ \hline
		Validazione e collaudo&2504,00&2501,00  \\ \hline
		&Preventivo \euro{}&Preventivo a finire \euro{}  \\ \hline
		Totale&17762,00&17887,00 \\ \hline
		Rendicontato&13417,00&13412,00 \\ \hline
	\end{tabular}
	\caption{Consuntivo conclusivo}
\end{table}

\begin{figure}[H]
	\centering
	\includegraphics[width=1\linewidth]{img/grafici/CostiConsuntivi/consuntivo-costo-verifica}
	\caption{Confronto grafico dei costi nei consuntivi a seguito del periodo di Verifica e collaudo}
	\label{fig:consuntivo-costi-verifica}
\end{figure}

\subsection{Conclusione}
Con la conclusione dell'ultimo periodo di Validazione e collaudo si può vedere come il
costo finale si sia mantenuto entro il valore preventivato, nonostante durante i vari periodi
si siano verificate diverse criticità che hanno richiesto lievi correzioni alla distribuzione
delle risorse orarie. Nel complesso la pianificazione è stata rispettata il meglio possibile,
concludendo con una media di 103 ore per persona.
Infine, il costo totale rendicontato risulta di 13412,00 \euro{} e quindi inferiore al preventivo
iniziale, comportando un risparmio complessivo di 5 \euro{}.


\appendix


\appendix

\chapter{Esito della rilevazione dei rischi}

\section{Introduzione}

In questa appendice si intende riportare l'esito della rilevazione dei rischi concretizzatisi durante il progetto. L'incorrere dei rischi viene monitorato costantemente e questa appendice viene incrementata ogni qualvolta ne venga rilevato uno. \\
Nelle sezioni successive per ogni periodo si inserisce una tabella che riporta le seguenti voci:

\begin{itemize}
	\item \textbf{Rischio}: il codice del rischio concretizzatosi durante il periodo del progetto;
	\item \textbf{Riscontro effettivo}: una breve descrizione del modo in cui il rischio si è concretizzato;
	\item \textbf{Contromisure}: le contromisure adottate al fine di mitigare e/o eliminare le conseguenze della succitata concretizzazione del rischio.
\end{itemize}

\section{Analisi}

\setlength\LTleft{-5.5mm}

\begin{longtable}{|p{15mm}|p{60mm}|p{60mm}|}
	\caption{Esito della rilevazione dei rischi - \textit{Analisi}} \\
	\hline \textbf{Rischio} & \textbf{Riscontro effettivo} & \textbf{Contromisure} \\
	
	\hline RT0 & Alcuni membri del gruppo non erano pratici con Git e \LaTeX &  È stato spiegato loro l'utilizzo di queste tecnologie \\
	
	\hline RS0 & Sono stati riscontrati alcuni guasti hardware da parte di alcuni componenti del gruppo & Nel ripristino del funzionamento dei dispositivi è stato possibile riscaricare dalla \textit{repository Github} i file utili per lavorare da dove era stato lasciato il lavoro precedentemente svolto \\
	
	\hline RS1 & Nessuno & Nessuna \\
	
	\hline RG0 & A causa delle festività invernali e/o problemi personali, alcuni membri del gruppo si sono assentati per alcuni giorni & Le eventuali assenze sono state comunicate con largo anticipo, ciò ha permesso di organizzare il lavoro in modo ottimale tenendo conto delle assenze \\
	
	\hline RG1 & Nessuno & Nessuna \\
	
	\hline RR0 & Il gruppo di analisti ha sollevato vari dubbi prima di iniziare il proprio lavoro & È stato organizzato un incontro con il proponente per risolvere prontamente ogni dubbio sollevato \\
	
	\hline RO0 & La sottostima dei tempi necessari ha causato ritardi nel compimento di alcuni task & Si è deciso di allungare il lasso di tempo di slack relativo a ciascun task in maniera proporzionale all'entità dello stesso \\
	
	\hline
\end{longtable}

\section{Consolidamento dei requisiti}

\setlength\LTleft{-5.5mm}

\begin{longtable}{|p{15mm}|p{60mm}|p{60mm}|}
	\caption{Esito della rilevazione dei rischi - \textit{Consolidamento dei requisiti}} \\
	\hline \textbf{Rischio} & \textbf{Riscontro effettivo} & \textbf{Contromisure} \\
	
	\hline RT0 & Nessuno & Nessuna \\
		
	\hline RS0 & Il portale che ospita documentazione e download di Speect è stato offline per due giorni & Una volta ripristinatosi il funzionamento del portale, tutto il materiale reperibile online è stato scaricato in locale e condiviso tra i membri così da tutelarsi nel caso il problema dovesse occorrere nuovamente \\
	
	\hline RS1 & Nessuno & Nessuna \\
	
	\hline RG0 & Nessuno & Nessuna \\
	
	\hline RG1 & Nessuno & Nessuna \\
	
	\hline RR0 & Il gruppo di analisti ha sollevato vari dubbi & \'E stato anche creato un canale Slack con la Proponente affinchè siano risolti dei futuri dubbi nello sviluppo del software \\
	
	\hline RO0 & Nessuno & Nessuna \\
	
	\hline
\end{longtable}

\section{Consolidamento delle tecnologie}

\setlength\LTleft{-5.5mm}

\begin{longtable}{|p{15mm}|p{60mm}|p{60mm}|}
	\caption{Esito della rilevazione dei rischi - \textit{Consolidamento delle tecnologie}} \\
	\hline \textbf{Rischio} & \textbf{Riscontro effettivo} & \textbf{Contromisure} \\
	
	\hline RT0 & Alcuni membri del gruppo non erano pratici con CMAKE e Travis CI &  È stato spiegato loro l'utilizzo di queste tecnologie \\
	
	\hline RS0 & Sono stati riscontrati alcuni guasti hardware da parte di alcuni componenti del gruppo & Nel ripristino del funzionamento dei dispositivi è stato possibile riscaricare dalla \textit{repository Github} i file utili per lavorare da dove era stato lasciato il lavoro precedentemente svolto \\
	
	\hline RS1 & Nessuno & Nessuna \\
	
	\hline RG0 & A causa delle festività pasquali e/o problemi personali, alcuni membri del gruppo si sono assentati per alcuni giorni & Le eventuali assenze sono state comunicate con largo anticipo, ciò ha permesso di organizzare il lavoro in modo ottimale tenendo conto delle assenze \\
	
	\hline RG1 & Nessuno & Nessuna \\
	
	\hline RR0 & Nessuno & Nessuna \\
	
	\hline RO0 & La sottostima dei tempi necessari ha causato ritardi nel compimento di alcuni task & Si è deciso di allungare il lasso di tempo di slack relativo a ciascun task in maniera proporzionale all'entità dello stesso \\
	
	\hline
\end{longtable}

\section{Progettazione e codifica}

\setlength\LTleft{-5.5mm}

\begin{longtable}{|p{15mm}|p{60mm}|p{60mm}|}
	\caption{Esito della rilevazione dei rischi - \textit{Progettazione e codifica}} \\
	\hline \textbf{Rischio} & \textbf{Riscontro effettivo} & \textbf{Contromisure} \\
	
	\hline RT0 & Nessuno & Nessuna \\
	
	\hline RS0 & Nessuno & Nessuna \\
	
	\hline RS1 & Il metodo e le conoscenze necessarie a comporre una architettura software non erano sufficienti per poter iniziare i lavori di progettazione & Ogni elemento del gruppo ha svolto un lavoro approfondito di studio di ogni tipologia di diagramma necessario a descrivere in modo chiaro l'architettura software da comporre per il progetto \\
	
	\hline RG0 & L'attività di colloqui lavorativi svolta da tutti gli elementi del gruppo per la ricerca di un luogo adatto per poter svolgere uno stage ha richiesto numerose ore di tempo & Si è cercato di svolgere tutti i lavori più onerosi nel periodo subito precedente a quello in cui sono stati effettuati i colloqui da parte dei componenti del gruppo \\
	
	\hline RG1 & Nessuno & Nessuna \\
	
	\hline RR0 & La progettazione del software ha fatto emergere dubbi riguardo all'effettiva comprensione di alcuni requisiti & Si è deciso di contattare prontamente il Proponente per dipanare i dubbi sorti \\
	
	\hline RO0 & Nessuno & Nessuna \\
	
	\hline
\end{longtable}

\section{Validazione e collaudo}

\setlength\LTleft{-5.5mm}

\begin{longtable}{|p{15mm}|p{60mm}|p{60mm}|}
	\caption{Esito della rilevazione dei rischi - \textit{Validazione e collaudo}} \\
	\hline \textbf{Rischio} & \textbf{Riscontro effettivo} & \textbf{Contromisure} \\
	
	\hline RT0 & Nessuno & Nessuna \\
	
	\hline RS0 & Nessuno & Nessuna \\
	
	\hline RS1 & Nessuno & Nessuna \\
	
	\hline RG0 & A causa delle festività e/o problemi personali, alcuni membri del gruppo si sono assentati per alcuni giorni & Le eventuali assenze sono state comunicate con largo anticipo, ciò ha permesso di organizzare il lavoro in modo ottimale tenendo conto delle assenze \\
	
	\hline RG1 & Nessuno & Nessuna \\
	
	\hline RR0 & La validazione del software ha fatto emergere dubbi riguardo un possibile raffinamento di alcuni requisiti & Si è deciso di contattare prontamente il Proponente per dipanare i dubbi sorti \\
	
	\hline RO0 & La sottostima dei tempi necessari ha causato ritardi nel compimento di alcuni task & Si è deciso di allungare il lasso di tempo di slack relativo a ciascun task in maniera proporzionale all'entità dello stesso \\
	
	\hline
\end{longtable}

\newpage

\section{Riepilogo}

\setlength\LTleft{40mm}

\begin{longtable}{|c|c|}
	\caption{Riepilogo dei rischi riscontrati durante il progetto} \\
	\hline \textbf{Rischio} & \textbf{Numero riscontri} \\
	
	\hline RT0 & 2 \\
	
	\hline RS0 & 3 \\
	
	\hline RS1 & 1 \\
	
	\hline RG0 & 4 \\
	
	\hline RG1 & 0 \\
	
	\hline RR0 & 4 \\
	
	\hline RO0 & 3 \\
	
	\hline
\end{longtable}

\chapter{Organigramma}

\section{Redazione}
\begin{table}[H]
	\centering
	\begin{tabular}{|c|c|c|}
		\hline
		Nome&Data&Firma \\ \hline
		Matteo Rizzo& 12-12-2017 &\includegraphics[scale=0.5]{img/firme/RizzoMatteo} \\
		\hline
	\end{tabular}
	\caption{Redazione}
\end{table}

\section{Approvazione}
\begin{table}[H]
	\centering
	\begin{tabular}{|c|c|c|}
		\hline
		Nome&Data&Firma \\ \hline
		Matteo Rizzo& 12-12-2017 & \includegraphics[scale=0.5]{img/firme/RizzoMatteo} \\
		\hline
	\end{tabular}
	\caption{Approvazione}
\end{table}

\section{Accettazione dei componenti}
\begin{table}[H]
	\centering
	\begin{tabular}{|c|c|c|}
		\hline
		Nome&Data&Firma \\ \hline
		Marco Focchiatti& 12-12-2017 &\includegraphics[scale=0.5]{img/firme/FocchiattiMarco} \\ \hline
		Samuele Modena& 12-12-2017 &\includegraphics[scale=0.5]{img/firme/ModenaSamuele} \\ \hline
		Matteo Rizzo& 12-12-2017 &\includegraphics[scale=0.5]{img/firme/RizzoMatteo} \\ \hline
		Giulio Rossetti&  12-12-2017 &\includegraphics[scale=0.5]{img/firme/RossettiGiulio} \\ \hline
		Kevin Silvestri& 12-12-2017 &\includegraphics[scale=0.5]{img/firme/SilvestriKevin} \\ \hline
		Manfredi Smaniotto& 12-12-2017 &\includegraphics[scale=0.5]{img/firme/SmaniottoManfredi} \\ \hline
		Cristiano Tessarolo& 12-12-2017 &\includegraphics[scale=0.5]{img/firme/TessaroloCristiano} \\  
		\hline
	\end{tabular}
	\caption{Accettazione dei componenti}
\end{table}

\section{Componenti}
\begin{table}[H]
	\begin{tabular}{|c|c|c|}
	\hline
	Nome&Matricola&Indirizzo email \\ \hline
	Marco Focchiatti&1121294&marco.focchiatti@studenti.unipd.it  \\ \hline
	Samuele Modena&1099080&samuele.modena@studenti.unipd.it \\ \hline
	Matteo Rizzo&1123496&matteo.rizzo.4@studenti.unipd.it \\ \hline
	Giulio Rossetti&1122603&giulio.rossetti@studenti.unipd.it \\ \hline
	Kevin Silvestri&1094138&kevin.silvestri@studenti.unipd.it \\ \hline
	Manfredi Smaniotto&1123057&manfredi.smaniotto@studenti.unipd.it \\ \hline
	Cristiano Tessarolo&1119924&cristiano.tessarolo@studenti.unipd.it \\  
	\hline
	\end{tabular}
\caption{Elenco dei componenti}
\end{table}

\newpage

\section{Definizione dei ruoli}
I membri del gruppo, durante lo svolgimento del progetto, andranno a ricoprire diversi ruoli. Questi ultimi rappresentano figure aziendali specializzate, alle quali corrisponde un costo orario espresso in euro. \\
Durante tutta la durata del progetto ogni componente del gruppo dovrà ricoprire almeno una volta ogni ruolo. Al fine di evitare il conflitto di interesse va certificato che non vi siano intervalli di tempo in cui una risorsa sia anche verificatrice di se stessa.

\end{document}