\documentclass[./PianodiProgetto.tex]{subfiles}

\begin{document}

\chapter{Consuntivo di periodo e \\ preventivo a finire}
In questa sezione verranno presentati i consuntivi dei vari periodi con una
breve valutazione degli stessi. Verrà inoltre presentato un preventivo a finire
che terrà conto dei soli periodi rendicontati. I valori presentati saranno:
\begin{itemize}
\item \textbf{Positivi:} se il preventivo è superiore ai valori del consuntivo;
\item \textbf{Negativi:} se il preventivo è inferiore ai valori del consuntivo.
\end{itemize}

\section{Periodo di analisi}
Essendo il periodo di Analisi considerato periodo di investimento, il consuntivo viene presentato a scopo informativo ma non conteggiato nel preventivo a finire.

\subsection{Consuntivo di periodo}
Di seguito è presentata la tabella contenente i dati del consuntivo per il
periodo di Analisi.

\begin{table}[H]
	\centering
	\begin{tabular}{|c|c|c|c|c|}
		\hline
	 	 & \multicolumn{2}{c|}{Ore} & \multicolumn{2}{c|}{Costo in \euro{}}  \\ \hline
		Ruolo&Preventivo&Consuntivo&Preventivo&Consuntivo \\ \hline
		Responsabile&24&24&720,00&720,00  \\ \hline
		Amministratore&20&19 (+1)&400,00&380,00 (+20,00)  \\ \hline
		Analista&61&67 (-6)&1525,00&1675,00 (-150,00)  \\ \hline
		Progettista& & & &  \\ \hline
		Programmatore& & & &  \\ \hline
		Verificatore&46&46&690,00&690,00  \\ \hline
		Totale&151&156&3335,00&3465,00  \\ \hline
		Differenza& \multicolumn{2}{c|}{-5 Ore} & \multicolumn{2}{c|}{-130,00 \euro{}} \\ \hline
	\end{tabular}
	\caption{Prospetto orario ed economico a consuntivo del periodo di Analisi}
\end{table}

\subsection{Conclusione}
Nell'esecuzione del primo periodo di Analisi è stato necessario usare più
ore del previsto per il ruolo di \textit{Analista} mentre si è riusciti a risparmiare nel ruolo di \textit{Amministratore}. Questo è dovuto
probabilmente ad una sottostima del carico di lavoro presentato dalla \textit{Analisi
dei Requisiti}. Le ore di verifica invece, si sono dimostrate sufficienti a svolgere
le attività previste. Il risultato del periodo è complessivamente di cinque ore
lavorative oltre il previsto, con una spesa aggiuntiva di 130,00 \euro{}.


\section{Periodo di consolidamento dei requisiti}
Essendo il periodo di Consolidamento dei requisiti considerato periodo di investimento, il consuntivo viene presentato a scopo informativo ma non conteggiato nel preventivo a finire.

\subsection{Consuntivo di periodo}
Di seguito è presentata la tabella contenente i dati del consuntivo per il
periodo di Consolidamento dei requisiti.

\begin{table}[H]
	\centering
	\begin{tabular}{|c|c|c|c|c|}
		\hline
		& \multicolumn{2}{c|}{Ore} & \multicolumn{2}{c|}{Costo in \euro{}}  \\ \hline
		Ruolo&Preventivo&Consuntivo&Preventivo&Consuntivo \\ \hline
		Responsabile&5&5&150,00&150,00  \\ \hline
		Amministratore&8&8&160,00&160,00  \\ \hline
		Analista&19&19&475,00&475,00  \\ \hline
		Progettista& & & &  \\ \hline
		Programmatore& & & &  \\ \hline
		Verificatore&15&15&225,00&225,00  \\ \hline
		Totale&47&47&1010,00&1010,00  \\ \hline
		Differenza& \multicolumn{2}{c|}{0 Ore} & \multicolumn{2}{c|}{0,00 \euro{}} \\ \hline
	\end{tabular}
	\caption{Prospetto orario ed economico a consuntivo del periodo di Consolidamento dei requisiti}
\end{table}

\subsection{Conclusione}
Nel periodo di Consolidamento dei requisiti il gruppo ha concluso le operazioni rispettando il prospetto orario. Le ore preventivate sono state rispettate per tutti i ruoli, questo ha portato ad un pareggio tra consuntivo e preventivo, come mostrato in tabella.


\section{Periodo di consolidamento delle tecnologie}
\subsection{Consuntivo di periodo}
Di seguito è presentata la tabella contenente i dati del consuntivo per il
periodo di Consolidamento delle tecnologie.

\begin{table}[H]
	\centering                   
       \begin{tabular}{|c|c|c|c|c|}
		\hline
		& \multicolumn{2}{c|}{Ore} & \multicolumn{2}{c|}{Costo in \euro{}}  \\ \hline
		Ruolo&Preventivo&Consuntivo&Preventivo&Consuntivo \\ \hline
		Responsabile&10&10&300,00&300,00  \\ \hline
		Amministratore&10&10&200,00&200,00 \\ \hline
		Analista&20&19 (+1)&500,00&475,00 (+25,00)  \\ \hline
		Progettista&97&96 (+1)&2134,00&2112,00 (+22,00) \\ \hline
		Programmatore&13&15 (-2)&195,00&225,00 (-30,00)  \\ \hline
		Verificatore&60&60&900,00&900,00  \\ \hline
		Totale&210&210&4229,00&4212,00  \\ \hline
		Differenza& \multicolumn{2}{c|}{0 Ore} & \multicolumn{2}{c|}{+17,00 \euro{}} \\ \hline
	\end{tabular}
	\caption{Prospetto orario ed economico a consuntivo del periodo di Consolidamento delle tecnologie}
\end{table}

\subsection{Conclusione}
Nel periodo di Consolidamento delle tecnologie si è presentata la necessità di ridistribuire alcune ore necessarie per la creazione del \textit{Proof of Concept}. E'
stato però possibile, tramite la comunicazione con Mivoq S.R.L. una miglior definizione dei requisiti. Questo ha lasciato la possibilità di ridurre le ore destinate all'analisi e alla progettazione piuttosto che ridurre le ore destinate alla verifica. Si è così riusciti a concludere con un consuntivo di periodo in positivo con un risparmio di 17,00 \euro{}.

\section{Preventivo a finire}
Viene qui presentata una tabella contenente l'attuale preventivo a finire.
Vengono inseriti i valori del periodo di Analisi e Consolidamento dei requisiti
a scopo riassuntivo, tuttavia essi non verranno conteggiati nel calcolo delle
ore rendicontate. Se il valore del consuntivo non fosse ancora presente, verrà
usato il valore del preventivo.

\begin{table}[H]
	\centering
	\begin{tabular}{|c|c|c|}
		\hline
		Periodo&Preventivo \euro{}&Consuntivo \euro{} \\ \hline
		Analisi&3335,00&3465,00  \\ \hline
		Consolidamento dei requisiti&1010,00&1010,00  \\ \hline
		\multicolumn{3}{|c|}{Rendicontato}  \\ \hline
		Consolidamento delle tecnologie&4229,00&4212,00  \\ \hline
		Progettazione e codifica&6684,00&Non presente  \\ \hline
		Validazione e collaudo&2504,00&Non presente  \\ \hline
		 &Preventivo \euro{}&Preventivo a finire \euro{}  \\ \hline
		Totale&17762,00&17892,00 \\ \hline
		Rendicontato&13417,00&13400,00 \\ \hline
	\end{tabular}
	\caption{Preventivo a finire}
\end{table}

\subsection{Conclusione}
A seguito dell'ultimo periodo terminato si è verificata una diminuzione del preventivo a finire, che permette di avere del lieve margine utilizzabile nel prossimo periodo nella codifica, tutto ciò rimanendo entro i costi inizialmente preventivati.

\appendix


\appendix

\chapter{Esito della rilevazione dei rischi}

\section{Introduzione}

In questa appendice si intende riportare l'esito della rilevazione dei rischi concretizzatisi durante il progetto. L'incorrere dei rischi viene monitorato costantemente e questa appendice viene incrementata ogni qualvolta ne venga rilevato uno. Nella sezione successiva si riporta dunque una tabella riepilogativa che riporta le seguenti voci:

\begin{itemize}
	\item \textbf{Codice}: il codice del rischio concretizzatosi durante il progetto;
	\item \textbf{Descrizione}: una breve descrizione del modo in cui il rischio si è concretizzato;
	\item \textbf{Contromisure}: le contromisure adottate al fine di mitigare e/o eliminare le conseguenze della succitata concretizzazione del rischio.
\end{itemize}

\section{Analisi e contromisure}

\setlength\LTleft{-5.5mm}

\begin{longtable}{|p{15mm}|p{60mm}|p{60mm}|}
	\hline \textbf{Codice} & \textbf{Descrizione} & \textbf{Contromisure} \\
	\hline RT0 & Alcuni membri del gruppo non erano pratici con Git e \LaTeX &  È stato spiegato loro l'utilizzo di queste tecnologie \\
	\hline RS0 & Sono stati riscontrati alcuni guasti hardware da parte di alcuni componenti del gruppo & Nel ripristino del funzionamento dei dispositivi è stato possibile riscaricare dalla \textit{repository Github} i file utili per lavorare da dove era stato lasciato il lavoro precedentemente svolto \\
	\hline RS0 & Il portale che ospita documentazione e download di Speect è stato offline per due giorni & Una volta ripristinatosi il funzionamento del portale, tutto il materiale reperibile online è stato scaricato in locale e condiviso tra i membri così da tutelarsi nel caso il problema dovesse occorrere nuovamente \\
	\hline RG0 & A causa delle festività invernali e/o problemi personali, alcuni membri del gruppo si sono assentati per alcuni giorni & Le eventuali assenze sono state comunicate con largo anticipo, ciò ha permesso di organizzare il lavoro in modo ottimale tenendo conto delle assenze \\
	\hline RR0 & Il gruppo di analisti ha sollevato vari dubbi prima di iniziare il proprio lavoro & È stato organizzato un incontro con il proponente per risolvere prontamente ogni dubbio sollevato; è stato anche creato un canale Slack con la Proponente affinchè siano risolti dei futuri dubbi nello sviluppo del software \\
	\hline RO0 & La sottostima dei tempi necessari ha causato ritardi nel compimento di alcuni task & Si è deciso di allungare il lasso di tempo di slack relativo a ciascun task in maniera proporzionale all'entità dello stesso \\
	\hline
	\caption{Riepilogo dell'esito della rilevazione dei rischi}
\end{longtable}

\chapter{Organigramma}

\section{Redazione}
\begin{table}[H]
	\centering
	\begin{tabular}{|c|c|c|}
		\hline
		Nome&Data&Firma \\ \hline
		Matteo Rizzo& 12-12-2017 &\includegraphics[scale=0.5]{img/firme/RizzoMatteo} \\
		\hline
	\end{tabular}
	\caption{Redazione}
\end{table}

\section{Approvazione}
\begin{table}[H]
	\centering
	\begin{tabular}{|c|c|c|}
		\hline
		Nome&Data&Firma \\ \hline
		Matteo Rizzo& 12-12-2017 & \includegraphics[scale=0.5]{img/firme/RizzoMatteo} \\
		\hline
	\end{tabular}
	\caption{Approvazione}
\end{table}

\section{Accettazione dei componenti}
\begin{table}[H]
	\centering
	\begin{tabular}{|c|c|c|}
		\hline
		Nome&Data&Firma \\ \hline
		Marco Focchiatti& 12-12-2017 &\includegraphics[scale=0.5]{img/firme/FocchiattiMarco} \\ \hline
		Samuele Modena& 12-12-2017 &\includegraphics[scale=0.5]{img/firme/ModenaSamuele} \\ \hline
		Matteo Rizzo& 12-12-2017 &\includegraphics[scale=0.5]{img/firme/RizzoMatteo} \\ \hline
		Giulio Rossetti&  12-12-2017 &\includegraphics[scale=0.5]{img/firme/RossettiGiulio} \\ \hline
		Kevin Silvestri& 12-12-2017 &\includegraphics[scale=0.5]{img/firme/SilvestriKevin} \\ \hline
		Manfredi Smaniotto& 12-12-2017 &\includegraphics[scale=0.5]{img/firme/SmaniottoManfredi} \\ \hline
		Cristiano Tessarolo& 12-12-2017 &\includegraphics[scale=0.5]{img/firme/TessaroloCristiano} \\  
		\hline
	\end{tabular}
	\caption{Accettazione dei componenti}
\end{table}

\section{Componenti}
\begin{table}[H]
	\begin{tabular}{|c|c|c|}
	\hline
	Nome&Matricola&Indirizzo email \\ \hline
	Marco Focchiatti&1121294&marco.focchiatti@studenti.unipd.it  \\ \hline
	Samuele Modena&1099080&samuele.modena@studenti.unipd.it \\ \hline
	Matteo Rizzo&1123496&matteo.rizzo.4@studenti.unipd.it \\ \hline
	Giulio Rossetti&1122603&giulio.rossetti@studenti.unipd.it \\ \hline
	Kevin Silvestri&1094138&kevin.silvestri@studenti.unipd.it \\ \hline
	Manfredi Smaniotto&1123057&manfredi.smaniotto@studenti.unipd.it \\ \hline
	Cristiano Tessarolo&1119924&cristiano.tessarolo@studenti.unipd.it \\  
	\hline
	\end{tabular}
\caption{Elenco dei componenti}
\end{table}

\section{Definizione dei ruoli}
I membri del gruppo, durante lo svolgimento del progetto, andranno a ricoprire diversi ruoli. Questi ultimi rappresentano figure aziendali specializzate, alle quali corrisponde un costo orario espresso in euro. \\
Durante tutta la durata del progetto ogni componente del gruppo dovrà ricoprire almeno una volta ogni ruolo. Al fine di evitare il conflitto di interesse va certificato che non vi siano intervalli di tempo in cui una risorsa sia anche verificatrice di se stessa.

\end{document}