\documentclass[../AnalisideiRequisiti.tex]{subfiles}

\begin{document}
	\chapter{Requisiti}
	\section{Classificazione dei requisiti}
	I requisiti sono identificati da un codice descritto nelle NP (§2.2.3.3).
	
	\section{Requisiti funzionali}
	\begin{longtable}{| p{2cm} | p{2.5cm} |p{5cm} | p{2.5cm} |}
		\hline
		\textbf{Codice} & \textbf{Importanza} & \textbf{Descrizione} & \textbf{Fonti}\\
		\hline
		\endhead
		\newline ROF0&
		\newline Obbligatorio&
		\newline L'utente può avviare DeSpeect visualizzandone la pagina iniziale&
		\newline \refer{UC1} \newline Capitolato
		\\[1em]
		\hline
		\newline ROF1&
		\newline Obbligatorio&
		\newline L'utente può accedere al menu file&
		\newline Interno
		\\[1em]	
		
		\hline
			
		\newline ROF2&
		\newline Obbligatorio&
		\newline L'utente può caricare un file \verb|.json|&
		\newline \refer{UC3} \newline Capitolato
		\\[1em]	
			\hline	
			
		\newline ROF2.1&
		\newline Obbligatorio&
		\newline L'utente può visualizzare il percorso del file \verb|.json| caricato&
		\newline \refer{UC3} \newline  VE-2017-12-15
		\\[1em]	
		\hline	
		
		\newline RFF2.2&
		\newline Facoltativo&
		\newline L'utente può modificare il file \verb|.json| cambiando l'ordine o rimuovendo gli utterance processor nell'utterance type&
	 	\newline \refer{UC6.2} \newline \refer{UC6.3} \newline Capitolato
		\\[1em]	
		\hline
				
		\newline RFF2.2.1&
		\newline Facoltativo&
		\newline L'utente può salvare nel file \verb|.json| le modifiche agli utterance processor&
		\newline \refer{UC21} \newline  VE-2017-12-15
		\\[1em]	
		\hline
		
		\newline RFF2.2.1.1&
		\newline Facoltativo&
		\newline Il sistema deve visualizzare un errore nel caso il salvataggio fallisca e ripristinare uno stato funzionante&
		\newline \refer{UC22} \newline Interno
		\\[1em]	
		\hline

		\newline ROF3&		
		\newline Obbligatorio&
		\newline L'utente può inizializzare Speect con il file \verb|.json|&
		\newline \refer{UC3} \newline  VE-2017-12-15
		\\[1em]	
			\hline	
		
		\newline ROF3.1&\newline Obbligatorio&
		\newline Il sistema deve visualizzare un errore in caso Speect fallisca l'inizializzazione&
		\newline \refer{UC4} \newline Interno
		\\[1em]		
		\hline
		
		\newline ROF4&\newline Obbligatorio&
		\newline L'utente può salvare l'audio risultante con estensione \verb|.wav|&
		\newline \refer{UC15} \newline Interno
		\\[1em]
			\hline
		
		\newline ROF4.1&\newline Obbligatorio&
		\newline L'utente può selezionare dove salvare il file&
		\newline \refer{UC15} \newline Interno
		\\[1em]
		
		\hline	
		\newline ROF4.1.1&\newline Obbligatorio&
		\newline L'utente può scrivere il nome del file da salvare&
		\newline \refer{UC15} \newline Interno
		\\[1em]
		
		\hline
		\newline ROF4.2&\newline Obbligatorio&
		\newline Il sistema deve visualizzare un errore in caso il salvataggio dell'audio fallisca&
		\newline \refer{UC16} \newline Interno
		\\[1em]
		\hline
		
		\newline RFF4.3&\newline Facoltativo&
		\newline L'utente può ascoltare l'audio prima di salvarlo&
		\newline Interno
		\\[1em]
		\hline
		
		\newline ROF6&\newline Obbligatorio&
		\newline L'utente può selezionare l'utterance type&
		\newline \refer{UC5} \newline  VE-2017-12-15
		\\[1em]
		\hline
				
		\newline RDF6.1&\newline Desiderabile&
		\newline L'utente può modificare gli utterance processor di un'utterance type&
		\newline \newline \refer{UC6}  \newline \refer{UC6.2} \newline \refer{UC6.3} \newline  VE-2017-12-15
		\\[1em]
		\hline	
				
		\newline RDF6.1.1&\newline Desiderabile&
		\newline L'utente può spostare gli utterance processor di un'utterance type&
		\newline \refer{UC6.1} \newline \refer{UC6.2} \newline Interno
		\\[1em]
		\hline	
				
		\newline RDF6.1.2&\newline Desiderabile&
		\newline L'utente può rimuovere gli utterance processor di un'utterance type&
		\newline \refer{UC6.1} \newline \refer{UC6.3} \newline Interno
		\\[1em]
		\hline	
		
		\newline ROF7&\newline Obbligatorio&
		\newline L'utente può inserire un testo da tradurre in voce&
		\newline \refer{UC7} \newline Capitolato
		\\[1em]
		
		\hline
		\newline ROF8&\newline Obbligatorio&
		\newline L'utente può processare il testo inserito&
		\newline \refer{UC7} \newline \refer{UC9} \newline \refer{UC10} \newline Capitolato
		\\[1em]
		\hline
		\newline ROF8.1&\newline Obbligatorio&
		\newline Il sistema visualizza l'errore di esecuzione se Speect fallisce l'esecuzione&
		\newline \refer{UC8} \newline Interno
		\\[1em]
		\hline
		
		\newline ROF9&\newline Obbligatorio&
		\newline L'utente può visualizzare il grafo ottenuto eseguendo Speect&
		\newline \refer{UC11} \newline Capitolato
		\\[1em]
		\hline
		
			
		\newline ROF9.1&\newline Obbligatorio&
		\newline L'utente può visualizzare l'informazione generale di ogni nodo sul grafo&
		\newline \refer{UC11} \newline Capitolato
		\\[1em]
		\hline
		
		\newline ROF9.2&\newline Obbligatorio&
		\newline L'utente vede ogni relazione del grafo di un colore diverso, relativo al colore in legenda&
		\newline  VE-2018-01-03  \newline Capitolato
		\\[1em]
		\hline
		
		\newline RDF9.2.1&\newline Desiderabile&
		\newline L'utente può cambiare il colore delle relazioni in legenda&
		\newline  VE-2018-01-03
		\\[1em]
		\hline
		
		\newline ROF9.3&\newline Obbligatorio&
		\newline L'utente può selezionare il nodo del grafo tramite click&
		\newline \refer{UC12.1} \newline Capitolato
		\\[1em]
		\hline
		
			\newline ROF9.3.1&\newline Obbligatorio&
		\newline L'utente può visualizzare tutte le informazioni del nodo selezionato&
		\newline \refer{UC12.1} \newline Capitolato
		\\[1em]
		\hline
			
		\newline RDF9.3.1.1&\newline Desiderabile&
		\newline L'utente può modificare il name del nodo selezionato&
		\newline \refer{UC12.3} \newline  VE-2017-12-15 \newline Capitolato
		\\[1em]
		\hline
		
			\newline RDF9.3.1.2&\newline Desiderabile&
		\newline L'utente può modificare il PoS del nodo selezionato&
		\newline  VE-2017-12-15 \newline Capitolato
		\\[1em]
		\hline
		
		\newline RDF9.4&\newline Desiderabile&
		\newline L'utente può testare se un percorso porta ad un nodo esistente&
		\newline \refer{UC13} \newline  VE-2017-12-15 \newline Capitolato
		\\[1em]
		\hline
		
		\newline RDF9.4.1&\newline Desiderabile&
		\newline L'utente può evidenziare un nodo del grafo tramite percorso partendo da un nodo selezionato&
		\newline \refer{UC13} \newline \refer{UC12.1} \newline  VE-2018-01-03 \newline Capitolato
		\\[1em]
		\hline
		
		\newline RDF9.4.2&\newline Desiderabile&
		\newline Il sistema visualizza un errore se il path porta fuori dal grafo&
		\newline \refer{UC14} \newline Interno
		\\[1em]
		\hline
		
		\newline ROF9.5&\newline Obbligatorio&
		\newline I nodi selezionati dall'utente vengono evidenziati&
		\newline \refer{UC12.1} \newline  VE-2017-12-15 \newline Capitolato
		\\[1em]
		\hline
		
		\newline RDF9.5.1&\newline Desiderabile&
		\newline L'utente può modificare il colore con cui si evidenzia il focus&
		\newline Interno
		\\[1em]
		\hline
		
		\newline ROF9.6&\newline Obbligatorio&
		\newline L'utente può spostare i nodi del grafo graficamente&
		\newline \refer{UC12.2} \newline VE-2018-01-03
		\\[1em]
		\hline
		
		\newline ROF9.7&\newline Obbligatorio&
		\newline L'utente può visualizzare gli strati di relazione del grafo selezionati&
		\newline \refer{UC12.4} \newline Capitolato
		\\[1em]
		\hline
	
		\newline RFF9.8&\newline Facoltativo&
		\newline L'utente può modificare gli archi dei nodi del grafo&
		\newline  VE-2017-12-15 \newline Capitolato
		\\[1em]
		\hline
		
		\newline RFF9.8.1&\newline Facoltativo&
		\newline L'utente può cancellare gli archi dei nodi del grafo&
		\newline Interno \newline Capitolato
		\\[1em]
		\hline
		
		\newline RFF9.8.2&\newline Facoltativo&
		\newline L'utente può aggiungere archi a dei nodi del grafo&
		\newline Interno \newline Capitolato
		\\[1em]
		\hline
		
		\newline ROF9.9&\newline Obbligatorio&
		\newline L'utente può modificare il grafo ottenuto eseguendo Speect&
		\newline \refer{UC12} \newline Capitolato
		\\[1em]
		\hline	
		
		\newline RFF10&\newline Facoltativo&
		\newline L'utente può eseguire ogni utterance processor singolarmente&
		\newline \refer{UC10} \newline Capitolato
		\\[1em]
		\hline
	
		
		\newline RFF11&\newline Facoltativo&
		\newline L'utente può salvare il grafo&
		\newline \refer{UC17} \newline  VE-2017-12-15 
		\\[1em]
		\hline


		\newline RFF11.1&\newline Facoltativo&
		\newline Il sistema deve visualizzare un errore se non riesce a salvare il grafo&
		\newline \refer{UC18} \newline  VE-2017-12-15 
		\\[1em]
		\hline
		
		\newline RFF12&\newline Facoltativo&
		\newline L'utente può caricare un grafo&
		\newline \refer{UC19} \newline  VE-2017-12-15
		\\[1em]
		\hline
		
		\newline RFF12.1&\newline Facoltativo&
		\newline Il sistema deve visualizzare un errore se non riesce a caricare il grafo&
		\newline \refer{UC20} \newline  VE-2017-12-15 
		\\[1em]
		\hline
		
		\newline RFF12.2&\newline Facoltativo&
		\newline L'utente può confrontare due strati di relazione automaticamente&
		\newline Capitolato
		\\[1em]
		\hline
		
		
		\newline RFF13&\newline Facoltativo&
		\newline L'utente può eseguire Speect dato un grafo&
		\newline \refer{UC7} \newline  VE-2017-12-15
		\\[1em]
		\hline
		
	
		
		\newline ROF14&\newline Obbligatorio&
		\newline L'utente può chiudere l'applicazione&
		\newline \refer{UC2} \newline Interno
		\\[1em]
		\hline
		
		
	\end{longtable}
	\section{Requisiti di qualità}
			\begin{longtable}{| p{2cm} | p{2.5cm} |p{5cm} | p{2.5cm} |}
			\hline
			\textbf{Codice} & \textbf{Importanza} & \textbf{Descrizione} & \textbf{Fonti}\\
			\hline
			\endhead
							
			\newline ROQ0&\newline Obbligatorio&
			\newline Deve essere fornito un manuale utente&
			\newline Capitolato
			\\[1em]
			\hline
			\newline ROQ0.1&\newline Obbligatorio&
			\newline Il manuale deve essere in lingua italiana&
			\newline Interno
			\\[1em]
			\hline
			\newline ROQ1&\newline Obbligatorio&
			\newline Lo sviluppo del prodotto deve rispettare i criteri definiti nei documenti NP e PQ&
			\newline Interno
			\\[1em]
			\hline
			\newline
			RDQ1&\newline Desiderabile&
			\newline 
			L'applicazione deve essere rilasciata con licenze open source&
			\newline Capitolato \newline Interno
			\\[1em]
			\hline	
			\newline
			RDQ1.1&\newline Desiderabile&
			\newline 
			L'applicazione deve essere rilasciata con licenze BSD/MIT &
			\newline Capitolato
			\\[1em]
			\hline	
			\newline
			ROQ2&\newline Obbligatorio&
			\newline 
			L'applicazione deve essere rilasciata con interfaccia grafica in lingua inglese&
			\newline VE-2018-03-17
			\\[1em]
			\hline	
			\newline
			ROQ3&\newline Obbligatorio&
			\newline 
			Il codice dell'applicazione deve essere rilasciato con commenti in lingua inglese&
			\newline VE-2018-03-17
			\\[1em]
			\hline	
			\newline
			ROQ4&\newline Obbligatorio&
			\newline 
			Il file \verb|README.md| del repository definitivo deve essere scritto in lingua inglese.&
			\newline VE-2018-04-06
			\\[1em]
			\hline
	\end{longtable}
\newpage
	\section{Requisiti di vincolo}
			\begin{longtable}{| p{2cm} | p{2.5cm} |p{5cm} | p{2.5cm} |}
			\hline
			\textbf{Codice} & \textbf{Importanza} & \textbf{Descrizione} & \textbf{Fonti}\\
			\hline
			\endhead
				\newline ROV0&\newline Obbligatorio&
			\newline 
			L'applicativo deve usare Speect modificato da Mivoq &
			\newline Capitolato
			\\[1em]
			\hline	
			\newline 
			ROV1&\newline Obbligatorio&
			\newline 
			L'applicativo deve essere sviluppato con QT 5.9 LTS &
			\newline Capitolato
			\newline Interno
			\\[1em]
			\hline
			\newline 
			ROV2&\newline Obbligatorio&
			\newline 
			L'applicativo deve essere utilizzabile su sistema operativo Linux Ubuntu 16.04 LTS&
			\newline Capitolato
			\\[1em]
			\hline
			\newline
			RDV2.1&\newline Desiderabile&
			\newline 
			L'applicativo deve essere utilizzabile su sistema operativo Windows 7 e successivi&
			\newline Capitolato
			\\[1em]
			\hline
	\end{longtable}
\newpage
	\section{Tracciamento fonte-requisiti}
	\begin{longtable}{| p{4cm} | p{4cm} |}
		\caption{Tracciamento fonte-requisiti} \\
		\hline
		\textbf{Fonte} & \textbf{Requisti} \\
			\hline
		\endhead
		\newline Capitolato & \newline ROF0 \newline ROF2 \newline RFF2.2 \newline ROF7 \newline ROF8 \newline ROF9 \newline ROF9.1 \newline ROF9.2 \newline ROF9.3 \newline ROF9.3 \newline ROF9.3.1 \newline ROF9.3.1.1 \newline ROF9.3.1.2 \newline RDF9.4 \newline RDF9.4.1 \newline RDF9.5 \newline ROF9.7 \newline RFF9.8 \newline RFF9.8.1 \newline RFF9.8.2 \newline ROF9.9 \newline RFF10 \newline RFF12.2 \newline ROQ0 \newline RDQ1 \newline RDQ1.1 \newline ROV0 \newline ROV1 \newline ROV2 \newline ROV2.1 \\[1em]
	\hline	
		\newline Interno & \newline ROF1 \newline RFF2.2.1.1 \newline ROF3.1 \newline ROF4 \newline ROF4.1  \newline ROF4.1.1  \newline ROF4.2  \newline RFF4.3 \newline RDF5.5  \newline RDF5.6  \newline RDF5.7 \newline ROF6.1 \newline RDF6.1.1 \newline RDF6.1.2 \newline RDF9.4.2 \newline ROF9.6  \newline ROF8.1  \newline RDF9.5.1  \newline RFF9.8.1  \newline RFF9.8.2 \newline ROF14 \newline ROQ0.1 \newline ROQ1 \newline RDQ1  \newline ROV1 \\[1em]
	\hline
		\newline  VE-2017-12-15 & \newline ROF2.1 \newline RFF2.2.1 \newline ROF5 \newline ROF5.2 \newline ROF5.3 \newline ROF5.4 \newline ROF6 \newline RDF6.1 \newline RDF9.3.1.1 \newline RDF9.3.1.2 \newline RDF9.4 \newline ROF9.5 \newline RFF9.8 \newline RFF11 \newline RFF11.1 \newline RFF12 \newline RFF12.1 \newline RFF13 \\[1em]
	\hline
		\newline VE-2018-01-03 & \newline ROF9.2 \newline RDF9.2.1 \newline RDF9.4.1 \newline ROF9.6 \\[1em]
	\hline
		\newline VE-2018-03-17 & \newline ROQ2 \newline ROQ3 \\[1em]
	\hline
		\newline VE-2018-04-06 & \newline ROQ4 \\[1em]	
		

	\hline
		\newline UC1 &  \newline ROF0 \\[1em]
	\hline	
		\newline UC2 &  \newline ROF14 \\[1em]
	\hline	
		\newline UC3 &  \newline ROF2 \newline ROF2.1 \newline ROF3 \\[1em]	
		\hline
		\newline UC4 &  \newline ROF3.1 \\[1em]	
		\hline	
		\newline UC5 &  \newline ROF6 \\[1em]
		\hline	
		\newline UC6 &  \newline RDF6.1 \\[1em]
		\hline
		\newline UC6.1 &  \newline RDF6.1.1 \newline RDF6.1.2 \\[1em]
		\hline
		\newline UC6.2 &  \newline RFF2.2 \newline RDF6.1 \newline RDF6.1.1 \\[1em]
		\hline
		\newline UC6.3 &  \newline RFF2.2 \newline RDF6.1 \newline RDF6.1.2 \\[1em]
		\hline
		\newline UC7 &  \newline ROF7 \newline ROF8 \newline RFF13 \\[1em]
		\hline
		\newline UC8 &  \newline ROF8.1 \\[1em]
		\hline
		\newline UC9 &  \newline ROF8 \\[1em]
		\hline
		\newline UC10 &  \newline RFF10 \\[1em]
		\hline
		\newline UC11 &  \newline ROF9 \newline ROF9.1 \\[1em]
		\hline		
		\newline UC12 &  \newline ROF9.9 \\[1em]
		\hline			
		\newline UC12.1 &  \newline ROF9.3 \newline ROF9.3.1 \newline RDF9.1 \newline RDF9.4.1 \newline ROF9.5 \\[1em]
		\hline
		\newline UC12.2 &  \newline ROF9.6 \\[1em]
		\hline
		\newline UC12.3 &   \newline RDF9.3.1.1 \\[1em]
		\hline
		\newline UC12.4 &  \newline ROF9.7 \\[1em]
		\hline
		\newline UC13 &  \newline RDF9.4 \newline RDF9.4.1 \newline RDF9.4.2 \\[1em]
		\hline
		\newline UC14 &  \newline RDF9.4.2 \\[1em]
		\hline
		\newline UC15 &  \newline ROF4 \newline ROF4.1 \newline ROF4.1.1 \\[1em]
		\hline
		\newline UC16 &  \newline ROF4.2 \\[1em]
		\hline
		\newline UC17 &  \newline RFF11 \\[1em]
		\hline
		\newline UC18 &  \newline RFF11.1 \\[1em]
		\hline
		\newline UC19 &  \newline RFF12 \\[1em]
		\hline
		\newline UC20 &  \newline RFF12.1 \\[1em]
		\hline		
		\newline UC21 &  \newline RFF2.2.1 \newline RFF2.2.1.1\\[1em]
		\hline
		\newline UC22 &  \newline RFF9.2.1.1 \\[1em]
		\hline
		
	\end{longtable}
\newpage

	\section{Tracciamento requisito-fonti}
	\begin{longtable}{| p{4cm} | p{4cm} |}
	\caption{Tracciamento requisito-fonti} \\
	
	\hline
\textbf{Requisito} & \textbf{Fonti} \\
\hline
\endhead
	\newline ROF0&
	\newline \refer{UC1} \newline Capitolato
	\\[1em]
	\hline
	\newline ROF1&
	\newline Interno
	\\[1em]	
	
	\hline
	
	\newline ROF2&
	\newline \refer{UC3} \newline Capitolato
	\\[1em]	
	\hline	
	
	\newline ROF2.1&
	\newline \refer{UC3} \newline  VE-2017-12-15
	\\[1em]	
	\hline	
	
	\newline RFF2.2&
	\newline \refer{UC6.2} \newline \refer{UC6.3} \newline Capitolato
	\\[1em]	
	\hline
	
	\newline RFF2.2.1&
	\newline \refer{UC21} \newline  VE-2017-12-15
	\\[1em]	
	\hline
	
	\newline RFF2.2.1.1&
	\newline \refer{UC22} \newline Interno
	\\[1em]	
	\hline
	
	\newline ROF3&	
	\newline \refer{UC3} \newline  VE-2017-12-15
	\\[1em]	
	\hline	
	
	\newline ROF3.1&
	\newline \refer{UC4} \newline Interno
	\\[1em]		
	\hline
	
	\newline ROF4&
	\newline \refer{UC15} \newline Interno
	\\[1em]
	\hline
	
	\newline ROF4.1&
	\newline \refer{UC15} \newline Interno
	\\[1em]
	
	\hline	
	\newline ROF4.1.1&
	\newline \refer{UC15} \newline Interno
	\\[1em]
	
	\hline
	\newline ROF4.2&
	\newline \refer{UC16} \newline Interno
	\\[1em]
	\hline
	
	\newline RFF4.3&
	
	\newline Interno
	\\[1em]
	\hline
	
	\newline ROF6&
	
	\newline \refer{UC5} \newline  VE-2017-12-15
	\\[1em]
	\hline
	
	\newline RDF6.1&
	
	\newline \refer{UC6} \newline \refer{UC6.2} \newline \refer{UC6.3} \newline  VE-2017-12-15
	\\[1em]
	\hline	
	
	\newline RDF6.1.1&
	
	\newline \refer{UC6.1} \newline \refer{UC6.2} \newline Interno
	\\[1em]
	\hline	
	
	\newline RDF6.1.2&
	
	\newline \refer{UC6.1} \newline \refer{UC6.3} \newline Interno
	\\[1em]
	\hline	
	
	\newline ROF7&
	
	\newline \refer{UC7} \newline Capitolato
	\\[1em]
	
	\hline
	\newline ROF8&
	
	\newline \refer{UC7} \newline Capitolato
	\\[1em]
	\hline
	\newline ROF8.1&
	
	\newline \refer{UC8} \newline Interno
	\\[1em]
	\hline
	
	\newline ROF9&
	
	\newline \refer{UC11} \newline Capitolato
	\\[1em]
	\hline
	
	
	
	\newline ROF9.1&
	
	\newline \refer{UC11} \newline Capitolato
	\\[1em]
	\hline
	
	\newline ROF9.2&
	
	\newline  VE-2018-01-03  \newline Capitolato
	\\[1em]
	\hline
	
	\newline RDF9.2.1&
	
	\newline  VE-2018-01-03
	\\[1em]
	\hline
	
	\newline ROF9.3&
	
	\newline \refer{UC12.1} \newline Capitolato
	\\[1em]
	\hline
	
	\newline ROF9.3.1&
	
	\newline \refer{UC12.1} \newline Capitolato
	\\[1em]
	\hline
	
	\newline RDF9.3.1.1&
	
	\newline \refer{UC12.3} \newline  VE-2017-12-15 \newline Capitolato
	\\[1em]
	\hline
	
	\newline RDF9.3.1.2&
	
	\newline  VE-2017-12-15 \newline Capitolato
	\\[1em]
	\hline
	
	\newline RDF9.4&
	
	\newline \refer{UC13} \newline  VE-2017-12-15 \newline Capitolato
	\\[1em]
	\hline
	
	\newline RDF9.4.1&
	
	\newline \refer{UC13} \newline \refer{UC12.1} \newline  VE-2018-01-03 \newline Capitolato
	\\[1em]
	\hline
	
	\newline RDF9.4.2&
		\newline \refer{UC14} \newline Interno
	\\[1em]
	\hline
	
	\newline ROF9.5&
	\newline \refer{UC12.1} \newline  VE-2017-12-15 \newline Capitolato
	\\[1em]
	\hline
	
	\newline RDF9.5.1
	&\newline Interno
	\\[1em]
	\hline
	
	\newline ROF9.6&
	\newline \refer{UC12.2} \newline VE-2018-01-03
	\\[1em]
	\hline
	
	\newline ROF9.7&
	\newline \refer{UC12.4} \newline Capitolato
	\\[1em]
	\hline
	
	\newline RFF9.8&
	\newline VE-2017-12-15 \newline Capitolato
	\\[1em]
	\hline
	
	\newline RFF9.8.1&
	\newline Interno \newline Capitolato
	\\[1em]
	\hline
	
	\newline RFF9.8.2&
	\newline Interno \newline Capitolato
	\\[1em]
	\hline
	
	\newline ROF9.9&
	\newline \refer{UC12} \newline Capitolato
	\\[1em]
	\hline
	
	\newline RFF10&
	\newline \refer{UC10} \newline Capitolato
	\\[1em]
	\hline
	
	
	
	
	\newline RFF11&
	\newline \refer{UC17} \newline  VE-2017-12-15 
	\\[1em]
	\hline
	
	
	\newline RFF11.1&
	\newline \refer{UC18} \newline  VE-2017-12-15 
	\\[1em]
	\hline
	
	\newline RFF12&
	\newline \refer{UC19} \newline  VE-2017-12-15
	\\[1em]
	\hline
	\newline RFF12.1&
	\newline \refer{UC20} \newline  VE-2017-12-15 
	\\[1em]
	\hline
	
	\newline RFF12.2&
	\newline Capitolato
	\\[1em]
	\hline
	
	
	\newline RFF13&
	\newline \refer{UC7} \newline  VE-2017-12-15
	\\[1em]
	\hline
	
	
	
	\newline ROF14&	\newline \refer{UC2} \newline Interno
	\\[1em]
	\hline
	
	

\newline 
ROQ0&\newline Capitolato
\\[1em]
\hline
\newline 
ROQ0.1&\newline Interno
\\[1em]
\hline
\newline 
ROQ1&\newline Interno
\\[1em]
\hline
\newline
RDQ1&\newline Capitolato \newline Interno
\\[1em]
\hline	
\newline
RDQ1.1&\newline Capitolato
\\[1em]
\hline	
\newline
ROQ2&\newline VE-2018-03-17
\\[1em]
\hline	
\newline
ROQ3&\newline VE-2018-03-17
\\[1em]
\hline	
\newline
ROQ4&\newline VE-2018-04-06
\\[1em]
\hline
\newline 
ROV0&\newline Capitolato
\\[1em]
\hline	
\newline 
ROV1&\newline Capitolato \newline Interno
\\[1em]
\hline
\newline 
ROV2&\newline Capitolato
\\[1em]
\hline
\newline
RDV2.1&\newline Capitolato
\\[1em]
\hline
\end{longtable}

\newpage

\section{Riepilogo dei requisiti}
	\begin{longtable}{| c | c | c | c | c |}
	\caption{Riepilogo requisiti} \\
	\hline
	\textbf{Tipo} & \textbf{Obbligatorio}& \textbf{Facoltativo} &\textbf{Desiderabile} &\textbf{Totale} \\
	\hline
	\endhead
\textbf{Funzionale} &24& 14 & 10 & 48\\

\hline\textbf{Prestazionale} &0& 0& 0 &0  \\

\hline\textbf{Qualità} &6& 0&2&8 \\

\hline\textbf{Vincolo}&3& 0& 1 & 4 \\

\hline\textbf{Totale}&33& 14& 13 & 60 \\

\hline
\end{longtable}
\end{document}