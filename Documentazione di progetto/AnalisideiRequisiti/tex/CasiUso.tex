\documentclass[../AnalisideiRequisiti.tex]{subfiles}

\begin{document}
	% Il comando UserCase accetta primo una label nel caso serva un link verso di lui \refer{label} poi 
	% attore primario
	% attore secondario
	% Descrizione
	% Precondi
	% Post
	% Scenario principale
	% Scenari alternativi 

	\chapter{Casi d'uso}
	\section{Classificazione dei casi d'uso}
	I casi d'uso sono identificati da un codice descritto nelle NP alla sezione §2.2.3.4. 
	
 	
	\section{UC0: Visualizzazione pagina iniziale}
	\begin{figure}[H]
 
		\centering
 
		\includegraphics[width=\textwidth]{../img/UC0.png}
 
		\caption{UC0: Visualizzazione pagina iniziale}
 
	\end{figure}
	\UserCase
	{UC0}
	{Utente}
	{Non previsto}
	{L'attore visualizza la pagina iniziale di DeSpeect dalla quale può aprire il menu File, caricare un file JSon, selezionare l'Utterance Type, scrivere del testo in input, eseguire Speect, modificare l'Utterance Type e modificare il grafo}
	{Il sistema funziona correttamente e visualizza la pagina iniziale di Despeect}
	{Il sistema ha ricevuto gli input dall'attore}
	{\begin{itemize}
			\item{} L'attore può aprire il menu File \refer{UC1};
			\item{} L'attore può caricare un file JSon \refer{UC2};
			\item{} L'attore può selezionare l'Utterance Type \refer{UC12};
			\item{} L'attore può scrivere del testo in input;
			\item{} L'attore può eseguire Speect \refer{UC7};
			\item{} L'attore può modificare l'Utterance Type \refer{UC6}.
			\item{} L'attore può modificare il grafo \refer{UC13}.
	\end{itemize}}
	{Non previsti}
	
	
	\section{UC1: Apertura menu File}
	\begin{figure}[H]
		
		\centering
		
		\includegraphics[width=\textwidth]{../img/UC1.png}
		
		\caption{UC1: Apertura menu File}
		
	\end{figure}
	\UserCase
	{UC1}
	{Utente}
	{Non previsto}
	{L'attore visualizza il menu File dalla quale può caricare o salvare un file JSon, caricare o salvare un grafo, salvare l'audio prodotto, cercare il percorso di un nodo nel grafo e chiudere l'applicazione}
	{Il sistema funziona correttamente e visualizza il menu File}
	{Il sistema ha ricevuto gli input dall'attore}
	{	\begin{itemize}
		\item{} L'attore può caricare un file JSon \refer{UC2};
		\item{} L'attore può salvare le modifiche al file JSon \refer{UC11};
		\item{} L'attore può caricare un grafo \refer{UC8};
		\item{} L'attore può salvare un grafo \refer{UC9};
		\item{} L'attore può salvare l'audio prodotto da Speect \refer{UC4};
		\item{} L'attore può cercare il percorso di un nodo nel grafo \refer{UC10};
		\item{} L'attore può chiudere l'applicazione \refer{UC5}.
		\end{itemize}
	}
	{Non previsti}

	\section{UC2: Caricamento file JSon}
	\begin{figure}[H]
		\centering
		\includegraphics[width=\textwidth]{../img/UC2.png}
		\caption{UC2: Caricamento file JSon}
	\end{figure}
	\UserCase
	{UC2}
	{Utente}
	{Non previsto}
	{L'attore vuole caricare un file JSon}
	{L'attore ha selezionato la voce relativa nel menu \refer{UC1}}
	{Viene inizializzato Speect con il file JSon selezionato e aggiornata la GUI}
	{
		\begin{itemize}
			\item{} Viene aperto il file browser;
			\item{} L'attore seleziona il file;
			\item{} L'attore preme Carica;
			\item{} Il file viene dato a Speect che prova l'inizializzazione;
			\item{} Viene visualizzato il percorso del file nell'apposito spazio \ref{fig:GUI}.
		\end{itemize}
	}
	{Speect fallisce l'inizializzazione e l'attore visualizza il messaggio dell'errore relativo al file \refer{UC2.1}}
	
	\section{UC2.1: Errore caricamento file JSon}
	\UserCase
	{UC2.1}
	{Utente}
	{Non previsto}
	{Durante l'inizializzazione Speect fallisce ritornando un errore}
	{L'attore carica un file JSon non corretto}
	{L'errore è visualizzato a schermo e viene ripristinato lo stato precedente ridando controllo all'attore}
	{L'attore ha caricato un file JSon non corretto e viene visualizzato un messaggio di errore}
	{Non previsti}

\section{UC4: Salvataggio audio prodotto}
\begin{figure}[H]
	\centering
	\includegraphics[width=\textwidth]{../img/UC4.png}
	\caption{UC4: Salvataggio audio prodotto}
\end{figure}
\UserCase
{UC4}
{Utente}
{Non previsto}
{L'attore vuole salvare l'audio prodotto}
{Speect è inizializzato \refer{UC2}}
{L'audio è salvato in un file}
{
		\begin{itemize}
		\item{} Viene aperto il file browser;
		\item{} L'attore si sposta nella cartella di destinazione;
		\item{} L'attore scrive il nome del file nella barra di testo;
		\item{} L'attore preme su Salva;
		\item{} Speect compila producendo il file desiderato;
		\item{} Il file viene salvato nella destinazione con estensione .WAV .
		\end{itemize}
}
{Avviene un errore durante il salvataggio dell'audio e l'attore visualizza il messaggio di errore relativo \refer{UC4.1}}
		
\section{UC4.1: Errore salvataggio audio}
\UserCase
{UC4.1}
{Utente}
{Non previsto}
{Avviene un errore durante il salvataggio dell'audio}
{L'attore ha cercato di salvare il file audio prodotto} 
{Viene visualizzato l'errore e nessuna operazione viene eseguita}
{L'attore ha cercato di salvare il file audio prodotto e viene visualizzato un messaggio di errore}
{Non previsti}

\section{UC5: Uscita applicazione}
\UserCase
{UC5}
{Utente}
{Non previsto}
{L'attore che vuole chiudere l'applicazione, visualizza una finestra di conferma e conferma la chiusura dell'applicazione}
{L'applicazione è in esecuzione}
{L'attore conferma la chiusura dell'applicazione e l'applicazione viene terminata}
{Chiusura dell'applicazione}
{L'attore annulla la chiusura dell'applicazione}

\section{UC6: Modifica Utterance Type}
\begin{figure}[H]
	\centering
	\includegraphics[width=\textwidth]{../img/UC6.png}
	\caption{UC6: Modifica Utterance Type}
\end{figure}
\UserCase
{UC6}
{Utente}
{Non previsto}
{L'attore vuole modificare l'Utterance Type}
{E' presente almeno un'Utterance Type e questo è selezionato \refer{UC12}}
{L'Utterance Type è stato modificato e il file JSon viene aggiornato}
{
	\begin{itemize}
		\item{} L'attore seleziona un Utterance Processor;
		\item{} L'attore riordina o rimuove l'Utterance Processor;	
		\item{} Le operazioni vengono eseguite;
		\item{} Il file JSon relativo viene aggiornato.	
	\end{itemize}
}
{Non previsti}

\section{UC6.1: Selezione Utterance Processor}
\UserCase
{UC6.1}
{Utente}
{Non previsto}
{L'attore vuole selezionare un Utterance Processor per spostarlo}
{Un file JSon è stato caricato correttamente \refer{UC2}}
{Vengono visualizzati i bottoni per modificare tale Utterance Processor}
{
	\begin{itemize}
		\item{} L'attore clicca sul nome dell'Utterance Processor;
		\item{} Vengono visualizzati due bottoni che permettono lo spostamento grafico del Utterance Processor \refer{UC6.2} e un bottone che ne permette la rimozione \refer{UC6.3}. 		
	\end{itemize}
}
{Non previsti}

\section{UC6.2: Riordino Utterance Processor}
\UserCase
{UC6.2}
{Utente}
{Non previsto}
{L'attore vuole cambiare l'ordine degli Utterance Processor}
{L'attore ha selezionato un Utterance Processor \refer{UC6.1}}
{Il file JSon viene aggiornato}
{
	\begin{itemize}
		\item{} L'attore clicca sull'Utterance Processor \refer{UC6.1};
		\item{} L'attore riordina tramite i pulsanti forniti;
		\item{} Le operazioni vengono eseguite;
		\item{} Se esisteva un grafo, esso non viene modificato.
		
	\end{itemize}
}
{Non previsti}

\section{UC6.3: Rimozione Utterance Processor}
\UserCase
{UC6.3}
{Utente}
{Non previsto}
{L'attore vuole rimuovere un Utterance Processor}
{L'attore ha selezionato un Utterance Processor \refer{UC6.1}}
{Il file JSon viene aggiornato}
{
	\begin{itemize}
		\item{} L'attore clicca sull'Utterance Processor \refer{UC6.1};
		\item{} L'attore lo rimuove tramite il pulsante fornito;
		\item{} L'operazione viene eseguita;
		\item{} Se esisteva un grafo, esso non viene modificato.
		
	\end{itemize}
}
{Non previsti}

\section{UC7: Esecuzione Speect}
\begin{figure}[H]
	\centering
	\includegraphics[width=\textwidth]{../img/UC7.png}
	\caption{UC7: Esecuzione}
\end{figure}
\UserCase
{UC7}
{Utente}
{Non previsto}
{L'attore vuole eseguire Speect}
{Il file JSon è stato caricato correttamente \refer{UC2}}
{Speect elabora il testo selezionato e viene visualizzato il grafo}
{\begin{itemize}
		\item{} L'attore seleziona l'Utterance Type \refer{UC12};
		\item{} L'attore compila il campo di testo o inserisce un grafo hrg;
		\item{} L'attore preme sul tasto di esecuzione;
		\item{} Vengono eseguiti gli Utterance Processor designati dall'Utterance Type \refer{UC7.3};
		\item{} Viene mostrato il grafo risultante dall'esecuzione \refer{UC7.2}.
	\end{itemize}
}
{Speect ha fallito l'esecuzione e l'attore visualizza un messaggio di errore \refer{UC7.1}}

\section{UC7.1: Errore esecuzione Speect}
\UserCase
{UC7.1}
{Utente}
{Non previsto}
{L'attore visualizza a schermo l'errore di esecuzione di Speect}
{Speect ha fallito l'esecuzione}
{Viene visualizzato un messaggio di errore}
{L'attore ha provato ad eseguire Speect e viene visualizzato un messaggio di errore}
{Non previsti}

\section{UC7.2: Visualizzazione grafo}
\UserCase
{UC7.2}
{Utente}
{Non previsto}
{L'attore visualizza il grafo}
{Speect ha terminato l'esecuzione con successo \refer{UC7}}
{Viene visualizzato a schermo un grafo corretto con almeno un nodo cliccabile}
{
	L'attore visualizza il grafo corretto e può modificarlo \refer{UC13}
}
{Non previsti}

\section{UC7.3: Esecuzione singolo Utterance Processor}
\UserCase
{UC7.3}
{Utente}
{Non previsto}
{Speect esegue un singolo Utterance Processor}
{Un Utterance Type è stato selezionato \refer{UC12} }
{Viene eseguito l'Utterance Processor partendo dal grafo già presente o dal campo di testo scritto}
{
	\begin{itemize}
		\item{} L'attore seleziona l'Utterance Processor \refer{UC6.1};
		\item{} L'attore può compilare il campo di testo;
		\item{} L'attore preme sul tasto di esecuzione per il singolo processor \ref{fig:GUI}.
	\end{itemize}
}
{Speect ha fallito l'esecuzione}

\section{UC8: Esportazione grafo}
\begin{figure}[H]
	\centering
	\includegraphics[width=\textwidth]{../img/UC8.png}
	\caption{UC8: Esportazione grafo}
\end{figure}
\UserCase
{UC8}
{Utente}
{Non previsto}
{L'attore vuole esportare il grafo visualizzato}
{Esiste un grafo esportabile}
{Il grafo viene esportato in file}
{
	\begin{itemize}
			\item{} Viene aperto il file browser;
			\item{} L'attore si sposta nella cartella in cui salvare il grafo;
			\item{} L'attore scrive il nome del file nella barra di testo;
			\item{} L'attore preme su Salva Grafo.
	\end{itemize}
}
{L'esportazione fallisce e l'attore visualizza un messaggio di errore \refer{UC8.1}}

\section{UC8.1: Errore esportazione grafo}
\UserCase
{UC8.1}
{Utente}
{Non previsto}
{Avviene un errore durante l'esportazione}
{L'esportazione del grafo è fallita}
{Viene visualizzato un messaggio di errore e nessuna operazione viene eseguita}
{L'esportazione del grafo è fallita e viene visualizzato un messaggio di errore}
{Non previsti}

\section{UC9: Importazione grafo}
\begin{figure}[H]
	\centering
	\includegraphics[width=\textwidth]{../img/UC9.png}
	\caption{UC9: Importazione grafo}
\end{figure}
\UserCase
{UC9}
{Utente}
{Non previsto}
{L'attore vuole importare un grafo}
{Esiste un grafo e l'attore ha cliccato Carica Grafo}
{Il grafo viene importato da file}
{
	\begin{itemize}
			\item{} Viene aperto il file browser;
			\item{} L'attore seleziona il file da importare;
			\item{} L'attore preme su Apri Grafo.
	\end{itemize}
}
{L'importazione fallisce e l'attore visualizza un messaggio di errore \refer{UC9.1}}

\section{UC9.1: Errore importazione grafo}
\UserCase
{UC9.1}
{Utente}
{Non previsto}
{Avviene un errore durante l'importazione}
{L'importazione del grafo è fallita}
{Viene visualizzato l'errore e nessuna operazione viene eseguita}
{L'importazione del grafo è fallita e viene visualizzato un messaggio di errore}
{Non previsti}

\section{UC10: Ricerca path}
\begin{figure}[H]
	\centering
	\includegraphics[width=\textwidth]{../img/UC10.png}
	\caption{UC10: Ricerca path}
\end{figure}
\UserCase
{UC10}
{Utente}
{Non previsto}
{L'attore vuole cercare un nodo tramite un percorso nel grafo}
{Esiste un grafo corretto, l'attore ha selezionato un nodo e premuto Ricerca Path nel menu File}
{Se il path porta ad un nodo definito, esso viene evidenziato \refer{UC7.2.1}}
{
	\begin{itemize}
		\item{} Viene visualizzata una finestra con una casella di testo e un pulsante;
		\item{} L'attore inserisce il percorso da cercare;
		\item{} L'attore preme il pulsante di Ricerca;
		\item{} Se il percorso inizia dal nodo selezionato e finisce in un nodo esistente, il nodo di arrivo viene evidenziato \refer{UC7.2.1}.
 	\end{itemize}
}
{Il percorso inserito dall'attore non è corretto e viene visualizzato un errore \refer{UC10.1}}

\section{UC10.1: Errore ricerca path}
\UserCase
{UC10.1}
{Utente}
{Non previsto}
{L'attore vuole cercare un nodo tramite un percorso nel grafo}
{Il percorso inserito dall'attore è sintatticamente errato}
{Viene visualizzato l'errore a schermo e si riapre la finestra di Ricerca \refer{UC10}}
{Il percorso inserito dall'attore non è corretto e viene visualizzato un messaggio di errore}
{Non previsti}

\section{UC11: Salvataggio modifiche file JSon}
\begin{figure}[H]
	\centering
	\includegraphics[width=\textwidth]{../img/UC11.png}
	\caption{UC11: Salvataggio modifiche file JSon}
\end{figure}
\UserCase
{UC11}
{Utente}
{Non previsto}
{L'attore ha modificato gli Utterance Processor e vuole salvare il nuovo file JSon}
{Esiste un file Json correttamente caricato \refer{UC2} e l'attore ha modificato gli Utterance Processor \refer{UC6.2} \refer{UC6.3}}
{Le modifiche vengono salvate}
{
	\begin{itemize}
		\item{} L'attore apre il menu File \refer{UC1};
		\item{} L'attore preme su Salva File JSon.
	\end{itemize}
}
{L'operazione di salvataggio fallisce e viene visualizzato un errore \refer{UC11.1}}

\section{UC11.1: Errore salvataggio modifiche file JSon}
\UserCase
{UC11.1}
{Utente}
{Non previsto}
{L'attore ha provato a salvare il file JSon}
{L'operazione di salvataggio fallisce}
{Viene visualizzato l'errore e nessuna operazione viene eseguita}
{L'operazione di salvataggio fallisce e viene visualizzato un errore}
{Non previsti}

\section{UC12: Selezione Utterance Type}
\UserCase
{UC12}
{Utente}
{Non previsto}
{L'attore vuole selezionare l'Utterance Type desiderato}
{Un file JSon è stato caricato correttamente \refer{UC2}}
{Vengono mostrati gli Utterance Processors utilizzati da Speect per tale Utterance Type}
{
	\begin{itemize}
		\item{} L'attore apre il menu a tendina relativo;
		\item{} L'attore clicca sull'Utterance Type desiderato;
		\item{} Vengono mostrati a schermo i nomi degli Utterance Processor utilizzati, negli appositi spazi \ref{fig:GUI}.
	\end{itemize}
}
{Non previsti}

\section{UC13: Modifica grafo}
\begin{figure}[H]
	\centering
	\includegraphics[width=\textwidth]{../img/UC13.png}
	\caption{UC13: Modifica grafo}
\end{figure}
\UserCase
{UC13}
{Utente}
{Non previsto}
{L'attore vuole modificare il grafo}
{Viene visualizzato a schermo un grafo corretto con almeno un nodo cliccabile \refer{UC7.2}}
{Il grafo è stato modificato}
{
	L'attore per modificare un grafo può:
	\begin{itemize}
		\item{} selezionare un nodo \refer{UC13.1};
		\item{} spostare un nodo \refer{UC13.2};	
		\item{} modificare la visualizzazione delle relazioni \refer{UC13.5}.		
	\end{itemize}
}
{Non previsti}

\section{UC13.1: Selezione nodo}
\UserCase
{UC13.1}
{Utente}
{Non previsto}
{L'attore vuole selezionare un nodo per visualizzarne i dettagli}
{Viene visualizzato a schermo un grafo corretto con almeno un nodo cliccabile \refer{UC7.2}}
{Viene evidenziato il nodo del grafo e vengono mostrate le sue informazioni nella finestra apposita}
{
	\begin{itemize}
		\item{} L'attore clicca una volta sul nodo;
		\item{} Il nodo viene evidenziato con un contorno giallo;
		\item{} Nel riquadro apposito \ref{fig:GUI} vengono visualizzati i dati del grafo:
		\begin{enumerate}
			\item{} Name;
			\item{} Part of Speech;
		\end{enumerate}
		\item{} L'attore può modificare il name del nodo selezionato \refer{UC13.3};
		\item{} L'attore può modificare il PoS del nodo selezionato \refer{UC13.4}.
	\end{itemize}
}
{Non previsti}

\section{UC13.2: Spostamento nodo}
\UserCase
{UC13.2}
{Utente}
{Non previsto}
{L'attore vuole spostare graficamente un nodo}
{Un nodo è selezionato \refer{UC13.1}}
{Il nodo viene spostato}
{
	\begin{itemize}
		\item{} L'attore trascina il nodo cliccando senza rilasciare;
		\item{} Il nodo si sposta;
		\item{} L'attore rilascia il click;
		\item{} Il nodo rimane nella nuova posizione.
	\end{itemize}
}
{Non previsti}

\section{UC13.3: Modifica name nodo}
\UserCase
{UC13.3}
{Utente}
{Non previsto}
{L'attore vuole modificare il name del nodo selezionato}
{Un nodo è selezionato \refer{UC13.1}}
{Il nodo cambia name}
{
	\begin{itemize}
		\item{} L'attore seleziona la casella di testo del name;
		\item{} L'attore cancella il name precedente;
		\item{} L'attore rimuove il focus dalla casella di testo;
		\item{} Il name viene aggiornato;
		\item{} Il grafo viene aggiornato e ristampato a schermo \refer{UC7.2}.
	\end{itemize}
}
{Non previsti}

\section{UC13.4: Modifica PoS nodo}
\UserCase
{UC13.4}
{Utente}
{Non previsto}
{L'attore vuole modificare il PoS del nodo selezionato}
{Un nodo è selezionato \refer{UC13.1}}
{Il nodo cambia PoS}
{
	\begin{itemize}
		\item{} L'attore seleziona la casella di testo del PoS;
		\item{} L'attore cancella il PoS precedente;
		\item{} L'attore rimuove il focus dalla casella di testo;
		\item{} Il PoS viene aggiornato;
		\item{} Il grafo viene aggiornato e ristampato a schermo \refer{UC7.2}.
	\end{itemize}
}
{Non previsti}

\section{UC13.5: Modifica visualizzazione relazione}
\UserCase
{UC13.5}
{Utente}
{Non previsto}
{L'attore vuole filtrare le relazioni del grafo}
{Un Utterance Type è stato scelto \refer{UC12}}
{Vengono mostrati nel grafo tutti i layer di relazione selezionati}
{
	\begin{itemize}
		\item{} L'attore seleziona/deseleziona una select box adiacente ad una relazione;
		\item{} La relazione in questione viene visualizzata/nascosta;
		\item{} Il grafo viene aggiornato e ristampato a schermo \refer{UC7.2}.
	\end{itemize}
}
{Non previsti}

\end{document}
