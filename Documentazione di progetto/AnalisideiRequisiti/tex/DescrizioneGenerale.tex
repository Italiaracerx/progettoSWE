\documentclass[../AnalisideiRequisiti.tex]{subfiles}

\begin{document}
	
\chapter{Descrizione generale}

\section{Obiettivo del prodotto}

Lo scopo del progetto è la realizzazione di un applicativo software di supporto allo sviluppo di applicazioni che sfruttano la tecnologia \glossario{Speect}{Speect}. L’applicazione da produrre è un'interfaccia grafica che aiuti in special modo i programmatori nello sviluppo dei plug-in per Speect. Nell’interfaccia utente si devono poter visualizzare e modificare i grafi delle \glossario{utterance}{utterance} di Speect. 


\section{Funzioni del prodotto}
L’interfaccia grafica permetterà di:
\begin{itemize}
	\item{} Caricare i file \glossario{json}{json} utili all’inizializzazione di Speect;
	\item{} Aggiungere, modificare e eliminare gli archi dei nodi;
	\item{} Modificare i campi dei nodi;
	\item{} Disporre graficamente i nodi per permettere una lettura semplificata;
	\item{} Riordinare e rimuovere utterance processors;
	\item{} Stampare i grafi su schermo;
	\item{} Visualizzare passo passo i grafi delle varie utterance in modo sequenziale, cioè l’utente potrà decidere quando eseguire e visualizzare il grafo della successiva utterance;
	\item{} Importare ed esportare i grafi;	
	\item{} Restituire il file audio generato da Speect e salvarlo all'interno della cartella desiderata;
	\item{} Salvare le modifiche fatte ai file \verb|.json|.
\end{itemize}


\section{Caratteristiche degli utenti}
Il software si rivolge a programmatori esperti che si occupano di sviluppare plug-in per Speect. Per poter fruire correttamente del prodotto, l'utente deve dunque possedere un'approfondita conoscenza di Speect e delle sue componenti.

\section{Piattaforma di esecuzione}
Sarà garantita l'esecuzione del software su tutte le macchine desktop e laptop con sistema operativo Linux su cui siano presenti \glossario{CMAKE}{CMAKE}, \glossario{GCC}{GCC} e le librerie di \glossario{QT}{Qt}. Verranno comunque utilizzate tecnologie presenti anche su sistemi Windows, il che renderà possibile la compilazione anche in questo ambiente. Per quest'ultima piattaforma, tuttavia, non verrà fornito un manuale di installazione.

\section{Vincoli generali}
Il software realizzato deve fare uso della tecnologia Speect offerta dalla Proponente, e deve essere utilizzabile su sistema operativo Linux Ubuntu 16.04 \glossario{LTS}{LTS}.