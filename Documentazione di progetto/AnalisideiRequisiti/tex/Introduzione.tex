\documentclass[./AnalisideiRequisiti.tex]{subfiles}



\begin{document}
	
\chapter{Introduzione}
\section{Scopo del documento}
Il presente documento si pone l’obbiettivo di trattare in modo esaustivo l’esposizione dei \glossario{\textit{casi d’uso}}{caso d'uso} e di tutti quei \glossario{\textit{requisiti}}{requisiti} che si sono palesati in seguito ad un’attenta analisi del \glossario{\textit{capitolato}}{capitolato} d’appalto \textit{DeSpeect: interfaccia grafica per Speect} (C3) e a riunioni interne ed esterne verbalizzate.
\\ \noindent Nella scelta dei casi d'uso, vengono seguite le indicazioni date dalla \glossario{\textit{proponente}}{proponente} Mivoq S.R.L.

\section{Scopo del prodotto}

Lo scopo del \glossario{\textit{prodotto}}{prodotto} è quello di fornire un \glossario{\textit{interfaccia grafica}}{interfaccia grafica} utilizzabile come strumento di supporto all'utilizzo di \glossario{\textit{plugin}}{plugin} sulla piattaforma Speect. 
\\ \noindent L'utente avrà anche la possibilità di salvare i grafi generati a schermo dall'applicazione.
\\ \noindent Il funzionamento dell'applicazione sarà garantito su un sistema \glossario{\textit{Linux Ubuntu}}{Linux Ubuntu} versione 16.04 o superiore.

\section{Ambiguità}
Per evitare ogni tipo di incomprensione riguardo al linguaggio presente nei documenti viene fornito il \textit{Glossario v2.0.0} contenente la definizione dei termini in corsivo marcati con una G al pedice.

\section{Riferimenti}
\subsection*{Normativi}
\begin{itemize}
	\item \textit{Norme di Progetto v2.0.0};
	\item Capitolato: \url{http://www.math.unipd.it/~tullio/IS-1/2017/Progetto/C3.pdf}
\end{itemize}
\subsection*{Informativi}
\begin{itemize}
	\item Presentazione capitolato d'appalto: \\ \url{http://www.math.unipd.it/~tullio/IS-1/2017/Progetto/C3.pdf}
	\item Slide del corso "Ingegneria del Software" riguardanti l'Analisi dei Requisiti: \\ \url{http://www.math.unipd.it/~tullio/IS-1/2017/Dispense/L08.pdf}
	\item Slide del corso "Ingegneria del Software" riguardanti i Diagrammi dei casi d'uso: \\\ \url{http://www.math.unipd.it/~tullio/IS-1/2017/Dispense/E02.pdf}
\end{itemize}
\end{document}