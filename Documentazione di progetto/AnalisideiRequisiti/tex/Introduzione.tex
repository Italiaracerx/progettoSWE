\documentclass[./AnalisideiRequisiti.tex]{subfiles}



\begin{document}
	
\chapter{Introduzione}
\section{Scopo del documento}
Il presente documento si pone l’obiettivo di trattare in modo esaustivo l’esposizione dei \glossario{\textit{casi d’uso}}{caso d'uso} e di tutti quei \glossario{\textit{requisiti}}{requisiti} che si sono palesati in seguito ad un’attenta analisi del \glossario{\textit{capitolato}}{capitolato} d’appalto \textit{DeSpeect: interfaccia grafica per Speect} (C3) e a riunioni interne ed esterne verbalizzate.
\\ \noindent Nella scelta dei casi d'uso, vengono seguite le indicazioni date dalla \glossario{\textit{proponente}}{proponente} Mivoq S.R.L.

\section{Scopo del prodotto}

Lo scopo del \glossario{\textit{prodotto}}{prodotto} è quello di fornire un \glossario{\textit{interfaccia grafica}}{interfaccia grafica} utilizzabile come strumento di supporto all'utilizzo di \glossario{\textit{plugin}}{plugin} sulla piattaforma Speect. 
\\ \noindent L'utente avrà anche la possibilità di salvare i grafi generati a schermo dall'applicazione.
\\ \noindent Il funzionamento dell'applicazione sarà garantito su un sistema \glossario{\textit{Linux Ubuntu}}{Linux Ubuntu} versione 16.04 o superiore.

\section{Ambiguità}
Per evitare ogni tipo di incomprensione riguardo al linguaggio presente nei documenti viene fornito il \textit{Glossario v3.0.0} contenente la definizione dei termini in corsivo marcati con una G al pedice.

\section{Riferimenti}
\subsection*{Normativi}
\begin{itemize}
	\item \textbf{Norme di Progetto v3.0.0}: documento \textit{Norme di Progetto v3.0.0};
	\subitem §2.2.3 "Analisi dei requisiti";
	\subitem §3.1 "Documentazione".
	\item \textbf{Capitolato d'appalto C3}: \textit{"DeSpeect: un'interfaccia grafica per Speect"} \url{http://www.math.unipd.it/~tullio/IS-1/2017/Progetto/C3.pdf};
	\subitem Capitolato d'appalto per il progetto \textit{"DeSpeect: un'interfaccia grafica per Speect"}.

\end{itemize}
\subsection*{Informativi}
\begin{itemize}
	\item \textbf{Analisi dei Requisiti - Slide del corso}: \\ \url{http://www.math.unipd.it/~tullio/IS-1/2017/Dispense/L08.pdf};
	\subitem Definizione di attività di analisi e modalità di tracciamento e classificazione dei requisiti. 
	\item \textbf{Diagrammi dei casi d'uso - Slide del corso}: \\\ \url{http://www.math.unipd.it/~tullio/IS-1/2017/Dispense/E02.pdf};
	\subitem Definizione e sintassi dei casi d'uso.
\end{itemize}
\end{document}