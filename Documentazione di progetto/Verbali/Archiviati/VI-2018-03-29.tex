\documentclass[openany,12pt,a4paper]{article} 
\usepackage{subfiles} 
\usepackage[]{graphicx} 
\usepackage{float} 
\graphicspath{{./img/}{./../img/}} 
\usepackage{../StileDoc} 
\title{Verbale Interno} 
\author{} 
\newcommand{\versione}{} 
\loadglsentries{../Glossario/Definizioni} 
 
\begin{document} 
  \makeatletter 
  \begin{titlepage} 
    \setlength{\headsep}{0pt}   
    \begin{center} 
      \includegraphics[width=0.5\linewidth]{Logo.png}\\[1em] 
      {\huge \bfseries  \@title }\\[10ex] 
      \textbf{\Large Informazioni Documento} \\[2em] 
      \bgroup 
      \def\arraystretch{1.5} 
      \begin{tabular}{l|l} 
        \textbf{Data approvazione} & 29 marzo 2018 \\ 
        \textbf{Responsabile} & Manfredi Smaniotto \\ 
        \textbf{Redattore} & Samuele Modena \\ 
        \textbf{Verificatore} & Giulio Rossetti \\ 
        \textbf{Distribuzione} & Prof. Tullio Vardanega \\ 
         & Prof. Riccardo Cardin \\ 
         & Gruppo Graphite \\ 
        \textbf{Uso} & Interno \\ 
        \textbf{Recapito} & graphite.swe@gmail.com \\ 
      \end{tabular} 
    \egroup 
    \end{center} 
  \end{titlepage} 
  \makeatother 
 
  \thispagestyle{empty} 
  \newpage 
   
  \tableofcontents 
  \newpage 
   
  \section{Informazioni generali} 
   
  \subsection{Informazioni sull'incontro} 
   
  \begin{itemize}  
      \item \textbf{Luogo:} Videochiamata;
      \item \textbf{Data:} 2018-03-29; 
      \item \textbf{Orari} 9:30 - 11.00;
      \item \textbf{Partecipanti del gruppo:} Marco Focchiatti, Samuele Modena, Matteo Rizzo, Giulio Rossetti, Kevin Silvestri, Manfredi Smaniotto, Cristiano Tessarolo; 
      \item \textbf{Partecipanti esterni:} nessuno. 
  \end{itemize} 
 
  \subsection{Ragioni dell'incontro} 
  Confronto su stato di avanzamento lavori per quanto riguarda progettazione e codifica in relazione alla \textit{Product Baseline}.
 
  \section{Resoconto} 
  Nel corso dell'incontro sono state discusse alcune scelte progettuali inerenti l'uso di design pattern, nonché la coerenza in termini di relazione tra la parte di presentazione e di modello dell'architettura. Nello specifico, sono stati affrontati i seguenti punti:
	
  \begin{itemize}
	\item Riepilogo dell'architettura progettata e dello stato della stessa;
	\item Confronto sulle scelte progettuali attuate in termini di design pattern;
	\item Esternazione di dubbi sull'architettura \textit{Model–View–ViewModel};
	\item Richiesta di ulteriori pareri o suggerimenti all'interno del gruppo.
  \end{itemize}

  \noindent In particolare, sono stati evidenziati errori concettuali dovuti ad una parziale incomprensione del design pattern \textit{Model–View–ViewModel} da parte di alcuni componenti del gruppo. L'esternazione del problema ha permesso di risolverlo grazie ad un confronto interno e alla condivisone di maggiore documentazione a riguardo. 
 
  \section{Tracciamento delle decisioni} 
   
  \begin{itemize} 
      \item \textbf{VI-2018-03-29.1:} Rivisitazione della progettazione relativa alla parte \textit{View–ViewModel} secondo indicazioni contenute nella documentazione condivisa a questo proposito;
      \item \textbf{VI-2018-03-29.2:} Adozione del design pattern \textit{Command} all'interno del \textit{Model}. 
  \end{itemize} 
   
  \end{document}
 \ No newline at end of file 
