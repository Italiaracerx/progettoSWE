\documentclass[openany,12pt,a4paper]{article} 
\usepackage{subfiles} 
\usepackage[]{graphicx} 
\usepackage{float} 
\graphicspath{{./img/}{./../img/}} 
\usepackage{../StileDoc} 
\title{Verbale Esterno} 
\author{} 
\newcommand{\versione}{} 
\loadglsentries{../Glossario/Definizioni} 
 
\begin{document} 
  \makeatletter 
  \begin{titlepage} 
    \setlength{\headsep}{0pt}   
    \begin{center} 
      \includegraphics[width=0.5\linewidth]{Logo.png}\\[1em] 
      {\huge \bfseries  \@title }\\[10ex] 
      \textbf{\Large Informazioni Documento} \\[2em] 
      \bgroup 
      \def\arraystretch{1.5} 
      \begin{tabular}{l|l} 
        \textbf{Data approvazione} & 17 marzo 2018 \\ 
        \textbf{Responsabile} & Marco Focchiatti \\ 
        \textbf{Redattore} & Matteo Rizzo \\ 
        \textbf{Verificatore} & Kevin Silvestri \\ 
        \textbf{Distribuzione} & Prof. Tullio Vardanega \\ 
         & Prof. Riccardo Cardin \\ 
         & Gruppo Graphite \\ 
        \textbf{Uso} & Esterno \\ 
        \textbf{Recapito} & graphite.swe@gmail.com \\ 
      \end{tabular} 
    \egroup 
    \end{center} 
  \end{titlepage} 
  \makeatother 
 
  \thispagestyle{empty} 
  \newpage 
   
  \tableofcontents 
  \newpage 
   
  \section{Informazioni generali} 
   
  \subsection{Informazioni sull'incontro} 
   
  \begin{itemize}  
      \item \textbf{Luogo:} Videochiamata;
      \item \textbf{Data:} 2018-03-17; 
      \item \textbf{Orari} 14:30 - 15:30;
      \item \textbf{Partecipanti del gruppo:} Marco Focchiatti, Samuele Modena, Matteo Rizzo, Giulio Rossetti, Kevin Silvestri, Manfredi Smaniotto, Cristiano Tessarolo; 
      \item \textbf{Partecipanti esterni:} Dottor Giulio Paci. 
  \end{itemize} 
 
  \subsection{Ragioni dell'incontro} 
  Presentazione del \textit{Proof of Concept} e richiesta di chiarimenti relativi a problemi emersi. 
 
  \section{Resoconto} 
  Nel corso dell'incontro sono stati chiariti i dubbi sorti nel gruppo a seguito della realizzazione del \textit{Proof of Concept}, nello specifico sono stati affrontati i seguenti punti:
	
  \begin{itemize}
	\item Presentazione del \textit{Proof of Concept} e delle sue funzionalità;
	\item Richiesta di delucidazioni relative a dubbi tecnologici;
	\item Richiesta di ulteriori pareri o suggerimenti.
  \end{itemize}
	
  \noindent Seguono le principali domande poste durante l'incontro e la sintesi delle relative risposte ricevute:

  \begin{enumerate}
  	
  	\item \textbf{Condividiamo con lei il codice sorgente relativo al PoC e le mostriamo un'esecuzione del software illustrando le funzionalità finora sviluppate. Che cosa ne pensa di quanto realizzato?} \\
  	La struttura dell'interfaccia si avvicina molto a quella da me proposta e sembrate aver compreso il funzionamento di Speect, la strada intrapresa sembrerebbe corretta;
  	
  	\item \textbf{Durante la realizzazione del PoC è emerso un problema relativo alla generazione dell'audio da parte di Speect. Tale problema sembrerebbe prescindere dal codice prodotto dal gruppo ed essere bensì interno a Speect. Come potremmo risolvere?} \\  
  	Il problema sembrerebbe effettivamente interno a Speect, che molto probabilmente interferisce in qualche modo con le librerie di Qt. Sarà mia premura trovare una soluzione e notificarvela; \\
  	
  	N.B.: La precedente risposta è stata ottenuta successivamente ad un'analisi del codice interessato da parte del dottor Paci, che ha poi provveduto tempestivamente a fornire tramite Slack una soluzione al problema.
  	
  	\item \textbf{Ha ulteriori critiche o suggerimenti relativi a quanto illustrato?} \\
  	Se possibile, sarebbe bene che l'interfaccia grafica e i commenti al codice fossero scritti in lingua inglese.
  \end{enumerate} 
 
  \section{Tracciamento delle decisioni} 
   
  \begin{itemize} 
      \item \textbf{VE-2018-03-17.1:} Aggiunta di un requisito di qualità relativo alla necessità che l'interfaccia grafica sia in lingua inglese; 
      \item \textbf{VE-2018-03-17.2:} Aggiunta di un requisito di qualità relativo alla necessità che i commenti al codice siano in lingua inglese.
  \end{itemize} 
   
  \end{document}
 \ No newline at end of file 
