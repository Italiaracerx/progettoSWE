\documentclass[openany,12pt,a4paper]{article} 
\usepackage{subfiles} 
\usepackage[]{graphicx} 
\usepackage{float} 
\graphicspath{{./img/}{./../img/}} 
\usepackage{../StileDoc} 
\title{Verbale Esterno} 
\author{} 
\newcommand{\versione}{} 
\loadglsentries{../Glossario/Definizioni} 
 
\begin{document} 
  \makeatletter 
  \begin{titlepage} 
    \setlength{\headsep}{0pt}   
    \begin{center} 
      \includegraphics[width=0.5\linewidth]{Logo.png}\\[1em] 
      {\huge \bfseries  \@title }\\[10ex] 
      \textbf{\Large Informazioni Documento} \\[2em] 
      \bgroup 
      \def\arraystretch{1.5} 
      \begin{tabular}{l|l} 
        \textbf{Data approvazione} & 30 aprile 2018 \\ 
        \textbf{Responsabile} & Manfredi Smaniotto \\ 
        \textbf{Redattore} & Matteo Rizzo \\ 
        \textbf{Verificatore} & Cristiano Tessarolo \\ 
        \textbf{Distribuzione} & Prof. Tullio Vardanega \\ 
         & Prof. Riccardo Cardin \\ 
         & Gruppo Graphite \\ 
        \textbf{Uso} & Esterno \\ 
        \textbf{Recapito} & graphite.swe@gmail.com \\ 
      \end{tabular} 
    \egroup 
    \end{center} 
  \end{titlepage} 
  \makeatother 
 
  \thispagestyle{empty} 
  \newpage 
   
  \tableofcontents 
  \newpage 
   
  \section{Informazioni generali} 
   
  \subsection{Informazioni sull'incontro} 
   
  \begin{itemize}  
      \item \textbf{Luogo:} videochiamata;
      \item \textbf{Data:} 2018-04-30; 
      \item \textbf{Orari} 16:00 - 16:45;
      \item \textbf{Partecipanti del gruppo:} Marco Focchiatti, Samuele Modena, Matteo Rizzo, Giulio Rossetti, Kevin Silvestri, Manfredi Smaniotto, Cristiano Tessarolo; 
      \item \textbf{Partecipanti esterni:} Dottor Giulio Paci. 
  \end{itemize} 
 
  \subsection{Ragioni dell'incontro} 
  Richiesta di chiarimenti relativi al raffinamento di alcuni requisiti, selezione dei requisiti non obbligatori da implementare e modalità di collaudo. 
 
  \section{Resoconto} 
  Nel corso dell'incontro con il dottor Giulio Paci è emerso un costruttivo confronto sullo stato di avanzamento del progetto, sull'effettivo soddisfacimento dei requisiti e dei quali, tra i non obbligatori, il soddisfacimento sarebbe prioritario per la Proponente. Si è poi passati a trattare le modalità di collaudo del prodotto. Nello specifico, sono stati trattati i seguenti argomenti: 
  
  \begin{itemize}
  	\item Presentazione dello stato attuale di \textit{DeSpeect} e delle sue funzionalità;
    \item Chiarimenti sulle modalità implementative di alcuni requisiti;
    \item Selezione dei requisiti non obbligatori di priorità maggiore per la Proponente;
    \item Modalità di collaudo;
    \item Richiesta di ulteriori pareri o suggerimenti sull'andamento del progetto.
    
  \end{itemize}
  
  \noindent Seguono le principali domande, con relativa sintesi delle risposte, fatte dal gruppo alla Proponente nel corso dell'incontro.
	
  \begin{itemize}
	
	\item \textbf{Condividiamo con lei il codice sorgente relativo a DeSpeect e le mostriamo un'esecuzione del software illustrando le funzionalità sviluppate. Che cosa ne pensa di quanto realizzato?} \\
	Sono soddisfatto dell'andamento del progetto e del prodotto realizzato. Noto con piacere che avete apportato le modifiche da me richieste nel corso del precedente incontro;
	
	\item \textbf{Inizialmente si era pensato di mostrare gli errori sotto forma di pop up invasi. Data la natura dell'applicazione, abbiamo pensato che un log di errori visualizzabile in qualsiasi momento dalla finestra dell'applicazione potrebbe tuttavia risultare più utile. Quale soluzione sarebbe a lei più congeniale?} \\
	Data la natura dell'applicazione, concordo sulla scelta del log.
	
	\item \textbf{Quali sarebbero i requisiti non obbligatori che, qualora soddisfatti,  secondo lei apporterebbero maggior valore all'applicazione?} \\
	Il requisito non obbligatorio per me prioritario è la ricerca e relativa evidenziazione di un nodo del grafo (RDF9.4.1);
	
	\item \textbf{La consegna del progetto è attualmente prevista per il giorno 2018-05-11, con collaudo in data 2018-05-14. Gradisce concordare un collaudo precedente a quello ufficiale per darci ulteriore feedback?} \\
	Dato il soddisfacente stato del prodotto, il collaudo potrà tenersi in data 14 maggio. Avendo accesso al repository contenente il codice da voi prodotto, continuerò comunque a seguirne le evoluzioni dandovi ulteriore feedback di tanto in tanto.
	
  \end{itemize}
 
  \section{Tracciamento delle decisioni} 
   
  \begin{itemize} 
  	
      \item \textbf{VE-2018-04-30.1:} Soddisfacimento del requisito RDF9.4.1;
      \item \textbf{VE-2018-04-30.2:} Piccole modifiche all'interfaccia grafica su richiesta della Proponente;
      \item \textbf{VE-2018-04-30.3:} Collaudo in data 2018-05-14.
      
  \end{itemize} 
   
  \end{document}
 \ No newline at end of file 
