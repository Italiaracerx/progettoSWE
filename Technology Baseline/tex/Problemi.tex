\documentclass[./../Technology Baseline.tex]{subfiles}

\begin{document}
	
\chapter{Possibili criticità di sviluppo legate alle tecnologie}

\section{Classificazione delle criticità}

Le criticità di seguito esposte vengono catalogate secondo il seguente codice:

\begin{center}
	C[Ambito][Priorità][ID]
\end{center}

Dove:

\begin{itemize}
	\item \textbf{C}: indica che si tratta di una criticità;
	\item \textbf{Ambito}: indica l'ambito a cui appartiene la criticità in esame. Può assumere i seguenti valori:
	\begin{itemize}
		\item \textbf{S}: indica che la criticità è inerente lo \textit{sviluppo} del prodotto;
		\item \textbf{P}: indica che la criticità è inerente la \textit{progettazione} del prodotto;
		\item \textbf{O}: indica che la criticità è inerente l'\textit{organizzazione} del gruppo.
	\end{itemize}
	\item \textbf{Priorità}: indica la priorità di risoluzione della criticità. Può assumere i seguenti valori:
	\begin{itemize}
		\item \textbf{B}: priorità bassa;
		\item \textbf{M}: priorità media;
		\item \textbf{A}: priorità alta.
	\end{itemize}
	\item \textbf{ID}: rappresenta un codice numerico incrementale atto all'identificazione univoca della criticità.
\end{itemize}

\noindent Le criticità sono presentate secondo il seguente schema: 

\begin{itemize}
	\item Nome;
	\item Codice;
	\item Descrizione;
	\item Tecnologia risolutiva proposta;
	\item Tecnologie alternative scartate;
	\item \textit{Proof of Concept} (laddove lo si ritenesse necessario).
\end{itemize}

\section{Dettaglio delle criticità}

\subsection{Compilazione in C++ di QT e Speect via CMAKE}

\subsubsection{Codice}

\subsubsection{Descrizione}

\subsubsection{Tecnologia risolutiva proposta}

\subsubsection{Tecnologie alternative scartate}

\subsubsection{Proof of Concept}

\subsection{Configurazione di Speect}

\subsubsection{Codice}

\subsubsection{Descrizione}

\subsubsection{Tecnologia risolutiva proposta}

\subsubsection{Tecnologie alternative scartate}

\subsubsection{Proof of Concept}

\subsection{Manipolazione della voice configurata}

\subsubsection{Codice}

\subsubsection{Descrizione}

\subsubsection{Tecnologia risolutiva proposta}

\subsubsection{Tecnologie alternative scartate}

\subsubsection{Proof of Concept}

\subsection{Disegno e manipolazione di elementi grafici attraverso il cursore}

\subsubsection{Codice}

\subsubsection{Descrizione}

\subsubsection{Tecnologia risolutiva proposta}

\subsubsection{Tecnologie alternative scartate}

\subsubsection{Proof of Concept}

\subsection{Efficienza delle operazioni di salvataggio e ripristino di uno stato di Speect}

\subsubsection{Codice}

\subsubsection{Descrizione}

\subsubsection{Tecnologia risolutiva proposta}

\subsubsection{Tecnologie alternative scartate}

\subsubsection{Proof of Concept}

\subsection{Incapsulamento di Speect tramite oggetti}

\subsubsection{Codice}

\subsubsection{Descrizione}

\subsubsection{Tecnologia risolutiva proposta}

\subsubsection{Tecnologie alternative scartate}

\subsubsection{Proof of Concept}

\subsection{Corretta implementazione dei software per il testing automatico in relazione alle librerie QT e Speect}

\subsubsection{Codice}

\subsubsection{Descrizione}

\subsubsection{Tecnologia risolutiva proposta}

\subsubsection{Tecnologie alternative scartate}

\subsubsection{Proof of Concept}

\section{Tabella riepilogativa delle criticità}

\end{document}