\chapter{Introduzione}
\section{Scopo del Documento}
Il presente documento ha l'obiettivo di trattare in modo esaustivo l’esposizione e la motivazione delle tecnologie, dei framework e delle librerie selezionate per lo sviluppo del prodotto \textit{DeSpeect}, nonché di dimostrarne l'adeguatezza e il grado di integrazione tramite \textit{Proof of Concept} correlato agli obiettivi di progetto. Il documento analizza inoltre possibili criticità di sviluppo, progettuali o organizzative, proponendo delle soluzioni a supporto della bontà delle scelte tecnologiche intraprese. 

\section{Scopo del Prodotto}

Lo scopo del \glossario{\textit{prodotto}}{prodotto} è quello di fornire un \glossario{\textit{interfaccia grafica}}{interfaccia grafica} utilizzabile come strumento di supporto all'utilizzo di \glossario{\textit{plugin}}{plugin} sulla piattaforma Speect. 
\\ \noindent L'utente avrà anche la possibilità di salvare i grafi generati a schermo dall'applicazione.
\\ \noindent Il funzionamento dell'applicazione sarà garantito su un sistema \glossario{\textit{Linux Ubuntu}}{Linux Ubuntu} versione 16.04 o superiore.

\section{Ambiguità}
Per evitare ogni tipo di incomprensione riguardo al linguaggio presente nei documenti viene fornito il \textit{Glossario v1.0.0} contenente la definizione dei termini in corsivo marcati con una G al pedice.

\section{Riferimenti}
\subsection{Normativi}
\begin{itemize}
	\item \textit{Norme di Progetto v2.0.0};
	\item Capitolato: \url{http://www.math.unipd.it/~tullio/IS-1/2017/Progetto/C3.pdf}
\end{itemize}
\subsection{Informativi}
\begin{itemize}
	\item Presentazione capitolato d'appalto: \\ \url{http://www.math.unipd.it/~tullio/IS-1/2017/Progetto/C3.pdf}
	\item Slide del corso "Ingegneria del Software" riguardanti le regole di progetto: \\\ \url{http://www.math.unipd.it/~tullio/IS-1/2017/Dispense/P01.pdf}
\end{itemize}